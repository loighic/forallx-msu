%!TEX root = forallxyyc.tex
\graphicspath{{figures--tt/}}
\part{Truth tables}
\label{ch.TruthTables}
\addtocontents{toc}{\protect\mbox{}\protect\hrulefill\par}

\chapter{Characteristic truth tables}
\label{s:CharacteristicTruthTables}

Any non-atomic sentence of TFL is composed of atomic sentences with sentential connectives. The truth value of the compound sentence depends on the truth value of the atomic sentences that comprise it. To know the truth value of `$(D \eand E)$', for instance, you need to know the truth value of `$D$', the truth value of `$E$', and the rule for when a conjunction is true and when it is false. 

We introduced five connectives in chapter \ref{s:TFLConnectives}, and now we need to explain when sentences using each connective are true and when they are false. For convenience, we will abbreviate `true' and `false'---which are the two truth values---with `T' and `F'.

\begin{factboxy}{Truth values}
\textit{Truth values} are the logical values that a sentence can have: \textit{true} and \textit{false}.
\end{factboxy}


\paragraph{Conjunction.}\label{conjunction-tt}  For any sentences \meta{A} and \meta{B}, the conjunction (\meta{A}\eand\meta{B}) is true if and only if both \meta{A} and \meta{B} are true. If one or both of \meta{A} and \meta{B} are false, then the sentence (\meta{A}\eand\meta{B}) is false. We can summarize this in the {characteristic truth table} for conjunction:
\begin{center}
\begin{tabular}{c c |c}
\meta{A} & \meta{B} & $\meta{A}\eand\meta{B}$\\
\hline
T & T & T\Tstrut\\
T & F & F\\
F & T & F\\
F & F & F
\end{tabular}
\end{center}

Now, let's briefly go over what is in a truth table. At the top (above the horizontal line) we have, on the right, the sentence whose truth and falsity we are investigating. On the left, in alphabetical order, are all of the atomic sentences that appear in the sentence on the right. 

Below the horizontal line on the left are different combinations of \textit{true} and \textit{false} for the atomic sentences. On the first line, there is a `T' below \meta{A} and a `T' below \meta{B}. So, this line represents the situation where \meta{A} is true and \meta{B} is true. What does that mean for the sentence $(\meta{A}\eand\meta{B})$? Of course, you know. But you can look on the right side of the truth table, on line 1, and see that, in this situation, $(\meta{A}\eand\meta{B})$ is true. Likewise, on line 2, there is a `T' below \meta{A} and an `F' below \meta{B}. Checking the right side, we see that when \meta{A} is true and \meta{B} is false, $(\meta{A}\eand\meta{B})$ is false. That's the basic idea, although we will discuss this further in the next chapter. 

Note that conjunction is \emph{symmetrical}. The truth value for $\meta{A} \eand \meta{B}$ is always the same as the truth value for $\meta{B} \eand \meta{A}$.  

\paragraph{Negation.} For any sentence \meta{A}: If \meta{A} is true, then \enot\meta{A} is false. If \meta{A} is false, then \enot\meta{A} is true. We can summarize this in the characteristic truth table for negation:
\begin{center}
\begin{tabular}{c|c}
\meta{A} & \enot\meta{A}\\
\hline
T & F\Tstrut\\
F & T 
\end{tabular}
\end{center}


\paragraph{Disjunction.} Recall that `$\eor$' always represents the inclusive-or. So, for any sentences \meta{A} and \meta{B}, the disjunction $(\meta{A}\eor \meta{B})$ is true when \meta{A} is true or \meta{B} is true or both are true. The only instance when $(\meta{A}\eor \meta{B})$ is false is when both \meta{A} and \meta{B} are false. We can summarize this in the characteristic truth table for disjunction:
\begin{center}
\begin{tabular}{c c|c}
\meta{A} & \meta{B} & $\meta{A}\eor\meta{B}$ \\
\hline
T & T & T\Tstrut\\
T & F & T\\
F & T & T\\
F & F & F
\end{tabular}
\end{center}

This is a good time to explain another point. We are, in this chapter, simply defining the characteristic truth table for each connective. We have reasons for defining them these ways, and there is a consensus that these are the best definitions. But, in the end, these are the correct truth tables for each connective because these are the ways that we have chosen to set them. Conceivably, we could say that $(\meta{A}\eor \meta{B})$ is false when both \meta{A} and \meta{B} are false \textit{and} when both \meta{A} and \meta{B} are true. That would agree with the way that we, at least some of the time, use \textit{or} in English. But that's not what we've chosen to do, and so the way that $(\meta{A}\eor \meta{B})$ is defined in the truth table above is going to apply from this point forward (and similarly for all of the other connectives). 


\paragraph{Conditional.} The conditional is interesting and, for some, philosophically contentious. One way to think about the conditional is as rule: if the antecedent happens, then the consequent has to happen. So, for instance, take this conditional: 

\begin{ebullet}	
		\item[] If it is Wednesday, then I am on campus by 10:00 am.
	\end{ebullet}

\noindent This sentence is obviously true when (1) it is Wednesday and I am on campus by 10:00 am. Conversely, this sentence is false when (2) it is Wednesday, but I am not on campus by 10:00 am. (If that happens, the rule has been broken.) Those two scenarios are represented by lines 1 and 2 in the characteristic truth table for the conditional, which is as follows.

\begin{center}\label{characteristic-tt-conditional}
\begin{tabular}{c c|c}
\meta{A} & \meta{B} & $\meta{A}\eif\meta{B}$\\
\hline
T & T & T\Tstrut\\
T & F & F\\
F & T & T\\
F & F & T
\end{tabular}
\end{center}

For the other two scenarios, we have to concentrate a bit. 

\begin{ebullet}	
\item[(3)] Our conditional is also true when it is not Wednesday (let's say it's Tuesday), but I'm on campus by 10:00 am. In this case, the rule \textit{if it is Wednesday, then I am on campus by 10:00 am} hasn't been broken; it just doesn't apply. So, when the antecedent is false and the consequent is true, the conditional is true. That's represented by line 3 of the characteristic truth table for the conditional. 
\item[(4)] Similarly, when it is not Wednesday, and I am not on campus by 10:00 am, the rule hasn't been broken. It is still in force. It just hasn't been invoked at all. So even though the antecedent didn't happen and the consequent didn't happen, the conditional is still true. (Again, it's still true that \textit{if it is Wednesday, then I am on campus by 10:00 am}, but since, let's say, it's Saturday and, at 10:00 am, I am still at home in bed, it's false that `it is Wednesday' and it's false that `I am on campus by 10:00 am'.) This scenario is represented on line 4 of the characteristic truth table.
\end{ebullet}

Hopefully, you understood the explanation just given for each of the four scenarios. The conditional is philosophically contentious, however, because every conditional is not as simple and straightforward as `if it is Wednesday, then I am on campus by 10:00 am'. Take a conditional where the antecedent is always false: `if the queen of England is on the moon, then Mississippi State University is in Starkville.' This isn't much of a rule, but it is a conditional. And since the antecedent is false and the consequent is true, the sentence is true. Even stranger, consider this conditional: `if the queen of England is on the moon, then pigs can fly.' Now, the antecedent is false and the consequent is false, but, as is shown on line 4 of the characteristic truth table, the sentence is true. 

Sometimes the truth values for the antecedent, the consequent, and the whole conditional make sense (as in our first example) and sometimes they seem odd. That has generated philosophical debate, but it actually does not present a problem for us. The conditional is precisely defined by its characteristic truth table. We, then, simply use that definition, and we don't have to make any decisions about whether a particular conditional is odd or should really be true or false. 

Finally, notice that, unlike the conjunction and the disjunction, the conditional is \emph{asymmetrical}. You cannot switch the antecedent and consequent without changing the meaning of the sentence. This is because $\meta{A}\eif\meta{B}$ (`if it is Wednesday, then I am on campus by 10:00 am') has a different truth table than $\meta{B}\eif\meta{A}$ (if I am on campus by 10:00 am, then it is Wednesday').


\paragraph{Biconditional.}\label{biconditional-2} As we said in section \ref{s:biconditional-1}, the biconditional is equivalent to the conjunction of a conditional running in each direction---that is, to $(\meta{A} \eif \meta{B}) \eand (\meta{A}\eif\meta{B})$. Consequently, on every line where both $\meta{A}\eif\meta{B}$ is true and $\meta{B}\eif\meta{A}$ is true, $\meta{A}\eiff\meta{B}$ is true. On every line where either $\meta{A}\eif\meta{B}$ is false or $\meta{B}\eif\meta{A}$ is false, $\meta{A}\eiff\meta{B}$ is false. That yields the following characteristic truth table for the biconditional.

\begin{center}
\begin{tabular}{c c|c}
\meta{A} & \meta{B} & $\meta{A}\eiff\meta{B}$\\
\hline
T & T & T\Tstrut\\
T & F & F\\
F & T & F\\
F & F & T
\end{tabular}
\end{center}


%%%%%%%%%%%%%%%%%%%%%%%%%%%%%%%%%%%%%%%%%%%%%%
%%%%%%%%%%%%%%%%%%%%%%%%%%%%%%%%%%%%%%%%%%%%%%

%%%%%%%%%%%%%%%%%%%%%%%%%%%%%%%%%%%%%%%%%%%%%%
%%%%%%%%%%%%%%%%%%%%%%%%%%%%%%%%%%%%%%%%%%%%%%


\chapter{Complete truth tables}
\label{s:CompleteTruthTables}

In chapter \ref{s:CharacteristicTruthTables}, we examined the characteristic truth table for each connective. Those truth tables define the truth values for each connective. Now that we have those definitions, we can investigate when other, more complex sentences are true and false---for instance, ones like `$(H\eand I)\eif H$' and `$(M \eand (N \eor P))$', which we will go through in this chapter. Once we understand how to create truth tables, we can investigate other properties of sentences, which we will do in chapters \ref{s:SemanticConcepts} and \ref{c:tt-validity}. 

Before we begin, we will define \define{valuation}.

\begin{factboxy}{Valuation}
A \textit{valuation} is any assignment of truth values to particular atomic sentences of TFL. Each row of a truth table represents a possible valuation. The entire truth table represents all possible valuations.
\end{factboxy}

\noindent Thus, the truth table provides us with a way of finding the truth values of complex sentences on each possible valuation---that is, for every combination of `true' and `false' for every atomic sentence. 

\section{An example}\label{s:tt-example}
Consider the sentence `$(H\eand I)\eif H$', which contains three atomic sentences, although only two different ones. We set up the truth table for this sentence by putting $H$ and $I$ on the left side of the vertical line and `$(H\eand I)\eif H$' on the right. (Although $H$ appears twice in `$(H\eand I)\eif H$', we only need one $H$ on the left.) Below the $H$ and $I$ on the left side, we put every combination of `T' and `F'. That is, on one line we want to have T and T; on another we want to have T and F; on another F and T; and on another F and F. 

Since we have two atomic sentences on the left, there are four combinations of true and false. For consistency, the Ts and Fs should always be listed this way: (a) in the first column (the one next to the vertical line), they alternate T, F, T, F; (b) in the second column, they alternate in pairs, T, T, F, F; and (c) if there are more than two atomic sentences , then more columns and more rows are needed, but the pattern remains the same. 
\begin{center}
\begin{tabular}{c c|d e e e f}
$H$&$I$&$(H$&\eand&$I)$&\eif&$H$\\
\hline
 T & T\Tstrut\\
 T & F\\
 F & T\\
 F & F
\end{tabular}
\end{center}

Once the left side of the truth table is completed, we begin on the right side. First, we copy the truth values for the atomic sentences to the right-side of the truth table. For the $H$, that gives us:
\begin{center}
\begin{tabular}{c c|d e e e f}
$H$&$I$&$(H$&\eand&$I)$&\eif&$H$\\
\hline
 T & T & {T} & &  & & {T}\Tstrut\\
 T & F & {T} & &  & & {T}\\
 F & T & {F} & &  & & {F}\\
 F & F & {F} & &  & & {F}
\end{tabular}
\end{center}
And then we add the truth values for $I$:
\begin{center}
\begin{tabular}{c c|d e e e f}
$H$&$I$&$(H$&\eand&$I)$&\eif&$H$\\
\hline
 T & T & {T} & & {T} & & {T}\Tstrut\\
 T & F & {T} & & {F} & & {T}\\
 F & T & {F} & & {T} & & {F}\\
 F & F & {F} & & {F} & & {F}
\end{tabular}
\end{center}

Now consider the subsentence `$(H\eand I)$'. This is a conjunction, and our next step is to determine the truth values  for just this subsentence. For this, we turn to the characteristic truth table for conjunction (p.~\pageref{conjunction-tt}). On the first line `$H$' and `$I$' are both true, and so we put `T' on the first line below the `$\eand$'. 

\begin{center}
\begin{tabular}{c c|d e e e f}
$\textcolor{light-gray}{H}$&$\textcolor{light-gray}{I}$&$(H$&\eand&$I)$&\textcolor{light-gray}{\eif}&\textcolor{light-gray}{$H$}\\
\hline
 \textcolor{light-gray}{T} & \textcolor{light-gray}{T} & T & \textbf{\textcolor{red2}{T}} & T & & \textcolor{light-gray}{T}\Tstrut\\
 \textcolor{light-gray}{T} & \textcolor{light-gray}{F} & \textcolor{light-gray}{T} &  & \textcolor{light-gray}{F} & & \textcolor{light-gray}{T}\\
 \textcolor{light-gray}{F} & \textcolor{light-gray}{T} & \textcolor{light-gray}{F} &  & \textcolor{light-gray}{T} & & \textcolor{light-gray}{F}\\
 \textcolor{light-gray}{F} & \textcolor{light-gray}{F} & \textcolor{light-gray}{F} &  & \textcolor{light-gray}{F} & & \textcolor{light-gray}{F}
\end{tabular}
\end{center}

\noindent On the second line, `$H$' is true and `$I$' is false. That means that `$(H\eand I)$' is false, and so we put `F' on the second line below the `$\eand$'. 

\begin{center}
\begin{tabular}{c c|d e e e f}
$\textcolor{light-gray}{H}$&$\textcolor{light-gray}{I}$&$(H$&\eand&$I)$&\textcolor{light-gray}{\eif}&\textcolor{light-gray}{$H$}\\
\hline
 \textcolor{light-gray}{T} & \textcolor{light-gray}{T} & T & \textbf{T} & T & & \textcolor{light-gray}{T}\Tstrut\\
 \textcolor{light-gray}{T} & \textcolor{light-gray}{F} & T & \textbf{\textcolor{red2}{F}} & F & & \textcolor{light-gray}{T}\\
 \textcolor{light-gray}{F} & \textcolor{light-gray}{T} & \textcolor{light-gray}{F} &  & \textcolor{light-gray}{T} & & \textcolor{light-gray}{F}\\
 \textcolor{light-gray}{F} & \textcolor{light-gray}{F} & \textcolor{light-gray}{F} &  & \textcolor{light-gray}{F} & & \textcolor{light-gray}{F}
\end{tabular}
\end{center}

\noindent Using to the characteristic truth table for conjunction (p.~\pageref{conjunction-tt}), we fill in the truth values for the third and fourth lines, and that completes the column under the `$\eand$'.

\begin{center}
\begin{tabular}{c c|d e e e f}
$\textcolor{light-gray}{H}$&$\textcolor{light-gray}{I}$&$(H$&\eand&$I)$&\textcolor{light-gray}{\eif}&\textcolor{light-gray}{$H$}\\
\hline
 \textcolor{light-gray}{T} & \textcolor{light-gray}{T} & T & \textbf{T} & T & & \textcolor{light-gray}{T}\Tstrut\\
 \textcolor{light-gray}{T} & \textcolor{light-gray}{F} & T & \textbf{F} & F & & \textcolor{light-gray}{T}\\
 \textcolor{light-gray}{F} & \textcolor{light-gray}{T} & F & \textbf{\textcolor{red2}{F}} & T & & \textcolor{light-gray}{F}\\
 \textcolor{light-gray}{F} & \textcolor{light-gray}{F} & F & \textbf{\textcolor{red2}{F}} & F & & \textcolor{light-gray}{F}
\end{tabular}
\end{center}

The full sentence, `$(H\eand I)\eif H$', is a conditional, which means that the `\eif' is the main logical operator and the column under the `\eif' is the one we fill in last. `$(H \eand I)$' is the antecedent and `$H$' is the consequent, and so we need to look at the truth values below the `\eif' and the `$H$' and refer to the characteristic truth table for the conditional (p.~\pageref{characteristic-tt-conditional}). On the first line, `$(H\eand I)$' is true and `$H$' is true, and so we put a `T' beneath the `\eif'.

\begin{center}
\begin{tabular}{c c|d e e e f}
$\textcolor{light-gray}{H}$&$\textcolor{light-gray}{I}$&$(H$&\eand&$I)$&\eif&$H$\\
\hline
 \textcolor{light-gray}{T} & \textcolor{light-gray}{T} & \textcolor{light-gray}{T} & T & \textcolor{light-gray}{T} & \textbf{\textcolor{red2}{T}} & T\Tstrut\\
 \textcolor{light-gray}{T} & \textcolor{light-gray}{F} & \textcolor{light-gray}{T} & \textcolor{light-gray}{F} & \textcolor{light-gray}{F} & & \textcolor{light-gray}{T}\\
 \textcolor{light-gray}{F} & \textcolor{light-gray}{T} & \textcolor{light-gray}{F} & \textcolor{light-gray}{F} & \textcolor{light-gray}{T} & & \textcolor{light-gray}{F}\\
 \textcolor{light-gray}{F} & \textcolor{light-gray}{F} & \textcolor{light-gray}{F} & \textcolor{light-gray}{F} & \textcolor{light-gray}{F} & & \textcolor{light-gray}{F}
\end{tabular}
\end{center}

On the second row, `$(H\eand I)$' is false and `$H$' is true. (That's the scenario on the third line of the characteristic truth table for the conditional (p.~\pageref{characteristic-tt-conditional}), not the second.) A conditional is, actually, always true when the antecedent is false, and so we put a `T' in the second row beneath the conditional symbol. 

\begin{center}
\begin{tabular}{c c|d e e e f}
$\textcolor{light-gray}{H}$&$\textcolor{light-gray}{I}$&$(H$&\eand&$I)$&\eif&$H$\\
\hline
 \textcolor{light-gray}{T} & \textcolor{light-gray}{T} & \textcolor{light-gray}{T} & T & \textcolor{light-gray}{T} & \textbf{T} & T \Tstrut\\
 \textcolor{light-gray}{T} & \textcolor{light-gray}{F} & \textcolor{light-gray}{T} & F & \textcolor{light-gray}{F} & \textbf{\textcolor{red2}{T}} & T\\
 \textcolor{light-gray}{F} & \textcolor{light-gray}{T} & \textcolor{light-gray}{F} & \textcolor{light-gray}{F} & \textcolor{light-gray}{T} & & \textcolor{light-gray}{F}\\
 \textcolor{light-gray}{F} & \textcolor{light-gray}{F} & \textcolor{light-gray}{F} & \textcolor{light-gray}{F} & \textcolor{light-gray}{F} & & \textcolor{light-gray}{F}
\end{tabular}
\end{center}

\noindent On the third and fourth rows, `$(H\eand I)$' is false, and so again, we put `T' below the `\eif' on each line. (On both of these lines, the antecedent is false and the consequent is false, and so that corresponds to line four in the characteristic truth table for the conditional.)

\begin{center}
\begin{tabular}{c c|d e e e f}
$\textcolor{light-gray}{H}$&$\textcolor{light-gray}{I}$&$(H$&\eand&$I)$&\eif&$H$\\
\hline 
 \textcolor{light-gray}{T} & \textcolor{light-gray}{T} & \textcolor{light-gray}{T} & T & \textcolor{light-gray}{T} & \textbf{T} & T \Tstrut\\
 \textcolor{light-gray}{T} & \textcolor{light-gray}{F} & \textcolor{light-gray}{T} & F & \textcolor{light-gray}{F} & \textbf{T} & T\\
 \textcolor{light-gray}{F} & \textcolor{light-gray}{T} & \textcolor{light-gray}{F} & F & \textcolor{light-gray}{T} & \textbf{\textcolor{red2}{T}} & F\\
 \textcolor{light-gray}{F} & \textcolor{light-gray}{F} & \textcolor{light-gray}{F} & F & \textcolor{light-gray}{F} & \textbf{\textcolor{red2}{T}} & F
\end{tabular}
\end{center}


Since the `\eif' is the main logical operator, we've now determine the truth values for this sentence. The column of `T's beneath the `\eif' tells us that the sentence `$(H \eand I)\eif H$' is true regardless of the truth values of `$H$' and `$I$'. Those atomic sentences can be true or false in any combination, and the full sentence, `$(H \eand I)\eif H$', remains true. Since we have considered all four possible assignments of truth and falsity to `$H$' and `$I$', we can say that `$(H \eand I)\eif H$' is true on every \textit{valuation}.

Most of the columns underneath the sentence are only there for bookkeeping purposes. Technically, one could skip filling in the columns under the atomic sentences (i.e., the letters) and just consult the columns on the left side of the truth table---although trying that is liable to invite mistakes. In any case, the column that matters most is the column beneath the \emph{main logical operator} for the sentence, since this tells us the truth value of the entire sentence. We have emphasized it in the last truth table above, by putting this column in bold. When you work through truth tables yourself, you should similarly emphasize it (perhaps by underlining or circling it).

\section{Building complete truth tables}\label{s:tt-example2}
A \define{complete truth table} has a line for every possible combination of \textit{true} and \textit{false} for the atomic sentences that compose the full sentence. Each line represents a \emph{valuation}, and a complete truth table has a line for all the different valuations. 

\newglossaryentry{complete truth table}
{
name=complete truth table,
description={A table that gives all the possible \glspl{truth value} for a \gls{sentence of TFL} or sentences in TFL, with a line for every possible \gls{valuation} of all atomic sentences}
}

The size of the complete truth table depends on the number of different atomic sentences in the table. A sentence that contains only one atomic sentence requires only two rows, as in the characteristic truth table for negation. This is true even if the same letter is repeated many times, as in the sentence
`$[(C\eiff C) \eif C] \eand \enot(C \eif C)$'.
The complete truth table requires only two lines because there are only two possibilities: `$C$' can be true or it can be false. The truth table for this sentence looks like this:
\begin{center}
\begin{tabular}{c| d e e e e e e e e e e e e e e f}
$C$&$[($&$C$&\eiff&$C$&$)$&\eif&$C$&$]$&\eand&\enot&$($&$C$&\eif&$C$&$)$\\
\hline
 T &    & T &  T  & T &   & T  & T & &\TTbf{\textcolor{red2}{F}}&  F& &   T &  T  & T & \Tstrut\\
 F &    & F &  T  & F &   & F  & F & &\TTbf{\textcolor{red2}{F}}&  F& &   F &  T  & F &   \\
\end{tabular}
\end{center}
Looking at the column underneath the main logical operator, we see that the sentence is false on both rows of the table; i.e., the sentence is false regardless of whether `$C$' is true or false. It is false on every valuation.

A sentence that contains two atomic sentences requires four lines for a complete truth table, as in the characteristic truth tables for our binary connectives, and as in the complete truth table for `$(H \eand I)\eif H$'.

A sentence that contains three atomic sentences requires eight lines, as  shown in the table right below. Notice that the `T's and `F's in the columns below $N$ and $P$ (on the left side) follow the same pattern as the example in the previous section. The column under the $P$, meanwhile, has four `T's and then four `F's.  
\begin{center}
\begin{tabular}{c c c|d e e e f}
$M$&$N$&$P$&$M$&\eand&$(N$&\eor&$P)$\\
\hline
T & T & T & T & \TTbf{\textcolor{red2}{T}} & T & T & T\Tstrut\\
T & T & F & T & \TTbf{\textcolor{red2}{T}} & T & T & F\\
T & F & T & T & \TTbf{\textcolor{red2}{T}} & F & T & T\\
T & F & F & T & \TTbf{\textcolor{red2}{F}} & F & F & F\\\arrayrulecolor{light-gray}\hline
F & T & T & F & \TTbf{\textcolor{red2}{F}} & T & T & T\Tstrut\\
F & T & F & F & \TTbf{\textcolor{red2}{F}} & T & T & F\\
F & F & T & F & \TTbf{\textcolor{red2}{F}} & F & T & T\\
F & F & F & F & \TTbf{\textcolor{red2}{F}} & F & F & F
\end{tabular}
\end{center}
From this table, we know that the sentence `$M\eand(N\eor P)$' can be true or false, depending on the truth values of `$M$', `$N$', and `$P$'.

A complete truth table for a sentence that contains four different atomic sentences requires 16 lines. If the sentence has five different letters, the truth table will have 32 lines. If it has six letters, it will have 64 lines, and so on. The rule here is this: for $n$ different atomic sentences, the truth table for the sentence must have $2^n$ lines.

But whether a truth table has four lines or 64 lines, every truth table for the same sentence should be the same. Hence, the columns on the left side have to be set as follows. Below the last letter (the one next to the vertical line), alternate between `T' and `F'. In the next column to the left, write two `T's, then two `F's, and repeat. For the third atomic sentence, write four `T's followed by four `F's. This yields an eight line truth table like the one above. For a 16 line truth table, the next column of atomic sentences should have eight `T's followed by eight `F's. For a 32 line table, the next column would have 16 `T's followed by 16 `F's, and so on.


\section{Some more examples}

\begin{earg}
\item[\ex{9.3.1}] To create a truth table for `$(P \eiff Q) \eif (P \eor Q)$', we first, as always, fill in the columns below each $P$ and $Q$. Next, we fill in the columns under the `$\eiff$' and the `$\eor$' (in either order). 
\begin{center}
\begin{tabular}{c c|d e e e e e f}
$P$ &$Q$&  ($P$& $\eiff$& $Q$)& $\eif$& ($P$& $\eor$& $Q$)\\
\hline
T& T&      \textcolor{light-gray}{T}& T&  \textcolor{light-gray}{T}&   &   \textcolor{light-gray}{T}& T& \textcolor{light-gray}{T}\Tstrut\\  
T& F&      \textcolor{light-gray}{T}& F&  \textcolor{light-gray}{F}&   &   \textcolor{light-gray}{T}& T& \textcolor{light-gray}{F}\\     
F& T&      \textcolor{light-gray}{F}& F&  \textcolor{light-gray}{T}&   &   \textcolor{light-gray}{F}& T& \textcolor{light-gray}{T}\\     
F& F&      \textcolor{light-gray}{F}& T&  \textcolor{light-gray}{F}&   &   \textcolor{light-gray}{F}& F&  \textcolor{light-gray}{F}\\ 
\end{tabular}
\end{center}
Once we have those columns complete, we finish the truth table by filling in the column under the `$\eif$', which we do by looking at the column under the `$\eiff$' and the column under the `$\eor$'.
\begin{center}
\begin{tabular}{c c|d e e e e e f}
$P$ &$Q$&  ($P$& $\eiff$& $Q$)& $\eif$& ($P$& $\eor$& $Q$)\\
\hline
T& T&      \textcolor{light-gray}{T}& T&  \textcolor{light-gray}{T}&   \TTbf{\textcolor{red2}{T}}&   \textcolor{light-gray}{T}& T& \textcolor{light-gray}{T}\Tstrut\\  
T& F&      \textcolor{light-gray}{T}& F&  \textcolor{light-gray}{F}&   \TTbf{\textcolor{red2}{T}}&   \textcolor{light-gray}{T}& T& \textcolor{light-gray}{F}\\     
F& T&      \textcolor{light-gray}{F}& F&  \textcolor{light-gray}{T}&   \TTbf{\textcolor{red2}{T}}&   \textcolor{light-gray}{F}& T& \textcolor{light-gray}{T}\\     
F& F&      \textcolor{light-gray}{F}& T&  \textcolor{light-gray}{F}&   \TTbf{\textcolor{red2}{F}}&   \textcolor{light-gray}{F}& F&  \textcolor{light-gray}{F}\\ 
\end{tabular}
\end{center}

\item[\ex{9.3.2}]To make a truth table for `$P \eand \enot Q$', we first (after filling in the columns below the $P$ and $Q$) fill in the column under the $\enot$. To do that, we look at the column under the $Q$.
\begin{center}
\begin{tabular}{c c|d e e f}
$P$& $Q$&  ($P$& $\eand$& $\enot$& $Q$)\\ 
\hline
T& T&    \textcolor{light-gray}{T}& & \TTbf{F}& T\Tstrut\\
T& F&    \textcolor{light-gray}{T}& & \TTbf{T}& F\\
F& T&    \textcolor{light-gray}{F}& & \TTbf{F}& T\\
F& F&    \textcolor{light-gray}{F}& & \TTbf{T}& F\\
\end{tabular}
\end{center}
Then, to complete the truth table, we fill in the column under the `$\eand$', which we do by looking at the column under the $P$ and the column under the `$\enot$'.
\begin{center}
\begin{tabular}{c c|d e e f}
$P$& $Q$&  ($P$& $\eand$& $\enot$& $Q$)\\ 
\hline
T& T&    T& \TTbf{\textcolor{red2}{F}}& F& \textcolor{light-gray}{T}\Tstrut\\
T& F&    T& \TTbf{\textcolor{red2}{T}}& T& \textcolor{light-gray}{F}\\
F& T&    F& \TTbf{\textcolor{red2}{F}}& F& \textcolor{light-gray}{T}\\
F& F&    F& \TTbf{\textcolor{red2}{F}}& T& \textcolor{light-gray}{F}\\
\end{tabular}
\end{center}

\item[\ex{9.3.3}]For `$\enot(P \eif Q)$', after we have filled in the columns under the `$P$' and the `$Q$', we fill in the column under the `$\eif$'.
\begin{center}
\begin{tabular}{c c|d e e f}
$P$& $Q$& $\enot$& ($P$& $\eif$& $Q$)\\ 
\hline
T& T&  &   \textcolor{light-gray}{T}& T& \textcolor{light-gray}{T}\Tstrut\\   
T& F&  &   \textcolor{light-gray}{T}& F& \textcolor{light-gray}{F}\\   
F& T&  &   \textcolor{light-gray}{F}& T& \textcolor{light-gray}{T}\\   
F& F&  &   \textcolor{light-gray}{F}& T& \textcolor{light-gray}{F}\\  
\end{tabular}
\end{center}
Then, to complete the table, we fill in the column under the `$\enot$'. To fill in that column, we look at the column under the `$\eif$'.
\begin{center}
\begin{tabular}{c c|d e e f}
$P$& $Q$& $\enot$& ($P$& $\eif$& $Q$)\\ 
\hline
T& T&  \TTbf{\textcolor{red2}{F}}&   \textcolor{light-gray}{T}& T& \textcolor{light-gray}{T}\Tstrut\\   
T& F&  \TTbf{\textcolor{red2}{T}}&   \textcolor{light-gray}{T}& F& \textcolor{light-gray}{F}\\   
F& T&  \TTbf{\textcolor{red2}{F}}&   \textcolor{light-gray}{F}& T& \textcolor{light-gray}{T}\\   
F& F&  \TTbf{\textcolor{red2}{F}}&   \textcolor{light-gray}{F}& T& \textcolor{light-gray}{F}\\  
\end{tabular}
\end{center}


\item[\ex{9.3.4}]In `$(P \eand \enot Q) \eor Q$', the `$\eor$' is the main logical operator, and so will fill in the column under it last. First (after we have filled in the columns under the `$P$' and the `$Q$'), we fill in the column under the `$\enot$'.
\begin{center}
\begin{tabular}{c c|d e e e e f}
$P$& $Q$&   ($P$& $\eand$& $\enot$& $Q$)& $\eor$& $Q$\\  
\hline
T& T&      \textcolor{light-gray}{T}& & F& \textcolor{light-gray}{T}&   & \textcolor{light-gray}{T}\Tstrut\\
T& F&      \textcolor{light-gray}{T}& & T& \textcolor{light-gray}{F}&   & \textcolor{light-gray}{F}\\   
F& T&      \textcolor{light-gray}{F}& & F& \textcolor{light-gray}{T}&   & \textcolor{light-gray}{T}\\   
F& F&      \textcolor{light-gray}{F}& & T& \textcolor{light-gray}{F}&   & \textcolor{light-gray}{F}\\ 
\end{tabular}
\end{center}
Next, while looking at the column under the `$P$' and under the `$\enot$', we fill in the column under the `$\eand$'.
\begin{center}
\begin{tabular}{c c|d e e e e f}
$P$& $Q$&   ($P$& $\eand$& $\enot$& $Q$)& $\eor$& $Q$\\  
\hline
T& T&      T& \TTbf{F}& F& \textcolor{light-gray}{T}&   & \textcolor{light-gray}{T}\Tstrut\\
T& F&      T& \TTbf{T}& T& \textcolor{light-gray}{F}&   & \textcolor{light-gray}{F}\\   
F& T&      F& \TTbf{F}& F& \textcolor{light-gray}{T}&   & \textcolor{light-gray}{T}\\   
F& F&      F& \TTbf{F}& T& \textcolor{light-gray}{F}&   & \textcolor{light-gray}{F}\\ 
\end{tabular}
\end{center}
Then last, we fill in the column under the `$\eor$' while looking at the column under the `$\eand$' and under the `$Q$'.
\begin{center}
\begin{tabular}{c c|d e e e e f}
$P$& $Q$&   ($P$& $\eand$& $\enot$& $Q$)& $\eor$& $Q$\\  
\hline
T& T&      \textcolor{light-gray}{T}& F& \textcolor{light-gray}{F}& \textcolor{light-gray}{T}&   \TTbf{\textcolor{red2}{T}}& T\Tstrut\\
T& F&      \textcolor{light-gray}{T}& T& \textcolor{light-gray}{T}& \textcolor{light-gray}{F}&   \TTbf{\textcolor{red2}{T}}& F\\   
F& T&      \textcolor{light-gray}{F}& F& \textcolor{light-gray}{F}& \textcolor{light-gray}{T}&   \TTbf{\textcolor{red2}{T}}& T\\   
F& F&      \textcolor{light-gray}{F}& F& \textcolor{light-gray}{T}& \textcolor{light-gray}{F}&   \TTbf{\textcolor{red2}{F}}& F\\ 
\end{tabular}
\end{center}

\end{earg}


\section{Truth tables in Carnap}\label{s:ttCarnap-intro}

You should practice making truth tables on paper, but you also need to make them using the software package Carnap (https://carnap.io/). Using Carnap is pretty straightforward, and it's made easier because the left side of the truth table is completed for you. (See figure \ref{fig:tt-1}.) On the right side, below each atomic sentence and connective, you have the option of selecting a `T' or an `F'. (See figure \ref{fig:tt-2}.)  

%%%%%%%%%%%%%%%%%%%%%%%%%%%%%%%%%%%%%%
\begin{figure}[h]
%\makebox[\textwidth][l]{		% lets the table go outside the margins
\begin{subfigure}[t]{0.47\textwidth}
\includegraphics[width=6.0cm]{tt-1.png}
\subcaption{}
\label{fig:tt-1}
\end{subfigure}
\hspace{.25cm}
\begin{subfigure}[t]{0.47\textwidth}
\includegraphics[width=5.95cm]{tt-2.png}
\subcaption{}
\label{fig:tt-2}
\end{subfigure}
\caption{}
%}
\end{figure} 
%%%%%%%%%%%%%%%%%%%%%%%%%%%%%%%%%%%%%%

Most often (although not always), the problems in Carnap will be set up so that you will only be able to submit your answers when they are correct. At those times, once the truth table is complete, you will select `Check'. Carnap will tell you ``Success!'' or ``Something's not quite right.'' It is easy to make a mistake when filling in a truth table, and so if something is not quite right, then you have to inspect every truth value until you find the mistake. Then select `Check' again. Once Carnap confirms that the truth table is correct, select `Submit'. \textbf{Don't forget to select to `submit' after you complete every truth table correctly.}

%%%%%%%%%%%%%%%%%%%%%%%%%%%
\begin{SCfigure}
\centering
\includegraphics[width=6.5cm]{tt-6a.png} % 2.5 inches
\caption{A completed and verified truth table in Carnap.}
\label{fig:tt-6}
\end{SCfigure}
%%%%%%%%%%%%%%%%%%%%%%%%%%%


%\newpage

\section{More about brackets}\label{s:MoreBracketingConventions}
Consider these two sentences:
	\begin{align*}
		((A \eand B) \eand C)\\
		(A \eand (B \eand C))
	\end{align*}
They are equivalent, and so, consequently, it will never make any difference from the perspective of their truth value---which is all that really matters for TFL---which of the two sentences we assert (or deny). But even though the order of the brackets does not matter to the truth of the sentence, we should not just drop them. The expression
	\begin{align*}
		A \eand B \eand C
	\end{align*}
is ambiguous between the two sentences above.  The same observation holds for disjunctions. The following sentences are logically equivalent:
	\begin{align*}
		((A \eor B) \eor C)\\
		(A \eor (B \eor C))
	\end{align*}
But we should not simply write:
	\begin{align*}
		A \eor B \eor C
	\end{align*}
In fact, it is a specific fact about the characteristic truth table of $\eor$ and $\eand$ that guarantees that any two conjunctions (or disjunctions) of the same sentences are truth functionally equivalent, however you place the brackets. \emph{But be careful}. These two sentences have \emph{different} truth tables:
	\begin{align*}
		((A \eif B) \eif C)\\
		(A \eif (B \eif C))
	\end{align*}
So if we were to write:
	\begin{align*}
		A \eif B \eif C
	\end{align*}
it would be dangerously ambiguous. So we must not do the same with conditionals. Equally, these sentences have different truth tables:
	\begin{align*}
		((A \eor B) \eand C)\\
		(A \eor (B \eand C))
	\end{align*}
So if we were to write:
	\begin{align*}
		A \eor B \eand C
	\end{align*}
it would be dangerously ambiguous. \emph{Never write this.} The moral is never drop brackets, unless there is no possibility of ambiguity.


%%%%%%%%%%%%%%%%%%%%%%%%%%%%%%%%%%%%%%%%%%%%%%
%%%%%%%%%%%%%%%%%%%%%%%%%%%%%%%%%%%%%%%%%%%%%%

%  exercises for `Complete truth tables'

%%%%%%%%%%%%%%%%%%%%%%%%%%%%%%%%%%%%%%%%%%%%%%
%%%%%%%%%%%%%%%%%%%%%%%%%%%%%%%%%%%%%%%%%%%%%%



%\practiceproblems
\section{Practice exercises}\label{pr.TT.TTorC}
\setcounter{ProbPart}{0}

\problempart
Make a complete truth table for each sentence.
\begin{earg}
\item $A \eif A$ %taut
\item $C \eif\enot C$ %contingent
\item $(A \eiff B) \eiff \enot(A\eiff \enot B)$ %tautology
\item $(A \eif B) \eor (B \eif A)$ % taut
\item $(A \eand B) \eif (B \eor A)$  %taut
\item $\enot(A \eor B) \eiff (\enot A \eand \enot B)$ %taut
\item $\bigl[(A\eand B) \eand\enot(A\eand B)\bigr] \eand C$ %contradiction
\item $[(A \eand B) \eand C] \eif B$ %taut
\item $\enot\bigl[(C\eor A) \eor B\bigr]$ %contingent
\end{earg}

\problempart
Check all the claims made in \S\ref{s:MoreBracketingConventions}. That is, show that each of these statements is true.
\begin{earg}
	\item `$((A \eand B) \eand C)$' and `$(A \eand (B \eand C))$' have the same truth table
	\item `$((A \eor B) \eor C)$' and `$(A \eor (B \eor C))$' have the same truth table
	\item `$((A \eor B) \eand C)$' and `$(A \eor (B \eand C))$' do not have the same truth table
	\item `$((A \eif B) \eif C)$' and `$(A \eif (B \eif C))$' do not have the same truth table
\end{earg}
Also, check whether:
\begin{earg}
	\item[5.] `$((A \eiff B) \eiff C)$' and `$(A \eiff (B \eiff C))$' have the same truth table
\end{earg}

\problempart
Make truth tables for the following sentences, and mark the column that represents the possible truth values for the whole sentence.

\begin{earg}

\item $\enot (N \eiff (P \eif N))$

\item $\enot [(X \eand Y) \eor (X \eor Y)]$

\item $(A \eif B) \eiff (\enot B\eiff \enot A)$

\item $[C \eiff (D \eor E)] \eand \enot C$

\item $\enot(C \eand (B \eand H)) \eiff (C \eor (B \eor H))$

\item	$(D \eand \enot D) \eif G $

\item	$(\enot P \eor \enot R) \eiff R $

\item	$\enot \enot (\enot A \eand \enot B)  $

\item 	$[(D \eand H) \eif J] \eif \enot(D \eor H) $

\item	$\enot [(D \eiff F) \eiff G] \eif (\enot D \eand F) $

\end{earg}


%%%%%%%%%%%%%%%%%%%%%%%%%%%%%%%%%%%%%%%%%%%%%%
%%%%%%%%%%%%%%%%%%%%%%%%%%%%%%%%%%%%%%%%%%%%%%

%  answers for `Complete truth tables'

%%%%%%%%%%%%%%%%%%%%%%%%%%%%%%%%%%%%%%%%%%%%%%
%%%%%%%%%%%%%%%%%%%%%%%%%%%%%%%%%%%%%%%%%%%%%%

\section{Answers}
\setcounter{ProbPart}{0}

\problempart\label{pr.TT.TTorC-a}

\begin{earg}

\item $A \eif A$ %taut
\myanswer{\begin{center}
\begin{tabular}{c | def}
$A$ & $A$&$\eif$&$A$\\
\hline
 T & T&\TTbf{T}&T\Tstrut\\
F & F&\TTbf{T}&F
\end{tabular}
\end{center}
}

\item $C \eif\enot C$ %contingent
\myanswer{\begin{center}
\begin{tabular}{c | d e e f}
$C$ & $C$&$\eif$&$\enot$&$C$\\
\hline
 T & T & \TTbf{F}& F& T\Tstrut\\
F & F & \TTbf{T}& T& F\\
\end{tabular}
\end{center}}

\item $(A \eiff B) \eiff \enot(A\eiff \enot B)$ %tautology
\myanswer{\begin{center}
\begin{tabular}{c c | d e e e e e e e f}
$A$ & $B$&$(A$&$\eiff$&$B)$&$\eiff$&$\enot$&$(A$&$\eiff$&$\enot$&$B)$ \\
\hline
T & T & T & T & T & \TTbf{T} & T & T & F & F & T\Tstrut\\
T & F & T & F & F & \TTbf{T} & F & T & T & T & F\\
F & T & F & F & T & \TTbf{T} & F & F & T & F & T\\
F & F & F & T & F & \TTbf{T} & T & F & F & T & F
 \end{tabular}
\end{center}}

\item $(A \eif B) \eor (B \eif A)$ % taut
\myanswer{
\begin{center}
\begin{tabular}{c c | d e e e e e f}
$A$ & $B$&$(A$&$\eif$&$B)$&$\eor$&$(B$&$\eif$&$A)$ \\
\hline
T & T & T & T & T & \TTbf{T} & T & T & T\Tstrut\\
T & F & T & F & F & \TTbf{T} & F & T & T\\
F & T & F & T & T & \TTbf{T} & T & F & F \\
F & F & F & T & F & \TTbf{T} & F & T & F
 \end{tabular}
\end{center}}

\item $(A \eand B) \eif (B \eor A)$  %taut
\myanswer{
\begin{center}
\begin{tabular}{c c | d e e e e e f}
$A$ & $B$&$(A$&$\eand$&$B)$&$\eif$&$(B$&$\eor$&$A)$ \\
\hline
T & T & T & T & T & \TTbf{T} & T & T & T\Tstrut\\
T & F & T & F & F & \TTbf{T} & F & T & T\\
F & T & F & F & T & \TTbf{T} & T & T & F \\
F & F & F & F & F & \TTbf{T} & F & F & F
 \end{tabular}
\end{center}}
\item $\enot(A \eor B) \eiff (\enot A \eand \enot B)$ %taut

\myanswer{\begin{center}
\begin{tabular}{c c | d e e e e e e e e f}
$A$ & $B$&$\enot$&$(A$&$\eor$&$B)$&$\eiff$&$(\enot$&$A$&$\eand$&$\enot$&$B)$\\
\hline
T & T & F & T & T & T & \TTbf{T} & F & T & F & F & T\Tstrut\\
T & F & F & T& T & F & \TTbf{T} & F & T & F & T & F\\
F & T & F & F & T & T & \TTbf{T} & T & F & F & F & T\\
F & F & T & F & F & F & \TTbf{T} & T & F & T & T & F
 \end{tabular}
\end{center}}


\item $\bigl[(A\eand B) \eand\enot(A\eand B)\bigr] \eand C$ %contradiction
\myanswer{\begin{center}
\begin{tabular}{c c c | d e e e e e e e e f}
$A$ & $B$&$C$&$\bigl[(A$&$\eand$&$B)$&$ \eand$&$\enot$&$(A$&$\eand$&$B)\bigr]$&$\eand$&$C$\\
\hline
T & T & T & T & T & T & F & F & T & T & T & \TTbf{F} & T\Tstrut\\
T & T & F & T& T & T & F & F & T & T & T & \TTbf{F}& F\\
T & F & T & T & F & F & F & T & T & F & F & \TTbf{F} & T\\
T & F & F & T & F & F & F & T & T & F & F & \TTbf{F} & F\\\arrayrulecolor{light-gray}\hline
F & T & T & F & F & T & F & T & F & F & T & \TTbf{F} & T\Tstrut\\
F & T & F & F & F & T & F & T & F & F & T & \TTbf{F} & F\\
F & F & T & F & F & F & F & T & F & F & F & \TTbf{F} & T\\
F & F & F & F & F & F & F & T & F & F & F & \TTbf{F} & F
\end{tabular}
\end{center}}

\item $[(A \eand B) \eand C] \eif B$ %taut
\myanswer{\begin{center}
\begin{tabular}{c c c | d e e e e e f}
$A$ & $B$&$C$&$[(A$&$\eand$&$B)$&$\eand$&$C]$&$\eif$&$B$\\
\hline
T & T & T & T & T & T & T & T & \TTbf{T} & T\Tstrut\\
T & T & F & T & T & T & F & F & \TTbf{T} & T\\
T & F & T & T & F & F & F & T & \TTbf{T} & F\\
T & F & F & T & F & F & F & F & \TTbf{T} & F\\\arrayrulecolor{light-gray}\hline
F & T & T & F & F & T & F & T & \TTbf{T} & T\Tstrut\\
F & T & F & F & F & T & F & F & \TTbf{T} & T\\
F & F & T & F & F & F & F & T & \TTbf{T} & F\\
F & F & F & F & F & F & F & F & \TTbf{T} & F\\
\end{tabular}
\end{center}}

\item $\enot\bigl[(C\eor A) \eor B\bigr]$ %contingent
\myanswer{\begin{center}
\begin{tabular}{c c c | d e e e e f}
$A$ & $B$&$C$&$\enot\bigl[($&$C$&$\eor$&$A)$&$\eor$&$B\bigr]$\\
\hline
T & T & T & \TTbf{F} & T & T & T & T & T\Tstrut\\
T & T & F & \TTbf{F} & F & T & T & T & T\\
T & F & T & \TTbf{F} & T & T & T & T & F\\
T & F & F & \TTbf{F} & F & T & T & T & F\\\arrayrulecolor{light-gray}\hline
F & T & T & \TTbf{F} & T & T & F & T & T\Tstrut\\
F & T & F & \TTbf{F} & F & F & F & T & T\\
F & F & T & \TTbf{F} & T & T & F & T & F\\
F & F & F & \TTbf{T} & F & F & F & F & F
\end{tabular}
\end{center}}
\end{earg}


\problempart

\begin{earg}
\item `$((A \eand B) \eand C)$' and `$(A \eand (B \eand C))$' have the same truth table
\myanswer{\begin{center}
\begin{tabular}{c c c | d e e e f | d e e e f }
$A$ & $B$ & $C$ & $(A$&$\eand$& $B)$ &$ \eand$ & $C$ & $A$ & $\eand$ & $(B$&$\eand$&$C)$\\
\hline
T & T & T & T & T & T &  \TTbf{T} & T &T & \TTbf{T} & T & T& T \Tstrut\\
T & T & F & T& T & T &  \TTbf{F} & F & T& \TTbf{F} & T & F& F\\
T & F & T & T & F & F &  \TTbf{F} & T & T & \TTbf{F} & F & F & T \\
T & F & F &  T & F & F &  \TTbf{F} & F & T& \TTbf{F} & F & F & F\\\arrayrulecolor{light-gray}\hline
F & T & T & F & F & T & \TTbf{F} & T & F& \TTbf{F} & T & T & T\Tstrut\\
F & T & F & F & F & T & \TTbf{F} & F & F& \TTbf{F} &  T & F & F\\
F & F & T & F & F & F & \TTbf{F} & T & F& \TTbf{F} & F& F & T\\
F & F & F & F & F & F & \TTbf{F} & F & F& \TTbf{F} &  F& F & F\\
\end{tabular}
\end{center}}

\item `$((A \eor B) \eor C)$' and `$(A \eor (B \eor C))$' have the same truth table
\myanswer{\begin{center}
\begin{tabular}{c c c | d e e e f | d e e e f }
$A$ & $B$ & $C$ & $(A$&$\eor$& $B)$ &$ \eor$ & $C$ & $A$ & $\eor$ & $(B$&$\eor$&$C)$\\
\hline
T & T & T & T & T & T &  \TTbf{T} & T &T & \TTbf{T} & T & T& T \Tstrut\\
T & T & F & T& T & T &  \TTbf{T} & F & T& \TTbf{T} & T & T& F\\
T & F & T & T & T & F &  \TTbf{T} & T & T & \TTbf{T} & F & T & T \\
T & F & F &  T & T& F &  \TTbf{T} & F & T& \TTbf{T} & F & F & F\\\arrayrulecolor{light-gray}\hline
F & T & T & F & T & T & \TTbf{T} & T & F& \TTbf{T} & T & T & T\Tstrut\\
F & T & F & F & T & T & \TTbf{T} & F & F& \TTbf{T} &  T & T & F\\
F & F & T & F & F & F & \TTbf{T} & T & F& \TTbf{T} & F& T & T\\
F & F & F & F & F & F & \TTbf{F} & F & F& \TTbf{F} &  F& F & F\\
\end{tabular}
\end{center}}

\item `$((A \eor B) \eand C)$' and `$(A \eor (B \eand C))$' do not have the same truth table
\myanswer{\begin{center}
\begin{tabular}{c c c | d e e e f | d e e e f }
$A$ & $B$ & $C$ & $(A$&$\eor$& $B)$ &$ \eand$ & $C$ & $A$ & $\eor$ & $(B$&$\eand$&$C)$\\
\hline
T & T & T & T & T & T &  \TTbf{T} & T &T & \TTbf{T} & T & T& T \Tstrut\\
T & T & F & T& T & T &  \TTbf{F} & F & T& \TTbf{T} & T & F& F\\
T & F & T & T & T & F &  \TTbf{T} & T & T & \TTbf{T} & F & F & T \\
T & F & F &  T & T& F &  \TTbf{F} & F & T& \TTbf{T} & F & F & F\\\arrayrulecolor{light-gray}\hline
F & T & T & F & T & T & \TTbf{T} & T & F& \TTbf{T} & T & T & T\Tstrut\\
F & T & F & F & T & T & \TTbf{F} & F & F& \TTbf{F} &  T & F & F\\
F & F & T & F & F & F & \TTbf{F} & T & F& \TTbf{F} & F& F & T\\
F & F & F & F & F & F & \TTbf{F} & F & F& \TTbf{F} &  F& F & F\\
\end{tabular}
\end{center}}

\item `$((A \eif B) \eif C)$' and `$(A \eif (B \eif C))$' do not have the same truth table
\myanswer{\begin{center}
\begin{tabular}{c c c | d e e e f | d e e e f }
$A$ & $B$ & $C$ & $(A$&$\eif$& $B)$ &$ \eif$ & $C$ & $A$ & $\eif$ & $(B$&$\eif$&$C)$\\
\hline
T & T & T & T & T & T &  \TTbf{T} & T &T & \TTbf{T} & T & T& T \Tstrut\\
T & T & F & T& T & T &  \TTbf{F} & F & T& \TTbf{F} & T & F& F\\
T & F & T & T & F & F &  \TTbf{T} & T & T & \TTbf{T} & F & T & T \\
T & F & F &  T & F& F &  \TTbf{T} & F & T& \TTbf{T} & F & T & F\\\arrayrulecolor{light-gray}\hline
F & T & T & F & T & T & \TTbf{T} & T & F& \TTbf{T} & T & T & T\Tstrut\\
F & T & F & F & T & T & \TTbf{F} & F & F& \TTbf{T} &  T & F & F\\
F & F & T & F & T & F & \TTbf{T} & T & F& \TTbf{T} & F& T & T\\
F & F & F & F & T & F & \TTbf{F} & F & F& \TTbf{T} &  F& T & F\\
\end{tabular}
\end{center}}

\item `$((A \eiff B) \eiff C)$' and `$(A \eiff (B \eiff C))$' have the same truth table.\\
\myanswer{\begin{center}
\begin{tabular}{c c c | d e e e f | d e e e f }
$A$ & $B$ & $C$ & $(A$&$\eiff$& $B)$ &$ \eiff$ & $C$ & $A$ & $\eiff$ & $(B$&$\eiff$&$C)$\\
\hline
T & T & T & T & T & T &  \TTbf{T} & T &T & \TTbf{T} & T & T& T \Tstrut\\
T & T & F & T& T & T &  \TTbf{F} & F & T& \TTbf{F} & T & F& F\\
T & F & T & T & F & F &  \TTbf{F} & T & T & \TTbf{F} & F & F & T \\
T & F & F &  T & F& F &  \TTbf{T} & F & T& \TTbf{T} & F & T & F\\\arrayrulecolor{light-gray}\hline
F & T & T & F & F & T & \TTbf{F} & T & F& \TTbf{F} & T & T & T\Tstrut\\
F & T & F & F & F & T & \TTbf{T} & F & F& \TTbf{T} &  T & F & F\\
F & F & T & F & T & F & \TTbf{T} & T & F& \TTbf{T} & F& F & T\\
F & F & F & F & T & F & \TTbf{F} & F & F& \TTbf{F} &  F& T & F\\
\end{tabular}
\end{center}}
\end{earg}

\problempart
Complete truth tables for each sentence with the column that represents the possible truth values for the whole sentence marked.

\begin{earg}

\item $\enot (N \eiff (P \eif N))$
\myanswer{
\begin{center}
\begin{tabular}{c c | |c|c c c c c c}
\cline{3-3}
$N$ & $P$ &	\enot 	&	(N 	&	\eiff	&	(P 	&	\eif	&	N))	\\ 
\hline
T & T	& 	F 		&	T	&	T	&	T	&	T	&	T\Tstrut\\
T & F	& 	F 		&	T	&	T	&	F	&	T	&	T	\\
F & T	& 	F 		&	F	&	T	&	T	&	F	&	F	\\
F & F	& 	T 		&	F	&	F	&	F	&	T	&	F	\\
\cline{3-3}
\end{tabular}
\end{center}
}


\item $\enot [(X \eand Y) \eor (X \eor Y)]$
\myanswer{
\begin{center}
\begin{tabular}{c c || c |ccccccc}
\cline{3-3}
$X$ & $Y$	&	\enot	&	 [(X 	&	\eand& 	Y) 	&	\eor 	&	(X 	&	\eor 	&	Y)] \\
\hline
T & T	&	F	&	T	&	T	&	T	&	T	&	T	&	T	&	T\Tstrut	\\
T & F &	F	&	T	&	F	&	F	&	T	&	T	&	T	&	F	\\
F & T	&	F	&	F	&	F	&	T	&	T	&	F	&	T	&	T	\\
F & F	&	T	&	F	&	F	&	F	&	F	&	F	&	F	&	F	\\
\cline{3-3}
\end{tabular}
\end{center}
}


\item $(A \eif B) \eiff (\enot B\eiff \enot A)$
\myanswer{
\begin{center}
\begin{tabular}{c c | ccc|c|ccccc}
\cline{6-6}
$A$ & $B$	&	(A 	&	\eif	&	B)	&	 \eiff 	&	(\enot&	B 	&	\eiff 	&	 \enot 	& 	 A) \\
\hline
T & T	&	T	&	T	&	T	&	T		&	F	 &	T	&	T	&	F		&	T\Tstrut	\\	
T & F	&	T	&	F	&	F	&	T		&	T	 &	F	&	F	&	F		&	T	\\
F & T	&	F	&	T	&	T	&	F		&	F	 &	T	&	F	&	T		&	F	\\
F & F	&	F	&	T	&	F	&	T		&	T	 &	F	&	T	&	T		&	F	\\
\cline{6-6}
\end{tabular}
\end{center}
}

\item $[C \eiff (D \eor E)] \eand \enot C$
\myanswer{
\begin{center}
\begin{tabular}{c c c | ccccc|c|cc}
\cline{9-9}
$C$ & $D$ & $E$	&	[C 	&	\eiff 	&	(D 	&	\eor 	&	E)] 	&	\eand 	&	 \enot 	&	 C \\
\hline
T& T& T	&	T	&	T	&	T	&	T	&	T	&	F		&	F		&	T\Tstrut\\
T& T& F	&	T	&	T	&	T	&	T	&	F	&	F		&	F		&	T	\\
T& F& T	&	T	&	T	&	F	&	T	&	T	&	F		&	F		&	T	\\
T& F& F	&	T	&	F	&	F	&	F	&	F	&	F		&	F		&	T	\\\arrayrulecolor{light-gray}\hline
F& T& T	&	F	&	F	&	T	&	T	&	T	&	F		&	T		&	F	\Tstrut\\
F& T& F	&	F	&	F	&	T	&	T	&	F	&	F		&	T		&	F	\\
F& F& T	&	F	&	F	&	F	&	T	&	T	&	F		&	T		&	F	\\
F& F&	 F   &	F	&	T	&	F	&	F	&	F	&	T		&	T		&	F	\\
\cline{9-9}
\end{tabular}
\end{center}
}

\item $\enot(C \eand (B \eand H)) \eiff (C \eor (B \eor H))$
\myanswer{
\begin{center}
\begin{tabular}{c c c | cccccc|c|ccccc}
\cline{10-10}
$C$& $B$& $H$	&\enot&	(C 	&\eand &	(B 	&	 \eand 	&	 H))	&	\eiff 	&	(C 	& \eor 	& (B 	& \eor	& H))	\\
\hline
T& T& T	&F	   &	T	&	  T &	T	&	T		&	T	&	F	&	T	&	T	&	T	&	T	&	T\Tstrut\\
T& T& F	&T	   &	T	&	  F &	T	&	F		&	F	&	T	&	T	&	T	&	T	&	T	&	F	\\	
T& F& T	&T	   &	T	&	 F  &	F	&	F		&	T	&	T	&	T	&	T	&	F	&	T	&	T	\\
T& F& F	&T	   &	T	&	 F  &	F	&	F		&	F	&	T	&	T	&	T	&	F	&	F	&	F	\\\arrayrulecolor{light-gray}\hline
F& T& T	&T	   &	F	&	F   &	T	&	T		&	T	&	T	&	F	&	T	&	T	&	T	&	T	\Tstrut\\
F& T& F	&T	   &	F	&	F   &	T	&	F		&	F	&	T	&	F	&	T	&	T	&	T	&	F	\\
F& F& T	&T	   &	F	&	F   &	F	&	F		&	T	&	T	&	F	&	T	&	F	&	T	&	T	\\
F& F& F	&T	   &	F	&	F   &	F	&	F		&	F	&	F	&	F	&	F	&	F	&	F	&	F	\\
\cline{10-10}
\end{tabular}
\end{center}
}

\vspace{1em}

\item	$(D \eand \enot D) \eif G $
%\vspace{1em}
\myanswer{
\begin{center}
\begin{tabular}{c c | cccc|c|c}
\cline{7-7}
$D$& $G$	&	(D 	&	 \eand 	& 	 \enot	&	 D) 	&	 \eif 	&	 G \\
 \hline
T&  T	&	T	&	F		&	F		&	T	&	T	&	T\Tstrut	\\
T&  F	&	T	&	F		&	F		&	T	&	T	&	F	\\
F&  T	&	F	&	F		&	T		&	F	&	T	&	T	\\
F&  F	&	F	&	F		&	T		&	F	&	T	&	F	\\
\cline{7-7}
\end{tabular}
\end{center}
}
\vspace{1em}


\item	$(\enot P \eor \enot R) \eiff R $
\myanswer{
\begin{center}
\begin{tabular}{c  c | ccccc|c|c}
\cline{8-8}
$P$& $R$	&	(\enot 	&	P 	&	\eor 	&	\enot 	& 	 R) 	& 	\eiff 	&	 R \\
\hline
T& T	&	F		&	T	&	F	&	F		&	T	&	F	&	T\Tstrut	\\
T& F	&	F		&	T	&	T	&	T		&	F	&	F	&	F	\\
F& T	&	T		&	F	&	T	&	F		&	T	&	T	&	T	\\
F& F	&	T		&	F	&	T	&	T		&	F	&	F	&	F	\\
\cline{8-8}
\end{tabular}
\end{center}
}
\vspace{1em}


\item	$\enot \enot (\enot A \eand \enot B)  $
\myanswer{
\begin{center}
\begin{tabular}{c c | |c|cccccc}
\cline{3-3}
$A$& $B$	&	\enot		&	 \enot 	&	(\enot 	& 	 A 	& \eand 	& 	\enot 	&	 B)  \\
 \hline
T& T	&	F		&	T		&	F		&	T	&	F	&	F		&	T\Tstrut	\\
T& F	&	F		&	T		&	F		&	T	&	F	&	T		&	F	\\
F& T	&	F		&	T		&	T		&	F	&	F	&	F		&	T	\\
F& F	&	T		&	F		&	T		&	F	&	T	&	T		&	F	\\
\cline{3-3}
\end{tabular}
\end{center}
}
\vspace{1em}


\item 	$[(D \eand H) \eif J] \eif \enot(D \eor H) $
\myanswer{
\begin{center}
\begin{tabular}{c  c  c | ccccc|c|cccc}
\cline{9-9}
$D$& $H$ & $J$ 	&	[(D 	& 	 \eand 	& 	 H)	& 	\eif 	&	J] 	&	\eif 	&	 \enot 	&	(D 	&	 \eor 	& H) \\
\hline
T&  T&  T	&	T	&	T		&	T	&	T	&	T	&	F	&	F		&	T	&	T		&T\Tstrut\\
T&  T&  F	&	T	&	T		&	T	&	F	&	F	&	T	&	F		&	T	&	T		&T	\\
T&  F&  T	&	T	&	F		&	F	&	T	&	T	&	F	&	F		&	T	&	T		&F	\\
T&  F&  F	&	T	&	F		&	F	&	T	&	F	&	F	&	F		&	T	&	T		&F	\\\arrayrulecolor{light-gray}\hline
F&  T&  T	&	F	&	F		&	T	&	T	&	T	&	F	&	F		&	F	&	T		&T	\Tstrut\\
F&  T&  F	&	F	&	F		&	T	&	T	&	F	&	F	&	F		&	F	&	T		&T	\\
F&  F&  T	&	F	&	F		&	F	&	T	&	T	&	T	&	T		&	F	&	F		&F	\\
F&  F&  F	&	F	&	F		&	F	&	T	&	F	&	T	&	T		&	F	&	F		&F	\\
\cline{9-9}
\end{tabular}
\end{center}
}
	
\vspace{1em}


\item	$\enot [(D \eiff F) \eiff G] \eif (\enot D \eand F) $
\myanswer{
\begin{center}
\begin{tabular}{c c  c | cccccc|c|cccc}
\cline{10-10}
$D$& $F$ & $G$ 	&	\enot 	&	[(D 	&	\eiff 	&	F) 	&	\eiff 	&	 G]	& 	\eif 	 &	(\enot 	& 	D 	 & 	 \eand &F) \\ 
\hline
T&  T&  T	&	F		&	T	&	T	&	T	&	T	&	T	&	T	&	F		&	T	&	F	&T	\Tstrut\\
T&  T&  F	&	T		&	T	&	T	&	T	&	F	&	F	&	F	&	F		&	T	&	F	&T	\\
T&  F&  T	&	T		&	T	&	F	&	F	&	F	&	T	&	F	&	F		&	T	&	F	&F	\\
T&  F&  F	&	F		&	T	&	F	&	F	&	T	&	F	&	T	&	F		&	T	&	F	&F	\\\arrayrulecolor{light-gray}\hline
F&  T&  T	&	T		&	F	&	F	&	T	&	F	&	T	&	T	&	T		&	F	&	T	&T	\Tstrut\\
F&  T&  F	&	F		&	F	&	F	&	T	&	T	&	F	&	T	&	T		&	F	&	T	&T	\\
F&  F&  T	&	F		&	F	&	T	&	F	&	T	&	T	&	T	&	T		&	F	&	F	&F	\\
F&  F&  F	&	T		&	F	&	T	&	F	&	F	&	F	&	F	&	T		&	F	&	F	&F	\\
\cline{10-10}
\end{tabular}
\end{center}
}
%\vspace{1em}
\end{earg}





%%%%%%%%%%%%%%%%%%%%%%%%%%%%%%%%%%%%%%%%%%%%%%
%%%%%%%%%%%%%%%%%%%%%%%%%%%%%%%%%%%%%%%%%%%%%%


\chapter{Semantic concepts}
\label{s:SemanticConcepts}

As we did in the previous chapter, we begin with the definition of \define{valuation}.

\begin{factboxy}{Valuation}
A \textit{valuation} is any assignment of truth values to particular atomic sentences of TFL. Each row of a truth table represents a possible valuation. The entire truth table represents all possible valuations.
\end{factboxy}

Let's say that we are going to create a truth table for `$P \eor \enot Q$'. On the first line of the truth table, you may recall, we make $P =$ `true' and $Q =$ `true'. That is one valuation. On the second line, we make $P =$ `true' and $Q =$ `false'. That is another valuation. The assignments of `true' and `false' to `$P$` and `$Q$' on lines 3 and 4, then, are the remaining possible valuations, when we have a sentence containing only two atomic sentences. (See figure \ref{fig:valuation}.)

%\begin{center}
%\begin{tabular}{c c|d e e f }
%$P$&$Q$& $P$& \eor& \enot& $Q$\\
%\hline
% T & T & T & \TTbf{T} & F & T\Tstrut\\
% T & F & T & \TTbf{T} & T & F \\
% F & T & F & \TTbf{F} & F & T \\
% F & F & F & \TTbf{T} & T & F
%\end{tabular}
%\end{center}


%%%%%%%%%%%%%%%%%%%%%%%%%%%
\begin{figure}
\centering
\includegraphics[width=6.5cm]{this_is_a_valuation.pdf}
\caption{For the TFL sentence `$P \eor \enot Q$', $P =$ `true' and $Q =$ `true' is one valuation. On this valuation, `$P \eor \enot Q$' is true.
The other valuations are $P =$ `true' and $Q =$ `false', $P =$ `false' and $Q =$ `true', and $P =$ `false' and $Q =$ `false'.}
\label{fig:valuation}
\end{figure}
%%%%%%%%%%%%%%%%%%%%%%%%%%%


In the previous chapter, we used truth tables to determine the truth value of any TFL sentence---for each possible valuation. In this chapter, we will extend this type of analysis. We will examine six properties that apply (or may apply) to either single TFL sentences (\textit{tautology}, \textit{contradiction}, and \textit{contingent}) or sets of TFL sentences (\textit{equivalent}, \textit{jointly consistent}, and \textit{jointly inconsistent}). For each, we use a truth table to determine which property applies.


\section{Tautologies and contradictions}
Recall from \S\ref{s:nec-truth} that a \emph{necessary truth} is a sentence that must be true, a \emph{necessary falsehood} is a sentence that must be false, and a sentence that is neither a necessary truth or a necessary falsehood we call \textit{contingent}. The first two have surrogates in TFL. We will start with a surrogate for necessary truth.

\begin{factboxy}{Tautology}
$\meta{A}$ (a sentence in TFL) is a \define{tautology} if and only if it is true on every valuation.

\tcblower

\footnotesize{Recall from \ref{s:Metavariables}, that $\meta{A}$ stands for any possible sentence of TFL (as do $\meta{B}$, $\meta{C}$, $\meta{D}$, etc.). Hence, `$\meta{A}$' can stand for `$P \eor Q$' or `$(P \eiff \enot R) \eand T$' or anything else.}
\end{factboxy}

We can determine whether a sentence is a tautology using a truth table. If the sentence is true on every line of a complete truth table (that is, if there are only `T's under the main connective), then it is true on every valuation. And if it is true on every valuation, it is a tautology. The example in \S\ref{s:tt-example}, `$(H \eand I) \eif H$', for instance, is a tautology. 

\begin{center}
\begin{tabular}{c c|d e e e f}
${H}$&${I}$&$(H$&\eand&$I)$&\eif&$H$\\
\hline 
{T} & {T} & {T} & T & {T} & \textbf{T} & T \Tstrut\\
{T} & {F} & {T} & F & {F} & \textbf{T} & T\\
{F} & {T} & {F} & F & {T} & \textbf{T} & F\\
{F} & {F} & {F} & F & {F} & \textbf{T} & F
\end{tabular}
\end{center}


\textit{Tautology} is only a surrogate, however, for \textit{necessary truth}. There are some necessary truths that we cannot adequately symbolize in TFL. An example is `$2 + 2 = 4$'. This \emph{must} be true, but if we try to symbolize it in TFL, the best we can offer is an atomic sentence, perhaps,
	\begin{ekey}
		\item[F] $2 + 2 = 4$ 
	\end{ekey}
But an atomic sentence by itself cannot be a tautology. (To see this, try making a truth table for just `$F$'.) Still, if we can adequately symbolize some English sentence using a TFL sentence, and that TFL sentence is a tautology, then the English sentence expresses a necessary truth.

We have a similar surrogate for \textit{necessary falsehood}.
\begin{factboxy}{Contradiction}
$\meta{A}$ is a \define{contradiction} if and only if it is false on every valuation.
\end{factboxy}

\noindent Again, we can determine whether a sentence is a contradiction with a truth table. If the sentence is false on every line of a complete truth table, then it is false on every valuation, and so it is a contradiction. The standard example of a contradiction is `$P \eand \enot P$'. Since we have only one letter in this sentence, it is only a two line truth table, but on each line, the sentence is false. 

\begin{center}
\begin{tabular}{c|d e e f }
$P$& $P$ & $\eand$ &\enot&$P$\\
\hline
 T & T & \TTbf{F} & F & T\Tstrut\\
 F & F & \TTbf{F} & T & F 
\end{tabular}
\end{center}
Similarly, although its truth table has four lines, `$(P \eor Q) \eiff (\enot P \eand \enot Q)$' is a contradiction. 
\begin{center}
\begin{tabular}{c c|d e e e e e e e f }
$P$& $Q$& $(P$ &\eor &$Q)$ &\eiff &(\enot &$P$ &\eand &\enot &$Q)$\\
\hline
T &T      &T &T &T   &\TTbf{F}    &F &T &F &F &T\Tstrut\\     
T &F      &T &T &F   &\TTbf{F}    &F &T &F &T &F\\     
F &T      &F &T &T   &\TTbf{F}    &T &F &F &F &T\\     
F &F      &F &F &F   &\TTbf{F}    &T &F &T &T &F  
\end{tabular}
\end{center}

In \S\ref{s:nec-truth}, we defined \define{contingent} as ``a sentence that is capable of being true and capable of being false (in different circumstances, of course).'' A truth table, then, provides us with those different circumstances. A sentence that is true on some valuations (or even just on one) and false on the others is contingent. Or, we can also say: any sentence that is neither a tautology nor a contradiction is contingent. $\enot(P \eor Q)$, for instance, is contingent.
\begin{center}
\begin{tabular}{c c|d e e f }
$P$&$Q$&\enot&$(P$&\eor&$Q)$\\
\hline
 T & T & \TTbf{F} & T & T & T\Tstrut\\
 T & F & \TTbf{F} & T & T & F \\
 F & T & \TTbf{F} & F & T & T \\
 F & F & \TTbf{T} & F & F & F
\end{tabular}
\end{center}



\section{Equivalence}\label{equivalence--tt}
When we have two sentences, three possible logical relations can exist between the sentences. (Actually, there are more than three, but we'll focus on three.) The first is \define{equivalence}.

\begin{factboxy}{Equivalence}
$\meta{A}$ and $\meta{B}$ are \define{equivalent} if and only if, for every valuation, their truth values agree (that is, if and only if there is no valuation for which they have opposite truth values). 
		
Equivalently, if $(\meta{A} \eiff \meta{B})$ is a tautology, then $\meta{A}$ and $\meta{B}$ are \define{equivalent}.
\end{factboxy}

Consider the sentences `$\enot(P \eor Q)$' and `$\enot P \eand \enot Q$'. Are they logically equivalent? To find out, we construct a truth table containing both sentences.
\begin{center}
\begin{tabular}{c c|d e e f |d e e e f}
$P$&$Q$&\enot&$(P$&\eor&$Q)$&\enot&$P$&\eand&\enot&$Q$\\
\hline
 T & T & \TTbf{F} & T & T & T & F & T & \TTbf{F} & F & T\Tstrut\\
 T & F & \TTbf{F} & T & T & F & F & T & \TTbf{F} & T & F\\
 F & T & \TTbf{F} & F & T & T & T & F & \TTbf{F} & F & T\\
 F & F & \TTbf{T} & F & F & F & T & F & \TTbf{T} & T & F
\end{tabular}
\end{center}
Looking at the columns for the main logical operators (`\enot' for the first sentence, `\eand' for the second), we see that on the first three rows, both sentences are false. On the final row, both are true. Since they match on every row---that is, on every valuation for `$P$' and `$Q$'---the two sentences are logically equivalent.


\section{Consistency}\label{consistency--tt}
In \S\ref{s:joint-poss}, we said that sentences are jointly possible if and only if it is possible for all of them to be true at once. We have a surrogate for this notion too. 

\begin{factboxy}{Jointly consistent}
$\meta{A}$ and $\meta{B}$ are \define{jointly consistent} if and only if there is some valuation that makes them both true and they are not equivalent.
		\medskip
		
Equivalently, if
\begin{earg}
	\item[(1)] there is at least one valuation that makes ($\meta{A}\eand\meta{B}$) true, and
	\item[(2)] $(\meta{A} \eiff \meta{B})$ is \textit{not} a tautology,
\end{earg}		
then $\meta{A}$ and $\meta{B}$ are \define{jointly consistent}.
\end{factboxy}

\noindent The requirement that the two sentences not be equivalent is not always included, but we will distinguish between sentences that are jointly consistent from those that are equivalent. It is acceptable, however, to let the two concepts overlap---that is, allow that some sentences are jointly consistent and equivalent---in which case the definition would drop this requirement about not being equivalent.

This was one of the examples in \S\ref{s:joint-poss}:
\begin{ebullet}	
		\item[G1.] \label{MartianGiraffes} There are at least four giraffes at the wild animal park.
		\item[G2.] There are exactly seven gorillas at the wild animal park.
	\end{ebullet}
These are jointly possible because it is possible for them both to be true at the same time. It takes nothing away from their joint possibility that they can also be false at the same time or one can be false while the other is true. Applying that same observation to \textit{jointly consistent}, all we need is one line where both sentences are true. (More than one line is fine also, although the truth values for the two sentences shouldn't match on every line. If they do, then the sentences are equivalent.) $(P \eor Q)$ and  $(P \eand \enot Q)$ have one line where they are both true, and so they are jointly consistent: 

\begin{center}
\begin{tabular}{c c|d e f |d e e f}
$P$ &$Q$& $P$& ~\eor~ & $Q$& $P$& \eand& \enot& $Q$\\
\hline
 T & T &  T & \TTbf{T} & T & T & \TTbf{F} & F & T\Tstrut\\
 T & F &  T & \TTbf{T} & F & T & \TTbf{T} & T & F\\
 F & T &  F & \TTbf{T} & T & F & \TTbf{F} & F & T\\
 F & F &  F & \TTbf{F} & F & F & \TTbf{F} & T & F
\end{tabular}
\end{center}

Conversely, sentences are \define{jointly inconsistent} if there is no valuation that makes them all true. If we think about this definition for a moment, we see that there are three ways that two sentences can be jointly inconsistent. 
\begin{ebullet}	
\item[(1)] One each line, the truth value for one sentence is `T' and the truth value for the other sentence is `F'. For instance, the truth values for $P \eor Q$ and $\enot P \eand \enot Q$ never match. On each line, one is true and the other is false. Hence, for this relationship between two sentences, all of these criteria are satisfied: $\enot(\meta{A} \eand \meta{B})$ is a tautology; $\enot(\meta{A} \eiff \meta{B})$ is a tautology; and $(\meta{A} \eor \meta{B})$ is a tautology.
\begin{center}
\begin{tabular}{c c|d e f |d e e e f}
$P$& $Q$& $P$& ~\eor~ & $Q$& \enot& $P$& \eand& \enot& $Q$\\
\hline
 T & T &		 T & \TTbf{T} & T 		&	 F & T & \TTbf{F} & F & T\Tstrut\\
 T & F &		 T & \TTbf{T} & F 		&	 F & T & \TTbf{F} & T & F\\
 F & T &		 F & \TTbf{T} & T 		&	 T & F & \TTbf{F} & F & T\\
 F & F &		 F & \TTbf{F} & F 		&	 T & F & \TTbf{T} & T & F
\end{tabular}
\end{center}
\medskip
\item[(2)] When the truth value for one sentence is `T', then the truth value for the other sentence is `F', but both sentences can be false at the same time. For example, $\enot(\enot P \eor Q)$ and $(\enot P \eand \enot Q)$ are never both true on the same line, but they are false on the same line. For sentences that are jointly inconsistent in this way, only this criterion is satisfied: $\enot(\meta{A} \eand \meta{B})$ is a tautology.
\begin{center}
\begin{tabular}{c c|d e e e f |d e e e f}
$P$& $Q$&	\enot& (\enot	&$P$	&~\eor~	&$Q)$	&\enot 	&$P$ & \eand 	& \enot	& $Q$\\
\hline
T  &	 T & 		\TTbf{F} &	 ~F 		& T 	& T 	& T 		& F 		& T 	& \TTbf{F} 		& F 		& T\Tstrut\\
T  &	 F & 		\TTbf{T} &	 ~F 		& T 	& F 	& F 		& F 		& T 	& \TTbf{F} 		& T 		& F\\
F  &	 T & 		\TTbf{F} &	 ~T 		& F	& T 	& T 		& T 		& F 	& \TTbf{F} 		& F 		& T\\
F  &	 F & 		\TTbf{F} &	 ~T 		& F	& T 	& F 		& T 		& F 	& \TTbf{T} 		& T 		& F
\end{tabular}
\end{center}
\medskip
\item[(3)] Both sentences are false on every line. For example, the truth values for $\enot P \eand P$ and $\enot Q \eand Q$ are always the same. On each line, both sentences are false. So, for sentences that are jointly inconsistent in this way, both of these criteria are satisfied: $\enot(\meta{A} \eand \meta{B})$ is a tautology and $(\meta{A} \eiff \meta{B})$ is a tautology. (And the latter, recall, means that these sentences are equivalent, and so here we do have two concepts that overlap.) 
\begin{center}
\begin{tabular}{c c|d e e f |d e e e f}
$P$&$Q$&\enot&$P$&\eand&$P$&\enot&$Q$&\eand&&$Q$\\
\hline
 T & T &  F & T & \TTbf{F} & T & F & T & \TTbf{F} &  & T\Tstrut\\
 T & F &  F & T & \TTbf{F} & T & T & F & \TTbf{F} &  & F\\
 F & T &  T & F & \TTbf{F} & F & F & T & \TTbf{F} &  & T\\
 F & F &  T & F & \TTbf{F} & F & T & F & \TTbf{F} &  & F
\end{tabular}
\end{center} 
\end{ebullet}

%\bigskip 

%\begin{minipage}{.5\linewidth}

%\medskip
%\end{minipage}%
%\begin{minipage}{.5\linewidth}

%\end{minipage}
%\medskip


%%%%%%%%%%%%%%%%%%%%%%%%%%%%%%%%%%%%%%%%%%%%%%
%%%%%%%%%%%%%%%%%%%%%%%%%%%%%%%%%%%%%%%%%%%%%%

%  exercises for `Semantic concepts'

%%%%%%%%%%%%%%%%%%%%%%%%%%%%%%%%%%%%%%%%%%%%%%
%%%%%%%%%%%%%%%%%%%%%%%%%%%%%%%%%%%%%%%%%%%%%%


%\practiceproblems
\section{Practice exercises}
\setcounter{ProbPart}{0}

\problempart
Revisit your answers to the exercises in part A of chapter \ref{s:CompleteTruthTables}, and determine which sentences were tautologies, which were contradictions, and which were neither tautologies nor contradictions.\\


\problempart Create a truth table for each sentence, and then determine whether the sentence is a \textbf{tautology}, a \textbf{contradiction}, or is \textbf{contingent}.

\begin{earg}
\item $\enot B \eand B$ \vspace{.5ex}%contra
\item $\enot D \eor D$ \vspace{.5ex}%taut
\item $(A\eand B) \eor (B\eand A)$\vspace{.5ex} %contingent
\item $\enot[A \eif (B \eif A)]$\vspace{.5ex} %contra
\item $A \eiff [A \eif (B \eand \enot B)]$ \vspace{.5ex}%contra
\item $[(A \eand B) \eiff B] \eif (A \eif B)$ \vspace{.5ex}% contingent. 
\end{earg}


\problempart
\label{pr.TT.consistent}
For each set of sentences, create a truth table and then determine whether the sentences are \textbf{jointly consistent} or \textbf{jointly inconsistent}.
\begin{earg}
\item $A\eif A$, $\enot A \eif \enot A$, $A\eand A$, $A\eor A$ %consistent
\item $A\eor B$, $A\eif C$, $B\eif C$ %consistent
\item $B\eand(C\eor A)$, $A\eif B$, $\enot(B\eor C)$  %inconsistent
\item $A\eiff(B\eor C)$, $C\eif \enot A$, $A\eif \enot B$ %consistent

\item $A \eand \enot B$, $\enot(A \eif B)$, $B \eif A$\vspace{.5ex} %Consistent
\item $A \eor B$, $A \eif \enot A$, $B \eif \enot B$ \vspace{.5ex}%inconsistent. 
\item $\enot(\enot A \eor B) $, $A \eif \enot C$, $A \eif (B \eif C)$\vspace{.5ex} %Inconsistent
\item $A \eif B$, $A \eand \enot B$\vspace{.5ex} %Inconsistent
\item $A \eif (B \eif C)$, $(A \eif B) \eif C$, $A \eif C$\vspace{.5ex} % consistent. 

\item $\enot B$, $A \eif B$, $A$ \vspace{.5ex}%inconsistent.
\item $\enot(A \eor B)$, $A \eiff B$, $B \eif A$\vspace{.5ex} %Consistent
\item $A \eor B$, $\enot B$, $\enot B \eif \enot A$\vspace{.5ex} %Inconsistent
\item $A \eiff B$, $\enot B \eor \enot A$, $A \eif B$\vspace{.5ex} %consistent. 
\item $(A \eor B) \eor C$, $\enot A \eor \enot B$, $\enot C \eor \enot B$\vspace{.5ex} %consistent
\end{earg}


\problempart
\label{pr.TT.equiv}
For each pair of sentences, create a truth table and then determine whether the pair of sentences are \textbf{equivalent} or are not. 
\begin{earg}
\item $A$ and $\enot A$
\item $A \eand \enot A$ and $\enot B \eiff B$
\item $[(A \eor B) \eor C]$ and $[A \eor (B \eor C)]$
\item $A \eor (B \eand C)$ and $(A \eor B) \eand (A \eor C)$
\item $[A \eand (A \eor B)] \eif B$ and $A \eif B$


\item $A\eif A$ and $A \eiff A$
\item $\enot(A \eif B)$ and $\enot A \eif \enot B$
\item $A \eor B$ and $\enot A \eif B$
\item$(A \eif B) \eif C$ and $A \eif (B \eif C)$
\item $A \eiff (B \eiff C)$ and $A \eand (B \eand C)$
\end{earg}


\problempart
\label{pr.TT.concepts}
\begin{earg}
\item Suppose that \meta{A} and \meta{B} are logically equivalent. What can you say about $\meta{A}\eiff\meta{B}$?

\item Suppose that $\meta{A}$, $\meta{B}$ and $\meta{C}$  are jointly inconsistent. What can you say about $(\meta{A}\eand\meta{B}\eand\meta{C})$?

\item Suppose that \meta{A} and \meta{B} are logically equivalent. What can you say about $(\meta{A}\eor\meta{B})$?

\item Suppose that \meta{A} and \meta{B} are \emph{not} logically equivalent. What can you say about $(\meta{A}\eor\meta{B})$?

\item Consider this principle:
	\begin{ebullet}
		\item[] Suppose $\meta{A}$ and $\meta{B}$ are equivalent. Suppose an argument contains $\meta{A}$ (either as a premise, or as the conclusion). The validity of the argument would be unaffected, if we replaced $\meta{A}$ with $\meta{B}$.
	\end{ebullet}
Is this principle correct? Explain your answer.
\end{earg}


%%%%%%%%%%%%%%%%%%%%%%%%%%%%%%%%%%%%%%%%%%%%%%
%%%%%%%%%%%%%%%%%%%%%%%%%%%%%%%%%%%%%%%%%%%%%%

%  answers for `Semantic concepts'

%%%%%%%%%%%%%%%%%%%%%%%%%%%%%%%%%%%%%%%%%%%%%%
%%%%%%%%%%%%%%%%%%%%%%%%%%%%%%%%%%%%%%%%%%%%%%

\section{Answers}
\setcounter{ProbPart}{0}

\problempart
From chapter \ref{s:CompleteTruthTables}
\begin{earg}
\item $A \eif A$ \hfill \myanswer{tautology}
\item $C \eif\enot C$ \hfill \myanswer{neither}
\item $(A \eiff B) \eiff \enot(A\eiff \enot B)$  \hfill \myanswer{tautology}
\item $(A \eif B) \eor (B \eif A)$  \hfill \myanswer{tautology}
\item $(A \eand B) \eif (B \eor A)$  \hfill \myanswer{tautology}
\item $\enot(A \eor B) \eiff (\enot A \eand \enot B)$ \hfill \myanswer{tautology}
\item $\bigl[(A\eand B) \eand\enot(A\eand B)\bigr] \eand C$  \hfill \myanswer{contradiction}
\item $[(A \eand B) \eand C] \eif B$  \hfill \myanswer{tautology}
\item $\enot\bigl[(C\eor A) \eor B\bigr]$  \hfill \myanswer{neither}
\end{earg}


\problempart Use a truth table to determine whether each sentence is a tautology, a contradiction, or a contingent sentence.
\begin{earg}
\item $\enot B \eand B$ \vspace{.5ex} \hfill \myanswer{Contradiction}
\item $\enot D \eor D$ \vspace{.5ex} \hfill \myanswer{Tautology}
\item $(A\eand B) \eor (B\eand A)$\vspace{.5ex} \hfill \myanswer{Contingent}
\item $\enot[A \eif (B \eif A)]$\vspace{.5ex} \hfill \myanswer{Contradiction}
\item $A \eiff [A \eif (B \eand \enot B)]$ \vspace{.5ex} \hfill \myanswer{Contradiction}
\item $[(A \eand B) \eiff B] \eif (A \eif B)$ \vspace{.5ex} \hfill \myanswer{Contingent}
\end{earg}


\problempart
\label{pr.TT.consistent}
Use a truth table to determine whether each set of sentences are jointly consistent or jointly inconsistent.
\begin{earg}
\item $A\eif A$, $\enot A \eif \enot A$, $A\eand A$, $A\eor A$ \\
These sentences are jointly consistent. (See line 1.)
\myanswer{\begin{center}
\begin{tabular}{c | d e f | d e e e f | d e f | d e f}
$A$ &  $A$&$\eif$&$A$&$\enot$&$A$&$\eif$&$\enot$&$A$&$A$&$\eand$&$A$&$A$&$\eor$&$A$\\
\hline
T & T & \TTbf{T} & T& F & T & \TTbf{T} & F & T & T & \TTbf{T} & T & T & \TTbf{T} & T\Tstrut\\
F & F & \TTbf{T} & F& T & F & \TTbf{T} & T & F & F & \TTbf{F} & F & F & \TTbf{F} & F
\end{tabular}
\end{center}}

\item $A\eor B$, $A\eif C$, $B\eif C$ \\
These sentences are jointly consistent. (See line 1.)
\myanswer{\begin{center}
\begin{tabular}{c c c | d e f | d e f | d e f}
$A$ & $B$ & $C$ & $A$&$\eor$&$B$&$A$&$\eif$&$C$&$B$&$\eif$&$C$\\
\hline
T & T & T & T & \TTbf{T} & T & T & \TTbf{T} & T & T & \TTbf{T} & T\Tstrut\\
T & T & F & T & \TTbf{T} & T & T & \TTbf{F} & F & T & \TTbf{F} & F\\
T & F & T & T & \TTbf{T} & T & T & \TTbf{T} & T & F & \TTbf{T} & T\\
T & F & F & T & \TTbf{T} & F & T & \TTbf{F} & F & F & \TTbf{T} & F\\\arrayrulecolor{light-gray}\hline
F & T & T & F & \TTbf{T} & F & F & \TTbf{T} & T & T & \TTbf{T} & T\Tstrut\\
F & T & F & F & \TTbf{T} & T & F & \TTbf{T} & F & T & \TTbf{F} & F\\
F & F & T & F & \TTbf{F} & F & F & \TTbf{T} & T & F & \TTbf{T} & T\\
F & F & F & F & \TTbf{F} & F & F & \TTbf{T} & F & F & \TTbf{T} & F\\
\end{tabular}
\end{center}}

\item $B\eand(C\eor A)$, $A\eif B$, $\enot(B\eor C)$ \\
These sentences are jointly inconsistent.
\myanswer{\begin{center}
\begin{tabular}{c c c | d e e e f | d e f | d e e f}
$A$ & $B$ & $C$ & $B$&$\eand$&$(C$&$\eor$&$A)$&$A$&$\eif$&$B$&$\enot$&$(B$&$\eor$&$C)$\\
\hline
T & T & T & T & \TTbf{T} & T & T & T & T & \TTbf{T} & T & \TTbf{F} & T & T & T\Tstrut\\
T & T & F & T & \TTbf{T} & F & T & T & T & \TTbf{T} & T & \TTbf{F} & T &  T & F\\
T & F & T & F & \TTbf{F} & T & T & T & T & \TTbf{F} & F & \TTbf{F} & F &  T & T\\
T & F & F & F & \TTbf{F} & F & T & T & T & \TTbf{F} & F & \TTbf{T} & F & F & F\\\arrayrulecolor{light-gray}\hline
F & T & T & T & \TTbf{T} & T & T & F & F & \TTbf{T} & T & \TTbf{F} & T & T & T\Tstrut\\
F & T & F & T & \TTbf{F} & F & F & F & F & \TTbf{T} & T & \TTbf{F} & T & T & F\\
F & F & T & F & \TTbf{F} & T & T & F & F & \TTbf{T} & F & \TTbf{F} & F & T & T\\
F & F & F & F & \TTbf{F} & F & F & F & F & \TTbf{T} & F & \TTbf{T} & F & F & F\\
\end{tabular}
\end{center}}

\item $A\eiff(B\eor C)$, $C\eif \enot A$, $A\eif \enot B$  
These sentences are jointly consistent. (See line 8.)
\myanswer{\begin{center}
\begin{tabular}{c c c | d e e e f | d e e f | d e e f}
$A$ & $B$ & $C$ & $A$&$\eiff$&$(B$&$\eor$&$C)$&$C$&$\eif$&$\enot$&$A$&$A$&$\eif$&$\enot$&$B$\\
\hline
T & T & T & T & \TTbf{T} & T & T & T & T & \TTbf{F} & F & T & T & \TTbf{F} & F & T\Tstrut\\
T & T & F & T & \TTbf{T} & T & T & F & F & \TTbf{T} & F & T & T & \TTbf{F} & F & T\\
T & F & T & T & \TTbf{T} & F & T & T & T & \TTbf{F} & F & T & T & \TTbf{T} & T & F\\
T & F & F & T & \TTbf{F} & F & F & F & F & \TTbf{T} & F & T & T & \TTbf{T} & T & F\\\arrayrulecolor{light-gray}\hline
F & T & T & F & \TTbf{F} & T & T & T & T & \TTbf{T} & T & F & F & \TTbf{T} &  F & T\Tstrut\\
F & T & F & F & \TTbf{F} & T & T & F & F & \TTbf{T} & T & F & F & \TTbf{T} & F & T\\
F & F & T & F & \TTbf{F} & F & T & T & T & \TTbf{T} & T & F & F & \TTbf{T} &T & F\\
F & F & F & F & \TTbf{T} & F & F & F & F & \TTbf{T} & T & F & F & \TTbf{T} & T & F
\end{tabular}
\end{center}}

\item $A \eand \enot B$, $\enot(A \eif B)$, $B \eif A$\\
These sentences are jointly consistent. (See line 2.)
\myanswer{
\begin{center}
\begin{tabular}{c c | c c c c | c c c c | c c c } 
$A$ & $B$ 	&	A 	& \eand	&  \enot & B &  \enot &  (A &  \eif & B)	 & 	 B	 & 	\eif  & A   \\ 
\hline
T  & T	& 	T   & F     &   F	 & T  &  F	& 	T &   T	  & T 	  & 	 T	 & 	 T	  & T\Tstrut\\ 
T  & F	&  T & 	\textbf{T}	 & T	 & F &  \textbf{T}	 & 	 T	 & 	 F	 	 & 	 F	 	 	 & 	 F	 	 & 	 \textbf{T}	 	 & 	 T	 	  	  \\ 
F  & T	& 	 F	 				 & 	 F	 & 	 F	 & T & 		 F	 & 	 F	 & 	 T	 	 & 	 T	 	  	  & 	 T	 	 & 	 F	 	 & 	 F	 \\ 
F  & F	& 	 F	  				& 	 F	 & 	 T	 & 	F&    	 	 F	 & 	 F	 & 	 T	 	 & 	 F	 	  	  & 	 F	 	 & 	 T	 	 & 	 F	 \\ 
\end{tabular}
\end{center}
}

\item $A \eor B$, $A \eif \enot A$, $B \eif \enot B$\\
These sentences are jointly inconsistent.
\myanswer{
\begin{center}
\begin{tabular}{cc | ccc | cccc | cccc} 
$A$ & $B$  & A	 & \eor 	 & B 	  	 	 & A 	 & \eif 	 & 	\enot & A 	  	 	 & B 	 & \eif 	 & \enot	 & 	B \\ 
\hline
T & T   &	 T	 & 	 T	 &T  	  	 	 & T	 & 	 F	 & 	F 	 & T 	 & 	 	  	T 	 & 	F 	 & 	 F	 & 	T\Tstrut\\
T &  F &	 T	& 	 T	 & F 	  	 	 & 	T 	 & 	 F	 & 	 F	 & 	 T	 & 	 	  	F 	 & 	 T	 & 	 T	 & 	 F	 \\ 
F &  T  &	 F	& 	 T	 & 	 T	  	 	 & 	F 	 & 	 T	 & 	 T	 & 	F 	 & 	 	    T	 & 	 F	 & 	 F	 & 	 T	 \\ 
F &  F  &	 F	& 	 F	 & 	 F	  	 	 & 	 F	 & 	 T	 & 	 T	 & 	 F	 & 	 	  	 F	 & 	 T	 & 	 T	 & 	 F	 \\ 
\end{tabular}
\end{center}
}

\item $\enot(\enot A \eor B) $, $A \eif \enot C$, $A \eif (B \eif C)$\\
These sentences are jointly consistent.
\begin{center}
\begin{tabular}{ccc | ccccc | cccc | ccccc}
$A$ & $B$  & $C$ &   \enot & (\enot & A & \eor & B)  & A  & \eif 	 & \enot 	 & C & 	 A & \eif 	& (B & \eif & C)  \\ 
 \hline
T&	T&	T&	F 	& 	F	 & 	T & T	 & T  	  & T & F	 & 	 F & T 	  	 & T & T	 & T	 & T 	 & T\Tstrut\\ 
T&	T&	F&	   	 F	& 	F	 & 	T & T	 & T  	  & T & T	 & 	 T & F	  	 & T & F	 & T	 & F	 & F 	  \\ 
T&	F&	T&	   	 T & 	F 	& 	T & F	 & F  	  & T & F	 & 	 F & T	  	 & T & T	 & F	 & T	 & T 	  \\ 
T&	F&	F&	      \TTbf{T}		&  F	 & 	T & F	 & 	F   & 	T & \TTbf{T}	 & 	 T & F 		 & T & \TTbf{T}	 & F	 & T	 & F\\\arrayrulecolor{light-gray}\hline 
F&	T&	T&	   	 F	& 	T	 & 	F & T	 & 	T   & 	F & T	 & 	 F & T	  	 & F	 & F	 & T	 & T	 & T\Tstrut\\
F&	T&	F&	   	 F	& 	 T	& 	F & T	 & 	T   & 	F & T	 & 	T & F 	 	 & F	 & T	 & T	 & F 	 & F 	\\ 
F&	F&	T&	   	 F	& 	 T	& 	F & T	 & 	F   & 	F & T	 & 	F & T	  	 & F	 & T	 & F	 & T	 & T 	\\ 
F&	F&	F&	   	 F	& 	 T	& 	F & T	 & 	F   & 	F & T	 & 	T & F	  	 & F	 & T	 & F	 & T	 & F 	\\ 
\end{tabular}
\end{center}


\item $A \eif B$, $A \eand \enot B$ \hfill \myanswer{Inconsistent}

\item $A \eif (B \eif C)$, $(A \eif B) \eif C$, $A \eif C$ \hfill \myanswer{ Consistent} 

\item $\enot B$, $A \eif B$, $A$ \vspace{.5ex} \hfill \myanswer{Inconsistent}
\item $\enot(A \eor B)$, $A \eiff B$, $B \eif A$\vspace{.5ex} \hfill \myanswer{Consistent}
\item $A \eor B$, $\enot B$, $\enot B \eif \enot A$\vspace{.5ex} \hfill \myanswer{Inconsistent}
\item $A \eiff B$, $\enot B \eor \enot A$, $A \eif B$\vspace{.5ex} \hfill \myanswer{Consistent} 
\item $(A \eor B) \eor C$, $\enot A \eor \enot B$, $\enot C \eor \enot B$\vspace{.5ex} \hfill \myanswer{Consistent}
\end{earg}


\problempart
\label{pr.TT.equiv}
Use a truth table to determine whether each set of sentences are equivalent or not.
\begin{earg}
\item $A$ and $\enot A$
\item $A \eand \enot A$ and $\enot B \eiff B$
\item $[(A \eor B) \eor C]$ and $[A \eor (B \eor C)]$
\item $A \eor (B \eand C)$ and $(A \eor B) \eand (A \eor C)$
\item $[A \eand (A \eor B)] \eif B$ and $A \eif B$

\item $A\eif A$ and $A \eiff A$
\item $\enot(A \eif B)$ and $\enot A \eif \enot B$
\item $A \eor B$ and $\enot A \eif B$
\item$(A \eif B) \eif C$ and $A \eif (B \eif C)$
\item $A \eiff (B \eiff C)$ and $A \eand (B \eand C)$
\end{earg}


\problempart
\label{pr.TT.concepts}
\begin{earg}
\item Suppose that \meta{A} and \meta{B} are logically equivalent. What can you say about $\meta{A}\eiff\meta{B}$?
\begin{ebullet}
\item[] \myanswer{\meta{A} and \meta{B} have the same truth value on every line of a complete truth table, so $\meta{A}\eiff\meta{B}$ is true on every line. It is a tautology.}
\end{ebullet}

\item Suppose that $\meta{A}$, $\meta{B}$ and $\meta{C}$  are jointly inconsistent. What can you say about $(\meta{A}\eand\meta{B}\eand\meta{C})$?
\begin{ebullet}
\item[] \myanswer{Since the sentences are jointly inconsistent, there is no valuation on which they are all true. So their conjunction is false on every valuation. It is a contradiction}
\end{ebullet}

\item Suppose that \meta{A} and \meta{B} are logically equivalent. What can you say about $(\meta{A}\eor\meta{B})$?
\begin{ebullet}
\item[] \myanswer{Not much. Since $\meta{A}$ and $\meta{B}$ are true on exactly the same lines of the truth table, their disjunction is true on exactly the same lines. So, their disjunction is logically equivalent to them.}
\end{ebullet}

\item Suppose that \meta{A} and \meta{B} are \emph{not} logically equivalent. What can you say about $(\meta{A}\eor\meta{B})$?
\begin{ebullet}
\item[] \myanswer{\meta{A} and \meta{B} have different truth values on at least one line of a complete truth table, and $(\meta{A}\eor\meta{B})$ will be true on that line. On other lines, it might be true or false. So $(\meta{A}\eor\meta{B})$ is either a tautology or it is contingent; it is \emph{not} a contradiction.}
\end{ebullet}

\item Consider this principle:
\begin{ebullet}
\item[] Suppose $\meta{A}$ and $\meta{B}$ are logically equivalent. Suppose an argument contains $\meta{A}$ (either as a premise, or as the conclusion). The validity of the argument would be unaffected, if we replaced $\meta{A}$ with $\meta{B}$.
\end{ebullet}
Is this principle correct? Explain your answer.

\begin{ebullet}
\item[] The principle is correct. Since $\meta{A}$ and $\meta{B}$ are logically equivalent, they have the same truth table. So every valuation that makes $\meta{A}$ true also makes $\meta{B}$ true, and every valuation that makes $\meta{A}$ false also makes $\meta{B}$ false. So if no valuation makes all the premises true and the conclusion false, when $\meta{A}$ was among the premises or the conclusion, then no valuation makes all the premises true and the conclusion false, when we replace $\meta{A}$ with $\meta{B}$.
\end{ebullet}

\end{earg}

