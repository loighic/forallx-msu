\graphicspath{{figures--proofs/}}
\part{First-order logic}
\label{ch.FOL}
\addtocontents{toc}{\protect\mbox{}\protect\hrulefill\par}

\chapter{The basics of first-order logic}\label{FOL-basics}

\section{Introduction to first-order logic}

We have mastered truth functional logic. There may be difficult proofs that still confound us, but we have covered everything that there is to cover in this branch of logic. We now move on to the logic system that is typically studied immediately after truth functional logic: first-order logic (which is sometimes called \textit{predicate logic}). 

As you know, the study of truth functional logic is based on the relationships between atomic sentences and sentences that employ the connectives `and', `or', `if \ldots then \ldots', and `if and only if' plus the negation. Hence, we know that `if Mary was arrested, then Peter when to the police' and we know that `Mary was arrested', then it must follow that `Peter went to the police'. And if we chose to symbolize this in TFL, we can easily do so:
\begin{earg}
\item[1.] $M \eif P$
\item[2.] $M$
\item[3.] $P$
\end{earg}
Here, however, is a valid argument that cannot be represented using TFL.
\begin{earg}
\item[1.] All humans are mortal.
\item[2.] Peter is a human.
\item[3.] Therefore, Peter is mortal. 
\end{earg}
Since there are no logical operators in any of these three sentences, if we tried to symbolize the argument, we would just have three seemingly unrelated atomic sentences:
\begin{earg}
\item[1.] $H$
\item[2.] $P$
\item[3.] $M$
\end{earg}
So represented, this argument is invalid. First-order logic, however, will give us the means to represent it as a valid argument.


\section{The content of first-order logic}

In first-order logic (FOL), we still use the logical operators from TFL, the rules of derivation given in chapter 14, and parentheses. We will, however, replace the atomic sentences of TFL with formulas composed of the following.

\begin{ebullet}
	\item[(\textit{a})] predicates (or properties)
	\item[(\textit{b})] names for specifical individuals or things 
	\item[(\textit{c})] variables 
	\item[(\textit{d})] quantifiers
\end{ebullet}
We also add the identity symbol, ‘=’, to the symbols from TFL.  


\section{Names}

Names are simple. They designate specific individuals (or any specific entity). We represent them with the lowercase letters \textit{a} -- \textit{r}.

\section{Predicates}

Predicates---or properties---are attributes that individuals can have. These, for instance are predicates:

\begin{ebullet}
	\item[] \rule{1cm}{0.15mm} is tall.
	\item[] \rule{1cm}{0.15mm} is a mammal.
	\item[] \rule{1cm}{0.15mm} ate dinner.
\end{ebullet}

As you can see, we need to add a name to the predicate to form a complete English sentence. If we use the name `Carol’, then we will have ‘Carol is tall’, ‘Carol is a mammal’, and ‘Carol ate dinner’.
Predicates are symbolized with uppercase letters, and we can use any letter $A$ through $Z$. Symbolizing `is tall' as $T$, `is a mammal' as $M$, `ate dinner' as $D$, and `Carol' as $c$, we combine this name and these predicate in FOL this way:

\begin{ebullet}
	\item[] $T(c)$
	\item[] $M(c)$
	\item[] $D(c)$
\end{ebullet}

Those are \textit{one-place predicates} because each takes only one name. These are \textit{two-place predicates} because each takes two names:

\begin{ebullet}
	\item[] \rule{1cm}{0.15mm} is the sister of \rule{1cm}{0.15mm} .
	\item[] \rule{1cm}{0.15mm} is in love with \rule{1cm}{0.15mm} .
	\item[] \rule{1cm}{0.15mm} is taller than \rule{1cm}{0.15mm} .
\end{ebullet}
With names added, we can symbolize these sentences in the ways given. Notice that the order of the names matters.

\begin{ebullet}
	\item[] Abby is the sister of Carol: $S(a,c)$
	\item[] Carol is the sister of Abby: $S(c,a)$
	\item[] David is in love with Carol: $L(d,c)$
	\item[] Carol is not in love with David: $\enot L(c,d)$
\end{ebullet}
And likewise, we can have three- or four- (or more) place predicates. 


\section{Quantifiers and variables}

The most notable feature of first-order logic is the use of what are called \textit{quantifiers}. There are two in this logic system:

\begin{ebullet}
	\item[(\textit{a})] The universal quantifier, which is symbolized with `$\forall$' and can be translated as ``every'' or ``all.''
	\item[(\textit{b})] The existential quantifier, which is symbolized with `$\exists$' and can be translated as ``some.''
\end{ebullet}

When we use these quantifiers, we are not designating a specific individual and so, with them, we must use variables. Variables work with predicates the same way as names do, but when we have a variable, we must always have quantifier. We represent variables with the lowercase letters \textit{s} -- \textit{z}.

Here are two English sentences that are each translated to FOL with a quantifier, a predicate, and a variable:

\begin{ebullet}
	\item[] Everyone ate dinner: $\forall x D(x)$. 
	\item[] Someone ate dinner: $\exists x D(x)$.
\end{ebullet}

As we did with TFL, there will be a point where we won’t be as concerned with the English sentence that is represented by an expression in FOL. So, reading the first one, as an expression in FOL (and without a specific English translation for $D$), we say, ``for all $x, D(x)$.” And for the second one: “for some $x, D(x)$”. 


\section{Identity}

The identity symbol, `=’, is used to signify that two things are the same; that is, not just similar or equivalent, but the same. So, while two professors who both teach intermediate German might, for all relevant purposes, be equivalent, we wouldn’t use the identity symbol to indicate this. Rather, we use the identity symbol for cases such as `Mark Twain is identical to Samuel Clemens.’ 


\section{Domains}

A logic system must be precise, but `all’ or `every’, when taken literally and without any qualification, refer to a vast set of individuals or things. For instance, if `$H$' represents `is happy’, then ‘$\forall x H(x)$’ represents ‘Everyone is happy’. But when we say this in English, we are not referring to everyone now alive on Earth or everyone who ever was alive or who ever will be alive. Rather, we mean something like `everyone now in the room,’ or `everyone enrolled in this course’. 
So, to be precise when we use quantified expressions in FOL, we need to specify a \define{domain}. The domain is the collection of things about which we want to refer. If we want to talk about people in Starkville, we define the domain as \textit{people in Starkville}. We write this at the beginning of the symbolization key, like this:

\begin{earg}
\item[] domain: people in Starkville
\end{earg}
The quantifiers, then, apply to (or range over) the domain. Given this domain, ‘$\forall x$’ is to be read roughly as ‘For every person in Starkville \ldots ’, and ‘$\exists x$’ is to be read roughly as ‘For some person in Starkville, \ldots ’.

In FOL, the domain must always include at least one thing. Moreover, in English we can legitimately infer ‘someone is angry’ from ‘David is angry’. Likewise in FOL, we will want to be able to infer ‘$\exists xA(x)$’ from ‘$A(d)$’. So, when we are using a name (and not a quantifier), it still must be the case that each name picks out exactly one thing in the domain. (Although there can be members of a domain that don’t have names or have more than one name.)




