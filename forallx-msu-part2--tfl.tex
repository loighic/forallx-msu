%!TEX root = forallxyyc.tex
\part{Truth-functional logic}
\label{ch.TFL}
\addtocontents{toc}{\protect\mbox{}\protect\hrulefill\par}

\chapter{Form and symbolization}\label{symbolization}

\section{Validity in virtue of form}\label{s:ValidityInVirtueOfForm}
Consider the following two arguments. First this one:
	\begin{earg}
		\item[1.] If it is raining outside, then Mary is miserable.
		\item[2.] It is raining outside.
		\item[3.] Therefore, Mary is miserable.
	\end{earg}
and then this one:
	\begin{earg}
		\item[1.] If Mary is a student, then Leiser is a spy.
		\item[2.] Mary is a student.
		\item[3.] Therefore, Leiser is a spy.
	\end{earg}
Both arguments are valid, and, as perhaps you  can see, they share a common \textit{form}. We can represent the form by itself this way:
	\begin{earg}
		\item[1.] If A, then B
		\item[2.] A
		\item[3.] Therefore, B
	\end{earg}
Any argument with this form will be valid. It doesn't matter what English sentences are put in the places of A and B. 

Here is another valid argument:
	\begin{earg}
		\item[1.] Seoul is larger than London.
		\item[2.] London is larger than Chicago.
		\item[3.] Therefore, Seoul is larger than Chicago. 
	\end{earg}
This argument also has a particular form that makes it valid, and we can represent its form like this:
	\begin{earg}
		\item[1.] C is larger than D.
		\item[2.] D is larger than F.
		\item[3.] Therefore, C is larger than F. 
	\end{earg}
For the first argument form that we examined, A and B could be any sentences. Here, C, D, and F are names (not full sentences), and we can put any names (for anything) in the places of C, D, and F, and the argument will be valid.

In contrast, this argument is valid, but there is no particular form that makes it so.
	\begin{earg}
		\item[1.] Juanita is a vixen.
		\item[2.] Therefore, Juanita is a fox.
	\end{earg}
Unlike the previous three examples, this argument is valid, not because of the form of the argument, but because of the particular meanings of \textit{fox} and \textit{vixen}. 

These examples illustrate \emph{validity in virtue of form}. The arguments about Mary, Leiser, Seoul, London, and Chicago are valid, but---unlike the argument about Juanita---their being valid has nothing to do with the specific meaning of `Mary is miserable', `Leiser is a spy', `Seoul', `London', or `Chicago.' (Whether the arguments are sound depends on these meanings, but not whether the arguments are valid.) Instead, these arguments are valid in virtue of the meanings of just these words: \textit{if, then} and \textit{is larger than}.

\begin{factboxy}{valid in virtue of form}
Let us define \define{structure words} as \textit{if-then}, \textit{and}, \textit{or}, \textit{not}, \textit{if and only if}, and comparative adjectives followed by \textit{than} (e.g., \textit{larger than}, \textit{faster faster than}, \textit{older than}).

An argument is \define{valid in virtue its form} when it remains valid under these conditions: (1) structure words are always part of the argument and (2) with those words in place, sentences or names can be freely substituted into the argument and it will remain valid. 
\end{factboxy}

\noindent This is not a perfect definition. \textit{Form} in formal logic is much broader than just the use of these ``structure words.'' This is a good definition with which to start, however.

And, although valid for reasons \textit{other than} the argument's form is an interesting topic, our focus will be on arguments that are valid because of their form---and in fact, only valid in virtue of some forms. In part 5, we will broaden the analysis a bit, but in parts  \ref{ch.TFL}, \ref{ch.TruthTables}, and \ref{ch.NDTFL} of this textbook, we will be interested in arguments where the form is set by the use of  \textit{if-then}, \textit{and}, \textit{or}, \textit{not}, and \textit{if and only if}. 

\begin{notebox}
Going forward, we will set aside arguments that are valid because they employ comparative adjectives. To make a final point about them, though, arguments that use comparative adjectives are valid because these terms denote a \define{transative relation}. Such a relation exists when the relation between two elements in a series applies to any elements in the series as long as the elements are taken in order. 
\end{notebox}

Here are some examples of arguments that are valid in virtue of \textit{or}, \textit{not}, and \textit{and}.
This one:
	\begin{earg}
		\item[1.] Claire is either a student, or she is a spy.
		\item[2.] Claire is not a student.
		\item[3.] Therefore, Claire is a spy.
	\end{earg}
has this form:
	\begin{earg}
		\item[1.] G or H
		\item[2.] not G
		\item[3.] Therefore, H
	\end{earg}
And this argument:
	\begin{earg}
		\item[1.] It's not the case that Jeff both studies often and acts in lots of plays.
		\item[2.] Jeff acts in lots of plays.
		\item[3.] Therefore, Jeff does not study often.
	\end{earg}
has a form that we can represent like this:
	\begin{earg}
		\item[1.] not (K and L)
		\item[2.] L
		\item[3.] Therefore, not K
	\end{earg}


\section{Atomic sentences and symbolization}\label{s:atomic}

Consider this sentence again:
	\begin{earg}
		\item[(a)] If it is raining outside, then Mary is miserable.
	\end{earg}
`It is raining outside' and `Mary is miserable' are \textit{subsentences} of sentence (a). To specify the structure of the first argument in \S\ref{s:ValidityInVirtueOfForm}, we replaced the sentences and subsentences in it with individual letters. `It is raining outside' was replaced with `A', and `Mary is miserable' was replaced with `B'. This kind of representation---letters standing for sentences or subsentences---is one important part of the formal language that we develop in this book.

\begin{factboxy}{Atomic sentences}
An \define{atomic sentence} is a sentence that (1) can be true or false and (2) no smaller complete sentence can be extracted from it.\medskip

In English, an atomic sentence is a declarative sentence that does not contain any subsentences.\smallskip

In our logic system, an atomic sentence is a single capital letter in this font: $A, B, C, . . .$.
\end{factboxy}

Atomic sentences are the basic building blocks used to form more complex sentences. We will use uppercase Roman letters for atomic sentences in our logic system. If, by chance, we ever need more than twenty-six different atomic sentences, we can obtain additional ones by adding subscripts to letters. Here, for instance, are five different atomic sentences:
	$$M, P, P_{\text{\tiny 1}}, P_{\text{\tiny 2}}, M_{\text{\tiny 17}}$$

We will use atomic sentences to represent, or \emph{symbolize}, certain English sentences. To do this, we provide a \define{symbolization key}, such as the following.
	\begin{ekey}
		\item[A] It is raining outside
		\item[B] Mary is miserable
	\end{ekey}
When we do this, we are not fixing this symbolization once and for all. We are just saying that, for the time being, we will think of the atomic sentence `$A$' as symbolizing the English sentence `It is raining outside', and the atomic sentence, `$B$', as symbolizing the English sentence `Mary is miserable'. Later, when we are dealing with different sentences or different arguments, we can provide a new symbolization key; for instance, 
	\begin{ekey}
		\item[A] Jeff stole the document.
		\item[B] Jeff is in the safe house.
	\end{ekey}



%%%%%%%%%%%%%%%%%%%%%%%%%%%%%%%%%%%%%%%%
%%%%%%%%%%%%%%%%%%%%%%%%%%%%%%%%%%%%%%%%
% CHAPTER 5
%%%%%%%%%%%%%%%%%%%%%%%%%%%%%%%%%%%%%%%%
%%%%%%%%%%%%%%%%%%%%%%%%%%%%%%%%%%%%%%%%



\chapter{Logical operators}
\label{s:TFLConnectives}

\define{Truth-functional propositional logic} is a branch of logic that focuses on the relationships between atomic sentences. One part of truth-functional propositional logic (or `TFL' for short) is a formal language. This formal language consists of atomic sentences of TFL---the sentence letters that were introduced in \S \ref{s:atomic}---and the \define{logical operators} `and', `or', `not', `if \ldots, then \ldots' and `if and only if'. A logical operator is a word or phrase that modifies a sentence or connects two sentences to form a more complex sentence. We call these operators \textit{truth-functional} because the truth of the complex sentences depends entirely on the truth of the atomic sentences of which they are composed. (\textit{Logical operators} are also sometimes referred to as \textit{connectives} because, except in the case of `not', these operators connect two simpler sentences.)  

In addition to symbolizing English sentences with sentence letters, we also want to symbolize the truth-functional logical operators. The symbols that we will use are shown in table \ref{table.connectives}. The operators listed there are not the only ones that we have in English. Others are, for example, `unless', `neither \dots{} nor \dots', `necessarily', and `because'. As we will see, the first two can be expressed with the connectives that are in table \ref{table.connectives}. The last two, however, cannot. Although they are logical operators, `necessarily' and `because' are not truth functional.

\begin{table*}\centering\sffamily\footnotesize
\ra{1.25}
\begin{tabular}{@{}l l l@{}}\toprule
\textth{symbol} & \textth{the sentence's name} & \textth{its meaning}\\\midrule
	\enot&negation&`It is not the case that$\ldots$'\\
	\eand&conjunction&`Both$\ldots$\ and $\ldots$'\\
	\eor&disjunction&`Either$\ldots$\ or $\ldots$'\\
	\eif&conditional&`If $\ldots$\ then $\ldots$'\\
	\eiff&biconditional&`$\ldots$ if and only if $\ldots$'\\
\bottomrule
\end{tabular}
\caption{The logical operators of truth functional logic}\label{table.connectives}
\end{table*}
	
Once we have introduced these logical operators (in this chapter and in chapter \ref{s:CharacteristicTruthTables}) and have explained what can and cannot be a sentence in TFL (which we will do in chapter \ref{s:TFLSentences}), our formal language will be complete. Although the formal language is central, truth-functional propositional logic does not consist only of a formal language. There is also a \textit{deductive system}, which we will explore in part \ref{ch.NDTFL}. 

        
\section{Negation}

Consider how we might symbolize these sentences:
	\begin{earg}
	\item[\ex{not1}] Mary is in Barcelona.
	\item[\ex{not2}] It is not the case that Mary is in Barcelona.
	\item[\ex{not3}] Mary is not in Barcelona.
	\end{earg}
To begin, we need an atomic sentence. This will be our symbolization key:
	\begin{ekey}
		\item[B] Mary is in Barcelona.
	\end{ekey}
	
$B$ is sentence \ref{not1}, and so we don't need to do anything else there. 
The second sentence is partially symbolized as `It is not the case that $B$'. In order to complete the symbolization, we need a symbol for `it is not the case that'. Or, put differently, we need a symbol that, when added to $B$, will express `the negation of $B$'. We will use `\enot' and symbolize sentence \ref{not2} as `$\enot B$'.

Sentence \ref{not3} also contains the word `not', and it is equivalent to sentence \ref{not2}. As such, we can also symbolize it as `$\enot B$'.

\begin{factboxy}{Negation}
A sentence can be symbolized as $\enot\meta{A}$ if it can be paraphrased in English as `It is not the case that \ldots'
\end{factboxy}

Here are a few more examples:
	\begin{earg}
		\item[\ex{not4}] The cog can be replaced.
		\item[\ex{not5}] The cog is irreplaceable.
		\item[\ex{not5b}] The cog is not irreplaceable.
	\end{earg}
For these, we will use this symbolization key:
	\begin{ekey}
		\item[R] The cog is replaceable
	\end{ekey}
Sentence \ref{not4} is symbolized just by `$R$'. Sentence \ref{not5} can be reworded as \textit{it is not the case that the cog is replaceable}. So even though sentence \ref{not5} does not contain the word `not', we will symbolize it `$\enot R$'.
Sentence \ref{not5b}, you will notice, is the denial of sentence \ref{not5}. So, we symbolize \ref{not5b} as `$\enot\enot R$'.

Finally, consider these English sentences:
	\begin{earg}
		\item[\ex{not6}] Jane is happy.
		\item[\ex{not7}] Jane is unhappy.
	\end{earg}
If we use`$H$' stand for `Jane is happy', then we can symbolize sentence \ref{not6} as `$H$'. It would be a mistake, however, to symbolize sentence \ref{not7} with `$\enot{H}$'. 
`$\enot{H}$' means `Jane is not happy', but `Jane is not happy' does not have the same meaning as `Jane is unhappy'. After all, Jane might be neither happy nor unhappy; her mood might just be neutral. In order to symbolize sentence \ref{not7}, we would need a different sentence letter.


\section{Conjunction}
\label{s:ConnectiveConjunction}

Let's start with these sentences:
	\begin{earg}
		\item[\ex{and1}] Adam is athletic, and Barbara is also athletic.
		\item[\ex{and2}] Barbara and Adam are both athletic.
		\item[\ex{and3}] Adam is not athletic, but Barbara is.
	\end{earg}
We will need separate sentence letters to symbolize sentences \ref{and1} and \ref{and2}, and so we will use this symbolization key:
	\begin{ekey}
		\item[A] Adam is athletic.
		\item[B] Barbara is athletic.
	\end{ekey}
Sentence \ref{and1} can be partially symbolized as `A and B'. To symbolize the `and'. We will use `\eand', which is called the \textit{ampersand}. Thus, sentence \ref{and1} becomes `$(A\eand B)$'. When two sentences are connected with an `$\eand$', the resulting sentence is called a \define{conjunction}. The two sentences that are combined with the `$\eand$' are the \define{conjuncts} of the conjunction. So, `$A$' and `$B$' are the conjuncts of the conjunction `$(A\eand B)$'.

Although it is worded differently, sentence \ref{and2} has the same meaning as sentence \ref{and1}. Thus, it is also symbolized as `$(A\eand B)$'.
Notice that we don't symbolize the word `also' in sentence \ref{and1} or `both' in \ref{and2}. Words like `both' and `also' function to draw our attention to the fact that two sentences are being joined to form a conjunction. They may affect the emphasis of a sentence in English, but we don't (and can't) symbolize such terms in TFL. 

For sentence \ref{and3}, let's first symbolize `Adam is not athletic' as `$\enot A$'. `Barbara is' means `Barbara is athletic', and so that subsentence is symbolized as `$B$'. `But' may have a slightly different meaning in English than `and', but, grammatically, they serve the same role: to join two sentences to form a conjunction. Putting this altogether, sentence \ref{and3} is symbolized as `$(\enot A \eand B)$'.


\begin{factboxy}{Conjunction}
A sentence can be symbolized as $(\meta{A}\eand\meta{B})$ if it can be paraphrased any of these ways in English:
\vspace{-2mm}
\begin{earg}
\item[] `Both\ldots, and\ldots',
\item[] `\ldots, and\ldots',
\item[] `\ldots, but \ldots', 
\item[] `\ldots, although \ldots',
\item[] `\ldots, as well as \ldots'
\end{earg}
\end{factboxy}
	
\subsection{Parentheses}

Although we will relax this requirement later, a conjunction in TFL should be enclosed in parentheses. (A negation---for instance, `$\enot P$'---should not, though.) The purpose of the parentheses is to let us be perfectly explicit about how each logical operator is related to each sentence letter. We will see more illustrations of this once we have introduced all of the logical operators, but consider these two English sentences, and think about whether they have the same or different meanings:
\begin{earg}
	\item[\ex{par1}] Kate is not at school, and Sarah is sleeping.
	\item[\ex{par2}] It is not the case that both Kate is at school and Sarah is sleeping.
\end{earg}
We will use this symbolization key:
	\begin{ekey}
		\item[K] Kate is at school.
		\item[S] Sarah is sleeping.
	\end{ekey}
Sentences \ref{par1} and \ref{par2} do not have the same meaning, and so we can't symbolize them in exactly the same way.

In sentence \ref{par1}, the `not' only applies to `Kate is at school.' Thus, this sentence becomes `$(\enot K \eand S)$'. In sentence \ref{par2}, the `it is not the case that' applies to the whole `Kate is at school and Sarah is sleeping'. We can tell this because the `it is not the case that' is before the `both', which signals the beginning of the conjunction. Hence, for this sentence, we need to put the `$\enot$' outside the parentheses like this: `$\enot (K \eand S)$'.


\section{Disjunction}
\label{s:ConnectiveDisjunction}

We will start with these sentences:
	\begin{earg}
		\item[\ex{or1}] Either Amy is at the train station, or Kate driving to Santa Fe.
		\item[\ex{or2}] Amy or Sarah is at the train station. 
	\end{earg}
And we will use this symbolization key:
	\begin{ekey}
		\item[A] Amy is at the train station.
		\item[K] Kate is driving to Santa Fe.
		\item[S] Sarah is at the train station.
	\end{ekey}
To represent the `or' in sentences \ref{or1} and \ref{or2}, we will use the symbol `$\eor$' (which we call the \textit{wedge}, not \textit{v}). Sentence \ref{or1}, then, is written as `$(A \eor K)$'. When two sentences are connected with an `$\eor$', the resulting sentence is called a \define{disjunction}. `$A$' and `$K$' are the \define{disjuncts} of the disjunction `$(A \eor K)$'.

Sentence \ref{or2} is only slightly more complicated. We can paraphrase it as `Either Amy is at the train station, or Sarah is at the train station', and then we symbolize it as `$(A \eor S)$'.

\begin{factboxy}{Disjunction}
A sentence can be symbolized as $(\meta{A}\eor\meta{B})$ if it can be paraphrased in English as `Either\ldots, or\ldots.' Each of the disjuncts must be a sentence.
\end{factboxy}

\subsection{The inclusive or}\label{inclusive-or-1}

Sometimes in English, the word `or' is used in a way that excludes the possibility that both disjuncts are true. This is called an \define{exclusive or}.  An \emph{exclusive or} is clearly intended when it says on a restaurant menu ``Entrees come with either soup or salad.'' This means that, with your entree, you may have soup or you may have salad, but you cannot have both.

At other times, the word `or' allows for the possibility that both disjuncts might be true. For instance, Amy might say, ``I am going to get an A in Logic or an A in German III.'' She probably means that she will get an A in one or both of those courses. (After all, if she did end up getting an A in both, we wouldn't insist that she was wrong when she said, ``I am going to get an A in Logic or an A in German III.'')

When we mean that \textit{one} or the \textit{other} or \textit{both} of the disjuncts is true, then we are using the \define{inclusive or}. The TFL symbol `\eor' always symbolizes an \emph{inclusive or}.

\subsection{Negation and disjunction}

Think about these sentences:
	\begin{earg}
		\item[\ex{or3}] Either Amy is not at the train station, or Sarah is not at the train station.
		\item[\ex{or4}] Neither Amy nor Sarah is at the train station.
		\item[\ex{or.xor}] Either Amy is at the train station or Sarah is at the train station, but both are not.
	\end{earg}
Sentence \ref{or3} is symbolized as `$(\enot A \eor \enot S)$'. Sentences \ref{or4} and \ref{or.xor} are a little trickier. 

According to \ref{or4}, is either one at the train station? No. So, when when we paraphrase it we get this: 
\begin{ebullet}
	\item[] It is not the case that either Amy is at the train station or Sarah is at the train station. 
\end{ebullet}
As our paraphrased sentence shows, we are negating the entire disjunction. Hence, we symbolize sentence \ref{or4} as `$\enot (A \eor S)$'. 

Sentence \ref{or.xor} expresses the meaning of the \textit{exclusive-or}: one or the other, but not both. The `$\eor$', however, represents the \textit{inclusive-or}: one or the other, or both. Therefore, to represent \ref{or.xor} in TFL, we need to break the sentence into two parts. 

The first part, `Amy is at the train station or Sarah is at the train station', is symbolized as `$(A \eor S)$'. The second part, which states that both won't be there, is paraphrased this way: `It is not the case that both Amy is at the train station and Sarah is at the train station'. This, we symbolize as `$\enot(A \eand S)$'. We put the two parts together with an `and', and sentence \ref{or.xor} becomes `$((A \eor S) \eand  \enot(A \eand S))$'.

These last two examples demonstrate that we can sometimes symbolize English sentences that, at first, appear not to be using the logical operators of TFL. We can do this as long as we can figure out a way to paraphrase the English sentence so that it is using some combination of `\textit{and}', `\textit{or}' (i.e., the inclusive-or), `\textit{not}', `\textit{if \ldots, then \ldots}', and `\textit{if and only if}'. 


\section{Conditional}

We will start with this sentence:
	\begin{earg}
		\item[\ex{if1}] If Jean is in Paris, then Jean is in France.
	\end{earg}
And we will use this symbolization key:
	\begin{ekey}
		\item[P] Jean is in Paris.
		\item[F] Jean is in France
	\end{ekey}
Sentence \ref{if1} has this form: `if P, then F', and we call this type of sentence a \define{conditional}. We will use `\eif' to symbolize `if \ldots, then \ldots'. Thus, sentence \ref{if1} becomes `$(P\eif F)$'. 

In a conditional, what goes before the `$\eif$'  is called the \define{antecedent}, and what comes after the `$\eif$' is called the \define{consequent}. So, in sentence \ref{if1}, `Jean is in Paris' is the antecedent, and `Jean is in France' is the consequent.

\begin{factboxy}{Conditional}
A sentence can be symbolized as $\meta{A} \eif \meta{B}$ if it can be paraphrased in English as `If A, then B'.
\end{factboxy}

Many English expressions can be represented using the conditional, and the most common alternatives to `if $\meta{A}$, then $\meta{B}$' are listed in table \ref{table.conditional.English}. If you think about it, you'll see that all six of the sentences in the table have the same meaning, and so they can all be symbolized as `($\meta{A} \eif \meta{B}$)'. (Or, in this case, as `($P \eif F$)'.)


\begin{table*}\centering\sffamily\footnotesize
\ra{1.25}
\begin{tabular}{@{}l l@{}}\toprule
If Jean is in Paris, then she is in France & If $\meta{A}$, then $\meta{B}$.\\
Jean is in France if she is in Paris. 	&	$\meta{B}$ if $\meta{A}$.\\
Whenever Jean is in Paris, she is in France.  	&	Whenever $\meta{A}$, $\meta{B}$.\\
Jean is in France provided that she is in Paris. 	&	$\meta{B}$ provided that $\meta{A}$.\\
Provided that Jean is in Paris, she is in France. 	&	Provided that $\meta{A}$, $\meta{B}$.\\
Jean is in Paris only if she is in France. 	&	$\meta{A}$ only if $\meta{B}$.\\
\bottomrule
\end{tabular}
\caption{The most common way of expressing a conditional in English is as `If Jean is in Paris, then she is in France.' This table lists some alternative but equivalent ways of expressing the same sentence.}\label{table.conditional.English}
\end{table*}


\section{Biconditional}\label{s:biconditional-1}

All of the logical operators that we have discussed so far are ones with which you were already familiar because you are an English speaker. The biconditional, which is mostly commonly expressed as `\textit{\ldots if and only if \ldots}', is one that you might not have really noticed before---even if you have used it on occasion. We'll start with the basic case.
	\begin{earg}
		\item[\ex{iff1}] The Bearcats won if and only if they scored more points than the Razorbacks.
	\end{earg}
And this will be our symbolization key:
	\begin{ekey}
		\item[B] The Bearcats won.
		\item[R] The Bearcats scored more points than the Razorbacks.
	\end{ekey}
The symbol `\eiff' will stand for `if and only if', and so we can symbolize sentence \ref{iff1} as `$B \eiff R$'.

Now, let's probe a little further into the meaning of `if and only if' with a different example.
	\begin{earg}
		\item[\ex{iff2}] If Mary has a sunburn, then she went to the beach.
		\item[\ex{iff3}] If she went to the beach, then Mary has a sunburn. 
		\item[\ex{iff4}] If Mary has a sunburn, then she went to the beach, and if she went to the beach, then Mary has a sunburn.
		\item[\ex{iff5}] Mary has a sunburn if and only if she went to the beach.
	\end{earg}
We will use this symbolization key:
	\begin{ekey}
		\item[S] Mary has a sunburn.
		\item[B] Mary went to the beach.
	\end{ekey}
From the previous section, you know that we symbolize sentences \ref{iff2} and \ref{iff3} like this:
	\begin{earg}
		\item[\ref{iff2}.] $(S \eif B)$
		\item[\ref{iff3}.] $(B \eif S)$ 
	\end{earg}
Sentence \ref{iff4}, then, is a conjunction created by combining \ref{iff2} and \ref{iff3}: 
	\begin{earg}
		\item[\ref{iff4}.] $((S \eif B) \eand (B \eif S))$
 	\end{earg}
And sentence \ref{iff5}, is symbolized with the `$\eiff$':
	\begin{earg}
		\item[\ref{iff5}.] $(S \eiff B)$
	\end{earg}


Sentence \ref{iff4}, it turns out, has the same meaning as sentence \ref{iff5}. `$(S \eif B) \eand (B \eif S)$' is equivalent to `$(S \eiff B)$'. We call `$(S \eiff B)$' a \define{biconditional}, because it is equivalent to the two conditionals that have their antecedent and the consequent switched.

The expression `if and only if' occurs frequently in philosophy, mathematics, and logic, and sometimes you will see it abbreviated `iff'. (Although even when `iff' is written, we still say `if and only if.') 

\begin{factboxy}{Biconditional}
		A sentence can be symbolized as $\meta{A} \eiff \meta{B}$ if it can be paraphrased in English as `A iff B'---that is, as `A if and only if B'.
\end{factboxy}
	
%A word of caution. Ordinary speakers of English often use `if \ldots, then\ldots' when they really mean to use something more like `\ldots if and only if \ldots'. Perhaps your parents told you when you were a child: `if you don't eat your vegetables, you won't get any dessert'. Suppose that you ate your vegetables, but that your parents refused to give you any dessert, on the grounds that they were only committed to the \emph{conditional} (roughly `if you get dessert, then you will have eaten your vegetables'), rather than the biconditional (roughly, `you get dessert if and only if you eat your vegetables'). Despite the valuable lesson in truth functional propositional logic, you would have been upset. So, be aware of this when interpreting what people say, and in your own writing, make sure you use \textit{if and only if} if and only if you mean to use it.

\section{Unless}\label{s:unless}
We have now introduced all of the logical operators of TFL. We can use them together to symbolize many kinds of sentences. An especially difficult case is when we use the English-language connective `unless'. Take this sentence:

\begin{earg}
\item[\ex{unless1}] You will catch a cold unless you wear a jacket. 
\end{earg}
To symbolize \ref{unless1}, we will use this symbolization key:
	\begin{ekey}
		\item[J] You will wear a jacket.
		\item[D] You will catch a cold.
	\end{ekey}

One meaning of sentence \ref{unless1} is that if you do not wear a jacket, then you will catch a cold. This we symbolize as `$(\enot J \eif D)$'. Alternatively, the sentence can mean that if you do not catch a cold, then you must have worn a jacket. This is symbolized as `$(\enot D \eif J)$'. And, finally, it can also mean that either you will wear a jacket or you will catch a cold. This, we symbolize as `$(J \eor D)$'.

All three ways of symbolizing sentence \ref{unless1} are correct. Indeed, in chapter \ref{s:SemanticConcepts} we will see that all three symbolizations are equivalent in TFL. Following the somewhat standard practice, however, we will define \textit{unless} as a disjunction.
% TODO: it might be useful to reference exercise 11.F.3 explicitly
% here, since the point is not discussed in the main text
	
\begin{factboxy}{Unless}
If a sentence can be paraphrased as `A unless B,' then it can be symbolized as `$\meta{A}\eor\meta{B}$'.
\end{factboxy}

There is a complication with treating `unless' as a disjunction, however. As we said earlier, `or' has an inclusive and an exclusive meaning, but in TFL, `or' is always inclusive. Speakers of English, however, often use `unless' to mean something more like the exclusive-or. Suppose someone says: `I will go running unless it snows'. They probably mean `either I will go running or it will snow, but not both'. So, it can be argued that the conditional---i.e., `if it does not snow, then I will go running' ($\enot S \eif R$)---captures the meaning of `unless' better than does the disjunction.


\section{The turnstile}

The final symbol that we need is, technically, not a symbol of TFL, but it is useful to have when displaying arguments in TFL. The symbol `\proves' is called the \textit{turnstile}. The purpose of the turnstile is to separate the sentences that are the premises of an argument from the sentence that is the conclusion, and it can be read as \textit{therefore}. Here is an example,
$$(P \eif C), (P \eor D) \proves (\enot C \eif D)$$
In this argument, the premises are `$(P \eif C)$' and `$(P \eor D)$', and the conclusion is `$(\enot C \eif D)$'.


%\begin{notebox}
%Like the metavariables `$\meta{A}, \meta{B}, \meta{C}, \meta{D}, \ldots$', `\proves' is a symbol of our metalanguage, augmented English. The %difference between the object language and the metalanguage is explained in \S\ref{s:Metalanguage}.
%\end{notebox}


%\section{Expressions in TFL}

%Some expressions in TFL will use only one logical operator, but many will contain multiple logical operators. These, for instance, each contain two or three:

%\begin{table*}[h]
%\ra{1.25}
%\begin{tabular}{l  l}
%$\enot C \eor D$ \qquad \qquad \qquad \qquad  & ``Not $C$ or $D$.''\\
%$(B \eor D) \eif (C \eand F)$ & ``If $B$ or $D$, then $C$ and $F$.''\\
%$P \eiff (R \eand S)$ & ``$P$ if and only if both $R$ and $S$.''\\
%$\enot(Q \eand R)$ & ``It is not the case that both $Q$ and $R$.''\\
%&\textit{or} ``Not both $Q$ and $R$.'' 
%\end{tabular}
%\end{table*}

%\noindent Your first task is to recognize each logic operator and recall it's meaning. The next step is to translate expressions from English to TFL or vice versa when the expressions in TFL contain multiple logical operators. This takes practice, and you can consult the exercises in the next section to help you develop this skill.



%%%%%%%%%%%%%%%%%%%%%%%%%%%%%%%%%%
%%%%%%%%%%%%%%%%%%%%%%%%%%%%%%%%%%
% Exercises: logical operators chapter

%\filbreak

\section{Practice exercises}
\setcounter{ProbPart}{0}
%\practiceproblems

\problempart Using the symbolization key given, translate each English sentence into TFL.\label{pr.monkeysuits}
	\begin{ekey}
		\item[A] Those creatures are aliens. 
		\item[C] Those creatures are centaurs. 
		\item[V] Those creatures are vampires.
	\end{ekey}
Always use capital letters for the atomic sentences, and, in this case, be especially careful to distinguish between $V$ and $\eor$.

\begin{earg}
\item Those creatures are not aliens.
\item Those creatures are aliens, or they are not.
\item Those creatures are either vampires or centaurs.
\item Those creatures are neither vampires nor centaurs.
\item If those creatures are centaurs, then it is not the case that they are vampires or aliens.
\item Either those creatures are aliens, or they are both centaurs and vampires.
\end{earg}

\problempart Using the symbolization key given, translate each English sentence into TFL.
\begin{ekey}
\item[A] Mr. Adams was murdered.
\item[B] The butler did it.
\item[C] The cook did it.
\item[D] The Duchess is lying.
\item[E] Mr. Edwards was murdered.
\item[F] The murder weapon was a frying pan.
\end{ekey}

\begin{earg}
\item Either Mr. Adams or Mr. Edwards was murdered.
\item If Mr. Adams was murdered, then the cook did it.
\item If Mr. Edwards was murdered, then the cook did not do it.
\item Either the butler did it, or the Duchess is lying.
\item The cook did it only if the Duchess is lying. 
\item If the murder weapon was not a frying pan, then the cook did not do it.
\item If the murder weapon was not a frying pan, then either the cook or the butler did it.
\item Mr. Adams was murdered if and only if Mr. Edwards was not murdered.
\item It is not the case that either the Duchess is lying or Mr. Edwards was not murdered.
\item If Mr. Adams was murdered, he was killed with a frying pan.
\item The cook did it, and the butler did not.
\item Of course the Duchess is lying!
\end{earg}

\problempart Using the symbolization key given, translate each English sentence into TFL.\label{pr.avacareer}
	\begin{ekey}
		\item[E_1] Ava is an electrician.
		\item[E_2] Harrison is an electrician.
		\item[F_1] Ava is a firefighter.
		\item[F_2] Harrison is a firefighter.
		\item[S_1] Ava is satisfied with her career.
		\item[S_2] Harrison is satisfied with his career.
	\end{ekey}
\begin{earg}
\item Ava and Harrison are both electricians.
\item If Ava is a firefighter, then she is satisfied with her career.
\item Ava is a firefighter, unless she is an electrician.
\item Harrison is an unsatisfied electrician.
\item Neither Ava nor Harrison is an electrician.
\item Both Ava and Harrison are electricians, but Ava is satisfied with her career and Harrison is not satisfied with his career.
\item Harrison is satisfied with his career only if he is a firefighter.
\item If Ava is not an electrician, then neither is Harrison, but if she is, then he is too.
\item Ava is satisfied with her career if and only if Harrison is not satisfied with his.
\item If Harrison is both an electrician and a firefighter, then he is satisfied with his career.
\item It is not the case that Harrison is both an electrician and a firefighter.
\item Harrison and Ava are both firefighters if and only if neither of them is an electrician.
\end{earg}

\problempart
Using the symbolization key given, translate each English-language sentence into TFL.
\label{pr.jazzinstruments}
\begin{ekey}
\item[J_1] John Coltrane played tenor sax.
\item[J_2] John Coltrane played soprano sax.
\item[J_3] John Coltrane played tuba
\item[M_1] Miles Davis played trumpet
\item[M_2] Miles Davis played tuba
\end{ekey}

\begin{earg}
\item John Coltrane played tenor and soprano sax. 
\item Neither Miles Davis nor John Coltrane played tuba. 
\item John Coltrane did not play both tenor sax and tuba. 
\item John Coltrane did not play tenor sax unless he also played soprano sax. 
\item John Coltrane did not play tuba, but Miles Davis did. 
\item Miles Davis played trumpet only if he also played tuba. 

\item If Miles Davis played trumpet, then John Coltrane played at least one of these three instruments: tenor sax, soprano sax, or tuba. 

\item It is not the case that if John Coltrane played tuba then Miles Davis played trumpet or tuba. 

\item Miles Davis and John Coltrane both played tuba if and only if Coltrane did not play tenor sax and Miles Davis did not play trumpet. 
\end{earg}


\problempart
\label{pr.spies}
Give a symbolization key, and then translate the following English sentences into TFL.
\begin{earg}
\item It is not the case that Alice and Bob are both spies.
\item If either Alice or Bob is a spy, then the code has been broken.
\item If neither Alice nor Bob is a spy, then the code has not been unbroken.
\item The letter is in the German embassy, unless someone has broken the code.
\item Either the code has been broken or it has not, but the letter is in German embassy regardless.
\item Either Alice or Bob is a spy, but not both.
\end{earg}


\problempart
For each argument, first, make a symbolization key, and then translate all of the sentences of the argument into TFL.
\begin{earg}
\item If Dorothy plays the piano in the morning, then Roger wakes up cross. Dorothy plays piano in the morning unless she is distracted. So, if Roger does not wake up cross, then Dorothy must be distracted.

\item It will either rain or snow on Tuesday. If it rains, Neville will be gloomy. If it snows, Neville will be cold. Therefore, Neville will either be gloomy or cold on Tuesday.

\item If Zoey remembered to do her chores, then the house is clean but not neat. If she forgot, then the house is neat but not clean. Therefore, the house is either neat or clean; but not both.
\end{earg}


%\problempart
%We symbolized an \emph{exclusive or} using `$\eor$', `$\eand$', and `$\enot$'. How could you symbolize an \emph{exclusive or} using only two operators? Is there any way to symbolize an \emph{exclusive or} using only one operator?




%%%%%%%%%%%%%%%%%%%%%%%%%%%%%%%%%%%%%%%%%%%%%%%%%
% Answers

\section{Answers}
\setcounter{ProbPart}{0}

\problempart 
	\begin{ekey}
		\item[A] Those creatures are aliens. 
		\item[C] Those creatures are centaurs. 
		\item[V] Those creatures are vampires.
	\end{ekey}
\begin{earg}
\item Those creatures are not aliens.
\item[] \myanswer{$\enot A$}
\item Those creatures are aliens, or they are not.
\item[] \myanswer{$(A \eor \enot A$)} 
\item Those creatures are either vampires or centaurs.
\item[] \myanswer{$(V \eor C)$}
\item Those creatures are neither vampires nor centaurs.
\item[] \myanswer{$\enot (C \eor V)$}
\item If those creatures are centaurs, then it is not the case that they are vampires or aliens.
\item[] \myanswer{$(C \eif \enot(V \eor A))$}
\item Either those creatures are aliens, or they are both centaurs and vampires.
\item[] \myanswer{$(A \eor (C \eand V))$}
\end{earg}

\problempart 
\begin{ekey}
\item[A] Mr. Adams was murdered.
\item[B] The butler did it.
\item[C] The cook did it.
\item[D] The Duchess is lying.
\item[E] Mr. Edwards was murdered.
\item[F] The murder weapon was a frying pan.
\end{ekey}
\begin{earg}
\item Either Mr. Adams or Mr. Edwards was murdered.
\item[] \myanswer{$(A \eor E)$}
\item If Mr. Adams was murdered, then the cook did it.
\item[] \myanswer{$(A \eif C)$}
\item If Mr. Edwards was murdered, then the cook did not do it.
\item[] \myanswer{$(E \eif \enot C)$}
\item Either the butler did it, or the Duchess is lying.
\item[] \myanswer{$(B \eor D)$}
\item The cook did it only if the Duchess is lying. (See table \ref{table.conditional.English}.) 
\item[] $(C \eif D)$
\item If the murder weapon was not a frying pan, then the cook did not do it.
\item[] \myanswer{$(\enot F \eif \enot C)$}
\item If the murder weapon was not a frying pan, then either the cook or the butler did it.
\item[] \myanswer{$(\enot F \eif (C \eor B))$}
\item Mr. Adams was murdered if and only if Mr. Edwards was not murdered.
\item[] \myanswer{$(A \eiff \enot E)$}
\item It is not the case that either the Duchess is lying or Mr. Edwards was not murdered.
\item[] \myanswer{$\enot(D \eor \enot E)$}
\item If Mr. Adams was murdered, he was killed with a frying pan.
\item[] \myanswer{$(A \eif F)$}
\item The cook did it, and the butler did not.
\item[] \myanswer{$(C \eand \enot B)$}
\item Of course the Duchess is lying!
\item[] \myanswer{$D$}
\end{earg}


\problempart\label{pr.avacareer}
	\begin{ekey}
		\item[E_1] Ava is an electrician.
		\item[E_2] Harrison is an electrician.
		\item[F_1] Ava is a firefighter.
		\item[F_2] Harrison is a firefighter.
		\item[S_1] Ava is satisfied with her career.
		\item[S_2] Harrison is satisfied with his career.
	\end{ekey}
\begin{earg}
\item Ava and Harrison are both electricians.
\item[] \myanswer{$(E_1 \eand E_2)$}
\item If Ava is a firefighter, then she is satisfied with her career.
\item[] \myanswer{$(F_1 \eif S_1)$}
\item Ava is a firefighter, unless she is an electrician.
\item[] \myanswer{$(F_1 \eor E_1)$}
\item Harrison is an unsatisfied electrician.
\item[] \myanswer{$(E_2 \eand \enot S_2)$}
\item Neither Ava nor Harrison is an electrician.
\item[] \myanswer{$\enot (E_1 \eor E_2)$}
\item Both Ava and Harrison are electricians, but Ava is satisfied with her career and Harrison is not satisfied with his career.
\item[] \myanswer{$((E_1 \eand E_2) \eand (S_1 \eand \enot S_2))$}
\item Harrison is satisfied with his career only if he is a firefighter.
\item[] \myanswer{$(S_2 \eif F_2)$}
\item If Ava is not an electrician, then neither is Harrison, but if she is, then he is too.
\item[] \myanswer{$((\enot E_1 \eif \enot E_2) \eand (E_1 \eif  E_2))$}
\item Ava is satisfied with her career if and only if Harrison is not satisfied with his.
\item[] \myanswer{$(S_1 \eiff \enot S_2)$}
\item If Harrison is both an electrician and a firefighter, then he is satisfied with his career.
\item[] \myanswer{$((E_2 \eand F_2) \eif S_2)$}
\item It is not the case that Harrison is both an electrician and a firefighter.
\item[] \myanswer{$\enot (E_2 \eand F_2)$}
\item Harrison and Ava are both firefighters if and only if neither of them is an electrician.
\item[] \myanswer{$((F_2 \eand F_1) \eiff \enot(E_2 \eor E_1))$}
\end{earg}

\problempart
\label{pr.jazzinstruments}
\begin{ekey}
\item[J_1] John Coltrane played tenor sax.
\item[J_2] John Coltrane played soprano sax.
\item[J_3] John Coltrane played tuba
\item[M_1] Miles Davis played trumpet
\item[M_2] Miles Davis played tuba
\end{ekey}

\begin{earg}
\item John Coltrane played tenor and soprano sax. 
\item[~] \myanswer{$J_1 \eand J_2$} 
\medskip

\item Neither Miles Davis nor John Coltrane played tuba.
\item[~] \myanswer{$\enot(M_2 \eor J_3)$ or $\enot M_2 \eand \enot J_3$}
\medskip

\item John Coltrane did not play both tenor sax and tuba. 
\item[~] $\enot(J_1 \eand J_3)$ or $\enot J_1 \eor \enot J_3$
\medskip

\item If John Coltrane did not play tenor sax, then he played soprano sax. 
\item[~] \myanswer{$\enot J_1 \eif J_2$}
\medskip

\item John Coltrane did not play tuba, but Miles Davis did. 
\item[~] \myanswer{$\enot J_3 \eand M_2$}
\medskip

\item Miles Davis played trumpet only if he also played tuba. 
\item[~] \myanswer{$M_1 \eif M_2$} 
\medskip

\item If Miles Davis played trumpet, then John Coltrane played at least one of these three instruments: tenor sax, soprano sax, or tuba. 
\item[~] $M_1 \eif (J_1 \eor (J_2 \eor J_3))$ or $M_1 \eif ((J_1 \eor J_2) \eor J_3)$  
\medskip

\item It is not the case that if John Coltrane played tuba, then Miles Davis played trumpet or tuba. 
\item[~] $\enot(J_3 \eif (M_1 \eor M_2))$
\medskip

\item Miles Davis and John Coltrane both played tuba if and only if Coltrane did not play tenor sax and Miles Davis did not play trumpet. 
\item[~] \myanswer{$(J_3 \eand M_2) \eiff (\enot J_1 \eand \enot M_1)$ or $(J_3 \eand M_2) \eiff \enot (J_1 \eor M_1)$} 
\end{earg}

\problempart
\label{pr.spies}
\myanswer{\begin{ekey}
\item[A] Alice is a spy.
\item[B] Bob is a spy.
\item[C] The code has been broken.
\item[L] The letter is in German embassy.
\end{ekey}}
\begin{earg}
\item It is not the case that Alice and Bob are both spies.
\item[] \myanswer{$\enot(A \eand B)$}
\item If either Alice or Bob is a spy, then the code has been broken.
\item[] \myanswer{$((A \eor B) \eif C)$}
\item If neither Alice nor Bob is a spy, then the code has not been unbroken.
\item[] \myanswer{$\enot (A \eor B) \eif \enot C$}
\item The letter is in the German embassy, unless someone has broken the code.
\item[] \myanswer{$(L \eor C)$}
\item Either the code has been broken or it has not, but the letter is in German embassy regardless.
\item[] \myanswer{$((C \eor \enot C) \eand L)$}
\item Either Alice or Bob is a spy, but not both.
\item[] \myanswer{$((A \eor B) \eand \enot (A \eand B))$}
\end{earg}



\problempart
For each argument, write a symbolization key and symbolize all of the sentences of the argument in TFL.
\begin{earg}
\item If Dorothy plays the piano in the morning, then Roger wakes up cross. Dorothy plays piano in the morning unless she is distracted. So, if Roger does not wake up cross, then Dorothy must be distracted.
\myanswer{\begin{ekey}
\item[P] Dorothy plays the piano in the morning.
\item[C] Roger wakes up cross.
\item[D] Dorothy is distracted.
\end{ekey}}
\item[] \myanswer{$(P \eif C), (P \eor D) \proves (\enot C \eif D)$}
\bigskip

\item It will either rain or snow on Tuesday. If it rains, Neville will be gloomy. If it snows, Neville will be cold. Therefore, Neville will either be gloomy or cold on Tuesday.
\myanswer{\begin{ekey}
\item[T_1] It rains on Tuesday
\item[T_2] It snows on Tuesday
\item[G] Neville is gloomy on Tuesday
\item[C] Neville is cold on Tuesday
\end{ekey}}
\item[] \myanswer{$(T_1 \eor T_2), (T_1 \eif G), (T_2 \eif C) \proves (G \eor C)$}
\bigskip

\item If Zoey remembered to do her chores, then the house is clean but not neat. If she forgot, then the house is neat but not clean. Therefore, the house is either neat or clean; but not both.
\myanswer{\begin{ekey}
\item[Z] Zoey remembered to do her chores
\item[C] The house is clean.
\item[N] The house is neat.
\end{ekey}}
\item[] \myanswer{$(Z \eif (C \eand \enot N)), (\enot Z \eif (N \eand \enot C)) \proves ((N \eor C) \eand \enot (N \eand C))$.}
\end{earg}


%\problempart
%We symbolized an \emph{exclusive or} using `$\eor$', `$\eand$', and `$\enot$'. How could you symbolize an \emph{exclusive or} using only two connectives? Is there any way to symbolize an \emph{exclusive or} using only one connective?

%For two connectives, we could use any of the following: 
%\begin{center}
%$\enot(\meta{A} \eiff \meta{B})$\\
%$(\enot\meta{A} \eiff \meta{B})$\\
%$(\enot (\enot \meta{A} \eand \enot \meta{B}) \eand \enot (\meta{A} \eand \meta{B}))$
%\end{center}
%But if we want to symbolize the exclusive-or using only one connective, then we would have to introduce a new connective.




%%%%%%%%%%%%%%%%%%%%%%%%%%%%%%%%%%%%%%%%%%%%%%%%%
%%%%%%%%%%%%%%%%%%%%%%%%%%%%%%%%%%%%%%%%%%%%%%%%%

% Chapter: Sentences of TFL

%%%%%%%%%%%%%%%%%%%%%%%%%%%%%%%%%%%%%%%%%%%%%%%%%
%%%%%%%%%%%%%%%%%%%%%%%%%%%%%%%%%%%%%%%%%%%%%%%%%


\chapter{Sentences of TFL}\label{s:TFLSentences}

`Bring with thee airs from heaven or blasts from hell' is a sentence of English. `$(P \eor Q)$' is a sentence of TFL. Oddly, although we can identify sentences of English when we encounter them, there is not a formal definition of \textit{sentence of English} that will tell us, for any possible combination of words and punctuation, whether or not it is a sentence of English. It is possible, however, to provide such a definition for sentences of TFL, and we will examine that definition in this chapter. 

\section{Expressions}

\begin{table*}\centering\sffamily\footnotesize
\ra{1.25}
\begin{tabular}{@{}l l@{}}\toprule
atomic sentences & $A,B,C,\ldots,Z$\\
\enspace {with subscripts if needed} & $A_1, A_2, A_3,A_4, \ldots, J_{10}, J_{11}, \ldots$\\
logical operators & $\enot, \eand, \eor, \eif, \eiff$\\
parentheses &( , )\\
\bottomrule
\end{tabular}
\caption{The three types of symbols of TFL}\label{table.symbols-TFL}
\label{TFL-symbols}
\end{table*}

You have been introduced to the symbols of TFL in the previous two chapters. They are also summarized in table \ref{TFL-symbols}. We define an \define{expression of TFL} as any string of symbols of TFL. Take any of the symbols of TFL and write them down, in any order, and you have an expression of TFL.


\section{Sentences}\label{s:Sentences}
Many expressions of TFL will be total gibberish. We want to know when an expression of TFL amounts to a \emph{sentence}. To that end, we have the following seven rules, which are one part of the grammar of TFL.


%Instead, we will describe the process by which sentences can be constructed. Consider negation. Take any sentence of TFL and call it \meta{A}. Since \meta{A} is a sentence of TFL, putting an `$\enot$' before it will yield a new sentence of TFL: $\enot\meta{A}$.

%We can stipulate similar rules for each of the other logical operators. For instance, if \meta{A} and \meta{B} are sentences of TFL, then combining them with an `$\eand$' and brackets will yield a new sentence of TFL: $(\meta{A}\eand\meta{B})$. Providing rules like this for all of the logical operators, we arrive at the following formal definition for a \define{sentence of TFL}.

\begin{factboxy}{Sentences of TFL}\label{TFLsentences}
	\begin{enumerate}
		\item Every atomic sentence is a sentence.
		\item If \meta{A} is a sentence, then $\enot\meta{A}$ is a sentence.
		\item If \meta{A} and \meta{B} are sentences, then $(\meta{A}\eand\meta{B})$ is a sentence.
		\item If \meta{A} and \meta{B} are sentences, then $(\meta{A}\eor\meta{B})$ is a sentence.
		\item If \meta{A} and \meta{B} are sentences, then $(\meta{A}\eif\meta{B})$ is a sentence.
		\item If \meta{A} and \meta{B} are sentences, then $(\meta{A}\eiff\meta{B})$ is a sentence.
		\item Nothing else is a sentence.
	\end{enumerate}
%\tcblower
\hrule
\medskip
\footnotesize{Notice that \meta{A} and $A$ are different fonts. $A$ is an atomic sentence in TFL. \meta{A} is not, actually, part of TFL. Rather, it stands for any sentence in TFL. That sentence could be $A$ or it could be $(B \eif D)$ or anything else. This use of \textit{metavariables} is explained more fully in \S\ref{s:Metavariables}.}
\end{factboxy}

From the previous chapter, you have the basic idea about how the logical operators are used. The simplest cases are when `$\eand$', `$\eor$', `$\eif$', or `$\eiff$' only connect two atomic sentences or when `$\enot$' is before a single atomic sentence, as we find here:  
\begin{earg}
\item[] $(P \eiff Q)$
\item[] $\enot R$
\end{earg}
But our logic system must be more complex than this to be useful. To see how we can create more complex sentences in TFL, we will start with these two sentences:
\begin{earg}
\item[] $(P \eand S)$
\item[] $(R \eif T)$
\end{earg}
Now, let’s see how we would express ``$(P \eand S)$ \textit{or} $(R \eif T)$.'' Although 
`$(P \eand S)$' and
`$(R \eif T)$'
are both composed of two atomic sentences and a logical operator, we can treat each as a unit and combine them with a `$\eor$’ like this:
\begin{earg}
\item[]$((P \eand S) \eor (R \eif T))$
\end{earg}
\noindent Similarly, we can express ``\textit{not} $(P \eand S)$'' by treating the `$(P \eand S)$' as a unit and adding a `$\enot$’:
\begin{earg}
\item[] $\enot(P \eand S)$
\end{earg}
We can even, if we need to do so, express ``\textit{not} $\enot(P \eand S)$'' by implementing the same procedure again. Now, `$\enot(P \eand S)$' is the unit and we add a `$\enot$’ to it:
\begin{earg}
\item[] $\enot\enot(P \eand S)$
\end{earg}

These procedures can be used to form an infinite number of sentences of TFL. We just have to be sure that we are following rules 1 -- 7, given above.  And when following these rules, we must remember that $\meta{A}$ and $\meta{B}$ are units that can stand for either single atomic sentences or longer sentences. In fact, they can stand for sentences of any length.

Ultimately, you want to be able to just look at an expression and tell whether or not it is a correctly formed sentence of TFL, and with time you will be able to do so. Here are some examples of sentences of TFL:
\begin{earg}
\item $((P \eand R) \eif (S \eif T))$
\item $(P \eand (R \eif (S \eif T)))$
\item $((P \eiff \enot S) \eor \enot(T \eiff R))$
\item $((R \eiff T) \eand \enot(P \eor (Q \eor \enot T)))$
\item $\enot(P \eif \enot(R \eor (S \eiff T)))$
\item $((S \eor P) \eand \enot(R \eor \enot\enot R))$
\end{earg}
These, on the other hand, are \textit{\textbf{not}} sentences of TFL because each violates one or more of 1 -- 7:
\begin{earg}
\item $(P\enot \eand R)$
\item $((R \eand \eor S) \eif Q)$
\item $(R \eand Q \eif)$
\item $((P \enot Q) \eand (R \eiff T))$
\item $(P,Q \eiff T)$
\item $(P \eand Q \eand R)$
\end{earg}

You will learn to recognize sentences of TFL more quickly if you write neatly and space the atomic sentences, logical operators, and parentheses as is shown in this textbook. Spaces are not actually part of TFL, and so technically, you don't need to use them. But just as you would never add or drop spaces when writing sentences in English, you should not add or drop spaces when using TFL.


\section{The main logical operator}
\label{main_logical_operator}

Setting aside the `$\enot$' for a moment, each of the other logical operators, as you know, combine two sentences (which may, themselves, be composed of multiple sentences). The two sentences that are being combined by a logical operator comprise that operator's \define{scope}. 

\begin{factboxy}{Scope}
For `$\eand$', `$\eor$', `$\eif$', and `$\eiff$', the \define{scope} of the logical operator is the two sentences that the logical operator is conjoining. 
\smallskip

For `\enot', the \define{scope} is the sentence being negated by this operator.
\end{factboxy}

So, the scope of the `$\eand$' in `$(P \eand Q)$' is the `$P$' and the `$Q$'. And the scope of the `$\eand$' in `$((P \eand Q) \eif T)$' is still just the `$P$' and the `$Q$'. The scope of the `$\eif$' in `$((P \eand Q) \eif T)$', meanwhile, is `$((P \eand Q)$' and `$T$'.



\begin{comment}
Consider this sentence: 
\begin{earg}
\item[] {Dr. Wilson is in his  office, and if she is teaching today, then Dr. Cook is in her office}. 
\end{earg}
This sentence contains two logical operators, `and' and `if \ldots, then\ldots', and one of them is the \define{main logical operator} of the sentence. The main logical operator determines, at the most general level, what kind of sentence it is---a conjunction, a disjunction, a conditional, a biconditional, or a negation. The sentence above is a conjunction. Thus, the `and' is the main logic operator, and these are the two conjuncts: 
\begin{earg}
\item[1.] {Dr. Wilson is in his office}.
\item[2.] {If she is teaching today, then Dr. Cook is in her office}. 
\end{earg}
The second conjunct is a conditional, but the \textit{if\ldots, then\ldots} only applies to \textit{she is teaching today} and \textit{Dr. Cook is in her office}, not to the whole sentence. 

Now let's look at this sentence: 
\begin{ebullet}
	\item[] {If today is not Saturday, then Amy is at work and Kate is at school}. 
\end{ebullet}
	
\noindent This is a conditional. The antecedent is \textit{today is not Saturday}, and the consequent is \textit{Amy is at work and Kate is at school}. So, although there are three logical operators in this sentence, the main one is the \textit{if ..., then ...} (and so if we translated this sentence into TFL, the main logical operator would be the `$\eif$'). 

Even though the sentence is a conditional, the \textit{not} and the \textit{and} each have a role. But their roles are limited to only a part of the sentence. Take the \textit{not}. It applies to (i.e., negates) \textit{today is Saturday}.  
The \textit{and}'s job, meanwhile, is to create a conjunction by joining \textit{Amy is at work} and \textit{Kate is at school}. 
\end{comment}

Here is a more complex example:

$$(((T \eif P) \eand R) \eor (S \eiff Q))$$

\begin{ebullet}
	\item[] The scope of the `$\eif$' is `$T$' and `$P$'. 
	\item[] The scope of the `$\eand$' is `$(T \eif P)$' and `$R$'.
	\item[] The scope of the `$\eiff$' is `$S$' and `$Q$'.
	\item[] The scope of the `$\eor$' is `$((T \eif P) \eand R)$' and `$(S \eiff Q)$'.
\end{ebullet}

When the scope of a logical operator is the whole sentence (besides that operator itself), then that logical operator is the \define{main logical operator} for the sentence. So, in the previous example, the `$\eor$' is the main logical operator. This means that the sentence is a disjunction. (The subsentences are different kinds of sentences, but the whole sentence is a disjunction.)

Being able to identify the main logical operator is very important for what you will be learning in parts 3 and 4 of this textbook, and so when you see a sentence of TFL, you always want to determine the scope of each logical operator and the identify the main logical operator. 

\begin{comment}
Now, let's turn to expressions in TFL. Although identifying the main logical operator in a long expression in TFL can seem confusing at first, because we are using parentheses, you'll find that it's not too difficult. Let's start with this example: $((P \eand Q) \eor R)$. This is a disjunction. One disjunct is $(P \eand Q)$ and the other is $R$. Hence, the main logical operator is the `$\eor$'. 

Let's change the expression to $\enot((P \eand Q) \eor R)$. This is a negation, and so the main logical operator is the `$\enot$'. Notice that the `$\enot$' is outside of the brackets that enclose the entire `$(P \eand Q) \eor R$'. That means that the `$\enot$' applies to the entire sentence. Hence, it is the main logical operator. 
\end{comment}

Here are some more examples:
\begin{earg}
\item[\ex{logic-operator1}] $((P \eand R) \eif (\enot Q \eand S))$ ~~~The main logical operator is the `$\eif$'. 
\item[\ex{logic-operator2}] $(((S \eor N) \eiff Q) \eand (T \eif R))$ ~~~The main logical operator is the `$\eand$'.
\item[\ex{logic-operator3}] $(((\enot D \eor N) \eif R) \eiff T)$ ~~~The main logical operator is the`$\eiff$'.
\item[\ex{logic-operator4}] $(P \eand (T \eiff (Q \eor R)))$ ~~~The main logical operator is the `$\eand$'. 
\item[\ex{logic-operator5}] $((\enot E \eor F) \eiff G)$ ~~~The main logical operator is the `$\eiff$'.
\end{earg}



Unlike the other operators, the `$\enot$', doesn't connect two sentences; it just negates a sentence (which may be composed of multiple subsentences). Hence, the scope of the `$\enot$' is the sentence that is being negated. For each of these examples, the scope of the `$\enot$' is the whole sentence: 

\begin{earg}
\item[\ex{neg1}] $\enot Q$
\item[\ex{neg2}] $\enot(P \eor T)$
\item[\ex{neg3}] $\enot(R \eor (S \eif N))$
\item[\ex{neg4}] $\enot((P \eand R) \eor (S \eif T))$
\end{earg}
And since, in each of these examples, the scope of the `$\enot$' is the whole sentence, the `$\enot$' is the main logical operator in each case.


Once the main logical operator has been identified, we know what kind of sentence we have and what its components are. (This will be super important when we get to chapter \ref{s:BasicTFL}.)
If it is a conjunction, then part of the sentence will be one conjunct and the rest will be the other conjunct (and nothing will be left over). If it's a disjunction, then part of the sentence will be one disjunct and the rest will the other disjunct, again with nothing left over. If it's a conditional, then part of the sentence will be the antecedent and the rest will be the consequent. And if it's a negation, then the whole sentence (minus the `not' itself) is being negated.


\subsection{another way of indentifying the main logical operator}

Alternatively, when the sentence includes the outermost brackets, you can find the main logical operator by using this method:
\begin{ebullet}
	\item[(1)] If the first symbol in the sentence is `$\enot$', then that is the main logical operator.
	\item[(2)] Otherwise, start counting the brackets by following one of these two procedures. (The open-bracket is `(' and the closed bracket is `)'.) 
	\begin{earg}
	\item[(2a)] Start from the left, and begin counting. For each open-bracket add $1$, and for each closing-bracket, subtract $1$. When your count is at exactly $1$, the next operator you come to (\emph{apart} from a `$\enot$') is the main logical operator. 
	\item[(2b)] If starting at the left-side of the sentence doesn't seem to work, follow the same procedure, but begin at the far right and work left. % Is working in this direction the only way that an `$\enot$' would be the next logical operator, but not the main one?
	\end{earg}
\end{ebullet}

As we will discuss in the next section, in some cases, it is acceptable to omit the outermost brackets in a sentence of TFL. For instance, although it is not strictly allowable according to the rules given in \S\ref{s:Sentences}, because it will not introduce any confusion or ambiguity, we can write `$(P \eand R) \eif Q$' instead of `$((P \eand R) \eif Q)$'. When the outermost brackets are dropped, we add (3) and (4) to our method.

\begin{ebullet}
\item[(3)] When the outermost brackes are omitted, (2a) and (2b) can still be used, but stop when your count gets to zero instead of $1$.
\item[(4)] For sentences that contain two or more atomic sentences, if `$\enot$' is the main logical operator, then the outermost brackets have to be used. (When `$\enot$' is the main logical operator---as it is in this example: $\enot((P \eand Q) \eor R)$---the `$\enot$' will be outside the outermost brackets.) 
\item[(5)] For sentences that contain two or more atomic sentences, when the outermost brackets are omitted, (1) no longer applies and `$\enot$' won't be the main logical operator. 
\end{ebullet}

\begin{comment}
\subsection{Scope}

Finally, let's define the \define{scope} of a logical operator. Basically, the scope of a logical operator is the part of the sentence to which the operator applies (or as Lemmon says, ``what a particular occurrence of a connective controls''). We give the precise definition in terms of the main logical operator for the whole sentence and the main logical operators for any sub-sentences contained therein. 

\begin{factboxy}{Scope}
The \define{scope} of a logical operator is the sentence or sub-sentence for which that logical operator is the main logical operator.\\
Alternatively, Lemmon defines the \textit{scope of a logical operator} as `the shortest sentence in which the logical operator appears'.
\end{factboxy}

The scope of the main logical operator is always the entire sentence. The scope of every other logical operator is a sub-sentence. Consider this sentence:

$$(\enot(R \eand T) \eiff (P \eif \enot Q))$$

The main logical operator is the `$\eiff$'. Therefore, the scope of `$\eiff$' is the entire sentence. 
The scope of the `$\enot$' is `$\enot(R \eand T)$', which means that `$\enot$' is the main logical operator for that sub-sentence. 
Similarly, the `$\eand$' is the main logical operator for just the `$(R \eand T)$', and so the scope of the `$\eand$' is `$(R \eand T)$'. 
The `$\eif$' is the main logical operator for `$(P \eif \enot Q)$'. And the `$\enot$' is the main logical operator for `$\enot Q$'. Hence, the scope of each are those respective sub-sentences. 

\end{comment}


\section{Using parentheses}
\label{TFLconventions}

Parentheses are required for any sentence of TFL containing two or more atomic sentences. For instance, even in a simple sentence such as `$(Q \eand R)$', they are required. One reason for this is because the rules given in \S\ref{s:Sentences} require it. Those rules don't make a distinction between sentences containing only two atomic sentences and sentences containing more than two. They just tell us to use parentheses in any sentence containing a `$\eand$', `$\eor$', `$\eif$', or `$\eiff$'. Another reason for using parentheses is that we might make `$(Q \eand R)$' a sub-sentence in a more complex sentence. For example, we might want to negate `$(Q \eand R)$', which would give us `$\enot(Q \eand R)$'. If we just had `$Q \eand R$' without the parentheses and put a `$\enot$' in front of it, we would have `$\enot Q \eand R$', which has a different meaning than `$\enot(Q\eand R)$'. 

That said, there are some convenient conventions that we can use as long as we are careful. First, as long as the entire sentence is not about to become a sub-sentence,  we can omit the sentence's \emph{outermost} parentheses. Thus, we allow ourselves to write `$Q \eand R$' instead of `$(Q \eand R)$' when `$Q \eand R$' is the whole sentence. We must remember, however, to put parentheses around it when we want to embed the sentence into a more complex one.

Second, it can be a bit difficult to stare at long sentences with many nested pairs of parentheses. To make things a bit easier on the eyes, we will allow ourselves to use square brackets, `[' and `]', in addition to rounded ones. So, there is no logical difference, for example, between `$(P\eor Q)$' and `$[P\eor Q]$'. 

Combining these two conventions, we can rewrite this sentence:
$$(((H \eif I) \eor (I \eif H)) \eand (J \eor K))$$
like this:
$$[(H \eif I) \eor (I \eif H)] \eand (J \eor K)$$
The scope of each logical operator is now much easier to identify.



%%%%%%%%%%%%%%%%%%%%%%%%%%%%%%%%%%%%%%%%%%%
%%%%%%%%%%%%%%%%%%%%%%%%%%%%%%%%%%%%%%%%%%%
% Use and mention
%%%%%%%%%%%%%%%%%%%%%%%%%%%%%%%%%%%%%%%%%%%
%%%%%%%%%%%%%%%%%%%%%%%%%%%%%%%%%%%%%%%%%%%

\noindent\begin{minipage}{1.0\textwidth}
\section{Object language and metalanguage}\label{s:Metalanguage}

Consider these two sentences:
	\begin{earg}
		\item[\ex{JT1}] Justin Trudeau is a Prime Minister.
		\item[\ex{JT2}] \textit{Justin Trudeau} is the Canadian Prime Minister's name.
	\end{earg}
When we want to talk about the Prime Minister of Canada, which we are doing in sentence \ref{JT1}, we \textit{use} his name. When we want to talk about the Prime Minister's name, as we are in sentence \ref{JT2} we \emph{mention} the name.
\end{minipage}
\vspace{-3pt}

Since we are describing a formal language, truth-functional logic, we often \emph{mention} sentences of TFL. For example, we are mentioning `$P \eif Q$' in sentence \ref{TFL1}:
\begin{earg}
		\item[\ex{TFL1}] `$P \eif Q$' is a conditional.
\end{earg}

\noindent And likewise, we are mentioning TFL sentences in \ref{obj1} and \ref{obj2}. 
\begin{earg}
		\item[\ex{obj1}] `$D$' is an atomic sentence of TFL.
		\item[\ex{obj2}] `$\enot (\enot Q \eor R)$' is a sentence of TFL if `$(\enot Q \eor R)$' is a sentence of TFL.
\end{earg}

When we talk about a language (i.e., mention it), the language about which we are talking is called the \define{object language}. The language that we use to talk about the object language is called the \define{metalanguage}.\label{def.metalanguage} 

The object language that concerns us right now is TFL, but any language can be an object language. If, for instance, we want to talk about German, then German is the object language, and English---the language we are using to talk about German---is the metalanguage. Here is an example:
	\begin{earg}
		\item[\ex{Ger2}]`Schnee ist wei\ss' is a German sentence.
	\end{earg}
In sentence \ref{Ger2}, we are saying that the clause at the beginning of the sentence is a German sentence. 

Whenever we want to talk, in English, about some specific expression of TFL, we need to indicate that we are \textit{mentioning} the expression, rather than \textit{using} it. We can do this by using single quotation marks or italics (although italics are also sometimes used simply for emphasis and not to indicate that we are mentioning a term or sentence). In this textbook, we will also indicate that we are mentioning an expression by placing it centered on the page like this:
\vspace{-2mm}
$$\enot(\enot Q \eor R)$$

\section{Metavariables}\label{s:Metavariables}

Sometimes we refer to specific expressions of TFL like `$D$' and `$\enot (\enot Q \eor R)$'. Other times, however, we want to say something about an arbitrary expression of TFL, not a specific one. To do this, we use these uppercase letters:
\vspace{-2mm}
	$$\meta{A}, \meta{B}, \meta{C}, \meta{D}, \ldots$$
	
You probably noticed that we used these letters in our definition of a sentence of TFL in \S\ref{s:Sentences}. For instance, this is one rule in that definition:
\begin{earg}
\item[3.] If \meta{A} and \meta{B} are sentences of TFL, then $(\meta{A}\eand\meta{B})$ is a sentence of TFL.
\end{earg}
We use `$\meta{A}$' and `$\meta{B}$' (and other capital letters in this font) when we want the letter to stand for any possible sentence of TFL. Hence, `$\meta{A}$' can stand for `$A$' (or `$B$') or for `$(P \eor Q)$' or `$((R \eif T) \eand \enot Q)$' or anything else. 

`$\meta{A}, \meta{B}, \meta{C}, \meta{D}, \ldots$' are not actually part of TFL. Rather, they are part of the metalanguage---that is, English---that we use to talk about expressions of TFL. 
	
	\factoidbox{
	  A \define{metavariable} is a variable in the metalanguage (i.e., English) that represents any sentence in our formal language of TFL. The symbols $\meta{A}, \meta{B}, \meta{C}, \meta{D}, \ldots$ are used for the metavariables.}


%%%%%%%%%%%%%%%%%%%%%%%%%%%%%%%%%%%%
% Exercises for sentences of TFL chapter
%%%%%%%%%%%%%%%%%%%%%%%%%%%%%%%%%%%%

%\filbreak

\section{Practice exercises}
\setcounter{ProbPart}{0}

\problempart
\label{pr.wiffTFL}
For each of the following, (a) is it a sentence of TFL, strictly speaking, and (b) is it a sentence of TFL, allowing for our relaxed bracketing conventions? If, by either of those standards, it is a sentence of TFL, then (c) what is the main logical operator?
\begin{earg}
\item $(A)$
\item $J_{374} \eor \enot J_{374}$
\item $\enot \enot \enot \enot F$
\item $\enot \eand S$
\item $(G \eand \enot G)$
\item $(A \eif (A \eand \enot F)) \eor (D \eiff E)$
\item $[(Z \eiff S) \eif W] \eand [J \eor X]$
\item $(F \eiff \enot D \eif J) \eor (C \eand D)$
\end{earg}

%\problempart
%Are there any sentences of TFL that contain no atomic sentences? Explain your answer.\\

\problempart
What is the scope of each connective in this sentence?
$$[(H \eif I) \eor (I \eif H)] \eand (J \eor K)$$


%%%%%%%%%%%%%%%%%%%%%%%%%%%%%%%%%%%%%%%%%%%
% Answers

\section{Answers}
\setcounter{ProbPart}{0}

\problempart
\label{pr.wiffTFL}
For each of the following, (a) is it a sentence of TFL, strictly speaking, and (b) is it a sentence of TFL, allowing for our relaxed bracketing conventions? If, by either of those standards, it is a sentence of TFL, then (c) what is the main logical operator?
\begin{earg}
\item $(A)$\hfill \myanswer{(a) no (b) no}
\medskip

\item $J_{374} \eor \enot J_{374}$ \hfill \myanswer{(a) no (b) yes (c) the `$\eor$'}
\medskip

\item $\enot \enot \enot \enot F$ \hfill \myanswer{(a) yes (b) yes (c) the first `$\enot'$}
\medskip

\item $\enot \eand S$\hfill \myanswer{(a) no (b) no}
\medskip

\item $(G \eand \enot G)$\hfill \myanswer{(a) yes (b) yes (c) the `$\eand$'}
\medskip

\item $(A \eif (A \eand \enot F)) \eor (D \eiff E)$\hfill \myanswer{(a) no (b) yes (c) the `$\eor$'}
\medskip

\item $[(Z \eiff S) \eif W] \eand [J \eor X]$\hfill \myanswer{(a) no (b) yes (c) the `$\eand$'}
\medskip

\item $(F \eiff \enot D \eif J) \eor (C \eand D)$\hfill \myanswer{(a) no (b) no}
\medskip
\end{earg}

%\problempart
%Are there any sentences of TFL that contain no atomic sentences? Explain your answer.
%\\\myanswer{No. Atomic sentences contain atomic sentences (trivially). And every more complicated sentence is built up out of less complicated sentences, that were in turn built out of less complicated sentences, \ldots, that were ultimately built out of atomic sentences.}\\


\problempart $[(H \eif I) \eor (I \eif H)] \eand (J \eor K)$\medskip

\noindent The scope of the left-most `$\eif$' is `$(H \eif I)$'. 

\noindent The scope of the right-most `$\eif$' is `$(I \eif H)$'. 

\noindent The scope of the left-most `$\eor$ is `$[(H \eif I) \eor (I \eif H)]$'.

\noindent The scope of the right-most `$\eor$' is `$(J \eor K)$'.

\noindent The scope of the `$\eand$' is the entire sentence, and so the `$\eand$' is the main logical operator and the sentence is a conjunction.


