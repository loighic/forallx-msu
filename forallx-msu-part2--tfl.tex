%!TEX root = forallxyyc.tex
\part{Truth-functional logic}
\label{ch.TFL}
\addtocontents{toc}{\protect\mbox{}\protect\hrulefill\par}

\chapter{First steps to symbolization}

\section{Validity in virtue of form}\label{s:ValidityInVirtueOfForm}
Consider this argument:
	\begin{earg}
		\item[1.] It is raining outside.
		\item[2.] If it is raining outside, then Jenny is miserable.
		\item[3.] Therefore, Jenny is miserable.
	\end{earg}
and this one:
	\begin{earg}
		\item[1.] Jenny is a student.
		\item[2.] If Jenny is a student, then John is a spy.
		\item[3.] Therefore, John is a spy.
	\end{earg}
Both arguments are valid, and there is a straightforward sense in which we can say that they share a common structure. We might express the structure this way:
	\begin{earg}
		\item[1.] A
		\item[2.] If A, then C
		\item[3.] Therefore, C
	\end{earg}
This looks like an excellent argument \emph{structure}. Indeed, any argument with this structure or form will be valid. Now, consider this argument:
	\begin{earg}
		\item[1.] Jenny is either happy or sad.
		\item[2.] Jenny is not happy.
		\item[3.] Therefore, Jenny is sad.
	\end{earg}
Again, this argument is valid, and this is its structure:
	\begin{earg}
		\item[1.] A or B
		\item[2.] not A
		\item[3.] Therefore, B
	\end{earg}
Here is another example:
	\begin{earg}
		\item[1.] It's not the case that Jim both studied often and acted in lots of plays.
		\item[2.] Jim acted in lots of plays.
		\item[3.] Therefore, Jim did not study often.
	\end{earg}
This valid argument has a structure which we might represent this way:
	\begin{earg}
		\item[1.] not (A and B)
		\item[2.] A
		\item[3.] Therefore, not B
	\end{earg}
These examples illustrate an important idea, which we might describe as \emph{validity in virtue of form}. These arguments are valid, but in each case, that has nothing to do with the specific meaning of `Jenny is sad', `John is a spy', or `Jim acted in lots of plays'. Instead, these arguments are valid in virtue of the meanings of just these words: `and', `or', `not,' and `if\ldots, then\ldots'. 


\section{Validity for special reasons}
There are plenty of arguments that are valid, but not for reasons relating to their structure. This an example:
	\begin{earg}
		\item[1.] Juanita is a vixen
		\item[2.] Therefore, Juanita is a fox
	\end{earg}
It is impossible for the premise to be true and the conclusion false. So the argument is valid. However, the validity is not related to the form of the argument. Here is an invalid argument with the same form:
	\begin{earg}
		\item[1.] Juanita is a vixen
		\item[2.] Therefore, Juanita is a cathedral
	\end{earg}
This suggests that the validity of the previous argument \emph{is} keyed to the meaning of the words `vixen' and `fox'. But, whether or not that is right, it is not simply the structure of the argument that makes it valid. Equally, consider the argument:
	\begin{earg}
		\item[1.] The sculpture is green all over.
		\item[2.] Therefore, the sculpture is not red all over. 
	\end{earg}
Again, it seems impossible for the premise to be true and the conclusion false, for nothing can be both green all over and red all over. So the argument is valid, but here is an invalid argument with the same form:
	\begin{earg}
		\item[1.] The sculpture is green all over.
		\item[2.] Therefore, the sculpture is not shiny all over.
	\end{earg}
The argument is invalid, since it is possible to be green all over and shiny all over. Plausibly, the first argument about the sculpture is valid because of the way that colors (or color-words) interact, but, whether or not that is right, it is not simply the structure of the argument that makes it valid. 

The important point here is that we will be interested only in arguments that are valid or invalid because of their structure.


\section{Atomic sentences and symbolization}

We isolated the form of the arguments, in \S\ref{s:ValidityInVirtueOfForm} by replacing sentences and subsentences of sentences with individual letters. `It is raining outside' is a subsentence of `If it is raining outside, then Jenny is miserable', and we replaced that subsentence with `A'. 

This kind of representation---letters standing for sentences or subsentences---is central to the formal language that we develop in this book. We start with some \emph{atomic sentences}. Notice that if we extract `it is raining outside' and `Jenny is miserable' from `If it is raining outside, then Jenny is miserable', both `it is raining outside' and `Jenny is miserable' are, themselves, complete sentences. That is, they contain a subject, verb, and direct object. If we extract any part of `it is raining outside', however, we will not have a complete sentence. Thus, in terms of sentences `it is raining outside' is an atom, or, as we will call it, an \textit{atomic sentence}. It's the smallest collection of words that still constitute a sentence. 

Similarly, `Jenny is miserable', `Jenny is a student', `John is a spy', and `Jenny is happy' are atomic sentences. On the other hand, `If it is raining outside, then Jenny is miserable' and `Jenny is either happy or sad' are not atomic sentences. They are both sentences that are constructed out of two atomic sentences.

Atomic sentences are the basic building blocks used to form more complex sentences. We will use uppercase Roman letters for atomic sentences of TFL. There are only twenty-six letters of the alphabet, but there is no limit to the number of atomic sentences that we might want to consider. By adding subscripts to letters, we obtain new atomic sentences. Here, for instance, are five different atomic sentences of TFL:
	$$A, R, R_1, R_2, A_{234}$$
We will use atomic sentences to represent, or \emph{symbolize}, certain English sentences. To do this, we provide a \define{symbolization key}, such as the following:
	\begin{ekey}
		\item[A] It is raining outside
		\item[C] Jenny is miserable
	\end{ekey}
In doing this, we are not fixing this symbolization \emph{once and for all}. We are just saying that, for the time being, we will think of the atomic sentence `$A$' as symbolizing the English sentence `It is raining outside', and the atomic sentence of TFL, `$C$', as symbolizing the English sentence `Jenny is miserable'. Later, when we are dealing with different sentences or different arguments, we can provide a new symbolization key; as it might be:
	\begin{ekey}
		\item[A] Jenny is a student.
		\item[C] John is a spy.
	\end{ekey}



%%%%%%%%%%%%%%%%%%%%%%%%%%%%%%%%%%%%%%%%
%%%%%%%%%%%%%%%%%%%%%%%%%%%%%%%%%%%%%%%%
% CHAPTER 5
%%%%%%%%%%%%%%%%%%%%%%%%%%%%%%%%%%%%%%%%
%%%%%%%%%%%%%%%%%%%%%%%%%%%%%%%%%%%%%%%%



\chapter{Logical Operators}
\label{s:TFLConnectives}

At this point, we should clarify the task at hand. Truth-functional propositional logic is a branch of logic that focuses on the relationships between atomic sentences. One part of truth-functional propositional logic (or \textit{TFL} for short) is a formal language. This formal language consists of sentence letters, which stand for atomic sentences of English (although we won't always be concerned about the specific English sentences that they might represent), and the \textit{logical operators} `and', `or', `not', `if \ldots, then \ldots' and `if and only if'. A logical operator is a word or phrase that modifies a sentence or connects two sentences to form a more complex sentence. We call these operators \textit{truth-functional} because the truth of the complex sentences depends entirely on the truth of the atomic sentences of which they are composed. (They are also sometimes referred to as \textit{connectives} because, except in the case of `not', these operators connect two simpler sentences.)  

In addition to symbolizing English sentences with sentence letters, we also want to symbolize the truth-functional operators. The symbols that we will use are shown in table \ref{table.connectives}. The operators listed there are not the only connectives in English. Others are, for example, `unless', `neither \dots{} nor \dots', `necessarily', and `because'. As we will see, the first two can be expressed with the connectives that are in table \ref{table.connectives}. The last two, however, cannot. Although they are logical operators, `necessarily' and `because' are not truth functional.
	
\begin{table*}\centering\sffamily\footnotesize
\ra{1.25}
\begin{tabular}{@{}l l l@{}}\toprule
\textsc{symbol} & \textsc{the sentence's name} & \textsc{its meaning}\\\midrule
	\enot&negation&`It is not the case that$\ldots$'\\
	\eand&conjunction&`Both$\ldots$\ and $\ldots$'\\
	\eor&disjunction&`Either$\ldots$\ or $\ldots$'\\
	\eif&conditional&`If $\ldots$\ then $\ldots$'\\
	\eiff&biconditional&`$\ldots$ if and only if $\ldots$'\\
\bottomrule
\end{tabular}
\caption{}\label{table.connectives}
\end{table*}
	
Once we have introduced these logical operators (in this chapter and in chapter \ref{s:CharacteristicTruthTables}) and explained what can and cannot be a sentence in TFL (which we will do in chapter \ref{s:TFLSentences}) our formal language will be complete. Although the formal language is central, truth-functional propositional logic does not consist only of a formal language. There is also a \textit{deductive system}, which we will explore in part \ref{ch.NDTFL}. 

        
\section{Negation}

Consider how we might symbolize these sentences:
	\begin{earg}
	\item[\ex{not1}] Mary is in Barcelona.
	\item[\ex{not2}] It is not the case that Mary is in Barcelona.
	\item[\ex{not3}] Mary is not in Barcelona.
	\end{earg}
To begin, we need an atomic sentence. This will be our symbolization key:
	\begin{ekey}
		\item[B] Mary is in Barcelona.
	\end{ekey}
$B$ is sentence \ref{not1}. Sentence \ref{not2} is partially symbolized as `It is not the case that $B$'. In order to complete the symbolization, we need a symbol for `it is not the case that'. Or, in other words, a symbol that, when added to $B$ will express `the negation of $B$'. We will use `\enot' and symbolize sentence \ref{not2} as `$\enot B$'.

Sentence \ref{not3} also contains the word `not', and it is obviously equivalent to sentence \ref{not2}. As such, we can also symbolize it as `$\enot B$'.

\begin{factboxy}{Negation}
A sentence can be symbolized as $\enot\meta{A}$ if it can be paraphrased in English as `It is not the case that \ldots'
\end{factboxy}

Here are a few more examples:
	\begin{earg}
		\item[\ex{not4}] The cog can be replaced.
		\item[\ex{not5}] The cog is irreplaceable.
		\item[\ex{not5b}] The cog is not irreplaceable.
	\end{earg}
For these, we will use this representation key:
	\begin{ekey}
		\item[R] The cog is replaceable
	\end{ekey}
Sentence \ref{not4} is symbolized by `$R$'. Sentence \ref{not5} can be reworded as \textit{it is not the case that the cog is replaceable}. So even though sentence \ref{not5} does not contain the word `not', we will symbolize it `$\enot R$'.

Sentence \ref{not5b} can be paraphrased as `It is not the case that the cog is irreplaceable.' That sentence can then be paraphrased as `It is not the case that it is not the case that the widget is replaceable'. So we symbolize this English sentence as `$\enot\enot R$'.

But some care is needed when handling negations. Consider:
	\begin{earg}
		\item[\ex{not6}] Jane is happy.
		\item[\ex{not7}] Jane is unhappy.
	\end{earg}
If we `$H$' stand for `Jane is happy', then we can symbolize sentence \ref{not6} as `$H$'. It would be a mistake, however, to symbolize sentence \ref{not7} with `$\enot{H}$'. 
`$\enot{H}$' means `Jane is not happy', but `Jane is not happy' does not have the same meaning as `Jane is unhappy'. After all, Jane might be neither happy nor unhappy; her affect might just be neutral. In order to symbolize sentence \ref{not7}, then, we would need a different sentence letter.


\section{Conjunction}
\label{s:ConnectiveConjunction}

Let's start with these sentences:
	\begin{earg}
		\item[\ex{and1}]Adam is athletic.
		\item[\ex{and2}]Barbara is athletic.
		\item[\ex{and3}]Adam is athletic, and Barbara is also athletic.
	\end{earg}
We will need separate sentence letters to symbolize sentences \ref{and1} and \ref{and2}, and so we will use these:
	\begin{ekey}
		\item[A] Adam is athletic.
		\item[B] Barbara is athletic.
	\end{ekey}
Sentence \ref{and1} is symbolized as `$A$', and sentence \ref{and2} as `$B$'. Sentence \ref{and3} expresses `A and B'. To symbolize the `and'. We will use `\eand'. Thus, sentence \ref{and3} becomes `$(A\eand B)$'. When two sentences are connected with an `$\eand$', the resulting sentence is called a \define{conjunction}. The two sentences that are combined with the `$\eand$' are the \define{conjuncts} of the conjunction. So, `$A$' and `$B$' are the conjuncts of the conjunction `$(A\eand B)$'.

Notice that we don't need to symbolize the word `also' in sentence \ref{and3}. Words like `both' and `also' function to draw our attention to the fact that two things are being conjoined. Maybe they affect the emphasis of a sentence, but we will not (and cannot) symbolize such terms in TFL. 

Let's look at some trickier conjunctions.
	\begin{earg}
		\item[\ex{and4}]Barbara is athletic and smart.
		\item[\ex{and5}]Barbara and Adam are both athletic.
		\item[\ex{and6}]Although Lisa is smart, she is not athletic.
		\item[\ex{and7}]Adam is athletic, but Barbara is more athletic than him.
	\end{earg}
In each of these cases, we must, first, state each atomic sentence precisely, then it will be obvious what sentence letters we need and how to use the `$\eand$'. 

The first, `Barbara is athletic and smart' is actually expressing two atomic sentences: `Barbara is athletic' and `Barbara is smart'. Sentence \ref{and5} also contains two atomic sentences: `Barbara is athletic' and `Adam is athletic'. Notice that sentence \ref{and6} does not contain an `and' at all. \textit{Although} may have a slightly different meaning in English than \textit{and}, but broadly speaking, they have the same meaning and perform the same role in sentences. As far as TFL is concerned, they are both conjunctions. Here, the conjunction is combining these two atomic sentences: `Lisa is smart' and `Lisa is athletic'. When symbolizing this sentence, though, we will also have to include the `$\enot$' to symbolize the `not' in the second one.  

We will get to sentence \ref{and7} in a moment, but right now, this will be our expanded symbolization key:
	\begin{ekey}
		\item[A] Adam is athletic.
		\item[B] Barbara is athletic.
		\item[C] Barbara is smart.
		\item[D] Lisa is smart.
		\item[E] Lisa is athletic.
	\end{ekey}
With this key, we symbolize sentences \ref{and4} - \ref{and6} as follows.
	\begin{earg}
		\item[\ref{and4}.] $(B \eand C)$
		\item[\ref{and5}.] $(B \eand A)$
		\item[\ref{and6}.] $(D \eand \enot E)$ 
	\end{earg}
Notice that we have lost all sorts of nuance by expressing sentence \ref{and6} as a sentence in TFL. There is a distinct difference in tone between the English version of sentence \ref{and6} and $(D \eand \enot E)$, which is read as `Both Lisa is smart and it is not the case that Lisa is athletic'. TFL does not (and cannot) preserve those sorts of nuances.

Sentence \ref{and7} raises a different issue. You might think, at this point, that there is some trick to representing this sentence with two of the letters given in the symbolization key above. The first half of sentence \ref{and7} is symbolized as `$A$', but there is no way to use `$B$' for `Barbara is athletic' and then symbolize `more than him' separately in TFL. (We cannot write $B > A$ in TFL.) Instead, we need a new sentence letter. Let the TFL sentence `$F$' symbolize the English sentence `Barbara is more athletic than Adam'. Now we can symbolize sentence \ref{and7} by `$(A \eand F)$'.

\begin{factboxy}{Conjunction}
A sentence can be symbolized as $(\meta{A}\eand\meta{B})$ if it can be paraphrased any of these ways in English:
\vspace{-2mm}
\begin{earg}
\item[] `Both\ldots, and\ldots',
\item[] `\ldots, and\ldots',
\item[] `\ldots, but \ldots', 
\item[] `\ldots, although \ldots',
\item[] `\ldots, as well as \ldots'
\end{earg}
\end{factboxy}
	
\subsection{Parentheses}
	
You might be wondering why we put parentheses around the conjunctions. It is to help us make the meaning of the TFL expression precise. Consider these two sentences in English:
	\begin{earg}
		\item[\ex{negcon1}] It's not the case that you will get both soup and salad.
		\item[\ex{negcon2}] You will not get soup but you will get salad.
	\end{earg}
For these, we will use this symbolization key:
	\begin{ekey}
		\item[S_1] You will get soup.
		\item[S_2] You will get salad.
	\end{ekey}
Sentence \ref{negcon1} can be paraphrased as `This is not the case: you will get soup and you will get salad'. We can symbolize the \textit{you will get soup and you will get salad} part as `$(S_1 \eand S_2)$'. To symbolize the full sentence, we simply add the negation symbol \textit{outside} the parentheses: `$\enot (S_1 \eand S_2)$'. 

Sentence \ref{negcon2}, meanwhile, also includes a `not', but that `not' only applies to $S_1$. You \emph{will not} get soup, and you \emph{will} get salad. The first part, `you will not get soup' is symbolized as `$\enot S_1$', and the full sentence becomes `$(\enot S_1 \eand S_2)$'. 

Sentences \ref{negcon1} and \ref{negcon2} are different, and how we symbolize them differs accordingly. If we didn't use parentheses, then they would both be $\enot S_1 \eand S_2$, which obviously isn't what we want. With the parentheses, we can show that, in \ref{negcon1}, the entire conjunction is negated, while in \ref{negcon2} just one conjunct is negated. Brackets help us to keep track of the \emph{scope} of the negation. 

\section{Disjunction}

We will start with these sentences:
	\begin{earg}
		\item[\ex{or1}]Either Mary will play a video game, or she will watch a movie.
		\item[\ex{or2}]Either Mary or Omar will play a video game. 
	\end{earg}
And for these sentences, we will use this symbolization key:
	\begin{ekey}
		\item[F] Mary will play a video game.
		\item[O] Omar will play a video game.
		\item[M] Mary will watch a movie.
	\end{ekey}
To represent the `or' in sentences \ref{or1} and \ref{or2}, we will use the symbol `$\eor$'. Sentence \ref{or1}, then, is written as `$(F \eor M)$'. When two sentences are connected with an `$\eor$', the resulting sentence is called a \define{disjunction}. `$F$' and `$M$' are the \define{disjuncts} of the disjunction `$(F \eor M)$'.

Sentence \ref{or2} is only slightly more complicated. We can paraphrase it as `Either Mary will play a video game, or Omar will play a video game', and symbolize it as `$(F \eor O)$'.

\begin{factboxy}{Disjunction}
		A sentence can be symbolized as $(\meta{A}\eor\meta{B})$ if it can be paraphrased in English as `Either\ldots, or\ldots.' Each of the disjuncts must be a sentence.
\end{factboxy}

\subsection{The \textit{inclusive or}}

Sometimes in English, the word `or' is used in a way that excludes the possibility that both disjuncts are true. This is called an \define{exclusive or}.  An \emph{exclusive or} is clearly intended when it says, on a restaurant menu, `Entrees come with either soup or salad'. This means that, with your entree, you may have soup or you may have salad, but you cannot have both.

At other times, the word `or' allows for the possibility that both disjuncts might be true. If Mary doesn't spend too much time with video games or movies, then she might say, ``I will get an A in Logic or I will get an A in Twentieth Century U.S. History''. She probably means that she will get an A in at least one \textit{or both} of those courses. (After all, if she did end up getting an A in both, then we wouldn't insist that she was wrong when she said, ``I will get an A in Logic or I will get an A in Twentieth Century U.S. History''.)

When the intended meaning is that that one or the other or both of the disjuncts are true, then the \define{inclusive or} is being used. The TFL symbol `\eor' always symbolizes an \emph{inclusive or}.

\subsection{Negation and disjunction}

Last, let's look at these examples:
	\begin{earg}
		\item[\ex{or3}] Either you will not have soup, or you will not have salad.
		\item[\ex{or4}] You will have neither soup nor salad.
		\item[\ex{or.xor}] You can have either soup or salad, but not both.
	\end{earg}
Using $S_1$ and $S_2$ again, sentence \ref{or3} is symbolized by `$(\enot S_1 \eor \enot S_2)$'.

Sentences \ref{or4} and \ref{or.xor} are a little trickier. Sentence \ref{or4} can be paraphrased as `This is not the case: you have the soup or you will have the salad'. (If it helps, this is equivalent to `You will not have the soup and you will not have the salad'.) But sticking with the disjunction, as our paraphrased sentence shows, we are negating the entire disjunction. Hence, we symbolize sentence \ref{or4} as `$\enot (S_1 \eor S_2)$'. 

Because we are translating the sentence into TFL, the `or' in sentence \ref{or.xor} has to be interpreted as the inclusive-or. The full sentence, however, expresses the meaning of the exclusive-or: one or the other, but not both. So how do we express that in TFL? We can break the sentence into two parts. The first part, `you can have soup or you can have salad', is symbolized as `$(S_1 \eor S_2)$'. The second part says that you cannot have both. We can paraphrase this as: `This not the case: you can have soup and you can have salad'. This, we symbolize as `$\enot(S_1 \eand S_2)$'. Now we just need to put the two parts together. As we saw above, `but' can usually be symbolized with `$\eand$'. Therefore, sentence \ref{or.xor} is `$((S_1 \eor S_2) \eand \enot(S_1 \eand S_2))$'.

This last example demonstrates that, although the TFL symbol `\eor' always stands for \emph{inclusive or}, we can still express the \emph{exclusive or} in {TFL}. We just have to use `$\enot$', `$\eand$', and `$\eor$'.

\section{Conditional}

We will start with this sentence:
	\begin{earg}
		\item[\ex{if1}] If Jean is in Paris, then Jean is in France.
	\end{earg}
And we will use this symbolization key:
	\begin{ekey}
		\item[P] Jean is in Paris.
		\item[F] Jean is in France
	\end{ekey}
Sentence \ref{if1} has this form: `if P, then F', and we will use `\eif' to symbolize `if \ldots, then \ldots'. Thus, sentence \ref{if1} becomes `$(P\eif F)$'. 

This operator is called the \define{conditional}. In a conditional, what goes before the `$\eif$'  is called the \define{antecedent}, and what comes after the `$\eif$' is called the \define{consequent}. So, in sentence \ref{if1}, `Jean is in Paris' is the antecedent, and `Jean is in France' is the consequent.

\begin{factboxy}{Conditional}
		A sentence can be symbolized as $\meta{A} \eif \meta{B}$ if it can be paraphrased in English as `If A, then B'.
\end{factboxy}

Many English expressions can be represented using the conditional, and the most common alternatives to `if $\meta{A}$, then $\meta{B}$' are listed in table \ref{table.conditional.English}. If you think about it, you'll see that all six of the sentences in the table have the same meaning, and so they can all be symbolized as `$P \eif F$' (or generally, as `$\meta{A} \eif \meta{B}$'). 


\begin{table*}\centering\sffamily\footnotesize
\ra{1.25}
\begin{tabular}{@{}l l@{}}\toprule
If Jean is in Paris, then she is in France & If $\meta{A}$, then $\meta{B}$.\\
Jean is in France if she is in Paris. 	&	$\meta{B}$ if $\meta{A}$.\\
Whenever Jean is in Paris, she is in France.  	&	Whenever $\meta{A}$, $\meta{B}$.\\
Jean is in France provided that she is in Paris. 	&	$\meta{B}$ provided that $\meta{A}$.\\
Provided that Jean is in Paris, she is in France. 	&	Provided that $\meta{A}$, $\meta{B}$.\\
Jean is in Paris only if she is in France. 	&	$\meta{A}$ only if $\meta{B}$.\\
\bottomrule
\end{tabular}
\caption{The most common way of expressing a conditional in English is as `If Jean is in Paris, then she is in France.' This table lists some alternative but equivalent ways of expressing the same sentence.}\label{table.conditional.English}
\end{table*}


\section{Biconditional}\label{s:biconditional-1}

All of the logical operators that we have discussed so far are ones with which you were already familiar because you are an English speaker. The biconditional, which is mostly commonly expressed as \textit{\ldots if and only if \ldots}, is one that you might not have really noticed before---even if you used it on occasion. We'll start with the basic case.
	\begin{earg}
		\item[\ex{iff1}] The Bearcats won if and only if they scored more points than the Razorbacks.
	\end{earg}
And this will be our symbolization key:
	\begin{ekey}
		\item[B] The Bearcats won.
		\item[R] The Bearcats scored more points than the Razorbacks.
	\end{ekey}
The symbol `\eiff' will stand for `if and only if', and so we can symbolize sentence \ref{iff1} with the TFL sentence `$B \eiff R$'.

Now, let's probe a little further into the meaning of `if and only if' with a different example.
	\begin{earg}
		\item[\ex{iff2}] If Mary has a sunburn, then she went to the beach.
		\item[\ex{iff3}] If she went to the beach, then Mary has a sunburn. 
		\item[\ex{iff4}] If Mary has a sunburn, then she went to the beach, and if she went to the beach, then Mary has a sunburn.
		\item[\ex{iff5}] Mary has a sunburn if and only if she went to the beach.
	\end{earg}
We will use this symbolization key:
	\begin{ekey}
		\item[S] Mary has a sunburn.
		\item[B] Mary wen to the beach.
	\end{ekey}
From the previous section, you know how to symbolize sentences \ref{iff2} and \ref{iff3}. (But notice that sentences \ref{iff2} and \ref{iff3} have different meanings.)
	\begin{earg}
		\item[\ref{iff2}.] $(S \eif B)$
		\item[\ref{iff3}.] $(B \eif S)$ 
	\end{earg}
Sentence \ref{iff4}, then, is a conjunction created by combining \ref{iff2} and \ref{iff3}: 
	\begin{earg}
		\item[\ref{iff4}.] $(S \eif B) \eand (B \eif S)$
 	\end{earg}
Maybe it is apparent to you right away, or maybe you need to ponder it (and we will return to this in chapter \ref{s:CharacteristicTruthTables}), but sentence \ref{iff4} has the same meaning as sentence \ref{iff5}. Thus, $(S \eif B) \eand (B \eif S)$ is equivalent to $(S \eiff B)$. We call sentences that have the form $\meta{A} \eiff \meta{B}$ \define{biconditionals}, because they are equivalent to the conditional in both directions.

The expression `if and only if' occurs a lot in philosophy, mathematics, and logic, and sometimes you will see it abbreviated `iff'. (Although even when `iff' is written, we still say `if and only if.') So `if' with only \emph{one} `f' is the English conditional. But `iff' with \emph{two} `f's is the English biconditional.

\begin{factboxy}{Biconditional}
		A sentence can be symbolized as $\meta{A} \eiff \meta{B}$ if it can be paraphrased in English as `A iff B'---that is, as `A if and only if B'.
\end{factboxy}
	
A word of caution. Ordinary speakers of English often use `if \ldots, then\ldots' when they really mean to use something more like `\ldots if and only if \ldots'. Perhaps your parents told you when you were a child: `if you don't eat your vegetables, you won't get any dessert'. Suppose that you ate your vegetables, but that your parents refused to give you any dessert, on the grounds that they were only committed to the \emph{conditional} (roughly `if you get dessert, then you will have eaten your vegetables'), rather than the biconditional (roughly, `you get dessert if and only if you eat your vegetables'). Despite the valuable lesson in truth functional propositional logic, you would have been upset. So, be aware of this when interpreting what people say, and in your own writing, make sure you use \textit{if and only if} if and only if you mean to use it.

\section{Unless}\label{s:unless}
We have now introduced all of the logical operators of TFL. We can use them together to symbolize many kinds of sentences. An especially difficult case is when we use the English-language connective `unless'. Take this sentence:

\begin{earg}
\item[\ex{unless1}] Unless you wear a jacket, you will catch a cold. (Or equivalently, `You will catch a cold unless you wear a jacket'.) 
\end{earg}
To symbolize \ref{unless1}, we will use this symbolization key:
	\begin{ekey}
		\item[J] You will wear a jacket.
		\item[D] You will catch a cold.
	\end{ekey}
Sentence \ref{unless1} mean that if you do not wear a jacket, then you will catch a cold. With this in mind, we might symbolize it as `$\enot J \eif D$'. Alternatively, it means that if you do not catch a cold, then you must have worn a jacket. With this in mind, we can symbolize it as `$\enot D \eif J$'. And, finally, it also means that either you will wear a jacket or you will catch a cold. Hence, we can symbolize it as `$J \eor D$'.

All three ways of symbolizing sentence \ref{unless1} are correct. Indeed, in chapter \ref{s:SemanticConcepts} we will see that all three symbolizations are equivalent in TFL. Following the somewhat standard practice, however, we will define \textit{unless} as a disjunction.
% TODO: it might be useful to reference exercise 11.F.3 explicitly
% here, since the point is not discussed in the main text
	
\begin{factboxy}{Unless}
		If a sentence can be paraphrased as `Unless A, B,' then it can be symbolized as `$\meta{A}\eor\meta{B}$'.
\end{factboxy}

There is a complication with treating `unless' as a disjunction, however. As we said earlier, `or' has an inclusive and an exclusive meaning, but in TFL, `or' is always inclusive. Ordinary speakers of English, however, often use `unless' to mean something more like the exclusive-or. Suppose someone says: `I will go running unless it rains'. They probably mean `either I will go running or it will rain, but not both'. So, it can be argued that the conditional---e.g., `if it does not rain, then I will go running' ($\enot R_a \eif R_u$)---captures the meaning of `unless' better than does the disjunction.


%%%%%%%%%%%%%%%%%%%%%%%%%%%%%%%%%%
%%%%%%%%%%%%%%%%%%%%%%%%%%%%%%%%%%
% Exercises: logical operators chapter

\filbreak
\section{Practice exercises}
\setcounter{ProbPart}{0}
%\practiceproblems

\problempart Using the symbolization key given, translate each English sentence into TFL.\label{pr.monkeysuits}
	\begin{ekey}
		\item[A] Those creatures are aliens. 
		\item[C] Those creatures are centaurs. 
		\item[V] Those creatures are vampires.
	\end{ekey}
\begin{earg}
\item Those creatures are not aliens.
\item Those creatures are aliens, or they are not.
\item Those creatures are either vampires or centaurs.
\item Those creatures are neither vampires nor centaurs.
\item If those creatures are centaurs, then they are neither vampires nor aliens.
\item Unless those creatures are aliens, they are either centaurs or they are vampires.
\end{earg}

\problempart Using the symbolization key given, translate each English sentence into TFL.
\begin{ekey}
\item[A] Mr. Adams was murdered.
\item[B] The butler did it.
\item[C] The cook did it.
\item[D] The Duchess is lying.
\item[E] Mr. Edwards was murdered.
\item[F] The murder weapon was a frying pan.
\end{ekey}
\begin{earg}
\item Either Mr. Adams or Mr. Edwards was murdered.
\item If Mr. Adams was murdered, then the cook did it.
\item If Mr. Edwards was murdered, then the cook did not do it.
\item Either the butler did it, or the Duchess is lying.
\item The cook did it only if the Duchess is lying.
\item If the murder weapon was a frying pan, then the culprit must have been the cook.
\item If the murder weapon was not a frying pan, then the culprit was either the cook or the butler.
\item Mr. Adams was murdered if and only if Mr. Edwards was not murdered.
\item The Duchess is lying, unless it was Mr. Edwards who was murdered.
\item If Mr. Adams was murdered, he was done in with a frying pan.
\item Since the cook did it, the butler did not.
\item Of course the Duchess is lying!
\end{earg}


\problempart Using the symbolization key given, translate each English sentence into TFL.\label{pr.avacareer}
	\begin{ekey}
		\item[E_1] Ava is an electrician.
		\item[E_2] Harrison is an electrician.
		\item[F_1] Ava is a firefighter.
		\item[F_2] Harrison is a firefighter.
		\item[S_1] Ava is satisfied with her career.
		\item[S_2] Harrison is satisfied with his career.
	\end{ekey}
\begin{earg}
\item Ava and Harrison are both electricians.
\item If Ava is a firefighter, then she is satisfied with her career.
\item Ava is a firefighter, unless she is an electrician.
\item Harrison is an unsatisfied electrician.
\item Neither Ava nor Harrison is an electrician.
\item Both Ava and Harrison are electricians, but neither of them find it satisfying.
\item Harrison is satisfied only if he is a firefighter.
\item If Ava is not an electrician, then neither is Harrison, but if she is, then he is too.
\item Ava is satisfied with her career if and only if Harrison is not satisfied with his.
\item If Harrison is both an electrician and a firefighter, then he must be satisfied with his work.
\item It cannot be that Harrison is both an electrician and a firefighter.
\item Harrison and Ava are both firefighters if and only if neither of them is an electrician.
\end{earg}

\problempart
Using the symbolization key given, translate each English-language sentence into TFL.
\label{pr.jazzinstruments}
\begin{ekey}
\item[J_1] John Coltrane played tenor sax.
\item[J_2] John Coltrane played soprano sax.
\item[J_3] John Coltrane played tuba
\item[M_1] Miles Davis played trumpet
\item[M_2] Miles Davis played tuba
\end{ekey}

\begin{earg}
\item John Coltrane played tenor and soprano sax. 
\item Neither Miles Davis nor John Coltrane played tuba. 
\item John Coltrane did not play both tenor sax and tuba. 
\item John Coltrane did not play tenor sax unless he also played soprano sax. 
\item John Coltrane did not play tuba, but Miles Davis did. 
\item Miles Davis played trumpet only if he also played tuba. 
\item If Miles Davis played trumpet, then John Coltrane played at least one of these three instruments: tenor sax, soprano sax, or tuba. 
\item If John Coltrane played tuba then Miles Davis played neither trumpet nor tuba. 
\item Miles Davis and John Coltrane both played tuba if and only if Coltrane did not play tenor sax and Miles Davis did not play trumpet. 
\end{earg}


\problempart
\label{pr.spies}
Give a symbolization key, and then translate the following English sentences into TFL.
\begin{earg}
\item Alice and Bob are both spies.
\item If either Alice or Bob is a spy, then the code has been broken.
\item If neither Alice nor Bob is a spy, then the code remains unbroken.
\item The letter is in German embassy, unless someone has broken the code.
\item Either the code has been broken or it has not, but the letter is in German embassy.
\item Either Alice or Bob is a spy, but not both.
\end{earg}


\problempart
For each argument, first, make a symbolization key, and then translate all of the sentences of the argument into TFL.
\begin{earg}
\item If Dorothy plays the piano in the morning, then Roger wakes up cranky. Dorothy plays piano in the morning unless she is distracted. So if Roger does not wake up cranky, then Dorothy must be distracted.
\item It will either rain or snow on Tuesday. If it rains, Neville will be gloomy. If it snows, Neville will be cold. Therefore, Neville will either be gloomy or cold on Tuesday.
\item If Zoey remembered to do her chores, then the house is clean but not neat. If she forgot, then the house is neat but not clean. Therefore, the house is either neat or clean; but not both.
\end{earg}


%\problempart
%We symbolized an \emph{exclusive or} using `$\eor$', `$\eand$', and `$\enot$'. How could you symbolize an \emph{exclusive or} using only two operators? Is there any way to symbolize an \emph{exclusive or} using only one operator?




%%%%%%%%%%%%%%%%%%%%%%%%%%%%%%%%%%%%%%%%%%%%%%%%%
% Answers

\section{Practice exercises}
\setcounter{ProbPart}{0}

\problempart 
	\begin{ekey}
		\item[A] Those creatures are aliens. 
		\item[C] Those creatures are centaurs. 
		\item[V] Those creatures are vampires.
	\end{ekey}
\begin{earg}
\item Those creatures are not aliens.
\item[] \myanswer{$\enot A$}
\item Those creatures are aliens, or they are not.
\item[] \myanswer{$(A \eor \enot A$)} 
\item Those creatures are either vampires or centaurs.
\item[] \myanswer{$(V \eor C)$}
\item Those creatures are neither vampires nor centaurs.
\item[] \myanswer{$\enot (C \eor V)$}
\item If those creatures are centaurs, then they are neither vampires nor aliens.
\item[] \myanswer{$(C \eif \enot(G \eor V))$}
\item Unless those creatures are aliens, they are either centaurs or they are vampires.
\item[] \myanswer{$(A \eor (C \eor V))$}
\end{earg}

\problempart 
\begin{ekey}
\item[A] Mr. Adams was murdered.
\item[B] The butler did it.
\item[C] The cook did it.
\item[D] The Duchess is lying.
\item[E] Mr. Edwards was murdered.
\item[F] The murder weapon was a frying pan.
\end{ekey}
\begin{earg}
\item Either Mr. Adams or Mr. Edwards was murdered.
\item[] \myanswer{$(A \eor E)$}
\item If Mr. Adams was murdered, then the cook did it.
\item[] \myanswer{$(A \eif C)$}
\item If Mr. Edwards was murdered, then the cook did not do it.
\item[] \myanswer{$(E \eif \enot C)$}
\item Either the butler did it, or the Duchess is lying.
\item[] \myanswer{$(B \eor D)$}
\item The cook did it only if the Duchess is lying.
\item[] \myanswer{$(C \eif D)$}
\item If the murder weapon was a frying pan, then the culprit must have been the cook.
\item[] \myanswer{$(F \eif C)$}
\item If the murder weapon was not a frying pan, then the culprit was either the cook or the butler.
\item[] \myanswer{$(\enot F \eif (C \eor B))$}
\item Mr. Adams was murdered if and only if Mr. Edwards was not murdered.
\item[] \myanswer{$(A \eiff \enot E)$}
\item The Duchess is lying, unless it was Mr. Edwards who was murdered.
\item[] \myanswer{$(D \eor E)$}
\item If Mr. Adams was murdered, he was done in with a frying pan.
\item[] \myanswer{$(A \eif F)$}
\item Since the cook did it, the butler did not.
\item[] \myanswer{$(C \eand \enot B)$}
\item Of course the Duchess is lying!
\item[] \myanswer{$D$}
\end{earg}


\problempart\label{pr.avacareer}
	\begin{ekey}
		\item[E_1] Ava is an electrician.
		\item[E_2] Harrison is an electrician.
		\item[F_1] Ava is a firefighter.
		\item[F_2] Harrison is a firefighter.
		\item[S_1] Ava is satisfied with her career.
		\item[S_2] Harrison is satisfied with his career.
	\end{ekey}
\begin{earg}
\item Ava and Harrison are both electricians.
\item[] \myanswer{$(E_1 \eand E_2)$}
\item If Ava is a firefighter, then she is satisfied with her career.
\item[] \myanswer{$(F_1 \eif S_1)$}
\item Ava is a firefighter, unless she is an electrician.
\item[] \myanswer{$(F_1 \eor E_1)$}
\item Harrison is an unsatisfied electrician.
\item[] \myanswer{$(E_2 \eand \enot S_2)$}
\item Neither Ava nor Harrison is an electrician.
\item[] \myanswer{$\enot (E_1 \eor E_2)$}
\item Both Ava and Harrison are electricians, but neither of them find it satisfying.
\item[] \myanswer{$((E_1 \eand E_2) \eand \enot (S_1 \eor S_2))$}
\item Harrison is satisfied only if he is a firefighter.
\item[] \myanswer{$(S_2 \eif F_2)$}
\item If Ava is not an electrician, then neither is Harrison, but if she is, then he is too.
\item[] \myanswer{$((\enot E_1 \eif \enot E_2) \eand (E_1 \eif  E_2))$}
\item Ava is satisfied with her career if and only if Harrison is not satisfied with his.
\item[] \myanswer{$(S_1 \eiff \enot S_2)$}
\item If Harrison is both an electrician and a firefighter, then he must be satisfied with his work.
\item[] \myanswer{$((E_2 \eand F_2) \eif S_2)$}
\item It cannot be that Harrison is both an electrician and a firefighter.
\item[] \myanswer{$\enot (E_2 \eand F_2)$}
\item Harrison and Ava are both firefighters if and only if neither of them is an electrician.
\item[] \myanswer{$((F_2 \eand F_1) \eiff \enot(E_2 \eor E_1))$}
\end{earg}

\problempart
\label{pr.jazzinstruments}
\begin{ekey}
\item[J_1] John Coltrane played tenor sax.
\item[J_2] John Coltrane played soprano sax.
\item[J_3] John Coltrane played tuba
\item[M_1] Miles Davis played trumpet
\item[M_2] Miles Davis played tuba
\end{ekey}

\begin{earg}
\item John Coltrane played tenor and soprano sax. 
\item[~] \myanswer{$J_1 \eand J_2$} 
\medskip

\item Neither Miles Davis nor John Coltrane played tuba.
\item[~] \myanswer{$\enot(M_2 \eor J_3)$ or $\enot M_2 \eand \enot J_3$}
\medskip

\item John Coltrane did not play both tenor sax and tuba. 
\item[~] \myanswer{$\enot(J_1 \eand J_3)$ or $\enot J_1 \eor \enot J_3$} 
\medskip

\item John Coltrane did not play tenor sax unless he also played soprano sax. 
\item[~] \myanswer{$\enot J_1 \eor J_2$}
\medskip

\item John Coltrane did not play tuba, but Miles Davis did. 
\item[~] \myanswer{$\enot J_3 \eand M_2$}
\medskip

\item Miles Davis played trumpet only if he also played tuba. 
\item[~] \myanswer{$M_1 \eif M_2$} 
\medskip

\item If Miles Davis played trumpet, then John Coltrane played at least one of these three instruments: tenor sax, soprano sax, or tuba. 
\item[~] \myanswer{$M_1 \eif (J_1 \eor (J_2 \eor J_3))$} 
\medskip

\item If John Coltrane played tuba then Miles Davis played neither trumpet nor tuba. 
\item[~] \myanswer{$J_3 \eif \enot(M_1 \eor M_2)$ or $J_3 \eif (\enot M_1 \eand \enot M_2)$} 
\medskip

\item Miles Davis and John Coltrane both played tuba if and only if Coltrane did not play tenor sax and Miles Davis did not play trumpet. 
\item[~] \myanswer{$(J_3 \eand M_2) \eiff (\enot J_1 \wedge \enot M_1)$ or $(J_3 \eand M_2) \eiff \enot (J_1 \eor M_1)$} 
\end{earg}

\problempart
\label{pr.spies}
\myanswer{\begin{ekey}
\item[A] Alice is a spy.
\item[B] Bob is a spy.
\item[C] The code has been broken.
\item[G] The letter is in German embassy.
\end{ekey}}
\begin{earg}
\item Alice and Bob are both spies.
\item[] \myanswer{$(A \eand B)$}
\item If either Alice or Bob is a spy, then the code has been broken.
\item[] \myanswer{$((A \eor B) \eif C)$}
\item If neither Alice nor Bob is a spy, then the code remains unbroken.
\item[] \myanswer{$(\enot (A \eor B) \eif \enot C)$}
\item The letter is in the German embassy, unless someone has broken the code.
\item[] \myanswer{$(G \eor C)$}
\item Either the code has been broken or it has not, but the letter is in German embassy regardless.
\item[] \myanswer{$((C \eor \enot C) \eand G)$}
\item Either Alice or Bob is a spy, but not both.
\item[] \myanswer{$((A \eor B) \eand \enot (A \eand B))$}
\end{earg}



\problempart
For each argument, write a symbolization key and symbolize all of the sentences of the argument in TFL.
\begin{earg}
\item If Dorothy plays the piano in the morning, then Roger wakes up cranky. Dorothy plays piano in the morning unless she is distracted. So if Roger does not wake up cranky, then Dorothy must be distracted.
\myanswer{\begin{ekey}
\item[P] Dorothy plays the piano in the morning.
\item[C] Roger wakes up cranky.
\item[D] Dorothy is distracted.
\end{ekey}}
\item[] \myanswer{$(P \eif C), (P \eor D) \therefore (\enot C \eif D)$}
\bigskip

\item It will either rain or snow on Tuesday. If it rains, Neville will be gloomy. If it snows, Neville will be cold. Therefore, Neville will either be gloomy or cold on Tuesday.
\myanswer{\begin{ekey}
\item[T_1] It rains on Tuesday
\item[T_2] It snows on Tuesday
\item[G] Neville is gloomy on Tuesday
\item[C] Neville is cold on Tuesday
\end{ekey}}
\item[] \myanswer{$(T_1 \eor T_2), (T_1 \eif G), (T_2 \eif C) \therefore (G \eor C)$}
\bigskip

\item If Zoey remembered to do her chores, then the house is clean but not neat. If she forgot, then the house is neat but not clean. Therefore, the house is either neat or clean; but not both.
\myanswer{\begin{ekey}
\item[Z] Zoey remembered to do her chores
\item[C] The house is clean.
\item[N] The house is neat.
\end{ekey}}
\item[] \myanswer{$(Z \eif (C \eand \enot N)), (\enot Z \eif (N \eand \enot C)) \therefore ((N \eor C) \eand \enot (N \eand C))$.}
\end{earg}


%\problempart
%We symbolized an \emph{exclusive or} using `$\eor$', `$\eand$', and `$\enot$'. How could you symbolize an \emph{exclusive or} using only two connectives? Is there any way to symbolize an \emph{exclusive or} using only one connective?

%For two connectives, we could use any of the following: 
%\begin{center}
%$\enot(\meta{A} \eiff \meta{B})$\\
%$(\enot\meta{A} \eiff \meta{B})$\\
%$(\enot (\enot \meta{A} \eand \enot \meta{B}) \eand \enot (\meta{A} \eand \meta{B}))$
%\end{center}
%But if we want to symbolize the exclusive-or using only one connective, then we would have to introduce a new connective.




%%%%%%%%%%%%%%%%%%%%%%%%%%%%%%%%%%%%%%%%%%%%%%%%%
%%%%%%%%%%%%%%%%%%%%%%%%%%%%%%%%%%%%%%%%%%%%%%%%%

% Chapter: Sentences of TFL

%%%%%%%%%%%%%%%%%%%%%%%%%%%%%%%%%%%%%%%%%%%%%%%%%
%%%%%%%%%%%%%%%%%%%%%%%%%%%%%%%%%%%%%%%%%%%%%%%%%


\chapter{Sentences of TFL}\label{s:TFLSentences}

``Bring with thee airs from heaven or blasts from hell'' is a sentence of English. `$(A \eor B)$' is a sentence of TFL. Although we can identify sentences of English when we encounter them, we do not have a formal definition of `sentence of English'. But in this chapter, we will offer a complete definition of `sentence of TFL'. This is one respect in which a formal language like TFL is more precise than a natural language like English.


\section{Expressions}

\begin{table*}\centering\sffamily\footnotesize
\ra{1.25}
\begin{tabular}{@{}l l@{}}\toprule
atomic sentences & $A,B,C,\ldots,Z$\\
\enspace \textit{with subscripts if needed} & $A_1, A_2, A_3,A_4, \ldots, J_{10}, J_{11}\ldots$\\
logical operators & $\enot,\eand,\eor,\eif,\eiff$\\
brackets &( , )\\
\bottomrule
\end{tabular}
\caption{The three types of symbols of TFL}\label{table.symbols-TFL}
\end{table*}

We define an \define{expression of TFL} as any string of symbols of TFL. Take any of the symbols of TFL and write them down, in any order, and you have an expression of TFL.


\section{Sentences}\label{s:Sentences}
Of course, many expressions of TFL will be total gibberish. We want to know when an expression of TFL amounts to a \emph{sentence}. It won't do to try to list every expression that could be a sentence of TFL since, although there are only five logical operators, there are an infinite number of atomic sentences and an infinite number of ways that they and the logical operators can be combined. (And, at the same time, there are also an infinite number of ways of combining the atomic sentences and logical operators to create expressions that are not sentences.)
% Obviously, individual atomic sentences like `$A$' and `$G_{13}$' is sentences. We can form more sentences out of these by using the various connectives. Using negation, we can get `$\enot A$' and `$\enot G_{13}$'. Using conjunction, we can get `$(A \eand G_{13})$', `$(G_{13} \eand A)$', `$(A \eand A)$', and `$(G_{13} \eand G_{13})$'. We could also apply negation repeatedly to get sentences like `$\enot \enot A$' or apply negation along with conjunction to get sentences like `$\enot(A \eand G_{13})$' and `$\enot(G_{13} \eand \enot G_{13})$'. The possible combinations are endless, even starting with just these two sentence letters, and there are infinitely many sentence letters. So there is no point in trying to list all the sentences one by one.
Instead, we will describe the process by which sentences can be constructed. Consider negation. Take any sentence of TFL and call it \meta{A}. Since \meta{A} is a sentence of TFL, putting an `$\enot$' before it will yield a new sentence of TFL: $\enot\meta{A}$.

\begin{notebox}
Notice that \meta{A} and $A$ are different fonts. $A$ is an atomic sentence in TFL. \meta{A} is not, actually, part of TFL. Rather, it stands for any sentence in TFL. That sentence could be $A$ or it could be $(B \eif D)$ or anything else. This use of \textit{metavariables} is explained more fully in \S\ref{s:Metavariables}.
\end{notebox}

We can stipulate similar rules for each of the other logical operators. For instance, if \meta{A} and \meta{B} are sentences of TFL, then combining them with an `$\eand$' and brackets will yield a new sentence of TFL: $(\meta{A}\eand\meta{B})$. Providing rules like this for all of the logical operators, we arrive at the following formal definition for a \define{sentence of TFL}.

\begin{factboxy2}{Sentences of TFL}\label{TFLsentences}
	\begin{enumerate}
		\item Every atomic sentence is a sentence.
		\item If \meta{A} is a sentence, then $\enot\meta{A}$ is a sentence.
		\item If \meta{A} and \meta{B} are sentences, then $(\meta{A}\eand\meta{B})$ is a sentence.
		\item If \meta{A} and \meta{B} are sentences, then $(\meta{A}\eor\meta{B})$ is a sentence.
		\item If \meta{A} and \meta{B} are sentences, then $(\meta{A}\eif\meta{B})$ is a sentence.
		\item If \meta{A} and \meta{B} are sentences, then $(\meta{A}\eiff\meta{B})$ is a sentence.
		\item Nothing else is a sentence.
	\end{enumerate}
\end{factboxy2}

Definitions like this are called \emph{recursive}. Recursive definitions begin with some specifiable base elements, and then present ways to generate indefinitely many more elements by combining previously established ones. To give you a better idea of what a recursive definition is, we can give a recursive definition of the idea of \emph{an ancestor of mine}. We specify a base clause.
	\begin{ebullet}
		\item[1.] My parents are ancestors of mine.
	\end{ebullet}
and then offer further clauses like:
	\begin{ebullet}
		\item[2.] If \textit{x} is an ancestor of mine, then \textit{x}'s parents are ancestors of mine.
		\item[3.] No one else is an ancestor of mine.
	\end{ebullet}
Using this definition, we can easily (sort of) determine whether someone---say, Hugh Bailey Johnson, Sr.---is my ancestor. 
	\begin{ebullet}
		\item[\textit{a}.] My father is an ancestor of mine. 
		\item[\textit{b}.] My father's father, Hugh Bailey Johnson, Jr., is an ancestor of mine. 
		\item[\textit{c}.] Hugh Bailey Johnson, Jr.'s father, Hugh Bailey Johnson, Sr., is an ancestor of mine. 
	\end{ebullet}
To apply this definition, we have to begin with me and work backwards, and so we can only determine that someone is my ancestor by starting with the right parent and tracing the correct route to the person. And we can only determine that someone is not my ancestor by failing to be able to trace a path to him or her. If we are unsure about the path to a potential ancestor, then this can take some trial and error. 

A similar, although actually easier, procedure works for our recursive definition of a sentence of TFL. Just as the recursive definition allows complex sentences to be built up from simpler parts, the definition allows us to decompose sentences into their simpler parts. Once we get down to atomic sentences, then we know that we have a sentence of TFL. Let's consider some examples.

	\begin{ebullet}
	\item[1.] Is `$\enot \enot \enot D$' a sentence of TFL? 
		\begin{ebullet}
		\item[\textit{a}.] According to (2) in the definition on p.~\pageref{TFLsentences}, `$\enot \enot \enot D$' is a sentence \textit{\textbf{if}} `$\enot \enot D$' is a sentence. 
		\item[\textit{b}.] Again, using (2) in the definition, `$\enot \enot D$' is a sentence \textit{\textbf{if}} `$\enot D$' is. 
		\item[\textit{c}.] Similarly, `$\enot D$' is a sentence \textit{\textbf{if}} `$D$' is a sentence. 
		\item[\textit{d}.] `$D$' is an atomic sentence of TFL. According (1) in the definition on p.~\pageref{TFLsentences}, every atomic sentence is a sentence of TFL. 
		\item[\textit{e}.] Hence, `$\enot \enot \enot D$' is a sentence of TFL.
		\end{ebullet}
	
	\item[2.] Is `$P \eand (R \enot T)$' a sentence of TFL?
		\begin{ebullet}
		\item[\textit{a}.] According to (3) in the definition, `$P \eand (R \enot T)$' is a sentence \textit{\textbf{if}} `$P$' is a sentence and \textit{\textbf{if}} `$(R \enot T)$' is a sentence.
		\item[\textit{b}.] `$P$' is an atomic sentence of TFL. According (1) in the definition, every atomic sentence is a sentence of TFL.
		\item[\textit{c}.] `$(R \enot T)$' does not satisfy any of the rules in the definition. Therefore, it is not a sentence. 
		\item[*] Rules 3 - 6 specify how two sentences of TFL can be combined, and `$(R \enot T)$' does not match any of them. (2) indicates how sentences are formed using `$\enot$', and `$(R \enot T)$' does not match what is given in that rule.
		\item[\textit{d}.] Hence, `$P \eand (R \enot T)$' is not a sentence of TFL. 
		\end{ebullet}

	\item[3.] Is `$\enot(P \eand \enot (\enot Q \eor R))$' a sentence of TFL?
		\begin{ebullet}
		\item[\textit{a}.] According to (2), `$\enot(P \eand \enot (\enot Q \eor R))$' is a sentence \textit{\textbf{if}} `$(P \eand \enot (\enot Q \eor R))$' is a sentence.
		\item[\textit{b}.] According to (3), `$(P \eand \enot (\enot Q \eor R))$' is a sentence \textit{\textbf{if}} `$P$' is a sentence and \textit{\textbf{if}} `$\enot (\enot Q \eor R)$' is a sentence.
		\item[\textit{c}.] `$P$' is an atomic sentence. According to (1), every atomic sentence is a sentence of TFL.
		\item[\textit{d}.] According to (2), `$\enot (\enot Q \eor R)$' is a sentence \textit{\textbf{if}} `$(\enot Q \eor R)$' is a sentence.
		\item[\textit{e}.] According to (4) `$(\enot Q \eor R)$' is a sentence \textit{\textbf{if}} `$\enot Q$' is a sentence and \textit{\textbf{if}} `$R$' is a sentence.
		\item[\textit{f}.] According to (2), `$\enot Q$' is a sentence \textit{\textbf{if}} `$Q$' is a sentence.
		\item[\textit{g}.] `$R$' is an atomic sentence. According (1), every atomic sentence is a sentence of TFL.
		\item[\textit{h}.] `$Q$' is an atomic sentence. According (1), every atomic sentence is a sentence of TFL.
		\item[\textit{d}.] Hence, `$\enot(P \eand \enot (\enot Q \eor R))$' is a sentence of TFL. 
		\end{ebullet}
	\end{ebullet}
Whew. Every step is simple, but, with a long expression, there will be a lot of steps. 

Ultimately, you want to be able to just look at an expression and tell whether or not it is a correctly formed sentence of TFL, and with time you will be able to do so. Related to that, you will keep yourself from mis-writing sentences of TFL if you write neatly and space the atomic sentences, logical operators, and parentheses appropriately. Spaces are not actually part of the formal language in TFL, and so technically, you don't need to use them. But just as you would never add or drop spaces when writing sentences in English, you should always do the same when using TFL.

The recursive structure of sentences in TFL will also be important when we consider the circumstances under which a particular sentence would be true or false. The sentence `$\enot \enot \enot D$' is true if and only if the sentence `$\enot \enot D$' is false, and so on through the structure of the sentence, until we arrive at the atomic components. We will return to this point in Part~\ref{ch.TruthTables}.


\section{The main logical operator}
\label{main_logical_operator}

Consider this sentence: \textit{Dr. Wilson is in his  office and Dr. Cook is not in her office}. This sentence contains two connectives, `and' and `not', and one of them is the \define{main logical operator} of the sentence. The main logical operator determines, at the most general level, what kind of sentence it is---a conjunction, a disjunction, a conditional, a biconditional, or a negation. The sentence \textit{Dr. Wilson is in his  office and Dr. Cook is not in her office} is a conjunction. Thus, the `and' is the main logic operator, and the two conjuncts are \textit{Dr. Wilson is in his office} and \textit{Dr. Cook is not in her office}. The second conjunct is  a negation (that is, it's the negation of `Dr. Cook is in her office'), but this negation is subordinate to full sentence. 

Now let's look at this sentence: 
\begin{ebullet}
	\item[] \textit{If today is not Saturday, then Amy is at work and Kate is at school}. 
\end{ebullet}
	
\noindent What kind of sentence is this? It's a conditional. The antecedent is \textit{today is not Saturday}, and the consequent is \textit{Amy is at work and Kate is at school}. So, although there are three logical operators in this sentence, the main one is the \textit{if ..., then ...} (and so if we translated this sentence into TFL, the main logical operator would be the `$\eif$'). 

Let's consider for a moment why \textit{If today is not Saturday, then Amy is at work and Kate is at school} isn't a negation or a conjunction, even though the sentence contains both of those operators. If we tried to explain the sentence as a negation, all that we could say is that it is the negation of \textit{today is Saturday}. The `not' doesn't apply to any other part of the sentence, and so we would leave the rest of the sentence out of the explanation. Similarly, if we tried to explain the sentence as a conjunction, then all we would be able to say is that one conjunct is \textit{Amy is at work} and the other is \textit{Kate is at school}. Again, we would leave part of the sentence completely out of our explanation. 

We will define the term \textit{scope} at the end of this section, but right you can see that, in \textit{if today is not Saturday, then Amy is at work and Kate is at school}, the scope of the `not' is only \textit{today is not Saturday} and the scope of the `and' is \textit{Amy is at work and Kate is at school}. The scope of the conditional, meanwhile, is the entire sentence.  

Now let's turn to expressions in TFL. Although identifying the main logical operator in a long expression in TFL can seem confusing at first, because we are using parentheses, you'll find that it's not too difficult. Let's start with this example: $((P \eand Q) \eor R)$. This is a disjunction. One disjunct is $(P \eand Q)$ and the other is $R$. Hence, the main logical operator is the `$\eor$'. (Notice that if we tried to explain it as a conjunction, we would only be able to say that one conjunct is $P$ and the other is $Q$. We wouldn't be able to include the $R$ in our analysis.)

Now, let's change the expression to $\enot((P \eand Q) \eor R)$. This is a negation, and so the main logical operator is the `$\enot$'. Notice that the `$\enot$' is outside of the brackets that enclose the entire `$(P \eand Q) \eor R$'. That means that the `$\enot$' ranges over the entire sentence (or in other words, the scope of the `$\enot$' is the entire sentence). Hence, it is the main logical operator. 

Here are some other examples:
\begin{earg}
\item[\ex{logic-operator1}] $((P \eand R) \eif (\enot Q \eand S))$ ~~~The main logical operator is the `$\eif$'. 
\item[\ex{logic-operator2}] $(((T \eif P) \eand R) \eor (S \eiff Q))$ ~~~The main logical operator is the `$\eor$'.
\item[\ex{logic-operator3}] $\enot\enot\enot D$ ~~~The main logical operator is the first `$\enot$'.
\item[\ex{logic-operator4}] $(P \eand \enot (\enot Q \eor R))$ ~~~The main logical operator is the `$\eand$'. 
\item[\ex{logic-operator5}] $((\enot E \eor F) \eif \enot G)$ ~~~The main logical operator is the `$\eif$'.
\end{earg}

Ultimately, you want to be able to identify the main logical operator by just looking at a sentence and seeing what kind of sentence it is. If it is a conjunction, then part of the sentence will be one conjunct and the rest will be the other conjunct (and nothing will be left over). If it's a disjunction, then part of the sentence will be one disjunct and the rest will the other disjunct, again with nothing left over. If it's a conditional, then part of the sentence will be the antecedent and the rest will be the consequent. And if it's a negation, then the whole sentence (minus the `not' itself) is being negated.

Alternatively, when the sentence includes the outermost brackets, you can find the main logical operator by using this method:
\begin{ebullet}
	\item[(1)] If the first symbol in the sentence is `$\enot$', then that is the main logical operator.
	\item[(2)] Otherwise, start counting the brackets by following one of these two procedures. (The open-bracket is `(' and the closed bracket is `)'.) 
	\begin{earg}
	\item[(2a)] Start from the left, and begin counting. For each open-bracket add $1$, and for each closing-bracket, subtract $1$. When your count is at exactly $1$, the next operator you come to (\emph{apart} from a `$\enot$') is the main logical operator. 
	\item[(2b)] If starting at the left-side of the sentence doesn't seem to work, follow the same procedure, but begin at the far right (and work left). % Is working in this direction the only way that an `$\enot$' would be the next logical operator, but not the main one?
	\end{earg}
\end{ebullet}

As we will discuss in the next section, in some cases, it is acceptable to omit the outermost brackets in a sentence of TFL. For instance, although it is not strictly allowable according to the rules given in \S\ref{s:Sentences}, because it will not introduce any confusion or ambiguity, we can write `$(P \eand R) \eif Q$' instead of `$((P \eand R) \eif Q)$'. 

\begin{ebullet}
\item[(3)] If `$\enot$' is the main logical operator, then the outermost brackets have to be used. (When `$\enot$' is the main logical operator---as it is in this example: $\enot((P \eand Q) \eor R)$---the `$\enot$' will be outside the outermost brackets.) In other words, when the outermost brackets are omitted, `$\enot$' won't be the main logical operator, and so (1) will not apply. 
\item[(4)] When the outermost brackes are omitted, (2a) and (2b) can still be used, but stop when your count gets to zero instead of $1$.
\end{ebullet}


\subsection{Scope}

Finally, let's define the \emph{scope} of a connective. The scope of the main logical operator is always the entire sentence. The scope of every other connective is the subsentence for which the connective is the main logical operator. Consider this sentence:

$$(\enot(R \eand B) \eiff (P \eand Q))$$

The main logical operator is the `$\eiff$'. The scope of the `$\enot$' is $\enot(R \eand B)$, which means that `$\enot$' is the main logical operator for that subsentence. Similarly, the `$\eand$' is the main logical operator for just the $(R \eand B)$, and so the scope of that `$\eand$' is $(R \eand B)$. The same holds for every connective and every sentence and subsentence, and so we have the following definition.

	\factoidbox{The \define{scope} of a connective (in a sentence) is the sentence or subsentence for which that connective is the main logical operator.}


\section{Bracketing conventions}
\label{TFLconventions}
Strictly speaking, the brackets in `$(Q \eand R)$' are required. One reason for this is because the rules for forming sentences in TFL state that two atomic sentences connected by a connective are enclosed in brackets. (See \S\ref{s:Sentences}.) Another reason is that we might use `$(Q \eand R)$' as a subsentence in a more complicated sentence. For example, we might want to negate `$(Q \eand R)$', which would give us `$\enot(Q \eand R)$'. If we just had `$Q \eand R$' without the brackets and put a negation in front of it, we would have `$\enot Q \eand R$'. But `$\enot Q \eand R$' is different than `$\enot(Q\eand R)$'. 

That said, there are some convenient conventions that we can use as long as we are careful. First,  we allow ourselves to omit the \emph{outermost} brackets of a sentence. Thus, we allow ourselves to write `$Q \eand R$' instead of `$(Q \eand R)$' when `$Q \eand R$' is the whole sentence. We must remember, however, to put brackets around it when we want to embed the sentence into a more complex sentence.

Second, it can be a bit difficult to stare at long sentences with many nested pairs of brackets. To make things a bit easier on the eyes, we will allow ourselves to use square brackets, `[' and `]', instead of rounded ones. So, there is no logical difference, for example, between `$(P\eor Q)$' and `$[P\eor Q]$'. 

Combining these two conventions, we can rewrite 
$$(((H \eif I) \eor (I \eif H)) \eand (J \eor K))$$
this way:
$$\bigl[(H \eif I) \eor (I \eif H)\bigr] \eand (J \eor K)$$
The scope of each connective is now much easier to pick out.



%%%%%%%%%%%%%%%%%%%%%%%%%%%%%%%%%%%%
% Exercises for sentences of TFL chapter
%%%%%%%%%%%%%%%%%%%%%%%%%%%%%%%%%%%%


\section{Practice exercises}
\setcounter{ProbPart}{0}

\problempart
\label{pr.wiffTFL}
For each of the following, (a) is it a sentence of TFL, strictly speaking, and (b) is it a sentence of TFL, allowing for our relaxed bracketing conventions? If, by either of those standards, it is a sentence of TFL, then (c) what is the main logical operator?
\begin{earg}
\item $(A)$
\item $J_{374} \eor \enot J_{374}$
\item $\enot \enot \enot \enot F$
\item $\enot \eand S$
\item $(G \eand \enot G)$
\item $(A \eif (A \eand \enot F)) \eor (D \eiff E)$
\item $[(Z \eiff S) \eif W] \eand [J \eor X]$
\item $(F \eiff \enot D \eif J) \eor (C \eand D)$
\end{earg}

%\problempart
%Are there any sentences of TFL that contain no atomic sentences? Explain your answer.\\

\problempart
What is the scope of each connective in this sentence?
$$\bigl[(H \eif I) \eor (I \eif H)\bigr] \eand (J \eor K)$$


%%%%%%%%%%%%%%%%%%%%%%%%%%%%%%%%%%%%%%%%%%%
% Answers

\section{Answers}
\setcounter{ProbPart}{0}


\problempart
\label{pr.wiffTFL}
For each of the following, (a) is it a sentence of TFL, strictly speaking, and (b) is it a sentence of TFL, allowing for our relaxed bracketing conventions?
\begin{earg}
\item $(A)$\hfill \myanswer{(a) no (b) no}
\medskip

\item $J_{374} \eor \enot J_{374}$ \hfill \myanswer{(a) no (b) yes (c) the `$\eor$'}
\medskip

\item $\enot \enot \enot \enot F$ \hfill \myanswer{(a) yes (b) yes (c) the first `$\enot'$}
\medskip

\item $\enot \eand S$\hfill \myanswer{(a) no (b) no}
\medskip

\item $(G \eand \enot G)$\hfill \myanswer{(a) yes (b) yes (c) the `$\eand$'}
\medskip

\item $(A \eif (A \eand \enot F)) \eor (D \eiff E)$\hfill \myanswer{(a) no (b) yes (c) the `$\eor$'}
\medskip

\item $[(Z \eiff S) \eif W] \eand [J \eor X]$\hfill \myanswer{(a) no (b) yes (c) the `$\eand$'}
\medskip

\item $(F \eiff \enot D \eif J) \eor (C \eand D)$\hfill \myanswer{(a) no (b) no}
\medskip
\end{earg}

%\problempart
%Are there any sentences of TFL that contain no atomic sentences? Explain your answer.
%\\\myanswer{No. Atomic sentences contain atomic sentences (trivially). And every more complicated sentence is built up out of less complicated sentences, that were in turn built out of less complicated sentences, \ldots, that were ultimately built out of atomic sentences.}\\


\problempart
$\bigl[(H \eif I) \eor (I \eif H)\bigr] \eand (J \eor K)$\\
The scope of the left-most `$\eif$' is `$(H \eif I)$'.\\
The scope of the right-most `$\eif$' is `$(I \eif H)$'.\\
The scope of the left-most `$\eor$ is `$\bigl[(H \eif I) \eor (I \eif H)\bigr]$'\\
The scope of the right-most `$\eor$' is `$(J \eor K)$'\\
The scope of the `$\eand$' is the entire sentence, and so the `$\eand$' is the main logical operator and the sentence is a conjunction.





%%%%%%%%%%%%%%%%%%%%%%%%%%%%%%%%%%%%%%%%%%%
%%%%%%%%%%%%%%%%%%%%%%%%%%%%%%%%%%%%%%%%%%%
% Chapter: Use and mention
%%%%%%%%%%%%%%%%%%%%%%%%%%%%%%%%%%%%%%%%%%%
%%%%%%%%%%%%%%%%%%%%%%%%%%%%%%%%%%%%%%%%%%%


\chapter{Use and mention}\label{s:UseMention}

\section{Quotation conventions}
Consider these two sentences:
	\begin{earg}
		\item[\ex{JT1}] Justin Trudeau is the Prime Minister.
		\item[\ex{JT2}] `Justin Trudeau' is composed of two uppercase letters and eleven lowercase letters
	\end{earg}
When we want to talk about the Prime Minister of Canada, which we are doing in sentence \ref{JT1}, we \textit{use} his name. When we want to talk about the Prime Minister's name, as we are in sentence \ref{JT2} we \emph{mention} that name.

There is a general point here. When we want to talk about things in the world, we just \emph{use} words. When we want to talk about words, we have to \emph{mention} those words. To indicate that we are mentioning them rather than using them, we put them in single quotation marks, or use italics. 

Let's compare sentences \ref{JT1} and \ref{JT2} to sentences \ref{JT1b} and \ref{JT2b}:
	\begin{earg}
		\item[\ex{JT1b}] `Justin Trudeau' is the Prime Minister.
		\item[\ex{JT2b}] Justin Trudeau is composed of two uppercase letters and eleven lowercase letters.
	\end{earg}
Sentence \ref{JT1} is correct. Justin Trudeau, the man, is the Prime Minister of Canada. According to sentence \ref{JT1b}, the phrase `Justin Trudeau' is the Prime Minister, which is false. The same problem is illustrated by sentences \ref{JT2} and \ref{JT2b}. Sentence \ref{JT2} is fine. According to \ref{JT2b}, however, Justin Trudeau (the man) is made of letters, which is false. 

%For those who aren't feint of heart, one more example:
%	\begin{ebullet}
%		\item ``\,`Justin Trudeau'\,'' is the name of `Justin Trudeau'.
%	\end{ebullet} 
%On the left-hand-side, here, we have the name of a name. On the right hand side, we have a name. Perhaps this kind of sentence only occurs in logic textbooks, but it is true nonetheless. 


\section{Object language and metalanguage}\label{s:Metalanguage}
Since we are describing a formal language, we are often \emph{mentioning} expressions from TFL. When we talk about a language, the language that we are talking about is called the \define{object language}. The language that we use to talk about the object language is called the \define{metalanguage}.\label{def.metalanguage} 

Imagine for a moment that we are talking about German. In that case, German is our object language, and English---the language we are using to talk about German---is the metalanguage.
	\begin{earg}
		\item[\ex{Ger1}] Schnee ist wei\ss\ is a German sentence.
		\item[\ex{Ger2}]`Schnee ist wei\ss' is a German sentence.
	\end{earg}
Sentence \ref{Ger2} is correct. There were are saying that the clause at the beginning of the sentence is a German sentence. You can probably tell that sentence \ref{Ger1} is attempting to express the same idea, but, as it is, it is just a sentence stating `Snow is white is a German sentence'---in two different languages, no less. 

Of course, we aren't concerned with German here. For the most part, the object language in this chapter has been the formal language of TFL. The metalanguage is English. And just as we saw with sentence \ref{Ger2}, when we are referring to a sentence in the object language, we need to indicate that we are mentioning it, not using it. 
	\begin{earg}
		\item[\ex{obj1}] `$D$' is an atomic sentence of TFL.
		\item[\ex{obj2}] `$\enot (\enot Q \eor R)$' is a sentence of TFL \textit{\textbf{if}} `$(\enot Q \eor R)$' is a sentence of TFL.
	\end{earg}

The general point is that, whenever we want to talk in English about some specific expression of TFL, we need to indicate that we are \emph{mentioning} the expression, rather than \emph{using} it. We can either deploy quotation marks, or we can adopt some similar convention, such as  placing it centrally in the page. 


\section{Metavariables}\label{s:Metavariables}

Sometimes we discuss specific expressions of TFL like `$D$' and `$\enot (\enot Q \eor R)$'. Other times, however, we want to say something about an arbitrary expression of TFL, not a specific one. To do that, we use these uppercase letters:
	
	$$\meta{A}, \meta{B}, \meta{C}, \meta{D}, \ldots$$
	
You probably noticed that we used these letters in our definition of a sentence of TFL in \S\ref{s:Sentences}. For instance, this is one rule in that definition:

\begin{earg}
\item[3.] If \meta{A} and \meta{B} are sentences of TFL, then $(\meta{A}\eand\meta{B})$ is a sentence of TFL.
\end{earg}

We used `$\meta{A}$' and `$\meta{B}$' because those symbols must stand for any possible sentence of TFL, not just `$A$' and `$B$'. For instance, `$\meta{A}$' might stand 
for `$(P \eor Q)$' or `$((R \eif T) \eand \enot Q)$' or anything else.

`$\meta{A}, \meta{B}, \meta{C}, \meta{D}, \ldots$' do not belong to TFL. Rather, they are part of the metalanguage---that is, English---that we use to talk about expressions of TFL. 
	
	\factoidbox{
	  A \define{metavariable} is a variable in the metalanguage (i.e., English) that represents any sentence in our formal language of TFL. The symbols $\meta{A}, \meta{B}, \meta{C}, \meta{D}, \ldots$ are used for the metavariables.}


\section{Quotation conventions for arguments}

One of our main purposes for using TFL is to study arguments, and that will be our concern in Parts \ref{ch.TruthTables} and \ref{ch.NDTFL}. In English, the premises of an argument are often expressed by individual sentences, and the conclusion by a further sentence. Since we can symbolize English sentences, we can symbolize English arguments using TFL. Thus we might ask whether the argument whose premises are the TFL sentences `$A$' and `$A \eif C$', and whose conclusion is the TFL sentence `$C$', is valid. However, it is quite a mouthful to write that every time. So instead we will introduce another bit of abbreviation. This:
	$$\meta{A}_1, \meta{A}_2, \ldots, \meta{A}_n \therefore \meta{C}$$
abbreviates:
	\begin{quote}
		the argument with premises $\meta{A}_1, \meta{A}_2, \ldots, \meta{A}_n$ and conclusion $\meta{C}$
	\end{quote}
To avoid unnecessary clutter, we will not regard this as requiring quotation marks around it. (Note, then, that `$\therefore$' is a symbol of our augmented \emph{metalanguage}, and not a new symbol of TFL.)
