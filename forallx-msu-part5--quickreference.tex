%!TEX root = forallx.tex
\chapter[Quick reference]{Quick reference}

%\addcontentsline{toc}{chapter}{C\ Quick Reference}
%\pagestyle{plain}
%{\LARGE \bf Quick Reference}

\section{Characteristic Truth Tables}
\label{app.CharacteristicTTs}
\bigskip

\hfill
\begin{tabular}{c|c}
\meta{A} & \enot\meta{A}\\
\hline
T & F\\
F & T 
\end{tabular}
\hfill
\begin{tabular}{c|c|c|c|c|c}
\meta{A} & \meta{B} & \meta{A}\eand\meta{B} & \meta{A}~\eor~\meta{B} & \meta{A}~\eif~\meta{B} & \meta{A}~\eiff~\meta{B}\\
\hline
T & T & T & T & T & T\\
T & F & F & T & F & F\\
F & T & F & T & T & F\\
F & F & F & F & T & T
\end{tabular}
\hfill

\vspace{4em}


\hfill
\begin{tabular}{c|c}
\meta{A} & \enot\meta{A}\\
\hline
1 & 0\\
0 & 1 
\end{tabular}
\hfill
\begin{tabular}{c|c|c|c|c|c}
\meta{A} & \meta{B} & \meta{A}\eand\meta{B} & \meta{A}~\eor~\meta{B} & \meta{A}~\eif~\meta{B} & \meta{A}~\eiff~\meta{B}\\
\hline
1 & 1 & 1 & 1 & 1 & 1\\
1 & 0 & 0 & 1 & 0 & 0\\
0 & 1 & 0 & 1 & 1 & 0\\
0 & 0 & 0 & 0 & 1 & 1
\end{tabular}
\hfill

\vfill


\section{Symbolization}
\medskip

\begin{center}
\label{app.symbolization}
\begin{tabular*}{\textwidth}{rl}
\multicolumn{2}{c}{\textsc{Sentential Connectives}}\\ \\
It is not the case that $P$. & $\enot P$\\
Either $P$, or $Q$. & $(P \eor Q)$\\
Neither $P$, nor $Q$. & $\enot(P \eor Q)$\ or \ $(\enot P \eand \enot Q)$\\
Both $P$, and $Q$. & $(P \eand Q)$\\
If $P$, then $Q$. & $(P \eif Q)$\\
$P$ only if $Q$. & $(P \eif Q)$\\
$P$ if and only if $Q$. & $(P \eiff Q)$\\
Unless $P$, $Q$. $P$ unless $Q$. & $(P \eor Q)$\\
\\
\multicolumn{2}{c}{\label{SymbolizingPredicates}\textsc{Predicates}}\\ \\
All $F$s are $G$s. & $\forall x(Fx \eif Gx)$\\
Some $F$s are $G$s. & $\exists x(Fx \eand Gx)$\\
Not all $F$s are $G$s. & $\enot\forall x(Fx \eif Gx)$\ or\ $\exists x(Fx \eand \enot Gx)$\\
No $F$s are $G$s. & $\forall x(Fx \eif\enot Gx)$\ or\ $\enot\exists x(Fx \eand Gx)$\\
\\
\multicolumn{2}{c}{\textsc{Identity}}\\ \\
Only $j$ is $G$. & $\forall x(Gx \eiff x=j)$\\
Everything besides $j$ is $G$. & $\forall x(x \neq j \eif Gx)$\\
%$j$ is more $R$ than anyone else. & $\forall x(x\neq j \eif Rjx)$\\
The $F$ is $G$. & $\exists x(Fx \eand \forall y(Fy \eif x=y) \eand Gx)$\\

%\multicolumn{2}{l}{`The F is not G' can be translated two ways:} \\
It is not the case that the F is G. (wide)& $\enot\exists x(Fx \eand \forall y(Fy \eif x=y) \eand Gx)$\\
The $F$ is non-$G$. (narrow) & $\exists x(Fx \eand \forall y(Fy \eif x=y) \eand \enot Gx)$
\end{tabular*}
\end{center}



% BEGIN: symbolizing cardinality

\newpage
\section*{Using identity to symbolize quantities}

\subsection*{There are at least \blank\ $F$s.}
\label{summary.atleast}

\begin{ekey}
\item[one] $\exists xFx$
\item[two] $\exists x_1\exists x_2(Fx_1 \eand Fx_2 \eand x_1 \neq x_2)$
\item[three] $\exists x_1\exists x_2\exists x_3(Fx_1 \eand Fx_2 \eand Fx_3 \eand x_1 \neq x_2 \eand x_1 \neq x_3 \eand x_2 \neq x_3)$
\item[four] $\exists x_1\exists x_2\exists x_3\exists x_4 (Fx_1 \eand Fx_2 \eand Fx_3 \eand Fx_4 \eand x_1 \neq x_2 \eand x_1 \neq x_3 \eand x_1 \neq x_4 \eand x_2 \neq x_3 \eand x_2 \neq x_4 \eand x_3 \neq x_4)$
\item[n] $\exists x_1\cdots\exists x_n(Fx_1 \eand\cdots\eand Fx_n \eand x_1 \neq x_2 \eand\cdots\eand x_{n-1}\neq x_n)$ 
\end{ekey}

\subsection*{There are at most \blank\ $F$s.}
\label{summary.atmost}

One way to say `at most $n$ things are $F$' is to put a negation sign in front of one of the symbolizations above and say $\enot$`at least $n+1$ things are $F$.' Equivalently:
\begin{ekey}
\item[one] $\forall x_1\forall x_2\bigl[(Fx_1 \eand Fx_2) \eif x_1=x_2\bigr]$
\item[two] $\forall x_1\forall x_2\forall x_3\bigl[(Fx_1 \eand Fx_2 \eand Fx_3) \eif (x_1=x_2 \eor x_1=x_3 \eor x_2=x_3)\bigr]$
\item[three] $\forall x_1\forall x_2\forall x_3\forall x_4\bigl[(Fx_1 \eand Fx_2 \eand Fx_3 \eand Fx_4) \eif (x_1=x_2 \eor x_1=x_3 \eor x_1=x_4 \eor x_2=x_3 \eor x_2=x_4 \eor x_3=x_4)\bigr]$
\item[n]$\forall x_1\cdots\forall x_{n+1}
\bigl[(Fx_1\eand \cdots \eand Fx_{n+1}) \eif (x_1=x_2 \eor \cdots \eor x_n=x_{n+1})\bigr]$ 
\end{ekey}

\subsection*{There are exactly \blank\ $F$s.}
\label{summary.exactly}

One way to say `exactly $n$ things are $F$' is to conjoin two of the symbolizations above and say `at least $n$ things are $F$' \eand\ `at most $n$ things are $F$.' The following equivalent formulae are shorter:
\begin{ekey}
\item[zero] $\forall x\enot Fx$
\item[one] $\exists x\bigl[Fx \eand \enot\exists y(Fy \eand x\neq y)\bigr]$
\item[two] $\exists x_1\exists x_2\bigl[Fx_1 \eand Fx_2 \eand x_1 \neq x_2 \eand \enot\exists y\bigl(Fy \eand y\neq x_1 \eand y \neq x_2\bigr) \bigr]$
\item[three] $\exists x_1\exists x_2\exists x_3\bigl[Fx_1 \eand Fx_2 \eand Fx_3 \eand x_1 \neq x_2 \eand x_1 \neq x_3 \eand x_2 \neq x_3 \eand\\
\enot\exists y(Fy \eand y \neq x_1 \eand y \neq x_2 \eand y\neq x_3) \bigr]$
\item[n] $\exists x_1\cdots\exists x_n\bigl[Fx_1 \eand\cdots\eand Fx_n  \eand x_1 \neq x_2 \eand\cdots\eand x_{n-1}\neq x_n \eand\\
 \enot\exists y(Fy \eand y\neq x_1 \eand \cdots \eand y\neq x_n)\bigr]$ 
%\item[one] $\exists x\forall y\bigl[Fx \eand (Fy \eif y = x)\bigr]$
%\item[two] $\exists x\exists y\forall z\Bigl(Fx \eand Fy \eand \bigl[Fz \eif (z=x \eor z=y)\bigr] \eand x \neq y\Bigr)$
%\item[three] $\exists x_1\exists x_2\exists x_3\forall y\Bigl(Fx_1 \eand Fx_2 \eand Fx_3 \eand [Fy \eif (y=x_1 \eor y=x_2 \eor y=x_3)] \eand x_1 \neq x_2 \eand x_1 \neq x_3 \eand x_2 \neq x_3\Bigr)$
%\item[n] $\exists x_1\cdots\exists x_n\forall y\Bigl(Fx_1 \eand\cdots\eand Fx_n \eand \bigl[Fy \eif (y=x_1 \eor \cdots \eor y=x_n)\bigr] \eand x_1 \neq x_2 \eand\cdots\eand x_{n-1}\neq x_n\Bigr)$ 
\end{ekey}

%\subsection*{Specifying the size of the UD}

%Removing $F$ from the symbolizations above produces sentences that talk about the size of the UD. For instance, `there are at least 2 things (in the UD)' may be symbolized as $\exists x\exists y(x \neq y)$.




%  BEGIN: Rules of proof

% change margins so that all the rules will fit
%\setlength{\topmargin}{0 in}
%\setlength{\headheight}{0 in}
%\setlength{\headsep}{0 in}
%\setlength{\textheight}{9 in}
%\setlength{\evensidemargin}{0.25 in}
%\setlength{\oddsidemargin}{0.25 in}
%\setlength{\textwidth}{6 in}

\newpage

% eliminate page numbers
%\pagestyle{empty}

\label{ProofRules} 
\section{Basic rules for TFL}

\begin{multicols}{2}
%\twocolumn

\noindent\textsc{Reiteration}

\begin{proof}
	\have[m]{a}{\meta{A}}
	\have[\ ]{c}{\meta{A}} \reit{a}
\end{proof}
\bigskip


\noindent\textsc{Conjunction Introduction}

\begin{proof}
	\have[m]{a}{\meta{A}}
	\have[n]{b}{\meta{B}}
	\have[\ ]{c}{\meta{A}\eand\meta{B}} \ai{a, b}
\end{proof}
\bigskip


\noindent\textsc{Conjunction Elimination}

\begin{proof}
	\have[m]{ab}{\meta{A}\eand\meta{B}}
	\have[\ ]{a}{\meta{A}} \ae{ab}
\end{proof}

\begin{proof}
	\have[m]{ab}{\meta{A}\eand\meta{B}}
	\have[\ ]{b}{\meta{B}} \ae{ab}
\end{proof}
\bigskip


\noindent\textsc{Disjunction Introduction}

\begin{proof}
	\have[m]{a}{\meta{A}}
	\have[\ ]{ab}{\meta{A}\eor\meta{B}}\oi{a}
\end{proof}

\begin{proof}
	\have[m]{a}{\meta{A}}
	\have[\ ]{ba}{\meta{B}\eor\meta{A}}\oi{a}
\end{proof}
\bigskip

\vfill\null
\columnbreak

\noindent\textsc{Disjunction Elimination}

\begin{proof}
	\have[m]{ab}{\meta{A}\eor\meta{B}}
	\have[n]{nb}{\enot\meta{B}}
	\have[\ ]{a}{\meta{A}} \oe{ab,nb}
\end{proof}

\begin{proof}
	\have[m]{ab}{\meta{A}\eor\meta{B}}
	\have[n]{na}{\enot\meta{A}}
	\have[\ ]{b}{\meta{B}} \oe{ab,nb}
\end{proof}
\bigskip


\noindent\textsc{Conditional Introduction}

\nopagebreak
\begin{proof}
	\open
		\hypo[m]{a}{\meta{A}} \as{}
		\have[n]{b}{\meta{B}}
	\close
	\have[\ ]{ab}{\meta{A}\eif\meta{B}}\ci{a-b}
\end{proof}
\bigskip


\noindent\textsc{Conditional Elimination}

\begin{proof}
	\have[m]{ab}{\meta{A}\eif\meta{B}}
	\have[n]{a}{\meta{A}}
	\have[\ ]{b}{\meta{B}} \ce{ab,a}
\end{proof}
\bigskip

\noindent\textsc{Biconditional Introduction}

\begin{proof}
	\open
		\hypo[m]{a1}{\meta{A}} \as{}
		\have[n]{b1}{\meta{B}}
	\close
	\open
		\hypo[p]{b2}{\meta{B}} \as{}
		\have[q]{a2}{\meta{A}}
	\close
	\have[\ ]{ab}{\meta{A}\eiff\meta{B}}\bi{a1-b1,b2-a2}
\end{proof}
\bigskip


\noindent\textsc{Biconditional Elimination}

\begin{proof}
	\have[m]{ab}{\meta{A}\eiff\meta{B}}
	\have[n]{a}{\meta{B}}
	\have[\ ]{b}{\meta{A}} \be{ab,a}
\end{proof}

\begin{proof}
	\have[m]{ab}{\meta{A}\eiff\meta{B}}
	\have[n]{a}{\meta{A}}
	\have[\ ]{b}{\meta{B}} \be{ab,a}
\end{proof}
\bigskip


\noindent\textsc{Negation Introduction}

\begin{proof}
	\open
		\hypo[m]{a}{\meta{A}} \as{}
		\have[n][-1]{b}{\meta{B}}
		\have{nb}{\enot\meta{B}}
	\close
	\have[\ ]{na}{\enot\meta{A}}\ni{a-nb}
\end{proof}
\bigskip


\noindent\textsc{Negation Elimination}

\begin{proof}
	\open
		\hypo[m]{na}{\enot\meta{A}} \as{}
		\have[n][-1]{b}{\meta{B}}
		\have{nb}{\enot\meta{B}}
	\close
	\have[\ ]{a}{\meta{A}}\ne{na-nb}
\end{proof}
\bigskip


\section{Derived rules for TFL}

\textsc{Dilemma}

\begin{proof}
	\have[m]{ab}{\meta{A}\eor\meta{B}}
	\have[n]{ac}{\meta{A}\eif\meta{C}}
	\have[p]{bc}{\meta{B}\eif\meta{C}}
	\have[\ ]{a}{\meta{C}} \by{DIL}{ab,ac,bc}
\end{proof}
\bigskip

\textsc{Modus Tollens}

\begin{proof}
	\have[m]{ab}{\meta{A}\eif\meta{B}}
	\have[n]{a}{\enot\meta{B}}
	\have[\ ]{b}{\enot\meta{A}} \by{MT}{ab,a}
\end{proof}
\bigskip

\textsc{Hypothetical Syllogism}

\begin{proof}
	\have[m]{ab}{\meta{A}\eif\meta{B}}
	\have[n]{bc}{\meta{B}\eif\meta{C}}
	\have[\ ]{ac}{\meta{A}\eif\meta{C}}\by{HS}{ab,bc}
\end{proof}

\end{multicols}

\newpage

\section{Quantifier rules}

\noindent\textsc{Existential Introduction}

\begin{proof}
	\have[m]{a}{\meta{A}\meta{c}}
	\have[\ ]{c}{\exists \meta{x}\meta{A}\meta{x}} \Ei{a}
\end{proof}

Note that \meta{x} may replace some or all occurrences of \meta{c} in \meta{A}\meta{c}.
\bigskip


\noindent\textsc{Existential Elimination}

\begin{proof}
	\have[m]{a}{\exists \meta{x}\meta{A}\meta{x}}
	\open	
		\hypo[n]{b}{\meta{A}\meta{c}^\ast}
		\have[p]{c}{\meta{B}}
	\close
	\have[\ ]{d}{\meta{B}} \Ee{a,b-c}
\end{proof}

$^\ast$ \meta{c} must not appear in $\exists\meta{x}\meta{A}\meta{x}$, in \meta{B}, or in any undischarged assumption.
\bigskip

\noindent\textsc{Universal Introduction}

\begin{proof}
	\have[m]{a}{\meta{A}\meta{c}^\ast}
	\have[\ ]{c}{\forall \meta{x}\meta{A}\meta{x}} \Ai{a}
\end{proof}

$^\ast$ \meta{c} must not occur in any undischarged assumptions.
\bigskip

\noindent\textsc{Universal Elimination}

\begin{proof}
	\have[m]{a}{\forall \meta{x}\meta{A}\meta{x}}
	\have[\ ]{c}{\meta{A}\meta{c}} \Ae{a}
\end{proof}

\newpage

\section{Identity rules}

\noindent\textsc{Identity introduction}

\begin{proof}
	\have[\ \,\,\,]{x}{\meta{c}=\meta{c}} \by{=I}{}
\end{proof}


\noindent\textsc{Identity elimination}

\begin{multicols}{2}
\begin{proof}
	\have[m]{e}{\meta{a}=\meta{b}}
	\have[n]{a}{\meta{A}(\ldots \meta{a} \ldots \meta{a}\ldots)}
	\have[\ ]{ea1}{\meta{A}(\ldots \meta{b} \ldots \meta{a}\ldots)} \by{=E}{e,a}
\end{proof}
\begin{proof}
	\have[m]{e}{\meta{a}=\meta{b}}
	\have[n]{a}{\meta{A}(\ldots \meta{b} \ldots \meta{b}\ldots)}
	\have[\ ]{ea2}{\meta{A}(\ldots \meta{a} \ldots \meta{b}\ldots)} \by{=E}{e,a}
\end{proof}
\end{multicols}



\section{Replacement rules}

{
\center

\textsc{Commutivity} (Comm)\\
$(\meta{A}\eand\meta{B}) \Longleftrightarrow (\meta{B}\eand\meta{A})$\\
$(\meta{A}\eor\meta{B}) \Longleftrightarrow (\meta{B}\eor\meta{A})$\\
$(\meta{A}\eiff\meta{B}) \Longleftrightarrow (\meta{B}\eiff\meta{A})$
\bigskip

\textsc{DeMorgan} (DeM)\\
$\enot(\meta{A}\eor\meta{B}) \Longleftrightarrow (\enot\meta{A}\eand\enot\meta{B})$\\
$\enot(\meta{A}\eand\meta{B}) \Longleftrightarrow (\enot\meta{A}\eor\enot\meta{B})$
\bigskip

\textsc{Double Negation} (DN)\\
$\enot\enot\meta{A} \Longleftrightarrow \meta{A}$
\bigskip

\textsc{Material Conditional} (MC)\\
$(\meta{A}\eif\meta{B}) \Longleftrightarrow (\enot\meta{A}\eor\meta{B})$\\
$(\meta{A}\eor\meta{B}) \Longleftrightarrow (\enot\meta{A}\eif\meta{B})$
\bigskip

\textsc{Biconditional Exchange} ({\eiff}{ex})\\
$[(\meta{A}\eif\meta{B})\eand(\meta{B}\eif\meta{A})] \Longleftrightarrow (\meta{A}\eiff\meta{B})$
\bigskip

\textsc{Quantifier Negation} (QN)\\
$\enot\forall\meta{x}\meta{A} \Longleftrightarrow \exists\meta{x}\enot\meta{A}$\\
$\enot\exists\meta{x}\meta{A} \Longleftrightarrow \forall\meta{x}\enot\meta{A}$

}



