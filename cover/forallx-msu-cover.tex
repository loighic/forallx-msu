
%% USE LUALATEX!!!

\documentclass[coverheight=229mm, coverwidth=152mm, spinewidth=15.09mm]{bookcover}
\usepackage{GS1,qrcode}   % for the barcode
\usepackage{xcolor}

\definecolor{forallx-orange}{RGB}{254, 102, 1}


% FONTS
\usepackage{fontspec}	% required for LuaLaTeX
\usepackage[letterspace=87]{microtype}  % to space caps with \textls{...}

\setmainfont{YaleNew-Roman.otf}[Ligatures=TeX, 
 	Numbers={OldStyle},
 	Ligatures={Common},
 	ItalicFont={YaleNew-Italic.otf},
 	BoldFont={YaleNew-Bold.otf}]

\setsansfont{lato-regular.ttf}[Ligatures=TeX,  
	Ligatures={Common},
	Numbers={OldStyle},
	Scale=1.0,
	ItalicFont={Lato-Italic.ttf},
	BoldFont={Lato-Bold.ttf}]
	
%% for math mode
\usepackage{unicode-math}
\setmathfont{Asana Math} % for math symbols, can be any other OpenType math font
\setmathfont[range=\mathup]{lato-regular.ttf}
\setmathfont[range=\mathbfup]{lato-regular.ttf}
%\setmathfont[range=\mathbfit]{Minion Pro Bold Italic}
\setmathfont[range=\mathit]{Lato-Italic.ttf}


% line spacing
\usepackage{setspace}
\renewcommand{\baselinestretch}{1.25} 


\begin{document}
\begin{bookcover}
\bookcovercomponent{color}{bg whole}{color=forallx-orange}

%TITLE
\bookcovercomponent{normal}{front}[20mm,,10mm,110mm]{\sffamily   %spacing: left, bottom, right, top 
\fontsize{44}{44}\selectfont\hfill \textbf{forall\textcolor{white}{x}}} 		% Baseline skip (the second value) isn't being used, and so I just made it the same as the font size.

% subtitle
\bookcovercomponent{normal}{front}[20mm,,10mm,128mm]{\sffamily
\fontsize{20}{20}\selectfont\hfill \textbf{\textls{\textcolor{white}{THE MISSISSIPPI STATE EDITION}}}}  


% Title on the spine
\bookcovercomponent{center}{spine}{
\rotatebox[origin=c]{-90}{\sffamily\LARGE \textbf{\textls{FORALL\textcolor{white}{X}~~~THE ~MISSISSIPPI ~STATE ~EDITION~~~~~\textcolor{white}{JOHNSON}}}}}


% back cover text
\bookcovercomponent{normal}{back}[20mm,,20mm,60mm]{\sffamily    %spacing: left, bottom, right, top 
\textcolor{white}{\textls{\small THIS INTRODUCTION TO FORMAL LOGIC}}~ covers truth-functional logic (which is also often called \textit{propositional logic}) and introduces first-order logic.

\quad The title \textit{forallx} (i.e., ``for all x'') is a reference to first-order logic. This is a symbolic expression in first-order logic: $\forall x(Kx \rightarrow Gx)$, and it is read, ``for all x, if x is K, then x is G.'' Hence, the name of the textbook. And if, for instance, we have \textit{K} stand for ``is a king,'' and \textit{G} stand for ``is greedy,'' then ``$\forall x(Kx \rightarrow Gx)$'' means ``for all x, if x is a king, then x is greedy,'' or ``everyone who is a king is greedy.''

\quad This book is based on a text---the original \textit{forallx}---written by P. D. Magnus and then revised and expanded by Tim Button, J. Robert Loftis, Aaron Thomas-Bolduc, and Richard Zach. It has been further revised specifically for the 1000-level logic course at Mississippi State University.

\quad This textbook is licensed under a Creative Commons Attribution 4.0 International (CC BY 4.0) license.  
}


%ISBN
\bookcovercomponent{normal}{back}[,1cm,,]{
\vfill
\centering
\savebox0{\EANBarcode[module_height=10mm]{ISBN 978-1-716-84837-7}}
\colorbox{white}{%
\usebox0
\raisebox{\depth}{\qrcode[height=\ht0]{https://www.lulu.com/shop/gregory-johnson/forallx/paperback/product-4emmmj.html?q=&page=1&pageSize=4}}}}

\end{bookcover}
\end{document}