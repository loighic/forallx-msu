%!TEX root = forallxyyc.tex
\part{Key concepts of logic}
\label{ch.intro}
\addtocontents{toc}{\protect\mbox{}\protect\hrulefill\par}


\chapter{Arguments}
\label{s:Arguments}

\section{We begin here.}

In everyday discourse, the word \textit{argument} typically refers to a verbal disagreement between two people. In logic and philosophy, however, it has a different and special meaning (although plenty of people do argue, in the everyday sense of the word, in logic and philosophy). We will use \textit{argument} to refer to a set of sentences like these:
	\begin{earg}\label{argRaining}
		\item[1.] It is raining heavily.
		\item[2.] If you do not take an umbrella, you will get soaked.
		      \vspace{0.1cm}
     			\hrule
      			\vspace{0.12cm}
		\item[3.] Therefore, you should take an umbrella.
	\end{earg}
In this set, the first two sentences support---or justify---the third sentence. The sentences providing support are the \emph{premises}. The sentence that is supported by (or justified by) the premises is the \emph{conclusion}. Together, \textit{premises} and a \textit{conclusion} comprise an \textit{argument}.  

\begin{factboxy}{Argument}
An \define{argument} is a set of sentences. One or more of the sentences provide support for another sentence in the set. The sentences providing support are \define{premises}. The sentence being supported is the \define{conclusion}.
\smallskip

We can also say that, in an argument, the conclusion \textit{follows from} the premises.
\end{factboxy}

That's the definition of an argument, but a broader analysis must include the idea that arguments can be good or bad---or  somewhere inbetween. A good argument is one in which the premises do, in fact, support the conclusion. For such an argument, if the premises are true, then we have good reason to believe that the conclusion is true. On the other hand, a bad (or a weak) argument is still an argument. It is just one in which the premises provide little support for the conclusion.

In the definition of an argument, we said that each premise and the conclusion is a sentence. And, as we saw, both premises and the conclusion in the example are individual sentences. All arguments can be expressed this way and many are, but a single sentence can also contain a complete argument, as is shown here:
	\begin{quote}
		 Joan was wearing sunglasses, and so it must have been sunny.
	\end{quote}
This argument has one premise and a conclusion. The premise and the conclusion could both be individual sentences, but here they are just independent clauses separated by the `and'. (The premise is before the `and', and the conclusion is after it.)

Many arguments also start with premises and end with a conclusion. But not all arguments are expressed in this order. For instance, here we have our first argument again, but the conclusion is at the beginning:
	\begin{quote}
		You should take an umbrella. After all, it is raining heavily. And if you do not take an umbrella, you will get soaked. 
	\end{quote}
We can also have the conclusion in the middle:
	\begin{quote}
		It is raining heavily. Accordingly, you should take an umbrella, given that if you do not take an umbrella, you will get soaked.
	\end{quote}

When approaching an argument, we want to know whether or not the conclusion is supported by the premises. So, first, we must identify the premise or premises (the sentences providing support) and the conclusion (and the sentence being supported). As a guide, these words are often used to indicate that a sentence or clause is the conclusion of an argument:
	\begin{quote}
		so, therefore, hence, thus, accordingly, consequently
	\end{quote}
By contrast, these expressions often indicate that we are dealing with a premise, rather than a conclusion:
	\begin{quote}
		since, because, given that
	\end{quote}

So that we can undertake a more detailed and precise analysis of some kinds of arguments, in chapter \ref{symbolization}, we will begin introducing a formal language: truth functional logic. But before we get there, in this chapter and chapter \ref{s:Valid}, we will cover some basic logical notions that apply to arguments in a natural language like English. Then, in chapter \ref{s:BasicNotions},  we will examine logical notions that apply to just sentences (not full arguments), and still in a natural language like English.


\section{Sentences}
\label{intro.sentences}

Only sentences that can be true or false can be the premises or the conclusion of an argument. The following types of sentences cannot be true or false, and so they cannot be part of an argument.

\paragraph{Questions} `Are you sleepy yet?' is, obviously, a question. Although you might be sleepy or you might be alert, the question itself is neither true nor false. For this reason, questions will not count as sentences in logic. 

\paragraph{Imperatives} Imperative sentences are, essentially, commands (although they can be nicer than what we usually think of as a command). For instance, `Wake up!', `Sit up straight', and `Please, tell me how to set the table' are all imperatives. Although it might be a good idea for you to sit up, and you may or may not do it, the command is neither true nor false. Note, however, that commands are not always phrased as imperatives.  As Cartman might say,``You will respect my authority.'' This is a command, but it is also true or false--- either you will or you will not respect Cartman's authority--- and so it counts as a sentence in logic.

\paragraph{Exclamations} Some exclamatory sentences can be true or false (and so they are also declarative sentences) and some cannot be. `It's Friday!' is an exclamation, and it is true or false. It can be part of an argument. On the other hand, a sentence such as `Ouch!' is neither true nor false, and so it cannot be part of an argument. 

\section{Truth values}
Going forward, by \define{sentence}, we will mean a declarative sentence. We impose this restriction because the premises and conclusion of an argument must be capable of having a \define{truth value}. That is, we must be able to assign a value about its truth to each sentence in an argument. Although more advanced ``non-classical'' logic systems introduce more options, the two truth values that concern us are just `true' and `false'. 

\begin{factboxy}{truth values}
\define{Truth values} are the logical values that a sentence can have, \textit{true} and \textit{false}.
\end{factboxy}

\practiceproblems

\noindent Highlight the phrase that expresses the conclusion of each of these arguments:
\begin{earg}
	\item It is sunny. So, I should take my sunglasses.
	\item It must have been sunny. I did wear my sunglasses, after all.
	\item No one but you has had their hands in the cookie-jar. And the scene of the crime is littered with cookie-crumbs. You're the culprit!
	\item Miss Scarlett and Professor Plum were in the study at the time of the murder. Reverend Green had the candlestick in the ballroom, and we know that there is no blood on his hands. Hence Colonel Mustard did it in the kitchen with the lead-piping. Recall, after all, that the gun had not been fired.
\end{earg}

%%%%%%%%%%%%%%%%%%%%%%%%%%%%%%%%
%%%%%%%%%%%%%%%%%%%%%%%%%%%%%%%%



\chapter{Validity and other standards}\label{s:Valid}

\section{Validity}\label{s:Valid-def}

Consider this argument:
	\begin{earg}
		\item[1.] You are reading this book.
		\item[2.] This is a logic book.
		\item[3.] Therefore, you are a logic student.
	\end{earg}

\begin{notebox}
When we list the premises and the conclusion of an argument this way, the final line is always the conclusion. However many lines there are before the final one are the premises.
\end{notebox}

\noindent If the premises of this argument are true---which, as it turns out, they are---it is very likely that the conclusion is true. But it is possible that someone besides a logic student is reading this book. If, say, the roommate of the book's owner picked it up and began looking through it, he or she would not immediately become a logic student. So, for this argument, we can say that, if the premises are true, then it is \textit{likely}, but not certain, that the conclusion is also true. 

Now, take this one:
	\begin{earg}\label{valid-bananas}
		\item[1.] Paris is in France, or it is in Germany.
		\item[2.] Paris is not in Germany.
		\item[3.] Therefore, Paris is in France.
	\end{earg}
For this argument, if the premises are true---which, again, they are---then the conclusion has to be true. There is no way for the premises to be true and the conclusion to be false. 

Here is another example,

\noindent\begin{minipage}{0.99\textwidth}
	\begin{earg}
		\item[1.] Paris is in Sweden, or it is in Spain.
		\item[2.] Paris is not in Sweden. 
		\item[3.] Therefore, Paris is in Spain.
	\end{earg}
\smallskip
\end{minipage}

\noindent Although this argument might strike you as a bit odd, we can say almost the exact same thing about this one as we did for the previous one:
\begin{quote}
In this argument, if the premises are true, then the conclusion has to be true. There is no way for the premises to be true and the conclusion to be false. 
\end{quote}
We have to drop the bit about the premises being true because the first one is false. But nonetheless, \textit{if the premises are true}, then the conclusion has to be true. 

This brings us to an important definition as well as an important point about doing logic. First the definition. 
\begin{factboxy}{Valid}
These are two equivalent definitions of \define{valid} (or \define{deductively valid}):
\begin{ebullet}
\vspace{-2mm}
\item[1.] An argument is \define{valid} when, and only when, it is the case that, if the premises are true, then the conclusion has to be true.

\item[2.] An argument \define{valid} when, and only when, it is impossible for all of the premises to be true and the conclusion to be false.
\end{ebullet}

\hrule
\medskip

Every argument that does not satisfy the definition of \textit{valid} is \define{invalid} (or \define{deductively invalid}).
\end{factboxy}
\noindent Typically, the study of logic focuses on determining when the conclusion of an argument follows from the premises with certainty. From the perspective of logic, whether the premises actually are true is less important. Of course, determining whether or not they are true can be important for many reasons, but this task is normally left to historians, scientists, or the Hardy boys.

We want to know whether, if all the premises \textit{were} true, would the conclusion also have to be true? Consider this argument:
	\begin{earg}
		\item[1.] Paris is a large city in France, or Paris is a large city on Jupiter.
		\item[2.] Paris is not a large city in France.
		\item[3.] Therefore, Paris is a large city on Jupiter.
	\end{earg}
This argument is valid. \emph{If} both premises are true (they're not, but if they were), then the conclusion has to be true. Now, let's think about this argument:
	\begin{earg}
		\item[1.] London is in England.
		\item[2.] Beijing is in China.
		\item[3.] Therefore, Paris is in France.
	\end{earg}
The premises and conclusion of this argument are all true, but the argument is invalid. If Paris were, somehow, to become independence from the rest of France, then the conclusion would be false, even though both of the premises would remain true. Thus, it is \emph{possible} for the premises of this argument to be true and the conclusion false. Hence, the argument is \textit{invalid}.

The important point to remember is that validity is not about the actual truth or falsity of the sentences in the argument. It is about whether it is \emph{possible} or \emph{impossible} for all of the premises to be true and the conclusion to be false. (Or, to say the same thing in a different way, whether or not the conclusion has to be true \textit{if} all of the premises are true.)

We can, however, classify the arguments that are valid and have all true premises. We call these \define{sound}\label{def-sound-arg}. 

\begin{factboxy}{Sound}
An argument is \define{sound} when, and only when, it is valid and has all true premises.
\end{factboxy}

The second argument on p.~\pageref{valid-bananas} is sound.


\section{Inductively strong arguments}
Many good arguments are invalid. Consider this one:
	\begin{earg}
		\item[1.] In January 2017, it rained in London.
		\item[2.] In January 2018, it rained in London.
		\item[3.] In January 2019, it rained in London.
		\item[4.] In January 2020, it rained in London.
		\item[5.] In January 2021, it rained in London.
		\item[6.] In January 2022, it rained in London.
		\item[7.] In January \the\year{}, it rained in London.
	\item[8.] Therefore, next January, it will rain in London.
\end{earg}
This argument generalizes from observations about several cases to a conclusion about all cases. This argument could be made stronger by adding additional premises, for instance: `In January 2016, it rained in London,' `In January 2015, it rained in London,' and so on. But, however many premises like this we add, the argument will remain invalid. Even if it has rained in London every January for the past 10,000 years, it remains \emph{possible} that it won't rain in London next January. Hence, this argument is invalid. But, at the same time, you might think, ``but it's still a good argument!'' It is, and we have a way of classifying such arguments.

\begin{factboxy}{Inductively strong}
An argument is \define{inductively strong} when (and only when) [1] it is not valid and [2] it is the case that if the premises are true, then their being true makes it likely that the conclusion is true.
\end{factboxy}

An inductively strong argument is one for which the conclusion has a high probability of being true (if the premises are). Arguments that are invalid, but not inductively strong, can have conclusions with every possible probability of being true (if the premises are true) from very high to zero. To simplify matters, we can say that the options are \textit{inductively strong}, \textit{medium}, or \textit{weak}. Since we are on a continuum, we could be much more fine grained than this. (But we won't. See table \ref{invalid-strong}.)

\begin{table*} %[hb]
\centering\sffamily\footnotesize
\ra{1.25}
\begin{tabular}{@{}m{4.75cm}  l@{}}\toprule
The premises being true, & {\rotatebox[origin=c]{90}{$\Lsh$}} These are all invalid.\\\midrule
make it very probable that the conclusion will be true. & inductively strong\\
\ldots & \\
make it somewhat probable that the conclusion will be true. & inductively medium\\
\ldots &\\
do not make it very likely that the conclusion will be true. & inductively weak\\
\bottomrule
\end{tabular}
\caption{Every argument is valid or invalid. Invalid arguments can have any degree of inductive strength, depending on how likely the conclusion is to be true given the premises.}\label{invalid-strong}
\end{table*}

And finally, we have a concept for inductively strong arguments that serves the role that \textit{sound} does for valid ones.

\begin{factboxy}{Reliable}
An argument is \define{reliable} when (and only when) it is inductively strong and has all true premises.
\end{factboxy}

In this textbook, however, we will set aside the analysis of inductively strong arguments and focus on just valid versus invalid ones.  

%%%%%%%%%%%%%%%%%%%%%%%%%%%%%%%%%

\practiceproblems
\problempart
Determine if each of the following arguments is valid or invalid.

\begin{enumerate}[(1)] %% The [(1)] puts the numbering for each argument in parentheses, e.g., (1), (2), etc.
%\noindent\begin{minipage}{0.99\textwidth}
\item 
\begin{earg}
\item Socrates is a man.
\item All men are carrots.
\item Therefore, Socrates is a carrot.
\end{earg}
%\vspace{2mm}
%\end{minipage}

\item
\begin{earg}
\item Either today is Labor Day, or the building is full.
\item The building isn’t full.
\item Therefore, today is Labor Day.
\end{earg}

\item
\begin{earg}
\item If the green van is missing, then Claire is at the beach.
\item The green van is missing.
\item Therefore, Claire is at the beach.
\end{earg}

\item\begin{earg}
\item[] If Jones decided that she is going to get divorced, then she called a lawyer. Jones just called a lawyer. Hence, she has decided that she’s going to get divorced. 
\end{earg}

\item\begin{earg}
\item Jeff is playing basketball, or Mary is watching television.
\item Mary is watching television.
\item Therefore, Jeff is playing basketball.
\end{earg}

\item\begin{earg}
\item 160 12th graders at Central High School were asked if they planned to go to college next year. 
\item 75 percent said that they were planning to go to college the following year. 
\item Therefore, about 75 percent of all the 12th graders at Central High School are probably going to college next year.
\end{earg}

\item\begin{earg}
\item If Mary stole the painting, then Jeff is in New Jersey.
\item Therefore, if Jeff is in New Jersey, then Mary stole the painting.
\end{earg}

\item\begin{earg}
\item As vacation destinations, Florence and Lisbon have many similarities: nice weather, historical attractions, and great restaurants.  
\item Sarah enjoyed visiting Florence. 
\item Therefore, Sarah will probably enjoy visiting Lisbon. 
\end{earg}

\item\begin{earg}
\item If Mary stole the painting, then Jeff is in New Jersey.
\item Therefore, if Jeff is not in New Jersey, then Mary did not steal the painting.
\end{earg}

\item\begin{earg}
\item Amy is on campus.
\item Therefore, Amy is on campus, or she is on the moon.
\end{earg}

\item\begin{earg}
\item Jack is taking a nap.
\item Therefore, Jack is taking a nap, and Kate is reading. 
\end{earg}

\item\begin{earg}
\item If Roger is in the bank, then Steven is waiting in the apartment.
\item Roger is not in the bank.
\item Therefore, Steven is not waiting in the apartment.
\end{earg}

\item\begin{earg}
\item If Joan is at work, then Kate is sleeping.
\item Therefore, if Kate is not sleeping, then Joan is not at work.
\end{earg}

\item\begin{earg}
\item If Mary is in the library, then Jeff is watching tv.
\item If Jeff is watching tv, then Claire is taking a nap.
\item Therefore, if Claire is taking a nap, then Mary is in the library.
\end{earg}

\item\begin{earg}
\item If Mary is in the library, then Jeff is watching tv.
\item If Jeff is watching tv, then Claire is taking a nap.
\item Therefore, if Mary is in the library, then Claire is taking a nap.
\end{earg}

\item\begin{earg}
\item If Mary is in the library, then Jeff is watching tv.
\item If Jeff is watching tv, then Claire is taking a nap.
\item Therefore, if Claire is not taking a nap, then Mary is not in the library.
\end{earg}

\item\begin{earg}
\item George is an architect, or Susan is a lawyer.
\item George is not an architect.
\item Therefore, Susan is a lawyer.
\end{earg}

\item\begin{earg}
\item Amy is walking in the park, or Sarah is playing basketball.
\item Amy is walking in the park.
\item Therefore, Sarah is not playing basketball.
\end{earg}

\item\begin{earg}
\item George is mowing the lawn.
\item Therefore, George is mowing the lawn, and Fred is looking for his coat.
\end{earg}

\item\begin{earg}
\item Almost all sea lions live in the Atlantic Ocean around New York and New Jersey. 
\item Sammy is a sea lion. 
\item Therefore, Sammy lives in the Atlantic Ocean around New York and New Jersey.
\end{earg}

\item\begin{earg}
\item All sea lions live in the Atlantic Ocean around New York and New Jersey. 
\item Sammy is a sea lion. 
\item Therefore, Sammy lives in the Atlantic Ocean around New York and New Jersey.
\end{earg}

\end{enumerate}


\problempart
\label{pr.EnglishCombinations}
For each statement, determine if it is possible or not. If it is possible, given an example as illustration. If it is not possible, then explain why it isn't.
	\begin{earg}
	\item A valid argument that has one false premise and one true premise
	\item A valid argument that has a false conclusion
	\item A valid argument that has only false premises
	\item A valid argument with only false premises and a false conclusion
	\item An invalid argument that can be made valid by the addition of a new premise
	\item A valid argument that can be made invalid by the addition of a new premise
	\end{earg}

%%%%%%%%%%%%%%%%%%%%%%%%%%%%
% Answers

\section{Answers}
\setcounter{ProbPart}{0}

\problempart
\begin{enumerate}[(1)] %% The [(1)] puts the numbering for each argument in parentheses, e.g., (1), (2), etc.
\item 
\begin{earg}
\item Socrates is a man.
\item All men are carrots.
\item Therefore, Socrates is a carrot.
\end{earg}
\noindent This argument is valid.

\item
\begin{earg}
\item Either today is Labor Day, or the building is full.
\item The building isn’t full.
\item Therefore, today is Labor Day.
\end{earg}
\noindent This argument is valid.

\item
\begin{earg}
\item If the green van is missing, then Claire is at the beach.
\item The green van is missing.
\item Therefore, Claire is at the beach.
\end{earg}
\noindent This argument is valid.

\item 
\begin{earg}
\item[] If Jones decided that she is going to get divorced, then she called a lawyer. Jones just called a lawyer. Hence, she has decided that she’s going to get divorced. 
\end{earg}
\noindent This argument is invalid.

\item
\begin{earg}
\item Jeff is playing basketball, or Mary is watching television.
\item Mary is watching television.
\item Therefore, Jeff is playing basketball.
\end{earg}
\noindent This argument is invalid.

\item
\begin{earg}
\item 240 12th graders at Central High School were asked if they planned to go to college next year. 
\item 75 percent said that they were planning to go to college the following year. 
\item Therefore, about 75 percent of all the 12th graders at Central High School are probably going to college next year.
\end{earg}
\noindent This argument is invalid.

\item
\begin{earg}
\item If Mary stole the painting, then Jeff is in New Jersey.
\item Therefore, if Jeff is in New Jersey, then Mary stole the painting.
\end{earg}
\noindent This argument is invalid.

\item
\begin{earg}
\item As vacation destinations, Florence and Lisbon have many similarities: nice weather, historical attractions, and great restaurants.  
\item Sarah enjoyed visiting Florence. 
\item Therefore, Sarah will probably enjoy visiting Lisbon.
\end{earg} 
\noindent This argument is invalid.

\item
\begin{earg}
\item If Mary stole the painting, then Jeff is in New Jersey.
\item Therefore, if Jeff is not in New Jersey, then Mary did not steal the painting.
\end{earg}
\noindent This argument is valid.

\item
\begin{earg}
\item Amy is on campus.
\item Therefore, Amy is on campus, or she is on the moon.
\end{earg}
\noindent This argument is valid.

\item
\begin{earg}
\item Jack is taking a nap.
\item Therefore, Jack is taking a nap, and Kate is reading. 
\end{earg}
\noindent This argument is invalid.

\item
\begin{earg}
\item If Roger is in the bank, then Steven is waiting in the apartment.
\item Roger is not in the bank.
\item Therefore, Steven is not waiting in the apartment.
\end{earg}
\noindent This argument is invalid.

\item
\begin{earg}
\item If Joan is at work, then Kate is sleeping.
\item Therefore, if Kate is not sleeping, then Joan is not at work.
\end{earg}
\noindent This argument is valid.

\item
\begin{earg}
\item If Mary is in the library, then Jeff is watching tv.
\item If Jeff is watching tv, then Claire is taking a nap.
\item Therefore, if Claire is taking a nap, then Mary is in the library.
\end{earg}
\noindent This argument is invalid.

\item
\begin{earg}
\item If Mary is in the library, then Jeff is watching tv.
\item If Jeff is watching tv, then Claire is taking a nap.
\item Therefore, if Mary is in the library, then Claire is taking a nap.
\end{earg}
\noindent This argument is invalid.

\item
\begin{earg}
\item If Mary is in the library, then Jeff is watching tv.
\item If Jeff is watching tv, then Claire is taking a nap.
\item Therefore, if Claire is not taking a nap, then Mary is not in the library.
\end{earg}
\noindent This argument is valid.

\item
\begin{earg}
\item George is an architect, or Susan is a lawyer.
\item George is not an architect.
\item Therefore, Susan is a lawyer.
\end{earg}
\noindent This argument is valid.

\item
\begin{earg}
\item Amy is walking in the park, or Sarah is playing basketball.
\item Amy is walking in the park.
\item Therefore, Sarah is not playing basketball.
\end{earg}
\noindent For the way that we will define ‘or’ in the logic system that is developed in this textbook, this argument is invalid.

\item
\begin{earg}
\item George is mowing the lawn.
\item Therefore, George is mowing the lawn, and Fred is looking for his coat.
\end{earg}
\noindent This argument is invalid.

\item
\begin{earg}
\item Almost all sea lions live in the Atlantic Ocean around New York and New Jersey. 
\item Sammy is a sea lion. 
\item Therefore, Sammy lives in the Atlantic Ocean around New York and New Jersey.
\end{earg}
\noindent This argument is invalid.

\item
\begin{earg}
\item All sea lions live in the Atlantic Ocean around New York and New Jersey. 
\item Sammy is a sea lion. 
\item Therefore, Sammy lives in the Atlantic Ocean around New York and New Jersey.
\end{earg}
\noindent This argument is valid.

\end{enumerate}

\problempart
\begin{enumerate}[(1)]
\item A valid argument that has one false premise and one true premise\\
\noindent Yes, this is possible. 
\begin{earg}
\item All whales are mammals. (\emph{true}) 
\item All mammals are plants. (\emph{false})
\item Therefore, all whales are plants.\smallskip
\end{earg}

\item A valid argument that has a false conclusion\\
Yes, this is possible. See example from previous exercise.\smallskip

\end{enumerate}

%%%%%%%%%%%%%%%%%%%%%%%%%%%%%%%%%%%
%%%%%%%%%%%%%%%%%%%%%%%%%%%%%%%%%%%

\chapter{Other concepts of logic}\label{s:BasicNotions}

The concept of a valid argument is central to logic. In this section, we will introduce some other important concepts that apply just to sentences, not to full arguments. 

\section{Joint possibility}\label{s:joint-poss}

Consider these two sentences:
	\begin{ebullet}
		\item[B1.] Jane's only brother is shorter than her.
		\item[B2.] Jane's only brother is taller than her.
	\end{ebullet}
Without knowing Jane and her brother, we have no way of knowing which, if either, of these sentences is true. Yet we can say that \emph{if} B1 is true, \emph{then} B2 must be false. Similarly, if B2 is true, then B1 must be false. It is impossible that both sentences are true at the same time. In other words, these sentences are inconsistent. On the other hand, G1 and G2 can both be true at the same time.
	\begin{ebullet}	
		\item[G1.] \label{MartianGiraffes} There are at least four giraffes at the wild animal park.
		\item[G2.] There are exactly seven gorillas at the wild animal park.
	\end{ebullet}
One of these sentences may be false and the other true, but it  is \textit{possible} that they are both true at the same time. 

\begin{factboxy}{jointly possible and impossible}
A set of sentences are \define{jointly possible} when, and only when, it is possible for them all to be true at the same time.
\medskip

A set of sentences are \define{jointly impossible} when, and only when, it is \textit{not} possible for them all to be true at the same time.
\end{factboxy}
\noindent So, G1 and G2 are \textit{jointly possible} while B1 and B2 are \emph{jointly impossible}.

We can investigate the joint possibility of any number of sentences. For example, let's add two more sentences to G1 and G2:
	\begin{ebullet}	
		\item[G1.] There are at least four giraffes at the wild animal park.
		\item[G2.] There are exactly seven gorillas at the wild animal park.
		\item[G3.] There are not more than two extra-terrestrials at the wild animal park.
		\item[G4.] Every giraffe at the wild animal park is an extra-terrestrial.
	\end{ebullet}
Together, G1 and G4 entail that there are at least four extra-terrestrials giraffes at the park. This conflicts with G3, which states that there are no more than two extra-terrestrials there. So, the sentences G1--G4 are jointly impossible. They cannot all be true together. (Notice also that just G1, G3 and G4 are jointly impossible, while G1, G2, and G3 are jointly possible.)


\section{Necessary equivalence}

Sentences G1 and G2---which we said were jointly possible---can both be true at the same time. They can also both be false, or one false and the other true. A stronger relationship holds between these two sentences:

\begin{earg}
\item[] John went to the store after he washed the dishes.
\item[] John washed the dishes before he went to the store.
\end{earg}
These two sentences must have the same truth value. That is, they must either both be true or both be false. It is impossible for one to be true and one to be false (at the same time).
When two sentences \textit{must} (or \textit{necessarily}) have the same truth value, they are \define{necessarily equivalent}.

\begin{factboxy}{necessarily equivalent}
Two sentences are \define{necessarily equivalent} if they must have the same truth value. (I.e., they must both be true or they both must be false.)
\end{factboxy}


\section[Necessary truths, falsehoods, and contingency]{Necessary truths, necessary falsehoods, and contingency}\label{s:nec-truth}

Consider these sentences:
	\begin{earg}
		\item[\textit{a.}] It is raining.
		\item[\textit{b.}] Either it is raining here, or it is not.
		\item[\textit{c.}] It is both raining here and not raining here.
	\end{earg}
In order to know if sentence (\textit{a}) is true, you would need to look outside or check a weather forcasting app. It might be true, or it might be false. A sentence that is capable of being true and capable of being false (in different circumstances, of course) is \define{contingent}.

Sentence (\textit{b}) is different. You do not need to look outside to know that it is true. Regardless of what the weather is, it is either raining or it is not. Thus, this sentence is a \define{necessary truth}. 

Similarly, you do not need to check the weather to determine whether or not sentence (\textit{c}) is true. It must be false, simply as a matter of logic. It might be raining here and not raining across town. It might be raining now but stop raining before you finish reading this sentence. It is impossible, however, for it to be both raining and not raining in the same place and at the same time. Therefore, this sentence is a \define{necessary falsehood}.

\begin{factboxy}{sentences: necessary and contingent}
\noindent A \define{necessary truth} is a sentence that must be true; it could not possibly be false.

\noindent A \define{necessary falsehood} is a sentence that must be false; it could not possibly be true.

\noindent A \define{contingent sentence} is neither a necessary truth nor a necessary falsehood. It may be true or it may not.
\end{factboxy}

Finally, a sentence might always be true and still be contingent. For instance, this sentence is true: 
	\begin{earg}
		\item[\textit{d.}] Mary Todd married Abraham Lincoln in 1842. 
	\end{earg}
And there is no way, now, that it will ever be false. But it could have been false. Todd and Lincoln could have gotten married in a different  year, or Todd could have married someone else or no one at all. A full analysis of this (and other) contingent truths would be too lengthy to undertake here, but hopefully you can see that things could have worked out in such a way that (\textit{d}) would be false. 

This is in contrast to a sentence like this one: `Today, in Starkville, Mississippi, it is Thursday, or it is not Thursday'. Or this one: `5 + 7 = 12'. These sentences cannot be false, and there is no way to imagine a possible series of events that would make them false. Hence, they are not contingent. They are necessary truths.    


%%%%%%%%%%%%%%%%%%%%%%%%%%%%%%
% Practice problems for the chapter on other logical notions

\section{Practice exercises}
\setcounter{ProbPart}{0}

\problempart
\label{pr.EnglishTautology2}
Determine if each sentence is a necessary truth, a necessary falsehood, or contingent.
\begin{earg}
\item Caesar crossed the Rubicon.
\item Someone once crossed the Rubicon.
\item No one has ever crossed the Rubicon.
\item If Caesar crossed the Rubicon, then someone has.
\item Even though Caesar crossed the Rubicon, no one has ever crossed the Rubicon.
\item If anyone has ever crossed the Rubicon, it was Caesar.

\item Elephants dissolve in water.
\item Wood is a light, durable substance useful for building things.
\item If wood is a good building material, it is useful for building things.
\item I live in a three story building that is two stories tall.
\item If gerbils are mammals, they nurse their young.
\end{earg}

\problempart Which of the following pairs of sentences are necessarily  equivalent? 
\begin{earg}
\item Elephants dissolve in water.
\item[]	If you put an elephant in water, it will dissolve.\smallskip
\item All mammals dissolve in water.
\item[]	If you put an elephant in water, it will dissolve.\smallskip 
\item George Bush was the 43rd president. 
\item[]	Barack Obama was the 44th president.\smallskip 
\item Barack Obama was the 44th president.
\item[]	Barack Obama was president immediately after the 43rd president.\smallskip 
\item Elephants dissolve in water. 
\item[]	All mammals dissolve in water.\smallskip

\item  Thelonious Monk played piano.
	\item[]	John Coltrane played tenor sax.\smallskip 
\item  Thelonious Monk played with John Coltrane.
	\item[]	John Coltrane played with Thelonious Monk.\smallskip
\item  All professional pianists begin playing as young children.
	\item[]	The professional pianist Bud Powell began playing as a young child.\smallskip
\item  Bud Powell suffered from severe mental illness.
	\item[]	All professional pianists suffer from severe mental illness.\smallskip
\item John Coltrane was deeply religious.	
\item[]	John Coltrane viewed music as an expression of spirituality. 
\end{earg}

\problempart  
\begin{earg}%[label=(\alph*)]
\item[G1.] \label{itm:at_least_four}There are at least four giraffes at the wild animal park.
\item[G2.] \label{itm:exactly_seven} There are exactly seven gorillas at the wild animal park.
\item[G3.] \label{itm:not_more_than_two} There are not more than two Martians at the wild animal park.
\item[G4.] \label{itm:martians} Every giraffe at the wild animal park is a Martian.
\end{earg}

\noindent Determine if each set of sentences is jointly possible or jointly impossible.
\begin{earg}
\item Sentences G2, G3, and G4
\item Sentences G1, G3, and G4
\item Sentences G1, G2, and G4
\item Sentences G1, G2, and G3
\end{earg}

\problempart 
\begin{earg}%[label=(\alph*)]
\item[M1.] \label{itm:allmortal} All people are mortal.
\item[M2.] \label{itm:socperson} Socrates is a person.
\item[M3.] \label{itm:socnotdie} Socrates will never die.
\item[M4.] \label{itm:socmortal} Socrates is mortal.
\end{earg}

\noindent Determine if each set of sentences is jointly possible or jointly impossible.
\begin{earg}
\item Sentences M1, M2, and M3
\item Sentences M2, M3, and M4
\item Sentences M2 and M3
\item Sentences M1 and M4
\item Sentences M1, M2, M3, and M4
\end{earg}

\problempart
\label{pr.EnglishCombinations2}
For each statement, determine whether or not it is possible. If it is possible, give an example that illustrates the statement. If it is not possible, explain why not.
\begin{earg}
\item A valid argument, the conclusion of which is a necessary falsehood
\item An invalid argument, the conclusion of which is a necessary truth
\item A necessary truth that is contingent
\item Two necessarily equivalent sentences, both of which are necessary truths
\item Two necessarily equivalent sentences, one of which is a necessary truth and one of which is contingent
\item Two necessarily equivalent sentences that together are jointly impossible
\item A jointly possible collection of sentences that contains a necessary falsehood
\item A jointly impossible set of sentences that contains a necessary truth

\item A valid argument with premises that are all necessary truths and with a conclusion that is contingent
\item A valid argument with true premises and a false conclusion
\item A jointly possible collection of sentences that contains two sentences that are not necessarily equivalent
\item A jointly possible collection of sentences, all of which are contingent
\item A false necessary truth
\item A valid argument with false premises
\item A necessarily equivalent pair of sentences that are not jointly possible
\item A necessary truth that is also a necessary falsehood
\item A jointly possible collection of sentences that are all necessary falsehoods
\end{earg}


%%%%%%%%%%%%%%%%%%%%%%%%%%%%
% Answers

\section{Answers}
\setcounter{ProbPart}{0}

\problempart
\label{pr.EnglishTautology}
For each of the following: Is it necessarily true, necessarily false, or contingent?
\begin{earg}
\item Caesar crossed the Rubicon.
\item[] Contingent\smallskip
\item Someone once crossed the Rubicon.
\item[] Contingent\smallskip
\item No one has ever crossed the Rubicon.
\item[] Contingent\smallskip
\item If Caesar crossed the Rubicon, then someone has.
\item[] Necessarily true\smallskip
\item Even though Caesar crossed the Rubicon, no one has ever crossed the Rubicon.
\item[] Necessarily false\smallskip
\item If anyone has ever crossed the Rubicon, it was Caesar.
\item[] Contingent\smallskip

\item Elephants dissolve in water.
\item[] Contingent\smallskip
\item Wood is a light, durable substance useful for building things.
\item[] Contingent\smallskip
\item If wood is a good building material, it is useful for building things.
\item[] Necessarily true\smallskip
\item I live in a three story building that is two stories tall.
\item[] Necessarily false\smallskip
\item If gerbils are mammals, they nurse their young.
\item[] This sentence is necessarily true. (\textit{Mammalia} is defined as the class of animals wherein the females have mammaries and nurse their young. Hence, `If gerbils are mammals, they nurse their young' is necessarily true.)
\end{earg}

\problempart 

\begin{earg}
\item Elephants dissolve in water.
\item[]	If you put an elephant in water, it will dissolve.
\item[]These sentences are necessarily equivalent.\smallskip

\item All mammals dissolve in water.
\item[]	If you put an elephant in water, it will dissolve.
\item[] These sentences are \textit{not} necessarily equivalent.\smallskip 

\item George Bush was the 43rd president. 
\item[]	Barack Obama was the 44th president.
\item[] These sentences are \textit{not} necessarily equivalent.\smallskip
 
\item Barack Obama was the 44th president.
\item[]	Barack Obama was president immediately after the 43rd president.
\item[] These sentences are necessarily equivalent.\smallskip 

\item Elephants dissolve in water. 
\item[]	All mammals dissolve in water.
\item[] These sentences are \textit{not} necessarily equivalent.\smallskip

\item  Thelonious Monk played piano.
	\item[]	John Coltrane played tenor sax.
	\item[] These sentences are \textit{not} necessarily equivalent.\smallskip 
\item  Thelonious Monk played with John Coltrane.
	\item[]	John Coltrane played with Thelonious Monk.
	\item[] These sentences are necessarily equivalent.\smallskip
	
\item  All professional pianists begin playing as young children.
	\item[]	The professional pianist Bud Powell began playing as a young child.
	\item[] These sentences are \textit{not} necessarily equivalent.\smallskip
\item  Bud Powell suffered from severe mental illness.
	\item[]	All professional pianists suffer from severe mental illness.
	\item[] These sentences are \textit{not} necessarily equivalent.\smallskip
\item John Coltrane was deeply religious.	
\item[]	John Coltrane viewed music as an expression of spirituality.
\item[] These sentences are \textit{not} necessarily equivalent. 
\end{earg}

\noindent
\problempart 
\label{pr.MartianGiraffes} 
\begin{earg}%[label=(\alph*)]
\item[G1.] \label{itm:at_least_four}There are at least four giraffes at the wild animal park.
\item[G2.] \label{itm:exactly_seven} There are exactly seven gorillas at the wild animal park.
\item[G3.] \label{itm:not_more_than_two} There are not more than two Martians at the wild animal park.
\item[G4.] \label{itm:martians} Every giraffe at the wild animal park is a Martian.
\end{earg}

\begin{earg}
\item Sentences G2, G3, and G4
\hfill \myanswer{Jointly possible}
\item Sentences G1, G3, and G4
\hfill \myanswer{Jointly impossible}
\item Sentences G1, G2, and G4
\hfill \myanswer{Jointly possible}
\item Sentences G1, G2, and G3
\hfill \myanswer{Jointly possible}
\end{earg}

\problempart 
\begin{earg}%[label=(\alph*)]
\item[M1.] \label{itm:allmortal} All people are mortal.
\item[M2.] \label{itm:socperson} Socrates is a person.
\item[M3.] \label{itm:socnotdie} Socrates will never die.
\item[M4.] \label{itm:socmortal} Socrates is mortal.
\end{earg}

\begin{earg}
\item Sentences M1, M2, and M3
\hfill Jointly impossible
\item Sentences M2, M3, and M4
\hfill Jointly impossible
\item Sentences M2 and M3
\hfill Jointly possible
\item[] \textit{Person}, at least in the philosophical sense, is different than \textit{human being} (although the two concepts generally overlap). \textit{Person} means, basically, \textit{moral agent}, and so, for instance, God, if he exists, is a person. Consequently, just the sentence `Socrates is a person' doesn't tell us whether or not Socrates will die.\smallskip
\item Sentences M1 and M4
\hfill Jointly possible
\item Sentences M1, M2, M3, and M4
\hfill Jointly impossible
\end{earg}

\problempart
\label{pr.EnglishCombinations2}
\begin{earg}
\item A valid argument, the conclusion of which is a necessary falsehood
\item[] \myanswer{Yes, this is possible. This is a valid argument, and the conclusion is a necessary falsehood:\\
P1. If today is Tuesday, then $1+1 = 3$.\\ 
P2. Today is Tuesday.\\
C. Therefore, $1+ 1= 3$.}\smallskip
\item An invalid argument, the conclusion of which is a necessary truth
\item[] \myanswer{No, this is not possible. If the conclusion is necessarily true, then there is no way to make it false, and hence no way to make it false whilst making all the premises true.}\smallskip 
\item A necessary truth that is contingent
\item[] \myanswer{No, this is not possible. If a sentence is a necessary truth, it cannot possibly be false, but a contingent sentence can be false.}\smallskip 
\item Two necessarily equivalent sentences, both of which are necessary truths
\item[] \myanswer{Yes, this is possible. `4 is even', `4 is divisible by 2'.}\smallskip 
\item Two necessarily equivalent sentences, one of which is a necessary truth and one of which is contingent
\item[] \myanswer{No, this is not possible.  A necessary truth cannot possibly be false,
  while a contingent sentence can be false.  So in any situation in
  which the contingent sentence is false, it will have a different
  truth value from the necessary truth. Thus, they will not necessarily
  have the same truth value, and so they will not be equivalent.}\smallskip
\item Two necessarily equivalent sentences that together are jointly impossible
\item[] \myanswer{Yes, this is possible. `$1+1=4$' and `$1+1=3$'.}\smallskip 
\item A jointly possible collection of sentences that contains a necessary falsehood
\item[] \myanswer{No, this is not possible. If a sentence is necessarily false, there is no way to make it true, let alone make it true along with all the other sentences.}
\item A jointly impossible set of sentences that contains a necessary truth
\item[] \myanswer{Yes, this is possible. `$1+1=4$' and `$1+1=2$'.}\smallskip

\item A valid argument with premises that are all necessary truths and with a conclusion that is contingent *
\item[] \myanswer{This is not possible. In a deductively valid argument, the information contained in the conclusion is information that is in the premises. If all of the premises are necessary truths, then the conclusion will be as well.}\smallskip

\item A valid argument with true premises and a false conclusion
\item[] \myanswer{This is not possible. A valid argument is one where if the premises are true, then the conclusion has to be true. Thus, if the premises are true, the conclusion has to be as well.}\smallskip

\item A jointly possible collection of sentences that contains two sentences that are not necessarily equivalent
\item[] \myanswer{Yes, this is possible. G1 and G2 on p.~\pageref{s:joint-poss} are jointly possible, but they are not necessarily equivalent.}\smallskip

\item A jointly possible collection of sentences, all of which are contingent
\item[] \myanswer{Yes, this is possible. G1 and G2 on p.~\pageref{s:joint-poss} are both contingent.}\smallskip

\item A false necessary truth
\item[] \myanswer{This is not possible. A necessary truth is a sentence that has to be true, and so it could not be false.}\smallskip

\item A valid argument with false premises
\item[] \myanswer{Yes, this is possible. This argument has false premises, and it is valid:\\ 
P1. Mississippi is in Canada.\\ P2. New York City is in Mississippi.\\ C. Therefore, New York City is in Canada.}
\smallskip

\item A necessarily equivalent pair of sentences that are not jointly possible
\item[] \myanswer{Yes, this is possible (although it isn't the standard case). These two sentences are both necessary falsehoods:\\
(a) It is November and it is not November.\\
(b) Jeff is in Texas and he is not in North America.\\ 
Since they are both always false, they are necessarily equivalent. But, at the same time, since they are both always false, they cannot be jointly possible.
 }\smallskip

\item A necessary truth that is also a necessary falsehood
\item[] \myanswer{This is not possible. A necessary truth is a sentence that must be (and is always) true. A necessary falsehood is a sentence that must be (and is always) false. Consequently, one sentence cannot be both.}\smallskip

\item A jointly possible collection of sentences that are all necessary falsehoods
\item[] \myanswer{This is not possible. See the answer to question 17.}\smallskip
\end{earg}
