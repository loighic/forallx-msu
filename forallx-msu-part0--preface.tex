\chapter{Preface}

I will begin by quoting E. J. Lemmon.
\begin{quote}
It is not easy, and perhaps not even useful, to explain briefly what logic is. Like most subjects, it comprises many different kinds of problem and has no exact boundaries; at one end, it shades off into mathematics, at another, into philosophy. The best way to find out what logic is is to do some. (1965, p. 1)
\end{quote}
He then continues, ``None the less, a few very general remarks about the subject may help to set the stage for the rest of this book,'' and so here are some general remarks. First, formal logic is the study of a formal language. Unlike natural languages (such as English, Spanish, Mandarin, and so forth), in a formal language, every part of the language—in particular, the content of the language and the rules for using the language—is precisely defined. Using a formal language limits what we can do. Natural languages are extremely flexible and adaptable; the formal language that we will use is not. The trade-off, however, is that our formal language is very precise, and it makes clear some of the fundamental aspects of human reasoning.

Second, and regarding those fundamental aspects of reasoning, the study of formal logic focuses on certain relationships between sentences, namely, \textit{consistency} and \textit{validity}. Consistency, as you might guess, concerns whether two or more sentences can all be  true at the same time (and some related notions  are also involved). Validity is about determining when (and, in some cases, why) a sentence follows, with certainty, from another sentence or set of sentences. For instance, if we know that `either \textit{A} or \textit{B} is true', then does it follow that `if \textit{B} is false, then \textit{A} is true'? (Yes, it  does.) 
\bigskip

\noindent This book is divided into four parts. Part~\ref{ch.intro} introduces the basic concepts of logic in an informal way, without introducing a formal language. Parts \ref{ch.TFL} -- \ref{ch.NDTFL} cover the formal language \textit{truth-functional logic} (TFL). (For reference, TFL also goes by other names: \textit{propositional logic} or \textit{sentential logic}.) In Part \ref{ch.TFL}, we begin with basic sentences. Basic sentences form more complex sentences with the logical operators ‘or’, ‘and’, ‘not’, ‘if \ldots then \ldots’, and `\ldots if and only if \ldots'. Once the logical operators have been introduced, we investigate validity in two ways: semantically, using the method of truth tables (in Part \ref{ch.TruthTables}) and proof-theoretically, using a system of formal derivations (in Part \ref{ch.NDTFL}). 

% In the appendices you’ll find a discussion of alternative notations for the language we discuss in this text and a ``quick reference'' listing most of the important rules and definitions. The central terms are listed in a glossary at the very end.

This book is based on a text originally written by P. D. Magnus and revised and expanded by Tim Button, J. Robert Loftis, Aaron Thomas-Bolduc, and Richard Zach. I have made additional revisions, taken out chapters that are not needed for the 1000-level logic course at Mississippi State University, and added instructions for using the logic software Carnap (http://carnap.io/), which can be used in conjunction with Parts \ref{ch.TruthTables} and \ref{ch.NDTFL}. The resulting text is licensed under a Creative Commons Attribution-ShareAlike 4.0 license.

Incidentally, the title \textit{forall\hspace{.10em}x} (i.e., ``for all x'') is a reference to \textit{first-order logic}---although this textbook does not cover first-order logic. In any event, this is a symbolic expression in first-order logic: $\forall x(Kx \eif Gx)$, and it is read, “for all \textit{x}, if \textit{x} is \textit{K}, then \textit{x} is \textit{G}.” Hence, the name of the textbook. (If, for instance, \textit{K} stands for ``is a king,'' and \textit{G} stands for ``is greedy,'' then $\forall x(Kx \eif Gx)$ means ``for all \textit{x}, if \textit{x} is a king, then \textit{x} is greedy,'' or ``everyone who is a king is greedy.'') 


\section{For instructors}

This textbook covers truth functional propositional logic only. The rules introduced in chapter \ref{s:BasicTFL} are similar to those used in Allen and Hand's \textit{Logic Primer}. (See section \ref{ProofRules} in the appendix for the exact list.) Proofs are constructed using Fitch notation, not the Lemmon-style system used by Allen and Hand.