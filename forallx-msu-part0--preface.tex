\chapter{Preface}

I will begin by quoting E. J. Lemmon,
\begin{quote}
It is not easy, and perhaps not even useful, to explain briefly what logic is. Like most subjects, it comprises many different kinds of problem and has no exact boundaries; at one end, it shades off into mathematics, at another, into philosophy. The best way to find out what logic is is to do some. (1965, p. 1)
\end{quote}
Nonetheless, there are a couple of things that will be useful to know before you begin. First, formal logic is the study of a formal language. Unlike a natural language (such as English, Spanish, Mandarin, and so forth), in formal languages every part of the language—in particular, the content of the language and the rules for using the language—is precisely defined. Second, the study of formal logic focuses on certain relationships between sentences, namely, consistency and entailment.  

The book is divided into four parts. Part~\ref{ch.intro} introduces the topic and basic concepts of logic in an informal way, without introducing a formal language. Parts \ref{ch.TFL} -- \ref{ch.NDTFL} cover the formal language \textit{truth-functional logic} (TFL). (For reference, TFL also goes by other names: \textit{propositional logic} or \textit{sentential logic}.) In Part \ref{ch.TFL}, we begin with basic sentences. Basic sentences form more complex sentences with the connectives ‘or’, ‘and’, ‘not’, ‘if \ldots then \ldots’, and `\ldots if and only if \ldots'. Once the connectives have been introduced, we investigate entailment in two ways: semantically, using the method of truth tables (in Part \ref{ch.TruthTables}) and proof-theoretically, using a system of formal derivations (in Part \ref{ch.NDTFL}). 

In the appendices you’ll find a discussion of alternative notations for the language we discuss in this text and a ``quick reference'' listing most of the important rules and definitions. The central terms are listed in a glossary at the very end.

This book is based on a text originally written by P. D. Magnus and revised and expanded by Tim Button, J. Robert Loftis, Aaron Thomas-Bolduc, and Richard Zach. I have made additional revisions, taken out chapters that I do not need for teaching the 1000-level logic course at Mississippi State University, and added instructions for using the logic software Carnap (http://carnap.io/), which I use in conjunction with Parts \ref{ch.TruthTables} and \ref{ch.NDTFL}. The resulting text is licensed under a Creative Commons Attribution-ShareAlike 4.0 license.

Incidentally, the title \textit{forall\hspace{.10em}x} (i.e., ``for all x'') is a reference to \textit{first-order logic}---although this version of the textbook does not, at least not right now, cover first-order logic. In any event, this is a symbolic expression in first-order logic: $\forall x(Kx \eif Gx)$, and it is read, “for all \textit{x}, if \textit{x} is \textit{K}, then \textit{x} is \textit{G}.” Hence, the name of the textbook. (If, for instance, \textit{K} stands for ``is a king,'' and \textit{G} stands for ``is greedy,'' then $\forall x(Kx \eif Gx)$ means ``for all \textit{x}, if \textit{x} is a king, then \textit{x} is greedy,'' or ``everyone who is a king is greedy.'') 