\chapter{Preface}

I will begin by quoting E. J. Lemmon.
\begin{quote}
It is not easy, and perhaps not even useful, to explain briefly what logic is. Like most subjects, it comprises many different kinds of problem and has no exact boundaries; at one end, it shades off into mathematics, at another, into philosophy. The best way to find out what logic is is to do some. (1965, p. 1)
\end{quote}
He then continues, ``None the less, a few very general remarks about the subject may help to set the stage for the rest of this book.'' Following his lead, here are some general remarks. First, formal logic is the study of formal languages. Unlike natural languages (such as English, Spanish, and Mandarin), in a formal language, every part of the language is precisely defined. Using a formal language limits what we can do. Natural languages are extremely flexible and adaptable; the formal language that we will use is not. The trade-off, however, is that the formal languages that we will study are very precise, and they make clear some of the fundamental aspects of human reasoning.

Second, this textbook is designed for a single semester introduction to formal (i.e., deductive) logic. It primarily covers \textit{truth-functional logic} (TFL). (For reference, TFL also goes by other names: \textit{propositional logic} or \textit{sentential logic}.) In truth functional logic, individual statements (e.g., ``the cat is on the mat'' or ``Westerby is talking with Ricardo'') are treated as units that can be combined into more complex statements with  ‘or’, ‘and’, ‘not’, ‘if \ldots then \ldots’, and `if and only if'. The study of truth functional logic, then, is the study of the properties of these more complex statements and the logical relationships between them. 

The final part of the book introduces \textit{first-order logic} (FOL), which includes, in addition to the connectives of truth-functional logic, names, variables, predicates, identity, and the so-called quantifiers. These additional elements of the language make it much more expressive than the truth-functional language, and so broaden the logical analysis that can be undertaken. 


Incidentally, the title \textit{forall\hspace{.10em}x} (i.e., ``for all x'') is a reference to \textit{first-order logic}. This is a symbolic expression in first-order logic: $\forall x(Kx \eif Gx)$, and it is read, “for all \textit{x}, if \textit{x} is \textit{K}, then \textit{x} is \textit{G}.” Hence, the name of the textbook. (If, for instance, \textit{K} stands for ``is a king,'' and \textit{G} stands for ``is greedy,'' then $\forall x(Kx \eif Gx)$ means ``for all \textit{x}, if \textit{x} is a king, then \textit{x} is greedy,'' or ``everyone who is a king is greedy.'') 


\section{For instructors}

This textbook covers truth functional propositional logic and introduces first-order logic. The rules introduced in chapter \ref{s:BasicTFL} are similar to those used in Allen and Hand's \textit{Logic Primer}. (See section \ref{ProofRules} in the appendix for the exact list.) Proofs are constructed using Fitch notation, not the Lemmon-style system used by Allen and Hand.

This book is based on a text originally written by P. D. Magnus and revised and expanded by Tim Button, J. Robert Loftis, Aaron Thomas-Bolduc, and Richard Zach. I have made additional revisions, taken out chapters that are not needed for the 1000-level logic course at Mississippi State University, and added instructions for using the logic software Carnap (http://carnap.io/), which can be used in conjunction with Parts \ref{ch.TruthTables} and \ref{ch.NDTFL}. The resulting text is licensed under a \href{https://creativecommons.org/licenses/by/4.0/}{Creative Commons Attribution 4.0 International (CC BY 4.0) license}.

 