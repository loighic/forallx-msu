\graphicspath{{figures--proofs/}}

\chapter{FOL: Translations and scope}\label{FOL-basics}

\section{Translations}

We will begin with some translations. Here is a translation scheme.

\begin{ekey}
	\item[\textrm{domain}] people in Starkville
	\item[B] \rule{1cm}{0.15mm} ate a burrito
	\item[N] \rule{1cm}{0.15mm} took a nap
	\item[L] \rule{1cm}{0.15mm} loves \rule{1cm}{0.15mm}
	\item[a] Abigail
	\item[c] Carol
	\item[d] David
\end{ekey}

\noindent We can then translate the following sentences from first-order logic to English as is shown. First, here are some sentences composed of predicates and names.

\begin{ebullet}
	\item[]$B(a) $: Abigail ate a burrito.\smallskip
	\item[]$N(d) $: David took a nap.\smallskip
	\item[]$\enot N(c)$: Carol did not take a nap.\smallskip
	\item[]$N(d) \eand B(a) $: David took a nap and Abigail ate a burrito.\smallskip
	\item[]$B(a) \eif N(a)$: If Abigail ate a burrito, then Abigail took a nap.\smallskip
	\item[]$L(d,c) \eand B(c)$: David loves Carol, and Carol ate a burrito.
\end{ebullet}

\noindent Here are some quantified expressions.

\begin{ebullet}
	\item[] $\forall y N(y)$: Everyone took a nap.\smallskip
	\item[] $\exists x B(x)$: Someone ate a burrito.\smallskip
	\item[] $\enot\forall xN(x)$: It is not the case that everyone took a nap.\smallskip
	\item[]$\forall x \enot N(x)$: Everyone did not take a nap.\smallskip
	\item[]$\enot\exists yB(y)$: It is not the case that someone ate a burrito.\smallskip
	\item[]$\exists y \enot B(y)$: Someone did not eat a burrito.\smallskip
	\item[]$\exists x[B(x) \eand N(x)]$: Someone ate a burrito and took a nap.\smallskip
\end{ebullet}
	
If you are stuck on a translation, you can always just translate $\forall x$ and $\exists x$ like this:
\begin{ebullet}
\item[] For all \textit{x}, \textit{x} is \ldots
\item[] There exists an \textit{x} such that \textit{x} is \ldots
\end{ebullet} 
So, to translate the expression, proceed this way. (1) Begin reading from the left. (2) Read the universal or existential quantifier and variable as ``For all \ldots'' or ``There exists \ldots''. (3) Read the predicates as is given in the translation scheme. (4) Read the logical operators as we did in TFL. (5) If there is a `not' sign somewhere, translate it right where it is as `it is not the case that'. Once you've translated the sentence this way, you may want to word it more naturally using `someone' or `everyone'.

These sentences contain either both a name and a variable or both of the quantifiers (and two different variables).
	
\begin{ebullet}
	\item[]$\exists x[B(x) \eand N(c)]$: Someone ate a burrito, and Carol took a nap.\smallskip
	\item[]$\exists xL(x,c)$: Someone loves Carol.\smallskip
	\item[]$\exists x \forall y[B(x) \eand N(y)]$: Someone ate a burrito and everyone took a nap.\smallskip
	\item[]$\forall x \exists yL(x,y)$: Everyone loves someone.\smallskip
	\item[]$\exists x \exists yL(x,y)$: Someone loves someone.
\end{ebullet}

\noindent The formats for common expressions that are translated into first-order logic as conditionals or conjunctions are listed in table \ref{FOL-translations}. 

%\begin{factboxy}{}
\begin{factboxy-width}[width=8cm]{equivalent quantified expressions}
$\exists \meta{x} ¬\meta{B(x)} $ is equivalent to $¬\forall \meta{x B(x)}$.\medskip

$\forall \meta{x} ¬\meta{D(x)}$ is equivalent to $¬\exists \meta{x D(x)}$.
\end{factboxy-width}


\begin{table*} %[hb]    % The \textcolor{white}{\LARGE{I}} is to make those row heights larger.
\centering\sffamily\footnotesize
\ra{1.10}
\begin{tabular}{@{}m{2.3cm}  l@{}}\arrayrulecolor{black}\hline
$\forall x[F(x) \eif G(x)]$:
&`For all x, if x is an F, then x is a G’ \textbf{or} 
`Every F is G.’\\  %\arrayrulecolor{light-gray}\hline

$\exists x[F(x) \eand G(x)]$:
&`There exists an x, such that x is an F and a G’ \textbf{or}\textcolor{white}{\LARGE{I}}\\ 
&Some F is G’.\\   %\arrayrulecolor{light-gray}\hline

`No F is G’:
&$¬\exists x[F(x) \eand G(x)] $ \textbf{or} $\forall x[F(x) \eif ¬G(x)]$ \textcolor{white}{\LARGE{I}}\\  %\arrayrulecolor{light-gray}\hline

‘Only Fs are Gs’:
&$\enot \exists x[G(x) \eand \enot F(x)]$ \textbf{or} $\forall x[G(x) \eif F(x)]$ \textcolor{white}{\LARGE{I}}\\\arrayrulecolor{black}\hline
\end{tabular}
\caption{}\label{FOL-translations}
\end{table*}


\section{Scope}

Just like the logical operators, the universal quantifier (`$\forall$’) and the existential quantifier (`$\exists$’) have a scope. For both, their scope is the part of the expression to which the quantifier applies---or, as we often say, the part of the sentence that it \textit{ranges over}. Basically, the scope of the quantifiers works like the scope of the negation (i.e., the `$\enot$’). 

\begin{ebullet}
\item[(1)] If there is \textit\textbf{not} a parenthesis between the quantifier and a predicate (e.g., `$\forall xF(x)$’ or `$\forall x \enot F(x)$’), then the scope of the quantifier is that predicate (or the predicate and a $\enot$) and its variable or variables.\smallskip
\item[(2)] If there is a parenthesis between the quantifier and the first predicate (e.g., `$\forall x[F(x) \eor G(x)]$’), then the scope of the quantifier is everything inside the parentheses.\smallskip
\item[] (Notice that this is all the same as the scope of the `$\enot$’.)
\end{ebullet}
