

%% Ch 1

\newglossaryentry{premise indicator word}
{
name=premise indicator,
description={a word or phrase such as ``because'' used to indicate that what follows is the premise of an argument}
}

\newglossaryentry{conclusion indicator word}
{
name=conclusion indicator,
description={a word or phrase such as ``therefore'' used to indicate that what follows is the conclusion of an argument}
}

\newglossaryentry{argument}
{
name=argument,
description={a connected series of sentences, divided into \gls{premise}s and \gls{conclusion}}
}

\newglossaryentry{premise}
{
name=premise,
description={a sentence in an \gls{argument} other than the \gls{conclusion}}
}

\newglossaryentry{conclusion}
{
name=conclusion,
description={the last sentence in an \gls{argument}}
}


%% Ch 2, Valid Arguments 

\newglossaryentry{valid}
{
name=valid,
description={A property of arguments where it is impossible for the premises to be true and the conclusion false}
}

\newglossaryentry{invalid}
{
name=invalid,
description={A property of arguments that holds when it is possible for the premises to be true without the conclusion being true; the opposite of \gls{valid}}
}

\newglossaryentry{sound}
{
name=sound,
description={A property of arguments that holds if the argument is valid and has all true premises}
}


%% Ch 3, Other Logical Notions

\newglossaryentry{possibility}
{
name=joint possibility,
text={jointly possible},
description={A property possessed by some sentences when they can all be true at the same time}
}

\newglossaryentry{contingent sentence}
{
name=contingent sentence,
description={A sentence that is neither a \gls{necessary truth} nor a \gls{necessary falsehood}; a sentence that in some situations is true and in others false}
}

\newglossaryentry{necessary truth}
{
name={necessary truth},
description={A sentence that must be true}
}

\newglossaryentry{necessary falsehood}
{
name={necessary falsehood},
description={A sentence that must be false}
}

\newglossaryentry{necessary equivalence}
{
name={necessary equivalence},
text={necessarily equivalent},
description={A property held by a pair of sentences that must always have the same truth value}
}


%% Ch 4, First Steps to Symbolization

\newglossaryentry{atomic sentence}
{
name={atomic sentence},
description={A sentence used to represent a basic sentence; a single letter in TFL, or a predicate symbol followed by names in FOL}
}

\newglossaryentry{symbolization key}
{
name={symbolization key},
description={A list that shows which English sentences are represented by which \glspl{atomic sentence} in TFL}
}

%% Ch 5, Logical operators

\newglossaryentry{connective}
{
name=connective,
description={A word or phrase used to modify a sentence, or a word or phrase used to combine two sentences into a more complex sentence} 
}

\newglossaryentry{negation}
{
name=negation,
description={The symbol \enot, used to represent words and phrases that function like the English word ``not''}
}

\newglossaryentry{conjunction}
{
name=conjunction,
description={The symbol \eand, used to represent words and phrases that function like the English word ``and''; or a sentence formed using that symbol}
}

\newglossaryentry{conjunct}
{
name=conjunct,
description={A sentence joined to another by a \gls{conjunction}}
}

\newglossaryentry{disjunction}
{
name=disjunction,
description={The connective \eor, used to represent words and phrases that function like the English word ``or'' in its inclusive sense; or a sentence formed by using this connective}
}

\newglossaryentry{disjunct}
{
name=disjunct,
description={A sentence joined to another by a \gls{disjunction}}
}

\newglossaryentry{conditional}
{
name={conditional},
description={The symbol \eif, used to represent words and phrases that function like the English phrase ``if \dots, then \dots''; a sentence formed by using this symbol}
}

\newglossaryentry{antecedent}
{
name=antecedent,
description={The sentence on the left side of a \gls{conditional}}
}

\newglossaryentry{consequent}
{
name=consequent,
description={The sentence on the right side of a \gls{conditional}}
}

\newglossaryentry{biconditional}
{
name=biconditional,
description={The symbol \eiff, used to represent words and phrases that function like the English phrase ``if and only if''; or a sentence formed using this connective.}
}


%% Ch 6, Sentences of TFL

\newglossaryentry{sentence of TFL}
{
name=sentence of TFL,
description={A string of symbols in TFL that can be built up according to the recursive rules given on p.~\pageref{TFLsentences}}
}

\newglossaryentry{scope}
{
name=scope,
description={A property of connectives. The sentence or subsentence for which that connective is the main logical operator}
}


%% Ch 7, Use and Mention

\newglossaryentry{object language}
{
name=object language,
description={A language that is constructed and studied by logicians. In this textbook,
 the object languages are TFL and FOL.}
}

\newglossaryentry{metalanguage}
{
name=metalanguage,
description={The language logicians use to talk about the object language. In this textbook, the metalanguage is English, supplemented by certain symbols like metavariables and technical terms like ``valid.''}
}

        \newglossaryentry{metavariables}
{
name=metavariables,
description={A variable in the metalanguage that can represent any sentence in the object language.}
}


%% Ch 8, Characteristic truth tables

\newglossaryentry{truth value}
{
name = truth value,
description = {One of the two logical values sentences can have: True and False}
}


%% Ch 9, Complete truth tables

\newglossaryentry{valuation}
{
name=valuation,
description={An assignment of \glspl{truth value} to particular atomic \glspl{sentence of TFL}}
}


%% Ch 10, Semantic concepts

\newglossaryentry{tautology}
{
name=tautology,
description={A sentence that has only Ts in the column under the main logical operator of its \gls{complete truth table}; a sentence that is true on every \gls{valuation}}
}

\newglossaryentry{contradiction of TFL}
{
  name=contradiction (of TFL),
  text = contradiction,
description={A sentence that has only Fs in the column under the main logical operator of its \gls{complete truth table}; a sentence that is false on every \gls{valuation}}
}


\newglossaryentry{equivalent}
{
  name=equivalence (in TFL),
  text = equivalent,
description={A property held by pairs of sentences if and only if the \gls{complete truth table} for those sentences has identical columns under the two main logical operators, i.e., if the sentences have the same truth value on every valuation}
}


\newglossaryentry{joint consistency in TFL}
{
  name=joint consistency (in TFL),
  text=jointly consistent,
description={A property held by sentences if and only if the \gls{complete truth table} for those sentences contains one line on which all the sentences are true, i.e., if some \gls{valuation} makes all the sentences true}
}


%% Ch 11, Entailment and Validity


\newglossaryentry{logically valid in TFL}
{
  name=logical validity (in TFL),
  text = logically valid,
description={A property held by arguments if and only if the \gls{complete truth table} for the argument contains no rows where the \glspl{premise} are all true and the \gls{conclusion} false, i.e., if no \gls{valuation} makes all premises true and the conclusion false}
}


%% Ch 14, The rules for TFL

\newglossaryentry{proof}
{
  name=proof,
  text = proof,
description={A sequence of sentences. The first sentences of the sequence are assumptions; these are the premises of the argument. Every sentence later in the sequence follows from earlier sentences by one of the rules of TFL. The final sentence of the sequence is the conclusion of the argument}
}


%% Ch 16, Proof-theoretic Concepts

\newglossaryentry{theorem}
{
name=theorem,
description={A sentence that can be proved without any premises}
}


\newglossaryentry{provably equivalent}
{
  name=provable equivalence,
  text = provably equivalent,
description={A property held by pairs of statements if and only if there is a derivation which takes you from each one to the other one}
}

\newglossaryentry{provably inconsistent}
{    name={provable inconsistency}, 
  description={Sentences are provably inconsistent iff a contradiction can be derived from them},
    text={provably inconsistent}
}

%% Ch 17, Soundness and Completeness

\newglossaryentry{soundness}
{
name=soundness,
description={A property held by logical systems if and only if p-valid implies tt-valid}
}


\newglossaryentry{completeness}
{
name=completeness,
description={A property held by logical systems if and only if tt-valid implies p-valid}
}



