\graphicspath{{figures--tt/}}

%%%%%%%%%%%%%%%%% CHAPTER 11

\chapter{Truth tables and validity}\label{c:tt-validity}

\section{Entailment}\label{s:tt-entailment}

Having examined the logical relations between two sentences in \S\ref{equivalence--tt} and \S\ref{consistency--tt}, we can now go a step further and consider the relationship between the premises and the conclusion of an argument. This begins with \define{entailment}.

\begin{factboxy}{Entailment}
The sentences $\meta{A}_1, \meta{A}_2, \ldots, \meta{A}_n$ \define{entail} the sentence $\meta{C}$ if there is no valuation of the atomic sentences that makes all of $\meta{A}_1, \meta{A}_2, \ldots, \meta{A}_n$ true and $\meta{C}$ false.
\end{factboxy}

\begin{notebox}
Recall that `$\proves$' is used to separate the premises from the conclusion in arguments in TFL. It can be read as \textit{therefore}.
\end{notebox}
 
Entailment is easy to check with a truth table. We will use `$\enot L \eif (M \eor L), \enot L \proves M$' as our example. Do `$\enot L \eif (M \eor L)$' and `$\enot L$' entail `$M$'? To find out, we check whether there is any valuation that makes both `$\enot L \eif (M \eor L)$' and `$\enot L$' true while making `$M$' false.  
\begin{center}
\begin{tabular}{c c|d e e e e f|d f|| c}
$M$&$L$&\enot&$L$&\eif&$(M$&\eor&$L)$&\enot&$L$&$M$\\
\hline
%J   L   -   L      ->     (J   v   L)
 T & T & F & T & \TTbf{T} & T & T & T & \TTbf{F} & T & \TTbf{T}\Tstrut\\
 T & F & T & F & \circled{\TTbf{T}} & T & T & F & \circled{\TTbf{T}} & F & \TTbf{T}\\
 F & T & F & T & \TTbf{T} & F & T & T & \TTbf{F} & T & \TTbf{F}\\
 F & F & T & F & \TTbf{F} & F & F & F & \TTbf{T} & F & \TTbf{F}
\end{tabular}
\end{center}
There is only one row where both `$\enot L \eif (M \eor L)$' and `$\enot L$' are true---the second row---and so that is the only row that concerns us. On that row, `$M$' is also true. Hence, `$\enot L \eif (M \eor L)$' and `$\enot L$' entail `$M$'. 

Next is this important observation: if $\meta{A}_1, \meta{A}_2, \ldots, \meta{A}_n$ entail $\meta{C}$, then $\meta{A}_1, \meta{A}_2, \ldots, \meta{A}_n \proves \meta{C}$ is valid. Just to remind ourselves, an argument is valid when it is the case that if the premises are true, then the conclusion has to be true. A different but equivalent way of wording this definition will be more useful to us here, though. 
\begin{factboxy}{Valid}
An argument is \define{valid} if and only if it is impossible for all of the premises to be true and the conclusion false.
\end{factboxy}

Here's why entailment equals validity. If $\meta{A}_1, \meta{A}_2, \ldots, \meta{A}_n$ entail $\meta{C}$, then there is no valuation that makes all of $\meta{A}_1, \meta{A}_2, \ldots, \meta{A}_n$ true while making $\meta{C}$ false. This means that it is \emph{impossible} for each of $\meta{A}_1, \meta{A}_2, \ldots, \meta{A}_n$ to be true and $\meta{C}$ to be false. And that is just what it takes for an argument with premises $\meta{A}_1, \meta{A}_2, \ldots, \meta{A}_n$ and conclusion $\meta{C}$ to be valid!

In short, we have a way to test whether an argument in English is valid. First, we symbolize the premises and conclusion in TFL. Then we test for entailment using truth tables. 


\section{Validity}\label{s:tt-validity}

When using a truth table to determine if an argument is valid, the premise or premises are listed first, the turnstile symbol (`\proves') is next, and then the conclusion is last. Once all of the truth table is completed except for the column under the turnstile, we check for lines that violate the definition of a valid argument. We'll call lines that violate that definition \textit{bad lines}. 
	\begin{earg}
		\item[(1)] Any line where all of the premises are true and the conclusion is false \textbf{is a bad line}.
		\item[(2)] Any line where all of the premises are true and the conclusion is true \textbf{is a good line}.
		\item[(3)] Any line where the conclusion is true cannot be a bad line. (So, whatever the case may be with the premises, \textbf{it's a good line}.) 
		\item[(4)] Any line where at least one premise is false cannot be a bad line. (So, whatever the case may be with the other premises and the 			 								conclusion, \textbf{it's a good line}.) 
	\end{earg}

Let's look at the truth table for our argument with one small (but significant) change: $\enot L \eif (M \eor L)$, $\enot L \proves \enot M$. The premises are the same, but now the conclusion is $\enot M$ instead of $M$. Here is the truth table:
\begin{center}
\begin{tabular}{c c|d e e e e f|d f| c | c}
$M$&$L$&\enot&$L$&\eif&$(M$&\eor&$L)$&\enot&$L$		&\proves		& $\enot M$\\
\hline
 T & T & F & T & \TTbf{T} & T & T & T 	& \TTbf{F} 	& T 		&	& \TTbf{F}\Tstrut\\
 T & F & T & F & \TTbf{T} & T & T & F 	& \TTbf{T} 	& F		&	& 	\TTbf{F}\\
 F & T & F & T & \TTbf{T} & F & T & T 	& \TTbf{F} 	& T 		&	& \TTbf{T}\\
 F & F & T & F & \TTbf{F} & F & F & F 	& \TTbf{T} 	& F 		&	& \TTbf{T}
\end{tabular}
\end{center}
The truth values for the premises are the same, and the truth values for the conclusion have, on each line, flipped from T to F or vice versa. Now, when we evaluate each line, what do we find? As before, on lines 1, 3, and 4, one of the premises is false, and so they are good lines. On line 2, the premises are true and the conclusion is false. That's a bad line! 
\begin{center}
\begin{tabular}{c c|d e e e e f|d f| c | c}
$M$&$L$&\enot&$L$&\eif&$(M$&\eor&$L)$&\enot&$L$	&\proves				& $\enot M$\\
\hline
 T & T & F & T & \TTbf{T} & T & T & T & \TTbf{F} & T 		&	\cm					& \TTbf{F}\Tstrut\\
 T & F & T & F & \circled{\TTbf{T}} & T & T & F & \circled{\TTbf{T}} & F 	& \xm	& \circled{\TTbf{F}}\\
 F & T & F & T & \TTbf{T} & F & T & T & \TTbf{F} & T 		&	\cm					& \TTbf{T}\\
 F & F & T & F & \TTbf{F} & F & F & F & \TTbf{T} & F 		&	\cm					& \TTbf{T}
\end{tabular}
\end{center}
This means that $\enot L \eif (M \eor L)$ and $\enot L$ do not entail $\enot M$ and the argument `$\enot L \eif (M \eor L)$, $\enot L \proves \enot M$' is not valid.


\section{Some examples}

Here are some examples using truth tables to determine whether an argument is valid. As a reminder, the definition of valid is given in \S\ref{s:tt-entailment}, and we can also use 1 - 4 on p.~\pageref{s:tt-validity} (which are consequences of the definition). We will begin with arguments that have only one premise and then do some with multiple premises.

\begin{earg}
\item[\ex{1P-1}] First is `$P \eand Q \proves Q$'. The premise, `$P \eand Q$', is only true on line 1. Since it is false on lines 2 - 4, we know that those are good lines. (See guideline 4.) On line 1, `$P \eand Q$' is true and the conclusion, `$Q$', is true, and so that is also a good line. (See guideline 2.) Since every line is a good line, this argument is valid.
\begin{center}
\begin{tabular}{c c|d e f| c | c}
$P$& $Q$& 	$P$& $\eand$& $Q$& $\proves$& $Q$\\
\hline
 T & T 	&   T& \TTbf{T} & T & \cm & \TTbf{T}\Tstrut\\
 T & F 	&   T& \TTbf{F} & F & \cm & \TTbf{F}\\
 F & T 	&   F& \TTbf{F} & T & \cm & \TTbf{T}\\
 F & F 	&   F& \TTbf{F} & F & \cm & \TTbf{F} 
\end{tabular}
\end{center}

\item[\ex{1P-2}]
In `$\enot (P\eor Q) \proves \enot P \eand Q$', the premise is false on lines 1 - 3, and so we know that those have to be good lines. On line 4, the premise is true and the conclusion is false, which means that line 4 is a bad line. (See guideline 1.) Since it has at least one bad line, this argument is not valid. 
\begin{center}
\begin{tabular}{c c|d e e f| c |d e e f}
$P$& $Q$& 	$\enot$& $(P$& $\eor$& $Q)$& $\proves$& $\enot$& $P$& $\eand$& $Q$\\
\hline
 T & T 	&   \TTbf{F} & T & T& T& 	\cm & F& T& \TTbf{F}& T\Tstrut\\
 T & F 	&   \TTbf{F} & T & T& F&	\cm & F& T& \TTbf{F}& F\\
 F & T 	&   \TTbf{F} & F & T& T&	\cm & T& F& \TTbf{T}& T\\
 F & F 	&   \TTbf{T} & F & F& F&	\xm & T& F& \TTbf{F}& F
\end{tabular}
\end{center}

\item[\ex{2P-1}] `$P \eif Q, \enot Q \proves \enot P$' contains two premises, `$P \eif Q$' and `$\enot Q$'. Since both premise are not true on lines 1, 2, and 3, those are all good lines. Both premises are true on line 4, and the conclusion is true on that line, and so that is a good line. Since every line is a good line, this argument is valid.
\begin{center}
\begin{tabular}{c c|d e f| d f | c |d f}
$P$ & $Q$ & $P$ & $\eif$ & $Q$ & $\enot$ & $Q$ & \proves & $\enot$ &$P$\\ 
\hline
T & T &   T &   \TTbf{T} &T   & \TTbf{F} &T & \cm &\TTbf{ F } &T\Tstrut\\ 
T & F &   T &   \TTbf{F} &F   & \TTbf{T} &F & \cm & \TTbf{F } &T\\ 
F & T &   F &   \TTbf{T} &T   & \TTbf{F} &T & \cm & \TTbf{T } &F\\ 
F & F &   F &   \TTbf{T} &F   & \TTbf{T} &F & \cm & \TTbf{T } &F\\ 
\end{tabular}
\end{center}


\item[\ex{2P-2}]
Next, consider `$P\eif Q, P \eif \enot Q \proves P$'. Since the second premise is false on line 1 and the first premise is false on line 2, those are good lines. On line 3, both of the premises are true and the conclusion is false. That's a bad line. And then the same is also the case on line 4, and so that is a bad line also. Since two of the lines in this truth table are bad lines, the argument is invalid.
\begin{center}
\begin{tabular}{c c|d e f 	 d e e f 	   c 	  c }
$P$ &$Q$ 	&$P$ & $\eif$ &$Q$,  	& $P$ & $\eif$ & $\enot$ &$Q$ & \proves	& $P$\\ 
\hline
T &T   &T &\TTbf{T} &T   &T &\TTbf{F} &F  &T & \cm &\TTbf{T}\Tstrut\\ 
T &F   &T &\TTbf{F} &F    &T &\TTbf{T} &T &F  & \cm &\TTbf{T}\\ 
F &T   &F &\TTbf{T} &T    &F &\TTbf{T} &F &T  & \xm &\TTbf{F}\\ 
F &F   &F &\TTbf{T} &F    &F &\TTbf{T} &T &F  & \xm &\TTbf{F}\\ 
\end{tabular}
\end{center}

\item[\ex{3P-1}] In the last argument, we have three premises. One of the premises is false on each of lines 1, 2, 4, 5, 7, and 8, and so all of those are good lines. On line 3, all of the premises are true and the conclusion is true, and so that is a good line. On line 6, all of the premises are true but the conclusion is false, and so that is a bad line. Since one of the lines is a bad line, this argument is invalid. 
\begin{center}
\begin{tabular}{c c c | d e f     d e f		 d e e f 	   c 	  c }
$P$& $Q$& $R$&  $P$& 	$\eor$& 	$Q$,&   $P$		&$\eif$	&$R$,		&$Q$		&$\eif$	&$\enot$	&$R$	&$\proves$& $R$\\ 
\hline
T& T& T &   T &\TTbf{T}& T   &   T &\TTbf{T}& T   &   T& \TTbf{F}& F& T&\cm   & \TTbf{T}\Tstrut\\ 
T& T& F &   T &\TTbf{T}& T   &   T &\TTbf{F}& F   &   T& \TTbf{T}& T& F&\cm   & \TTbf{F}\\ 
T& F& T &   T &\TTbf{T}& F   &   T &\TTbf{T}& T   &   F& \TTbf{T}& F& T&\cm   & \TTbf{T}\\ 
T& F& F &   T &\TTbf{T}& F   &   T &\TTbf{F}& F   &   F& \TTbf{T}& T& F &\cm  & \TTbf{F}\\\arrayrulecolor{light-gray}\hline
F& T& T &   F &\TTbf{T}& T   &   F &\TTbf{T}& T   &   T& \TTbf{F}& F& T&\cm   & \TTbf{T}\Tstrut\\ 
F& T& F &   F &\TTbf{T}& T   &   F &\TTbf{T}& F   &   T& \TTbf{T}& T& F &\xm & \TTbf{F}\\ 
F& F& T &   F &\TTbf{F}& F   &   F &\TTbf{T}& T   &   F& \TTbf{T}& F& T &\cm  & \TTbf{T}\\ 
F& F& F &   F &\TTbf{F}& F   &   F &\TTbf{T}& F   &   F& \TTbf{T}& T& F &\cm  & \TTbf{F}\\
\end{tabular}
\end{center}

\end{earg}


\section{`$\proves$' versus `$\eif$'}

When using truth tables to determine whether an argument is valid, it may help you to notice a similarity between `$\proves$' and `$\eif$'. As you know, a conditional is true under every circumstance except when the antecedent is true and the consequent if false. (So, when we have a `T' under the antecedent and an `F' under the consequent, we put an `F' under the `$\eif$'.) Meanwhile, in an argument, when all of  the premises are true and the conclusion is false, the argument is invalid. (So, for a specific line, when we have a `T' under every premise and an `F' under the conclusion, we put a `\xm' under the `$\proves$'.) 

The reasoning here is similar. In both cases, we are violating the principle---of either the conditional or of a valid argument---when we have a false sentence that follows from a sentence or a set of sentences that are all true. Thus, if $\meta{A} \eif \meta{C}$ is false, then $\meta{A} \proves \meta{C}$ is invalid (and if $\meta{A} \proves \meta{C}$ is invalid, then $\meta{A} \eif \meta{C}$ is false). Conversely, whenever $\meta{A} \eif \meta{C}$ is true, then $\meta{A} \proves \meta{C}$ is valid (and vice versa). 


\section{The limits of this type of analysis}\label{s:ParadoxesOfMaterialConditional}

We have seen in chapters \ref{s:SemanticConcepts} and \ref{c:tt-validity} that truth tables are a useful tool for analyzing sentences---whether those are individual sentences, pairs of sentences, or arguments. There are limitations to this type of analysis, however, and it worth understanding some of those limitations.

First, consider this argument:
	\begin{earg}
		\item Daisy has four legs. Therefore, Daisy has more than two legs.
	\end{earg}
To symbolize this argument in TFL, we would have to use two different atomic sentences---perhaps `$F$' for the premise  and `$T$' for the conclusion. The English version of this argument is clearly valid, but `$F \proves T$' is just as clearly invalid. 
\begin{center}
\begin{tabular}{c c|c c c}
$F$& $T$&  $T$& $\proves$& $F$\\ 
\hline
T& T& T&\cm&  T\Tstrut\\ 
T& F& F&\cm&  T\\ 
F& T& T&\xm&  F\\ 
F& F& F&\cm&  F\\
\end{tabular}
\end{center} 

Next, consider this sentence:
\begin{earg}
\setcounter{eargnum}{1}
\item\label{n:JohnBald} John is neither bald nor not-bald.
\end{earg}
This is symbolized in TFL as `$\enot(J \eor \enot J)$', and, as you can see  from the truth table, it is a contradiction. 
\begin{center}
\begin{tabular}{c | d e e e f}
$J$&  $\enot$& ($J$& $\eor$& $\enot$& $J$)\\ 
\hline
T&  \TTbf{F}&   T& T& F& T\Tstrut\\ 
F&  \TTbf{F}&   F& T& T& F\\  
\end{tabular}
\end{center} 
But sentence \ref{n:JohnBald} does not seem like a contradiction. After all, someone could very well add ``John is on the borderline of baldness,'' which would (it seems) mean that sentence \ref{n:JohnBald} is true.

Third, consider this sentence:
\begin{earg}
\setcounter{eargnum}{2}	
\item\label{n:GodParadox}	It's not the case that, if God exists, he answers evil prayers.
\end{earg}
Symbolizing this in TFL, we have `$\enot (G \eif E)$'. As we can see from the truth table, `$\enot (G \eif E)$' entails `$G$'. 
\begin{center}
\begin{tabular}{c c | d e e f c c}
$E$& $G$&  $\enot$& ($G$& $\eif$& $E$)& $\proves$& $G$\\ 
\hline
T& T&  \TTbf{F}&   T& T& T&\cm&    \TTbf{T}\Tstrut\\ 
T& F&  \TTbf{F}&   F& T& T&\cm&    \TTbf{F}\\ 
F& T&  \TTbf{T}&   T& F& F&\cm&    \TTbf{T}\\ 
F& F&  \TTbf{F}&   F& T& F&\cm&   \TTbf{F}\\ 
\end{tabular}
\end{center} 
So sentence \ref{n:GodParadox} seems to entail that God exists. But that's not what we expect. An atheist could believe that `It's not the case that, if God exists, he answers evil prayers' without accepting that God does, in fact, exist.

It might be that sentence \ref{n:GodParadox}, despite appearances, does not express what we mean. We can try rephrasing it this way: 
\begin{earg}
\setcounter{eargnum}{3}	
\item\label{n:GodParadox2} If God exists, he does not answer evil prayers.
\end{earg}
This we symbolize as `$G \eif \enot E$'. Now, as shown in the truth table on the left, `$G$' does not follow from this premise. (That is, the argument `$G \eif \enot E \proves G$' is invalid.) But, at the same time, from the premise `$\enot G$' (i.e., `God does not exist'), it follows that `if God exists, he answers evil prayers'.

\bigskip 
\noindent\begin{minipage}{.50\linewidth}
\begin{center}
\begin{tabular}{c c | d e e f c c} 
$E$& $G$&  ($G$& $\eif$& $\enot$& $E$)& \proves& $G$\\ 
\hline
T& T&    T& \TTbf{F}& F& T&\cm&    \TTbf{T}\Tstrut\\ 
T& F&    F& \TTbf{T}& F& T&\xm&    \TTbf{F}\\ 
F& T&    T& \TTbf{T}& T& F&\cm&    \TTbf{T}\\ 
F& F&    F& \TTbf{T}& T& F&\xm&    \TTbf{F}\\ 
\end{tabular}
\end{center} 
%\medskip
\end{minipage}
\begin{minipage}{.50\linewidth}
\begin{center}
\begin{tabular}{c c | df c d e f} 
$E$& $G$&  $\enot$& $G$& $\proves$& ($G$& $\eif$& $E$)\\ 
\hline
T& T&  \TTbf{F}& T&\cm&    T& \TTbf{T}& T\Tstrut\\   
T& F&  \TTbf{T}& F&\cm&    F& \TTbf{T}& T\\   
F& T&  \TTbf{F}& T&\cm&    T& \TTbf{F}& F\\   
F& F&  \TTbf{T}& F&\cm&    F& \TTbf{T}& F\\   
\end{tabular}
\end{center}  
\end{minipage}
\bigskip

(We can also put these final two points as follows. When `$G$' is false, `$G \eif \enot E$' is true, and so we don't have to be committed to the existence of God to accept that `If God exists, he does not answer evil prayers'. But if `$G$' is false, then `$G \eif E$'---i.e., `If God exists, he answers evil prayers'---is true.)
  
In different ways, these three examples illustrate some of the limitations of using a language like TFL that can only handle truth-functional connectives. These limitations give rise to some interesting questions in philosophical logic, however. The case of John's baldness (or non-baldness) raises the general question of what logic we should use when dealing with \emph{vague} discourse. The case of God answering evil prayers illustrates some of the \emph{paradoxes of material implication}. 

Part of the purpose of studying truth-functional propositional logic is to equip ourselves with the tools to explore these questions of philosophical logic. But we have to walk before we can run; and  so we have to become proficient using TFL before we can adequately discuss its limits and consider alternatives. 


%%%%%%%%%%%%%%%%%%%%%%%%%%%%%%%%%%%%%%%%%%%%%%
%%%%%%%%%%%%%%%%%%%%%%%%%%%%%%%%%%%%%%%%%%%%%%

%  exercises for `truth tables and validity'

%%%%%%%%%%%%%%%%%%%%%%%%%%%%%%%%%%%%%%%%%%%%%%
%%%%%%%%%%%%%%%%%%%%%%%%%%%%%%%%%%%%%%%%%%%%%%



%\practiceproblems
\section{Practice exercises}
\setcounter{ProbPart}{0}

\problempart
\label{pr.TT.valid}
Create a truth table for each argument and then determine if the argument is valid or invalid.
\begin{earg}
\item $A\eif A \proves A$\vspace{.5ex} %invalid
\item $A\eif(A\eand\enot A) \proves \enot A$\vspace{.5ex} %valid
\item $A\eor(B\eif A) \proves \enot A \eif \enot B$\vspace{.5ex} %valid
\item $A\eor B, B\eor C, \enot A \proves B \eand C$\vspace{.5ex} %invalid
\item $(B\eand A)\eif C, (C\eand A)\eif B \proves (C\eand B)\eif A$\vspace{.5ex} %invalid


\item $A\eif B$, $B \proves  A$\vspace{.5ex} %invalid
\item $A\eiff B$, $B\eiff C \proves A\eiff C$\vspace{.5ex} %valid
\item $A \eif B$, $A \eif C\proves B \eif C$\vspace{.5ex} %invalid. 
\item $A \eif B$, $B \eif A\proves A \eiff B$\vspace{.5ex} %valid. 


\item $A\eor\bigl[A\eif(A\eiff A)\bigr] \proves  A $\vspace{.5ex}%invalid
\item $A\eor B$, $B\eor C$, $\enot B \proves A \eand C$\vspace{.5ex} %valid
\item $A \eif B$, $\enot A\proves \enot B$ \vspace{.5ex}%invalid
\item $A$, $B\proves \enot(A\eif \enot B)$ \vspace{.5ex}%valid
\item $\enot(A \eand B)$, $A \eor B$, $A \eiff B\proves C$ \vspace{.5ex}%valid 
\end{earg}


\problempart
\begin{earg}
\item Suppose that $(\meta{A}\eand\meta{B})\eif\meta{C}$ is neither a tautology nor a contradiction. Is it possible to determine if $\meta{A}, \meta{B} \proves \meta{C}$ is valid or not? Explain.
\item Suppose that \meta{A} is a contradiction. Is $\meta{A}, \meta{B} \proves \meta{C}$ valid or invalid? Explain.
\item Suppose that \meta{C} is a tautology. Is $\meta{A}, \meta{B}\proves \meta{C}$ valid or invalid? Explain.
\end{earg}


%%%%%%%%%%%%%%%%%%%%%%%%%%%%%%%%%%%%%%%%%%%%%%
%%%%%%%%%%%%%%%%%%%%%%%%%%%%%%%%%%%%%%%%%%%%%%

%  answers for `truth tables and validity'

%%%%%%%%%%%%%%%%%%%%%%%%%%%%%%%%%%%%%%%%%%%%%%
%%%%%%%%%%%%%%%%%%%%%%%%%%%%%%%%%%%%%%%%%%%%%%

\section{Answers}
\setcounter{ProbPart}{0}


\problempart
\label{pr.TT.valid}
Use truth tables to determine whether each argument is valid or invalid.
\begin{earg}
\item $A\eif A \proves A$\\
This argument is invalid.
\myanswer{\begin{center}
\begin{tabular}{c | d e f   c   c}
$A$ &$A$&$\eif$&$A$& \proves   &$A$\\
\hline
 T & T & \TTbf{T} & T& \cm  & T\Tstrut\\
 F & F & \TTbf{T} & F& \xm  & F
 \end{tabular}
\end{center}}

\item $A\eif(A\eand\enot A) \proves \enot A$\\
This argument is valid.
\myanswer{\begin{center}
\begin{tabular}{c | d e e e e f   c  df}
$A$&$A$&$\eif$&$(A$&$\eand$&$\enot$&$A)$& \proves &$\enot$&$A$\\
\hline
 T & T & \TTbf{F} & T & F& F&T&\cm  &\TTbf{F}&T\Tstrut\\
 F & F & \TTbf{T} & F & F&T&F&\cm  &\TTbf{T}&F
\end{tabular}
\end{center}}

\item $A\eor(B\eif A) \proves \enot A \eif \enot B$\\
This argument is valid.
\myanswer{\begin{center}
\begin{tabular}{c c | d e e e f    c   d e e e f}
$A$ & $B$ & $A$&$\eor$&$(B$&$\eif$&$A)$& \proves &$\enot$&$A$&$\eif$&$\enot$&$B$\\
\hline
T & T & T & \TTbf{T} & T & T & T &\cm& F & T & \TTbf{T} & F & T\Tstrut\\
T & F & T & \TTbf{T} & F & T & T &\cm& F & T & \TTbf{T} & T & F \\
F & T & F & \TTbf{F} & T & F & F &\cm& T & F & \TTbf{F} & F & T \\
F & F & F & \TTbf{T} & F & T & F &\cm& T & F & \TTbf{T} & T & F
\end{tabular}
\end{center}}

\item $A\eor B, B\eor C, \enot A \proves B \eand C$\\
This argument is invalid.
\myanswer{\begin{center}
\begin{tabular}{c c c | d e f   d e f   d f   c  d e f}
$A$ & $B$ & $C$ & $A$&$\eor$&$B$,&$B$&$\eor$&$C$,&$\enot$&$A$& \proves &$B$&$\eand$&$C$\\
\hline
T & T & T & T & \TTbf{T} & T & T & \TTbf{T} & T & \TTbf{F} & T &\cm& T & \TTbf{T} & T\Tstrut\\
T & T & F & T & \TTbf{T} & T & T & \TTbf{T} & F & \TTbf{F} & T &\cm& T &\TTbf{F} & F \\
T & F & T & T & \TTbf{T} & F & F & \TTbf{T} & T & \TTbf{F} & T &\cm& F & \TTbf{F} & T \\
T & F & F & T & \TTbf{T} & F & F & \TTbf{F} & F & \TTbf{F} & T &\cm& F & \TTbf{F} & F\\\arrayrulecolor{light-gray}\hline
T & T & T & F & \TTbf{T} & T & T & \TTbf{T} & T & \TTbf{T} & F &\cm& T & \TTbf{T} & T\Tstrut\\
T & T & F & F & \TTbf{T} & T & T & \TTbf{T} & F & \TTbf{T} & F &\xm& T &\TTbf{F} & F \\
T & F & T & F & \TTbf{F} & F & F & \TTbf{T} & T & \TTbf{T} & F &\cm& F & \TTbf{F} & T \\
T & F & F & F & \TTbf{F} & F & F & \TTbf{F} & F & \TTbf{T} & F &\cm& F & \TTbf{F} & F
\end{tabular}
\end{center}}

\item $(B\eand A)\eif C, (C\eand A)\eif B \proves (C\eand B)\eif A$\\
This argument is invalid.
\myanswer{\begin{center}
\begin{tabular}{c c c | d e e e f   d e e e f  c  d e e e f}
$A$ & $B$ & $C$ & $(B$&$\eand$&$A)$&$\eif$&$C$,&$(C$&$\eand$&$A)$&$\eif$&$B$& \proves &$(C$&$\eand$&$ B)$&$\eif$&$A$\\
\hline
T & T & T & T & T & T & \TTbf{T} & T & T & T & T & \TTbf{T} & T &\cm& T & T & T & \TTbf{T} & T\Tstrut\\
T & T & F & T & T & T & \TTbf{F} & F & F & F & T & \TTbf{T} & T &\cm& F & F & T & \TTbf{T} & T\\
T & F & T & F & F & T & \TTbf{T} & T & T & T & T & \TTbf{F} & F &\cm& T & F & F & \TTbf{T} & T\\
T & F & F & F & F & T & \TTbf{T} & F & F & F & T & \TTbf{T} & F &\cm& F & F & F & \TTbf{T} & T\\\arrayrulecolor{light-gray}\hline
F & T & T & T & F & F & \TTbf{T} & T & T & F & F & \TTbf{T} & T &\xm& T & T & T & \TTbf{F} & F\Tstrut\\
F & T & F & T & F & F & \TTbf{T} & F & F & F & F & \TTbf{T} & T &\cm& F & F & T & \TTbf{T} & F\\
F & F & T & F & F & F & \TTbf{T} & T & T & F & F & \TTbf{T} & F &\cm& T & F & F & \TTbf{T} & F\\
F & F & F & F & F & F & \TTbf{T} & F & F & F & F & \TTbf{T} & F &\cm& F & F & F & \TTbf{T} & F
\end{tabular}
\end{center}}

\item $A\eif B$, $B \proves  A$ \hfill \myanswer{Invalid}
\item $A\eiff B$, $B\eiff C \proves A\eiff C$ \hfill \myanswer{Valid}
\item $A \eif B$, $A \eif C\proves B \eif C$ \hfill \myanswer{Invalid} 
\item $A \eif B$, $B \eif A\proves A \eiff B$ \hfill \myanswer{Valid} 

\item $A\eor\bigl[A\eif(A\eiff A)\bigr] \proves  A $\vspace{.5ex} \hfill \myanswer{Invalid}
\item $A\eor B$, $B\eor C$, $\enot B \proves A \eand C$\vspace{.5ex} \hfill \myanswer{Valid}
\item $A \eif B$, $\enot A\proves \enot B$ \vspace{.5ex} \hfill \myanswer{Invalid}
\item $A$, $B\proves \enot(A\eif \enot B)$ \vspace{.5ex} \hfill \myanswer{Valid}
\item $\enot(A \eand B)$, $A \eor B$, $A \eiff B\proves C$ \vspace{.5ex} \hfill \myanswer{Valid}
\end{earg}


\problempart
\begin{earg}
\item Suppose that $(\meta{A}\eand\meta{B})\eif\meta{C}$ is neither a tautology nor a contradiction. Is it possible to determine if $\meta{A}, \meta{B} \proves \meta{C}$ is valid or not?
\begin{ebullet}
\item[] \myanswer{Since the sentence $(\meta{A}\eand\meta{B})\eif\meta{C}$ is not a tautology, there is some line on which it is false. Since it is a conditional, on that line, \meta{A} and \meta{B} are true and \meta{C} is false. Hence, the argument, `$\meta{A}, \meta{B} \proves\meta{C}$', is invalid.}
\end{ebullet}

\item Suppose that \meta{A} is a contradiction. Is $\meta{A}, \meta{B} \proves \meta{C}$ valid or invalid?
\begin{ebullet}
\item[] \myanswer{Since \meta{A} is false on every line of a truth table, there is no line on which \meta{A} and \meta{B} are true and \meta{C} is false. Hence, the argument is  valid. (Although that would be kind of an odd argument since we know that one of the premises is a contradiction.)}
\end{ebullet}

\item Suppose that \meta{C} is a tautology. Is $\meta{A}, \meta{B} \proves \meta{C}$ valid or invalid?
\begin{ebullet}
\item[] \myanswer{Since \meta{C} is true on every line of a complete truth table, there is no line on which \meta{A} and \meta{B} are true and \meta{C} is false. Hence, the argument is valid.}
\end{ebullet}
\end{earg}




%%%%%%%%%%%%%%%%%%%%%%%%%%%%%%%%%%%%%%
%%%%%%%%%%%%%%%%%%%%%%%%%%%%%%%%%%%%%%

% Chapter 12

%%%%%%%%%%%%%%%%%%%%%%%%%%%%%%%%%%%%%%
%%%%%%%%%%%%%%%%%%%%%%%%%%%%%%%%%%%%%%

\chapter{Truth table shortcuts}
With practice, you will become adept at quickly filling out truth tables. There are, however, some shortcuts that will (1) save you some time and (2) reinforce the meaning the concepts that can be tested using truth tables. 

\section{Testing for validity}\label{test-valid}

As we said in \S\ref{s:tt-validity}, when we use truth tables to test for validity, we are checking for \emph{bad} lines: lines where the premises are all true and the conclusion is false. Consequently,
	\begin{earg}
		\item[\textbullet] Any line where the conclusion is true is not a bad line. 
		\item[\textbullet] Any line where some premise is false is not a bad line. 
	\end{earg}
Since \emph{all} we are doing is looking for bad lines, if we find a line where the conclusion is true, we do not need to evaluate anything else on that line. That line definitely isn't bad. Likewise, if we find a line where some premise is false, we do not need to evaluate anything else on that line. 

With this in mind, consider how we might investigate wither this argument is valid:
	$$\enot L \eif (J \eor L), \enot L \proves J$$
The first step is evaluating the conclusion. If we find that the conclusion is true on some line, then that is not a bad line, and so we can simply ignore the rest of the line.  \begin{center}
\begin{tabular}{c c|d e e e e f 		  d f   c  c}
$J$&$L$&\enot&$L$&\eif&$(J$&\eor&$L)$&\enot&$L$,&\proves&$J$\\
\hline
 T & T & &&&&&&&&\cm& {T}\Tstrut\\
 T & F & &&&&&&&&\cm& {T}\\
 F & T & &&?&&&&?&&?& {F}\\
 F & F & &&?&&&&?&&?& {F}
\end{tabular}
\end{center}
The blank spaces under $\enot L \eif (J \eor L)$ and $\enot L$  indicate that we are not going to bother doing any more investigation since the line is not bad. The question-marks indicate that we need to keep investigating. On those lines, it is possible that the premises are true and the conclusion is false. 

The easiest premise to evaluate is the second ($\enot L$), and so we do that next.
\begin{center}
\begin{tabular}{c c|d e e e e f    d f  c  c}
$J$&$L$&\enot&$L$&\eif&$(J$&\eor&$L)$,&\enot&$L$&\proves&$J$\\
\hline
 T & T & &&&&&&&&\cm& {T}\Tstrut\\
 T & F & &&&&&&&&\cm& {T}\\
 F & T & &&&&&&{F}&&\cm& {F}\\
 F & F & &&?&&&&{T}&&?& {F}
\end{tabular}
\end{center}
Now we see that we no longer need to consider the third line. It will not be a bad line, because at least one of the premises is false on that line, namely, $\enot L$. Finally, we complete the fourth line:
\begin{center}
\begin{tabular}{c c|d e e e e f    d f   c  c}
$J$&$L$&\enot&$L$&\eif&$(J$&\eor&$L)$,&\enot&$L$&\proves&$J$\\
\hline
 T & T & &&&&&&&&\cm& {T}\Tstrut\\
 T & F & &&&&&&&&\cm& {T}\\
 F & T & &&&&&&{F}& &\cm& {F}\\
 F & F & T &  & \TTbf{F} &  & F & & {T} & &\cm& {F}
\end{tabular}
\end{center}
Since the fourth line tells us that the first premise is false, the truth table has no bad lines. Hence, the argument is valid: any valuation for which all the premises are true is a valuation for which the conclusion is true.

Let us check whether the following argument is valid using the same method.
$$A\eor B, \enot (A\eand C), \enot (B \eand \enot D) \proves (\enot C\eor D)$$
Again, we first determine the truth value of the conclusion. Since this is a disjunction, it is true whenever either disjunct is true. We can speed things along by noting that the conclusion will be true whenever `$D$' is true. Then we only have to determine the truth value for `$\enot C$' on the lines where $D$ is false. 

Once we have the truth values for the conclusion, we can, as we did in the last example, ignore every line apart from the lines where the conclusion is false.
\begin{center}
\begin{tabular}[t]{c c c c | c   c   c  c  d e e f }
$A$ & $B$ & $C$ & $D$ & $A\eor B$, & $\enot (A\eand C)$, & $\enot (B\eand \enot D)$&\proves & $(\enot$ &$C$& $\eor$ & $D)$\\
\hline
T & T & T & T & & &			&\cm		&  &&  \TTbf{T} & \Tstrut\\
T & T & T & F & ? & ? & ? 	&?		& F & &  \circled{\TTbf{F}} & \\
T & T & F & T &  & &			&\cm  & & &  \TTbf{T} & \\
T & T & F & F &  &  &   		&\cm	& T & &  \TTbf{T} &\\\arrayrulecolor{light-gray}\hline
T & F & T & T &  &  &  		&\cm	& & &  \TTbf{T} &\Tstrut\\
T & F & T & F & ? & ? & ?  	&?	& F &  &  \circled{\TTbf{F}} &\\
T & F & F & T & & & 			&\cm	& & & \TTbf{T} &\\
T & F & F & F & & & 			&\cm	& T &  & \TTbf{T} & \\\arrayrulecolor{light-gray}\hline
F & T & T & T & & & 			&\cm	& & & \TTbf{T} & \Tstrut\\
F & T & T & F & ? & ? & ? 	&?	& F &  & \circled{\TTbf{F}} &\\
F & T & F & T & & &  			&\cm	& & & \TTbf{T} & \\
F & T & F & F & & & 			&\cm	&T & & \TTbf{T} & \\\arrayrulecolor{light-gray}\hline
F & F & T & T & & & 			&\cm	& & & \TTbf{T} & \Tstrut\\
F & F & T & F & ? & ? & ? 	&?	& F & & \circled{\TTbf{F}} & \\
F & F & F & T & & & 			&\cm	& & & \TTbf{T} & \\
F & F & F & F & & & 			&\cm	& T& & \TTbf{T} & \\
\end{tabular}
\end{center}
We must now evaluate the premises. The first premise is the simplest, and so we start there. Of the four lines where the conclusion is false, there are three where $A \eor B$ is true. So the truth values for the next premise have to be determined for those three lines. (The second premise is simpler to evaluate than the third, so it's next.) 

On those three lines, there is only one where the first two premise are true. With a little bit more work, we find that the third premise is false on that line. There is no line where the premises are true and the conclusion is false! The argument is valid.
\begin{center}
\begin{tabular}[t]{c c c c | d e f    d e e f    d e e e f    c    d e e f }
$A$ & $B$ & $C$ & $D$ & $A$ & $\eor$ & $B$, & $\enot$ & $(A$ &$\eand$ &$ C)$, & $\enot$ & $(B$ & $\eand$ & $\enot$ & $D)$&\proves & $(\enot$ &$C$& $\eor$ & $D)$\\
\hline
T & T & T & T & & && & && & && & & 								&\cm	& &  &  \TTbf{T} & \Tstrut\\ 
T & T & T & F & &\TTbf{T}& & \TTbf{F}& &T& & & & & & 		&\cm	& F & &  \circled{\TTbf{F}} & \\
T & T & F & T & & && & && & &&  & &   							&\cm	& & &  \TTbf{T} & \\
T & T & F & F & & && & && & &&  &  &   							&\cm	& T & &  \TTbf{T} & \\\arrayrulecolor{light-gray}\hline
T & F & T & T & & && & && & &&  &  &  							&\cm	& & &  \TTbf{T} & \Tstrut\\
T & F & T & F & &\TTbf{T}& &\TTbf{F}& &T& &  && & & 		&\cm	& F & & \circled{\TTbf{F}} & \\
T & F & F & T & & && & && & && & & 								&\cm	& & & \TTbf{T} & \\
T & F & F & F & & && & && & && & & 								&\cm	& T &  & \TTbf{T} & \\\arrayrulecolor{light-gray}\hline
F & T & T & T& & && & && & & & & & 								&\cm	& & & \TTbf{T} & \Tstrut\\
F & T & T & F & &\TTbf{T}& & \TTbf{T}& & F& & \TTbf{F}& & T& T&  			&\cm	& F &  & \circled{\TTbf{F}} & \\
F & T & F & T & & && & && & && & &  								&\cm	& & & \TTbf{T} & \\
F & T & F & F& & && & && & && & & 								&\cm	&T & & \TTbf{T} & \\\arrayrulecolor{light-gray}\hline
F & F & T & T & & && & && & && & & 								&\cm	& & & \TTbf{T} & \Tstrut\\
F & F & T & F & & \TTbf{F} & & & & & & &&  &  &  				&\cm	& F & & \circled{\TTbf{F}} & \\
F & F & F & T & & && & && & && & & 								&\cm	& & & \TTbf{T} & \\
F & F & F & F & & && & && & && & & 								&\cm	& T& & \TTbf{T} & \\
\end{tabular}
\end{center}
If we had used no shortcuts, we would have had to write 256 `T's or `F's on this table. Using shortcuts, we only had to write 37. We have saved ourselves a \emph{lot} of work.



\section{Partial truth tables }\label{s:PartialTruthTable}

In the previous section, we saw how an incomplete truth table---although one that still had all of the lines---could be enough to determine if an argument is valid or invalid. That's one method where we use less than the full truth table. Another is where we create a one line truth table. This is called a \define{partial truth table}.  

We can also use partial truth tables to determine if a sentence is not a tautology or is not a contradiction, and to determine if a set of sentences are not consistent or are consistent. 

\paragraph{Tautology} 
To show that a sentence is a tautology, we need to show that it is true on every valuation. That is to say, we need to know that it is true on every line of the truth table. To do that, we need a complete truth table. 

To show that a sentence is \emph{not} a tautology, however, we only need one line: a line on which the sentence is false. Therefore, in order to show that some sentence is not a tautology, it is enough to provide a single valuation---a single line of the truth table---that makes the sentence false. 

Suppose that we want to show that the sentence `$(U \eand T) \eif (S \eand W)$' is \emph{not} a tautology. We set up a \define{partial truth table}. We have only left space for one line, rather than 16, since we are only looking for one line on which the sentence is false. Let us suppose that the sentence is false
\begin{center}
\begin{tabular}{c c c c |d e e e e e f}
$S$&$T$&$U$&$W$&$(U$&\eand&$T)$&\eif    &$(S$&\eand&$W)$\\
\hline
   &   &   &   &    &   &    &\TTbf{F}&    &   &\Tstrut\\
\end{tabular}
\end{center}
The main logical operator of the sentence is a conditional. In order for the conditional to be false, the antecedent must be true and the consequent must be false. So we put those in the table.
\begin{center}
\begin{tabular}{c c c c |d e e e e e f}
$S$&$T$&$U$&$W$&$(U$&\eand&$T)$&\eif    &$(S$&\eand&$W)$\\
\hline
   &   &   &   &    &  T  &    &\TTbf{F}&    &   F &\Tstrut\\  
\end{tabular}
\end{center}
For the `$(U\eand T)$' to be true, both `$U$' and `$T$' must be true. Knowing that, we can set the truth values for these atomic sentences on the left side of the truth table.
\begin{center}
\begin{tabular}{c c c c|d e e e e e f}
$S$&$T$&$U$&$W$&$(U$&\eand&$T)$&\eif    &$(S$&\eand&$W)$\\
\hline
   & T & T &   &  T &  T  & T  &\TTbf{F}&    &   F &\Tstrut\\   
\end{tabular}
\end{center}
Now we just need to see if we can make `$(S\eand W)$' false, which requires at least one of `$S$' and `$W$' to be false. Since the truth values for `$S$' and `$W$' have not been set yet, we can make both `$S$' and `$W$' false if we want. With that, we finish the table in this way:
\begin{center}
\begin{tabular}{c c c c|d e e e e e f}
$S$&$T$&$U$&$W$&$(U$&\eand&$T)$&\eif    &$(S$&\eand&$W)$\\
\hline
 F & T & T & F &  T &  T  & T  &\TTbf{F}&  F &   F & F\Tstrut\\  
\end{tabular}
\end{center}
We now have a partial truth table that shows that `$(U \eand T) \eif (S \eand W)$' is not a tautology. Put otherwise, we have shown that there is a valuation that makes `$(U \eand T) \eif (S \eand W)$' false, namely, the valuation where `$S$' is false, `$T$' is true, `$U$' is true and `$W$' is false (which would be line 10 in a full truth table). 

To be clear, we use this method in an \textit{attempt} to show that a sentence is not a tautology. If a sentence is a tautology, then we won't be able to find an assignment of `true' and `false' for every sentence letter that makes the full sentence false.

\paragraph{Contradiction}
Showing that a sentence is a contradiction requires a complete truth table: we need to show that the sentence is false on every line of the truth table. 

On the other hand, to show that a sentence is \emph{not} a contradiction, all we need to do is find a valuation that makes the sentence true, and so a single line of a truth table will suffice. We can illustrate this with the same example.
\begin{center}
\begin{tabular}{c c c c|d e e e e e f}
$S$&$T$&$U$&$W$&$(U$&\eand&$T)$&\eif    &$(S$&\eand&$W)$\\
\hline
  &  &  &  &   &   &   &\TTbf{T}&  &  &\Tstrut\\
\end{tabular}
\end{center}
One way for this sentence to be true is for the antecedent to be false. Since the antecedent is a conjunction, we can just make one of the conjuncts false. Let's make `$U$' false. Then, we can assign whatever truth value we like to the other atomic sentences. With $S =$ \textit{false}, $T =$ \textit{true}, $U =$ \textit{false}, and $W =$ \textit{false}, we have shown that `$(U \eand T) \eif (S \eand W)$' is not contradiction.
\begin{center}
\begin{tabular}{c c c c|d e e e e e f}
$S$&$T$&$U$&$W$&$(U$&\eand&$T)$&\eif    &$(S$&\eand&$W)$\\
\hline
 F & T & F & F &  F &  F  & T  &\TTbf{T}&  F &   F & F\Tstrut\\
\end{tabular}
\end{center}

\paragraph{Equivalent}
To show that two sentences are logically equivalent, we must show that the sentences have the same truth value on every valuation. So this requires a  complete truth table.

To show that two sentences are \emph{not} logically equivalent, we only need to show that there is a valuation on which they have different truth values. So this requires only a one-line partial truth table. We make the table so that one sentence is true and the other false.

\paragraph{Consistent}
To show that some sentences are jointly consistent, we must show that there is a valuation that makes all of the sentences true. This requires only a partial truth table with a single line. 

To show that some sentences are jointly inconsistent, we must show that there is no valuation which makes all of the sentence true. So this requires a complete truth table: You must show that on every row of the table at least one of the sentences is false.

\begin{table*}\centering\sffamily\footnotesize
\ra{1.25}
\begin{tabular}{@{}l l l@{}}\toprule
\textsc{To check} & \textsc{that it is} & \textsc{that it is not}\\\midrule
tautology & complete & one-line partial \\
contradiction &  complete & one-line partial \\
equivalent & complete  & one-line partial \\
consistent & one-line partial & complete \\
valid & complete & one-line partial \\
\bottomrule
\end{tabular}
\caption{The kind of truth table required to check each of these logical notions.}\label{table.CompleteVsPartial}
\end{table*}

\paragraph{Valid}
To show that an argument is valid, we must show that there is no valuation that makes all of the premises true and the conclusion false. This requires a truth table with all of the requisite lines, although we can take the shortcuts that were described in the first section of this chapter.  

To show that argument is invalid, we must show that there is a valuation that makes all of the premises true and the conclusion false. So this requires only a one-line partial truth table where all of the premises are true and the conclusion is false.


%%%%%%%%%%%%%%%%%%%%%%%%%%%%%%%%%%%%
% Exercises for Partial truth tables chapter
%%%%%%%%%%%%%%%%%%%%%%%%%%%%%%%%%%%%

\section{Practice exercises}
\setcounter{ProbPart}{0}


\problempart
\label{pr.TT.equiv3}
If it is possible, use a partial truth table to show that the pair of sentences are \textbf{not equivalent}. If it can't be shown that they are not equivalent, then create a full truth table showing that they are equivalent. 
\begin{earg}
\item $A$, $\enot A$ %No
\item $A$, $A \eor A$ %Yes
\item $A\eif A$, $A \eiff A$ %Yes
\item $A \eor \enot B$, $A\eif B$ %No
\item $A \eand \enot A$, $\enot B \eiff B$ %Yes
\item $\enot(A \eand B)$, $\enot A \eor \enot B$ %Yes
\item $\enot(A \eif B)$, $\enot A \eif \enot B$ %No
\item $(A \eif B)$, $(\enot B \eif \enot A)$ %Yes
\item $ ((U \eif (X \eor X)) \eor U)$ and $\enot (X \eand (X \eand U))$
\item $ ((C \eand (N \eiff C)) \eiff C)$ and $(\enot \enot \enot N \eif C)$
\item $[(A \eor B) \eand C]$ and $[A \eor (B \eand C)]$
\item $((L \eand C) \eand I)$ and $L \eor C$
\end{earg}


\problempart
\label{pr.TT.consistent4}
If it is possible, use a partial truth table to show that the set of sentences are \textbf{consistent}. If it can't be shown that they are consistent, then create a full truth table showing that they are not consistent.
\begin{earg}
\item $A \eand B$, $C\eif \enot B$, $C$ %inconsistent
\item $A\eif B$, $B\eif C$, $A$, $\enot C$ %inconsistent
\item $A \eor B$, $B\eor C$, $C\eif \enot A$ %consistent
\item $A$, $B$, $C$, $\enot D$, $\enot E$, $F$ %consistent
\item $A \eand (B \eor C)$, $\enot(A \eand C)$, $\enot(B \eand C)$ %consistent
\item $A \eif B$, $B \eif C$, $\enot(A \eif C)$ %inconsistent
\item $A\eif A$, $\enot A \eif \enot A$, $A\eand A$, $A\eor A$ %consistent
\item $A \eif \enot A$, $\enot A \eif A$%inconsistent. 
\item $A\eor B$, $A\eif C$, $B\eif C$ %consistent
\item $A \eor B$, $A \eif C$, $B \eif C$, $\enot C$ %	Inconsistent
\item $B\eand(C\eor A)$, $A\eif B$, $\enot(B\eor C)$  %inconsistent
\item $(A \eiff B) \eif B$,  $B \eif \enot (A \eiff B)$, $A \eor B$  %	Consistent
\item $A\eiff(B\eor C)$, $C\eif \enot A$, $A\eif \enot B$ %consistent
\item  $A \eiff B$,  $\enot B \eor \enot A$,  $A \eif  B$ % Consistent
\item $A \eiff B$, $A \eif C$, $B \eif D$, $\enot(C \eor D)$ %consitent
\item $\enot (A \eand \enot B)$,  $B \eif \enot A$, $\enot B$   %Consistent
\end{earg}


\problempart
\label{pr.TT.valid4}
If it is possible, use a partial truth table to show that the argument is \textbf{invalid}. If it can't be shown that the argument is invalid, then, using the shortcuts explained in \S \ref{test-valid}, create a full truth table showing that it is valid.
\begin{earg}
\item $A\eor\bigl[A\eif(A\eiff A)\bigr] \proves A$ %invalid
\item $A\eiff\enot(B\eiff A) \proves A$ %invalid
\item $A\eif B, B \proves A$ %invalid
\item $A\eor B, B\eor C, \enot B \proves A \eand C$ %valid
\item $A\eiff B, B\eiff C \proves A\eiff C$ %valid
\item $A\eif(A\eand\enot A)\proves \enot A$% valid
\item $A \eor B$, $A \eif B$, $B \eif A \proves  A \eiff B$  % Valid
\item $A\eor(B\eif A)\proves \enot A \eif \enot B$ %valid
\item $A \eor B$, $A \eif B$, $ B \eif A \proves  A \eand B$ %valid
\item $(B\eand A)\eif C$, $(C\eand A)\eif B\proves (C\eand B)\eif A$ % invalid
\item $\enot (\enot A \eor \enot B)$, $A \eif \enot C \proves  A \eif (B \eif C)$ % invalid.
\item $A \eand (B \eif C)$, $\enot C \eand (\enot B \eif \enot A)\proves C \eand \enot C$ % valid
\item $A \eand B$, $\enot A \eif \enot C$, $B \eif \enot D \proves  A \eor B$ % Invalid
\item $A \eif B\proves (A \eand B) \eor (\enot A \eand \enot B)$ % invalid
\item $\enot A \eif B$,$ \enot B \eif C $,$ \enot C \eif A \proves  \enot A \eif (\enot B \eor \enot C) $% Invalid
\item $A\eiff\enot(B\eiff A)\proves A$ % invalid
\item $A\eor B$, $B\eor C$, $\enot A\proves B \eand C$ % invalid
\item $A \eif C$, $E \eif (D \eor B)$, $B \eif \enot D\proves (A \eor C) \eor (B \eif (E \eand D))$ % invalid
\item $A \eor B$, $C \eif A$, $C \eif B\proves A \eif (B \eif C)$ % invalid
\item $A \eif B$, $\enot B \eor A\proves A \eiff B$ % valid
\end{earg}

\problempart
\label{pr.TT.TTorC3}
If it is possible, use a partial truth table to show that the sentence is \textbf{not a tautology} or \textbf{not a contradiction}. If it can't be shown that it is not a tautology or not a contradiction, then give a full truth table showing that the sentence is a tautology, contradiction, or contingent. (Note that if the sentence is not a tautology \textit{and} not a contradiction, then it is contingent.) 

% truth tables in LaTeX generated by http://www.curtisbright.com/logic/. Be sure to give him a shout out.

\begin{earg}
\item  $A \eif \enot A$ \vspace{.5ex}							

%{\color{red}
%$
%\begin{array}{c|cccc}
%A&A&\eif&\enot&A\\\hline
%T&T&\mathbf{F}&F&T\\
%F&F&\mathbf{T}&T&F
%\end{array}
%$ 
%
%Contingent	 \vspace{6pt}
%}
%	T letter, 2 connectives
\item $A \eif (A \eand (A \eor B))$ \vspace{.5ex}	

%{\color{red}
%$
%\begin{array}{cc|ccc@{}ccc@{}ccc@{}c@{}c}
%A&B&A&\eif&(&A&\eand&(&A&\eor&B&)&)\\\hline
%T&T&T&\mathbf{T}&&T&T&&T&T&T&&\\
%T&F&T&\mathbf{T}&&T&T&&T&T&F&&\\
%F&T&F&\mathbf{T}&&F&F&&F&T&T&&\\
%F&F&F&\mathbf{T}&&F&F&&F&F&F&&
%\end{array}
%$
%
%Tautology \vspace{6pt}
%}
%			2 letters, 3 connectives

\item $(A \eif B) \eiff (B \eif A)$ 	\vspace{.5ex}				%
%
%{\color{red}
%$
%\begin{array}{cc|c@{}ccc@{}ccc@{}ccc@{}c}
%a&b&(&a&\rightarrow&b&)&\leftrightarrow&(&b&\rightarrow&a&)\\\hline
%T&T&&T&T&T&&\mathbf{T}&&T&T&T&\\
%T&F&&T&F&F&&\mathbf{F}&&F&T&T&\\
%F&T&&F&T&T&&\mathbf{F}&&T&F&F&\\
%F&F&&F&T&F&&\mathbf{T}&&F&T&F&
%\end{array}
%$
%
%Contingent \vspace{6pt}
%
%}
%		2 letters, 3 connectives

\item $A \eif \enot(A \eand (A \eor B)) $	\vspace{.5ex}	

%{\color{red}
%$
%\begin{array}{cc|cccc@{}ccc@{}ccc@{}c@{}c}
%a&b&a&\rightarrow&\enot&(&a&\eand&(&a&\eor&b&)&)\\\hline
%T&T&T&\mathbf{F}&F&&T&T&&T&T&T&&\\
%T&F&T&\mathbf{F}&F&&T&T&&T&T&F&&\\
%F&T&F&\mathbf{T}&T&&F&F&&F&T&T&&\\
%F&F&F&\mathbf{T}&T&&F&F&&F&F&F&&
%\end{array}
%$
%
%Contingent	\vspace{6pt}
%
%}
%
% 2 letters, 4 connectives

\item $\enot B \eif [(\enot A \eand A) \eor B]$\vspace{.5ex} 

%{\color{red}
%$
%\begin{array}{cc|cccc@{}c@{}cccc@{}ccc@{}c}
%a&b&\enot&b&\rightarrow&(&(&\enot&a&\eand&a&)&\eor&b&)\\\hline
%T&T&F&T&\mathbf{T}&&&F&T&F&T&&T&T&\\
%T&F&T&F&\mathbf{F}&&&F&T&F&T&&F&F&\\
%F&T&F&T&\mathbf{T}&&&T&F&F&F&&T&T&\\
%F&F&T&F&\mathbf{F}&&&T&F&F&F&&F&F&
%\end{array}
%$
%Contingent	 \vspace{6pt}
%
%}
%	2 letters, 5 connectives

\item $\enot(A \eor B) \eiff (\enot A \eand \enot B)$ \vspace{.5ex}

%{\color{red}
%$
%\begin{array}{cc|cc@{}ccc@{}ccc@{}ccccc@{}c}
%a&b&\enot&(&a&\eor&b&)&\leftrightarrow&(&\enot&a&\eand&\enot&b&)\\\hline
%T&T&F&&T&T&T&&\mathbf{T}&&F&T&F&F&T&\\
%T&F&F&&T&T&F&&\mathbf{T}&&F&T&F&T&F&\\
%F&T&F&&F&T&T&&\mathbf{T}&&T&F&F&F&T&\\
%F&F&T&&F&F&F&&\mathbf{T}&&T&F&T&T&F&
%\end{array}
%$
%
%Tautology \vspace{6pt}
%}
%2 letters, 6 connectives

\item $[(A \eand B) \eand C] \eif B$\vspace{.5ex}							
%
%{\color{red}
%$
%\begin{array}{ccc|c@{}c@{}ccc@{}ccc@{}ccc}
%a&b&c&(&(&a&\eand&b&)&\eand&c&)&\rightarrow&b\\\hline
%T&T&T&&&T&T&T&&T&T&&\mathbf{T}&T\\
%T&T&F&&&T&T&T&&F&F&&\mathbf{T}&T\\
%T&F&T&&&T&F&F&&F&T&&\mathbf{T}&F\\
%T&F&F&&&T&F&F&&F&F&&\mathbf{T}&F\\
%F&T&T&&&F&F&T&&F&T&&\mathbf{T}&T\\
%F&T&F&&&F&F&T&&F&F&&\mathbf{T}&T\\
%F&F&T&&&F&F&F&&F&T&&\mathbf{T}&F\\
%F&F&F&&&F&F&F&&F&F&&\mathbf{T}&F
%\end{array}
%$
%
%Tautology \vspace{6pt}
%}
%
%3 letters, 3 connectives

\item $\enot\bigl[(C\eor A) \eor B\bigr]$\vspace{.5ex} 						
%
%{\color{red}
%$
%\begin{array}{ccc|cc@{}c@{}ccc@{}ccc@{}c}
%a&b&c&\enot&(&(&c&\eor&a&)&\eor&b&)\\\hline
%T&T&T&\mathbf{F}&&&T&T&T&&T&T&\\
%T&T&F&\mathbf{F}&&&F&T&T&&T&T&\\
%T&F&T&\mathbf{F}&&&T&T&T&&T&F&\\
%T&F&F&\mathbf{F}&&&F&T&T&&T&F&\\
%F&T&T&\mathbf{F}&&&T&T&F&&T&T&\\
%F&T&F&\mathbf{F}&&&F&F&F&&T&T&\\
%F&F&T&\mathbf{F}&&&T&T&F&&T&F&\\
%F&F&F&\mathbf{T}&&&F&F&F&&F&F&
%\end{array}
%$
%
%Contingent \vspace{6pt}
%
%}
%	 	3 letters, 3 connectives

\item $\bigl[(A\eand B) \eand\enot(A\eand B)\bigr] \eand C$ \vspace{.5ex}	
%
%{\color{red}
%$
%\begin{array}{ccc|c@{}c@{}ccc@{}cccc@{}ccc@{}c@{}ccc}
%a&b&c&(&(&a&\eand&b&)&\eand&\enot&(&a&\eand&b&)&)&\eand&c\\\hline
%T&T&T&&&T&T&T&&F&F&&T&T&T&&&\mathbf{F}&T\\
%T&T&F&&&T&T&T&&F&F&&T&T&T&&&\mathbf{F}&F\\
%T&F&T&&&T&F&F&&F&T&&T&F&F&&&\mathbf{F}&T\\
%T&F&F&&&T&F&F&&F&T&&T&F&F&&&\mathbf{F}&F\\
%F&T&T&&&F&F&T&&F&T&&F&F&T&&&\mathbf{F}&T\\
%F&T&F&&&F&F&T&&F&T&&F&F&T&&&\mathbf{F}&F\\
%F&F&T&&&F&F&F&&F&T&&F&F&F&&&\mathbf{F}&T\\
%F&F&F&&&F&F&F&&F&T&&F&F&F&&&\mathbf{F}&F
%\end{array}
%$
%
%Contradiction \vspace{6pt}
%
%}
%
%% 	3 letters, 5 connectives
%
\item $(A \eand B) ]\eif[(A \eand C) \eor (B \eand D)]$ \vspace{.5ex}		
%
%{\color{red}
%$
%\begin{array}{cccc|c@{}c@{}ccc@{}c@{}ccc@{}c@{}ccc@{}ccc@{}ccc@{}c@{}c}
%a&b&c&d&(&(&a&\eand&b&)&)&\eif&(&(&a&\eand&c&)&\eor&(&b&\eand&d&)&)\\\hline
%T&T&T&T&&&T&T&T&&&\mathbf{T}&&&T&T&T&&T&&T&T&T&&\\
%T&T&F&F&&&T&T&T&&&\mathbf{F}&&&T&F&F&&F&&T&F&F&&\\
%\end{array}
%$
%
%Contingent \vspace{6pt}
%}
%
%	4 letters, 5 connectives

\item  $\enot (A \eor A)$\vspace{.5ex}							%	Contradiction		1 letter, 2 connectives
\item $(A \eif B) \eor (B \eif A)$\vspace{.5ex}					%	Tautology			2 letters, 2 connectives
\item $[(A \eif B) \eif A] \eif A$\vspace{.5ex}					%	Tautology			2 letters, 3 connectives
\item $\enot[( A \eif B) \eor (B \eif A)]$\vspace{.5ex}			%	Contradiction		2 letters, 4 connectives
\item $(A \eand B) \eor (A \eor B)$\vspace{.5ex} 				%	Contingent		2 letters, 5 connectives
\item $\enot(A\eand B) \eiff A$\vspace{.5ex} 					%contingent			2 letters, 3 connectives
\item $A\eif(B\eor C)$\vspace{.5ex} 							%contingent			3 letters, 2 connectives
\item $(A \eand\enot A) \eif (B \eor C)$\vspace{.5ex} 			%tautology			3 letters, 4 connectives 
\item $(B\eand D) \eiff [A \eiff(A \eor C)]$\vspace{.5ex}			%contingent			4 letters, 4 connectives
\item $\enot[(A \eif B) \eor (C \eif D)]$\vspace{.5ex} 			% Contingent. 		4 letters, 4 connectives
\end{earg}




