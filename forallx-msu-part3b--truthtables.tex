\graphicspath{{figures--tt/}}

%%%%%%%%%%%%%%%%% CHAPTER 11

\chapter{Entailment \& validity}\label{c:tt-validity}

\section{Entailment}\label{s:tt-entailment}

Having examined the logical relations between two sentences in \S\ref{equivalence--tt} and \S\ref{consistency--tt}, we can now go a step further and consider the relationship between the premises and the conclusion of an argument. This begins with \define{entailment}.

\factoidbox{
		The sentences $\meta{A}_1, \meta{A}_2, \ldots, \meta{A}_n$ \define{entail} the sentence $\meta{C}$ if there is no valuation of the atomic sentences that makes all of $\meta{A}_1, \meta{A}_2, \ldots, \meta{A}_n$ true and $\meta{C}$ false.
	}
 
Entailment is easy to check with a truth table. Do `$\enot L \eif (M \eor L)$' and `$\enot L$' entail `$M$'? To find out, we check whether there is any valuation that makes both `$\enot L \eif (M \eor L)$' and `$\enot L$' true while making `$M$' false.  
\begin{center}
\begin{tabular}{c c|d e e e e f|d f|| c}
$M$&$L$&\enot&$L$&\eif&$(M$&\eor&$L)$&\enot&$L$&$M$\\
\hline
%J   L   -   L      ->     (J   v   L)
 T & T & F & T & \TTbf{T} & T & T & T & \TTbf{F} & T & \TTbf{T}\Tstrut\\
 T & F & T & F & \circled{\TTbf{T}} & T & T & F & \circled{\TTbf{T}} & F & \TTbf{T}\\
 F & T & F & T & \TTbf{T} & F & T & T & \TTbf{F} & T & \TTbf{F}\\
 F & F & T & F & \TTbf{F} & F & F & F & \TTbf{T} & F & \TTbf{F}
\end{tabular}
\end{center}
There is only one row where both `$\enot L \eif (M \eor L)$' and `$\enot L$' are true---the second row---and so that is the only row that concerns us. On that row, `$M$' is also true. Hence, `$\enot L \eif (M \eor L)$' and `$\enot L$' entail `$M$'. 

Next is this important observation:
	\factoidbox{
		If $\meta{A}_1, \meta{A}_2, \ldots, \meta{A}_n$ entail $\meta{C}$, then $\meta{A}_1, \meta{A}_2, \ldots, \meta{A}_n \therefore \meta{C}$ is valid.
	}

Just to remind ourselves, an argument is valid when it is the case that if the premises are true, then the conclusion has to be true. A different but equivalent way of wording this definition will be more useful to us here, though. 
\factoidbox{
		An argument is \define{valid} if and only if it is impossible for all of the premises to be true and the conclusion false.
	}

Here's why entailment equals validity. If $\meta{A}_1, \meta{A}_2, \ldots, \meta{A}_n$ entail $\meta{C}$, then there is no valuation that makes all of $\meta{A}_1, \meta{A}_2, \ldots, \meta{A}_n$ true while making $\meta{C}$ false. This means that it is \emph{impossible} for $\meta{A}_1, \meta{A}_2, \ldots, \meta{A}_n$ to be true and $\meta{C}$ to be false. And that is just what it takes for an argument, with premises $\meta{A}_1, \meta{A}_2, \ldots, \meta{A}_n$ and conclusion $\meta{C}$, to be valid!

In short, we have a way to test whether an argument in English is valid. First, we symbolize the premises and conclusion in TFL. Then we test for entailment using truth tables. 

\section{Validity}\label{s:tt-validity}

When using a truth table to determine if an argument is valid, the premise or premises are listed first, followed by the conclusion. (In the example in \S\ref{s:tt-entailment}, `$\enot L \eif (M \eor L)$' and `$\enot L$' are the premises. $M$ is the conclusion.) We will also add the turnstile symbol, \proves, between the premise or premises and the conclusion, and give it a column in the truth table. Once the truth table is completed, we check for lines that violate the definition of a valid argument. We'll call lines that violate that definition \textit{bad lines}. 
	\begin{earg}
		\item[(1)] Any line where all of the premises are true and the conclusion is false \textbf{is a bad line}.
		\item[(2)] Any line where all of the premises are true and the conclusion is true \textbf{is a good line}.
		\item[(3)] Moreover, any line where the conclusion is true \textbf{cannot} be a bad line. (So, whatever the case may be with the premises, it's a good line.) 
		\item[(4)] And any line where at least one premise is false \textbf{cannot} be a bad line. (So, whatever the case may be with the other premises and the 			 								conclusion, it's a good line.) 
	\end{earg}

Let's look at the truth table for an argument with one small (but significant) change: $\enot L \eif (M \eor L)$, $\enot L \therefore \enot M$. The premises are the same, but now the conclusion is $\enot M$ instead of $M$. Here is the truth table:
\begin{center}
\begin{tabular}{c c|d e e e e f|d f| c | c}
$M$&$L$&\enot&$L$&\eif&$(M$&\eor&$L)$&\enot&$L$		&\proves		& $\enot M$\\
\hline
 T & T & F & T & \TTbf{T} & T & T & T 	& \TTbf{F} 	& T 		&	& \TTbf{F}\Tstrut\\
 T & F & T & F & \TTbf{T} & T & T & F 	& \TTbf{T} 	& F		&	& 	\TTbf{F}\\
 F & T & F & T & \TTbf{T} & F & T & T 	& \TTbf{F} 	& T 		&	& \TTbf{T}\\
 F & F & T & F & \TTbf{F} & F & F & F 	& \TTbf{T} 	& F 		&	& \TTbf{T}
\end{tabular}
\end{center}
The truth values for the premises are the same, and the truth values for the conclusion have, on each line, flipped from T to F or vice versa. Now, when we evaluate each line, what do we find? As before, on lines 1, 3, and 4, one of the premises is false, and so they are not bad lines. \textbf{On line 2, the premises are true and the conclusion is false. That's a bad line!} 
\begin{center}
\begin{tabular}{c c|d e e e e f|d f| c | c}
$M$&$L$&\enot&$L$&\eif&$(M$&\eor&$L)$&\enot&$L$	&\proves				& $\enot M$\\
\hline
 T & T & F & T & \TTbf{T} & T & T & T & \TTbf{F} & T 		&	\checkmark					& \TTbf{F}\Tstrut\\
 T & F & T & F & \circled{\TTbf{T}} & T & T & F & \circled{\TTbf{T}} & F 	& \ding{55}	& \circled{\TTbf{F}}\\
 F & T & F & T & \TTbf{T} & F & T & T & \TTbf{F} & T 		&	\checkmark					& \TTbf{T}\\
 F & F & T & F & \TTbf{F} & F & F & F & \TTbf{T} & F 		&	\checkmark					& \TTbf{T}
\end{tabular}
\end{center}
This means that $\enot L \eif (M \eor L)$ and $\enot L$ do not entail $\enot M$ and the argument `$\enot L \eif (M \eor L)$, $\enot L \therefore \enot M$' is not valid.  


\subsection{the turnstile}

As we mentioned in the previous section, the symbol `\proves' is called the \textit{turnstile}. Like the metavariables `$\meta{A}, \meta{B}, \meta{C}, \meta{D}, \ldots$', `\proves' is not a symbol of TFL. Rather, it is a symbol of our metalanguage, augmented English (recall the difference between object language and metalanguage from \S\ref{s:Metalanguage}). The purpose of the turnstile is to separate the sentences that are the premises of an argument from the sentence that is the conclusion, and it can be read as \textit{therefore}.


\subsection{checking for validity}

This section has some examples of using truth tables to determine whether an argument is valid. As a reminder, the definition of valid is given in \S\ref{s:tt-entailment}, and we can also use 1 - 4 on p.~\pageref{s:tt-validity} (which are consequences of the definition). We will begin with arguments that have only one premise and then do some with multiple premises.

\begin{earg}
	\item[\ex{1P-1}] The argument in the first truth table is \textbf{$P \eand Q \proves Q$}. The premise, `$P \eand Q$', is only true on line 1. Since it is false on lines 2 - 4, we know that those are good lines. (See guideline 4.) On line 1, `$P \eand Q$' is true and the conclusion, `$Q$', is true, and so that is also a good line. (See guideline 2.) Since every line is a good line, this argument is valid.
\begin{center}
\begin{tabular}{c c|d e f| c | c}
$P$& $Q$& 	$P$& $\eand$& $Q$& $\proves$& $Q$\\
\hline
 T & T 	&   T& \TTbf{T} & T & \checkmark & \TTbf{T}\Tstrut\\
 T & F 	&   T& \TTbf{F} & F & \checkmark & \TTbf{F}\\
 F & T 	&   F& \TTbf{F} & T & \checkmark & \TTbf{T}\\
 F & F 	&   F& \TTbf{F} & F & \checkmark & \TTbf{F} 
\end{tabular}
\end{center}

\item[\ex{1P-2}]
In this argument, the premise is false on lines 1 - 3, and so we know that those have to be good lines. On line 4, the premise is true and the conclusion is false, which means that line 4 is a bad line. (See guideline 1.) Since it has at least one bad line, this argument is not valid. 
\begin{center}
\begin{tabular}{c c|d e e f| c |d e e f}
$P$& $Q$& 	$\enot$& $(P$& $\eor$& $Q)$& $\proves$& $\enot$& $P$& $\eand$& $Q$\\
\hline
 T & T 	&   \TTbf{F} & T & T& T& 	\checkmark & F& T& \TTbf{F}& T\Tstrut\\
 T & F 	&   \TTbf{F} & T & T& F&	\checkmark & F& T& \TTbf{F}& F\\
 F & T 	&   \TTbf{F} & F & T& T&	\checkmark & T& F& \TTbf{T}& T\\
 F & F 	&   \TTbf{T} & F & F& F&	\ding{55} & T& F& \TTbf{F}& F
\end{tabular}
\end{center}

\item[\ex{2P-1}] This argument contains two premises, `$P \eif Q$' and `$\enot Q$'. Since both premise are not true on lines 1, 2, and 3, those are all good lines. Both premises are true on line 4, and the conclusion is true on that line, and so that is a good line. Since every line is a good line, this argument is valid.
\begin{center}
\begin{tabular}{c c|d e f| d f | c |d f}
$P$ & $Q$ & $P$ & $\eif$ & $Q$ & $\enot$ & $Q$ & \proves & $\enot$ &$P$\\ 
\hline
T & T &   T &   \TTbf{T} &T   & \TTbf{F} &T & \checkmark &\TTbf{ F } &T\Tstrut\\ 
T & F &   T &   \TTbf{F} &F   & \TTbf{T} &F & \checkmark & \TTbf{F } &T\\ 
F & T &   F &   \TTbf{T} &T   & \TTbf{F} &T & \checkmark & \TTbf{T } &F\\ 
F & F &   F &   \TTbf{T} &F   & \TTbf{T} &F & \checkmark & \TTbf{T } &F\\ 
\end{tabular}
\end{center}


\item[\ex{2P-2}]
Since the second premise is false on line 1 and the first premise is false on line 2, those lines have to be good lines. On line 3, both of the premises are true and the conclusion is false. That's a bad line. And then the same is also the case on line 4, and so that is a bad line also. Since two of the lines in this truth table are bad lines, the argument is invalid.
\begin{center}
\begin{tabular}{c c|d e f 	 d e e f 	   c 	  c }
$P$ &$Q$ 	&$P$ & $\eif$ &$Q$,  	& $P$ & $\eif$ & $\enot$ &$Q$ & \proves	& $P$\\ 
\hline
T &T   &T &\TTbf{T} &T   &T &\TTbf{F} &F  &T & \checkmark &\TTbf{T}\Tstrut\\ 
T &F   &T &\TTbf{F} &F    &T &\TTbf{T} &T &F  & \checkmark &\TTbf{T}\\ 
F &T   &F &\TTbf{T} &T    &F &\TTbf{T} &F &T  & \ding{55} &\TTbf{F}\\ 
F &F   &F &\TTbf{T} &F    &F &\TTbf{T} &T &F  & \ding{55} &\TTbf{F}\\ 
\end{tabular}
\end{center}

\item[\ex{3P-1}] Here we have three premises. One of the premises is false on each of lines 1, 2, 4, 5, 7, and 8, and so all of those have to be good lines. On line 3, all of the premises are true and the conclusion is true, and so that is a good line. On line 6, all of the premises are true but the conclusion is false, and so that is a bad line. Since one of the lines is a bad line, this argument is invalid. 
\begin{center}
\begin{tabular}{c c c | d e f     d e f		 d e e f 	   c 	  c }
$P$& $Q$& $R$&  $P$& 	$\eor$& 	$Q$,&   $P$		&$\eif$	&$R$,		&$Q$		&$\eif$	&$\enot$	&$R$	&$\proves$& $R$\\ 
\hline
T& T& T &   T &\TTbf{T}& T   &   T &\TTbf{T}& T   &   T& \TTbf{F}& F& T&\checkmark   & \TTbf{T}\Tstrut\\ 
T& T& F &   T &\TTbf{T}& T   &   T &\TTbf{F}& F   &   T& \TTbf{T}& T& F&\checkmark   & \TTbf{F}\\ 
T& F& T &   T &\TTbf{T}& F   &   T &\TTbf{T}& T   &   F& \TTbf{T}& F& T&\checkmark   & \TTbf{T}\\ 
T& F& F &   T &\TTbf{T}& F   &   T &\TTbf{F}& F   &   F& \TTbf{T}& T& F &\checkmark  & \TTbf{F}\\ 
F& T& T &   F &\TTbf{T}& T   &   F &\TTbf{T}& T   &   T& \TTbf{F}& F& T&\checkmark   & \TTbf{T}\\ 
F& T& F &   F &\TTbf{T}& T   &   F &\TTbf{T}& F   &   T& \TTbf{T}& T& F &\ding{55} & \TTbf{F}\\ 
F& F& T &   F &\TTbf{F}& F   &   F &\TTbf{T}& T   &   F& \TTbf{T}& F& T &\checkmark  & \TTbf{T}\\ 
F& F& F &   F &\TTbf{F}& F   &   F &\TTbf{T}& F   &   F& \TTbf{T}& T& F &\checkmark  & \TTbf{F}\\
\end{tabular}
\end{center}

\end{earg}


\section{`$\proves$' versus `$\eif$'}

When using truth tables to determine whether an argument is valid, it may help you to notice a similarity between `$\proves$' and `$\eif$'. As you know, a conditional is true under every circumstance except when the antecedent is true and the consequent if false. (So, when we have a `T' under the antecedent and an `F' under the consequent, we put an `F' under the `$\eif$'.) Meanwhile, in an argument, when all of  the premises are true and the conclusion is false, the argument is invalid. (So, for a specific line, when we have a `T' under every premise and an `F' under the conclusion, we put a `\ding{55} under the `$\proves$'.) 

The reasoning here is similar. In both cases, we are violating the principle---of either the conditional or of a valid argument---when we have a false sentence that follows from a sentence or a set of sentences that are all true. Thus, if $\meta{A} \eif \meta{C}$ is false, then $\meta{A} \proves \meta{C}$ is invalid, and vice versa. Conversely, whenever $\meta{A} \eif \meta{C}$ is true, then $\meta{A} \proves \meta{C}$ is valid (and vice versa). (There's much more to say about this, but I will just refer you back to  \S\ref{s:Valid-def} and p.~\pageref{characteristic-tt-conditional} in chapter \ref{s:CharacteristicTruthTables}.)


\section{The limits of these tests}\label{s:ParadoxesOfMaterialConditional}
We have reached an important milestone: a test for the validity of arguments! It is, however, important to understand the limits of this achievement. We will illustrate these limits with three examples.

First, consider the argument: 
	\begin{earg}
		\item Daisy has four legs. Therefore, Daisy has more than two legs.
	\end{earg}
To symbolize this argument in TFL, we would have to use two different atomic sentences---perhaps `$F$' for the premise  and `$T$' for the conclusion. The English version of this argument is clearly valid, but `$F \proves T$' is just as clearly invalid. 

Second, consider the sentence:
	\begin{earg}
\setcounter{eargnum}{1}
		\item\label{n:JohnBald} John is neither bald nor not-bald.
	\end{earg}
To symbolize this sentence in TFL, we would offer something like `$\enot J \eand \enot \enot J$'. This a contradiction (check this with a truth-table), but sentence \ref{n:JohnBald} does not seem like a contradiction; for we might have happily gone on to add ``John is on the borderline of baldness''!

Third, consider the following sentence:
	\begin{earg}
\setcounter{eargnum}{2}	
		\item\label{n:GodParadox}	It's not the case that, if God exists, he answers malevolent prayers.
%	Aaliyah wants to kill Zebedee. She knows that, if she puts chemical A into Zebedee's water bottle, Zebedee will drink the contaminated water and die. Equally, Bathsehba wants to kill Zebedee. She knows that, if she puts chemical B into Zebedee's water bottle, then Zebedee will drink the contaminated water and die. But chemicals A and B neutralize each other; so that if both are added to the water bottle, then Zebedee will not die.
	\end{earg}
        Symbolizing this in TFL, we would offer something like `$\enot (G \eif M)$'. Now, `$\enot (G \eif M)$' entails `$G$' (again, check this with a truth table). So if we symbolize sentence \ref{n:GodParadox} in TFL, it seems to entail that God exists. But that's strange: surely even an atheist can accept sentence \ref{n:GodParadox}, without contradicting herself!

        One lesson of this is that the symbolization of \ref{n:GodParadox} as `$\enot(G \eif M)$' shows that \ref{n:GodParadox} does not express what we intend. Perhaps we should rephrase it as
        	\begin{earg}
                  \setcounter{eargnum}{2}	
                \item\label{n:GodParadox2} If God exists, he does not answer malevolent prayers.
  \end{earg}
and symbolize \ref{n:GodParadox2} as `$G \eif \enot M$'.  Now, if atheists are right, and there is no God, then `$G$' is false and so `$G \eif \enot M$' is true, and the puzzle disappears. However, if `$G$' is false, then `$G \eif M$' (i.e., `If God exists, he answers malevolent prayers') is \emph{also} true!
                
In different ways, these four examples highlight some of the limits of working with a language like TFL that can \emph{only} handle truth-functional connectives. Moreover, these limits give rise to some interesting questions in philosophical logic. The case of John's baldness (or otherwise) raises the general question of what logic we should use when dealing with \emph{vague} discourse. The case of the atheist raises the question of how to deal with the (so-called) \emph{paradoxes of the material conditional}. Part of the purpose of this course is to equip you with the tools to explore these questions of \emph{philosophical logic}. But we have to walk before we can run; and  so we have to become proficient using TFL, before we can adequately discuss its limits and consider alternatives. 



\practiceproblems
\problempart
Revisit your answers to the exercises in part A of chapter \ref{s:CompleteTruthTables}, and determine which sentences were tautologies, which were contradictions, and which were neither tautologies nor contradictions.
\solutions

\

\problempart
\label{pr.TT.consistent}
Use truth tables to determine whether these sentences are jointly consistent, or jointly inconsistent:
\begin{earg}
\item $A\eif A$, $\enot A \eif \enot A$, $A\eand A$, $A\eor A$ %consistent
\item $A\eor B$, $A\eif C$, $B\eif C$ %consistent
\item $B\eand(C\eor A)$, $A\eif B$, $\enot(B\eor C)$  %inconsistent
\item $A\eiff(B\eor C)$, $C\eif \enot A$, $A\eif \enot B$ %consistent
\end{earg}


\solutions
\problempart
\label{pr.TT.valid}
Use truth tables to determine whether each argument is valid or invalid.
\begin{earg}
\item $A\eif A \therefore A$ %invalid
\item $A\eif(A\eand\enot A) \therefore \enot A$ %valid
\item $A\eor(B\eif A) \therefore \enot A \eif \enot B$ %valid
\item $A\eor B, B\eor C, \enot A \therefore B \eand C$ %invalid
\item $(B\eand A)\eif C, (C\eand A)\eif B \therefore (C\eand B)\eif A$ %invalid
\end{earg}

\problempart Determine whether each sentence is a tautology, a contradiction, or a contingent sentence, using a complete truth table.
\begin{earg}
\item $\enot B \eand B$ \vspace{.5ex}%contra


\item $\enot D \eor D$ \vspace{.5ex}%taut


\item $(A\eand B) \eor (B\eand A)$\vspace{.5ex} %contingent


\item $\enot[A \eif (B \eif A)]$\vspace{.5ex} %contra


\item $A \eiff [A \eif (B \eand \enot B)]$ \vspace{.5ex}%contra


\item $[(A \eand B) \eiff B] \eif (A \eif B)$ \vspace{.5ex}% contingent. 

\end{earg}



\noindent\problempart
\label{pr.TT.equiv}
Determine whether each the following sentences are logically equivalent using complete truth tables. If the two sentences really are logically equivalent, write ``equivalent.'' Otherwise write, ``Not equivalent.'' 
\begin{earg}
\item $A$ and $\enot A$
\item $A \eand \enot A$ and $\enot B \eiff B$
\item $[(A \eor B) \eor C]$ and $[A \eor (B \eor C)]$
\item $A \eor (B \eand C)$ and $(A \eor B) \eand (A \eor C)$
\item $[A \eand (A \eor B)] \eif B$ and $A \eif B$\end{earg}


\problempart
\label{pr.TT.equiv2}
Determine whether each the following sentences are logically equivalent using complete truth tables. If the two sentences really are equivalent, write ``equivalent.'' Otherwise write, ``not equivalent.''
\begin{earg}
\item $A\eif A$ and $A \eiff A$
\item $\enot(A \eif B)$ and $\enot A \eif \enot B$
\item $A \eor B$ and $\enot A \eif B$
\item$(A \eif B) \eif C$ and $A \eif (B \eif C)$
\item $A \eiff (B \eiff C)$ and $A \eand (B \eand C)$
\end{earg}


\problempart
\label{pr.TT.consistent2}
Determine whether each collection of sentences is jointly consistent or jointly inconsistent using a complete truth table. 
\begin{earg}
\item $A \eand \enot B$, $\enot(A \eif B)$, $B \eif A$\vspace{.5ex} %Consistent

%\begin{tabular}{ccccccccccccccc} 
%1. 	&	A 					 & \eand 		&  \enot & B & & \enot  		& 	 (A	  & 	 \eif	 	 & 	 B)		 & 	 & 	 B	 	 & 	\eif 	 	 & 	A 	 	 & 	 Consistent \\ 
%\cline{2-5} \cline{7-10}\cline{12-14} 
%	& 	T 					 & 	 F	 		&  F	 & T & & F	 		& 	 T	  & 	 T	 	 & 	T 	 	 & 	 & 	 T	 	 & 	 T	 	 & T	 	 	&	  \\ 
%\cline{2-14}
%	& \multicolumn{1}{|r}{T}& 	\textbf{T}	 & T	 & F & & \textbf{T}	 & 	 T	 & 	 F	 	 & 	 F	 	 & 	 & 	 F	 	 & 	 \textbf{T}	 	 & 	 \multicolumn{1}{r|}{T}	 	 & 	  \\ 
%\cline{2-14}
%	& 	 F	 				 & 	 F	 & 	 F	 & T & 	& 	 F	 & 	 F	 & 	 T	 	 & 	 T	 	 & 	  & 	 T	 	 & 	 F	 	 & 	 F	 	 & 	  \\ 
%	& 	 F	  				& 	 F	 & 	 T	 & 	F&  & 	 F	 & 	 F	 & 	 T	 	 & 	 F	 	 & 	  & 	 F	 	 & 	 T	 	 & 	 F	 	 & 	  \\ 
%\end{tabular}

\item $A \eor B$, $A \eif \enot A$, $B \eif \enot B$ \vspace{.5ex}%inconsistent. 

%\begin{tabular}{ccccccccccccccc} 
%2. &A	 & \eor 	 & B 	 & 	 	 & A 	 & \eif 	 & 	\enot & A 	 & 	 	 & B 	 & \eif 	 & \enot	 & 	B 	 & 	Inconsistent \\ 
%\cline{2-4}\cline{6- 9} \cline{11-14}
%   &	T	 & 	 T	 &T  	 & 	 	 & T	 & 	 F	 & 	F 	 & T 	 & 	 	 & 	T 	 & 	F 	 & 	 F	 & 	T 	 & 	 \\ 
%   &	 T	& 	 T	 & F 	 & 	 	 & 	T 	 & 	 F	 & 	 F	 & 	 T	 & 	 	 & 	F 	 & 	 T	 & 	 T	 & 	 F	 & 	 \\ 
%   &	 F	& 	 T	 & 	 T	 & 	 	 & 	F 	 & 	 T	 & 	 T	 & 	F 	 & 	 	 & 	 T	 & 	 F	 & 	 F	 & 	 T	 & 	 \\ 
%   &	 F	& 	 F	 & 	 F	 & 	 	 & 	 F	 & 	 T	 & 	 T	 & 	 F	 & 	 	 & 	 F	 & 	 T	 & 	 T	 & 	 F	 & 	 \\ 
%\end{tabular}

\item $\enot(\enot A \eor B) $, $A \eif \enot C$, $A \eif (B \eif C)$\vspace{.5ex} %Inconsistent

%3. &\enot & (\enot & A & \eor &B) &  &A  & \eif 	 &\enot 	 &C & 	 & A &\eif 	& (B 	 &\eif 	& C)	 &Consistent \\ 
%\cline{2-6}\cline{8-11} \cline{13-17} 
%   &	F 	& 	F	 & 	T & T	 & T & 	  & T & F	 & 	 F&T 	 & 	 &T & T	 & T	 &T 	 &T 	 & \\ 
%   &	 F	& 	F	 & 	T & T	 & T & 	  & T & T	 & 	 T& F	 & 	 &T & F	 & T	 & F	 &F 	 & \\ 
% 
%  &	 T & 	F 	& 	T & F	 & F & 	  & T & F	 & 	 F& T	 & 	 &T & T	 & F	 & T	 &T 	 & \\ 
%\cline{2-17}
%   &	 \multicolumn{1}{|r}{{\color{red}T}}		&  F	 & 	T & F	 & 	F &  & 	T & {\color{red}T}	 & 	 T&F 	& 	 &T & {\color{red}T}	 & F	 & T	 &\multicolumn{1}{r|}{F} 	 & \\ 
%\cline{2-17}
%   &	 F	& 	T	 & 	F & T	 & 	T &  & 	F & T	 & 	 F& T	 & 	 &F	 & F	 & T	 & T	 &T 	 & \\ 
%   &	 F	& 	 T	& 	F & T	 & 	T &  & 	F & T	 & 	T & F 	& 	 &F	 & T	 & T	 &F 	 &F 	 & \\ 
%   &	 F	& 	 T	& 	F & T	 & 	F &  & 	F & T	 & 	F & T	 & 	 &F	 & T	 & F	 & T	 &T 	 & \\ 
%   &	 F	& 	 T	& 	F & T	 & 	F &  & 	F & T	 & 	T & F	 & 	 &F	 & T	 & F	 & T	 &F 	 & \\ 
%\end{tabular}
%


\item $A \eif B$, $A \eand \enot B$\vspace{.5ex} %Inconsistent

\item $A \eif (B \eif C)$, $(A \eif B) \eif C$, $A \eif C$\vspace{.5ex} % consistent. 

\end{earg}

\noindent\problempart
\label{pr.TT.consistent3}
Determine whether each collection of sentences is jointly consistent or jointly inconsistent, using a complete truth table. 
\begin{earg}
\item $\enot B$, $A \eif B$, $A$ \vspace{.5ex}%inconsistent.
\item $\enot(A \eor B)$, $A \eiff B$, $B \eif A$\vspace{.5ex} %Consistent
\item $A \eor B$, $\enot B$, $\enot B \eif \enot A$\vspace{.5ex} %Inconsistent
\item $A \eiff B$, $\enot B \eor \enot A$, $A \eif B$\vspace{.5ex} %consistent. 
\item $(A \eor B) \eor C$, $\enot A \eor \enot B$, $\enot C \eor \enot B$\vspace{.5ex} %consistent
\end{earg}




\noindent\problempart
\label{pr.TT.valid2}
Determine whether each argument is valid or invalid, using a complete truth table. 
\begin{earg}
\item $A\eif B$, $B \therefore  A$ %invalid

\item $A\eiff B$, $B\eiff C \therefore A\eiff C$ %valid

\item $A \eif B$, $A \eif C\therefore B \eif C$ %invalid. 

\item $A \eif B$, $B \eif A\therefore A \eiff B$ %valid. 

\end{earg}

\noindent\problempart
\label{pr.TT.valid3}
Determine whether each argument is valid or invalid, using a complete truth table. 
\begin{earg}
\item $A\eor\bigl[A\eif(A\eiff A)\bigr] \therefore  A $\vspace{.5ex}%invalid
\item $A\eor B$, $B\eor C$, $\enot B \therefore A \eand C$\vspace{.5ex} %valid
\item $A \eif B$, $\enot A\therefore \enot B$ \vspace{.5ex}%invalid
\item $A$, $B\therefore \enot(A\eif \enot B)$ \vspace{.5ex}%valid
\item $\enot(A \eand B)$, $A \eor B$, $A \eiff B\therefore C$ \vspace{.5ex}%valid 
\end{earg}

\solutions
\problempart
\label{pr.TT.concepts}
Answer each of the questions below and justify your answer.
\begin{earg}
\item Suppose that \meta{A} and \meta{B} are logically equivalent. What can you say about $\meta{A}\eiff\meta{B}$?
%\meta{A} and \meta{B} have the same truth value on every line of a complete truth table, so $\meta{A}\eiff\meta{B}$ is true on every line. It is a tautology.
\item Suppose that $(\meta{A}\eand\meta{B})\eif\meta{C}$ is neither a tautology nor a contradiction. What can you say about whether $\meta{A}, \meta{B} \therefore\meta{C}$ is valid?
%The sentence is false on some line of a complete truth table. On that line, \meta{A} and \meta{B} are true and \meta{C} is false. So the argument is invalid.
\item Suppose that $\meta{A}$, $\meta{B}$ and $\meta{C}$  are jointly inconsistent. What can you say about $(\meta{A}\eand\meta{B}\eand\meta{C})$?
\item Suppose that \meta{A} is a contradiction. What can you say about whether $\meta{A}, \meta{B} \entails \meta{C}$?
%Since \meta{A} is false on every line of a complete truth table, there is no line on which \meta{A} and \meta{B} are true and \meta{C} is false. So the argument is valid.
\item Suppose that \meta{C} is a tautology. What can you say about whether $\meta{A}, \meta{B}\entails \meta{C}$?
%Since \meta{C} is true on every line of a complete truth table, there is no line on which \meta{A} and \meta{B} are true and \meta{C} is false. So the argument is valid.
\item Suppose that \meta{A} and \meta{B} are logically equivalent. What can you say about $(\meta{A}\eor\meta{B})$?
%Not much. $(\meta{A}\eor\meta{B})$ is a tautology if \meta{A} and \meta{B} are tautologies; it is a contradiction if they are contradictions; it is contingent if they are contingent.
\item Suppose that \meta{A} and \meta{B} are \emph{not} logically equivalent. What can you say about $(\meta{A}\eor\meta{B})$?
%\meta{A} and \meta{B} have different truth values on at least one line of a complete truth table, and $(\meta{A}\eor\meta{B})$ will be true on that line. On other lines, it might be true or false. So $(\meta{A}\eor\meta{B})$ is either a tautology or it is contingent; it is \emph{not} a contradiction.
\end{earg}
\problempart 
Consider the following principle:
	\begin{ebullet}
		\item Suppose $\meta{A}$ and $\meta{B}$ are logically equivalent. Suppose an argument contains $\meta{A}$ (either as a premise, or as the conclusion). The validity of the argument would be unaffected, if we replaced $\meta{A}$ with $\meta{B}$.
	\end{ebullet}
Is this principle correct? Explain your answer.



%%%%%%%%%%%%%%%%%%%%%%%%%%%%%%%%%%%%%%
%%%%%%%%%%%%%%%%%%%%%%%%%%%%%%%%%%%%%%

% Chapter 12

%%%%%%%%%%%%%%%%%%%%%%%%%%%%%%%%%%%%%%
%%%%%%%%%%%%%%%%%%%%%%%%%%%%%%%%%%%%%%

\chapter{Truth table shortcuts}
With practice, you will quickly become adept at filling out truth tables. In this section, we want to give you some permissible shortcuts to help you along the way. 

\section{Testing for validity}

As we said in \S\ref{s:tt-validity}, when we use truth tables to test for validity, we are checking for \emph{bad} lines: lines where the premises are all true and the conclusion is false. Consequently,
	\begin{earg}
		\item[\textbullet] Any line where the conclusion is true is not a bad line. 
		\item[\textbullet] Any line where some premise is false is not a bad line. 
	\end{earg}
Since \emph{all} we are doing is looking for bad lines, if we find a line where the conclusion is true, we do not need to evaluate anything else on that line. That line definitely isn't bad. Likewise, if we find a line where some premise is false, we do not need to evaluate anything else on that line. 

With this in mind, consider how we might test the following for validity:
	$$\enot L \eif (J \eor L), \enot L \proves J$$
The \emph{first} thing we should do is evaluate the conclusion. If we find that the conclusion is \emph{true} on some line, then that is not a bad line, and so we can simply ignore the rest of the line. After evaluating that much, we are left with something like this:
\begin{center}
\begin{tabular}{c c|d e e e e f 		  d f   c  c}
$J$&$L$&\enot&$L$&\eif&$(J$&\eor&$L)$&\enot&$L$,&\proves&$J$\\
\hline
 T & T & &&&&&&&&\checkmark& {T}\Tstrut\\
 T & F & &&&&&&&&\checkmark& {T}\\
 F & T & &&?&&&&?&&?& {F}\\
 F & F & &&?&&&&?&&?& {F}
\end{tabular}
\end{center}
The blank spaces under $\enot L \eif (J \eor L)$ and $\enot L$  indicate that we are not going to bother doing any more investigation (since the line is not bad). The question-marks indicate that we need to keep investigating. On those lines, it is possible that the premises are true and the conclusion is false. 

The easiest premise to evaluate is the second ($\enot L$), so we do that next:
\begin{center}
\begin{tabular}{c c|d e e e e f    d f  c  c}
$J$&$L$&\enot&$L$&\eif&$(J$&\eor&$L)$,&\enot&$L$&\proves&$J$\\
\hline
 T & T & &&&&&&&&\checkmark& {T}\Tstrut\\
 T & F & &&&&&&&&\checkmark& {T}\\
 F & T & &&&&&&{F}&&\checkmark& {F}\\
 F & F & &&?&&&&{T}&&?& {F}
\end{tabular}
\end{center}
Now we see that we no longer need to consider the third line. It will not be a bad line, because at least one of the premises is false on that line, $\enot L$. Finally, we complete the fourth line:
\begin{center}
\begin{tabular}{c c|d e e e e f    d f   c  c}
$J$&$L$&\enot&$L$&\eif&$(J$&\eor&$L)$,&\enot&$L$&\proves&$J$\\
\hline
 T & T & &&&&&&&&\checkmark& {T}\Tstrut\\
 T & F & &&&&&&&&\checkmark& {T}\\
 F & T & &&&&&&{F}& &\checkmark& {F}\\
 F & F & T &  & \TTbf{F} &  & F & & {T} & &\checkmark& {F}
\end{tabular}
\end{center}
Since the fourth line tells us that---for those valuations of $J$ and $L$---the first premise is false, the truth table has no bad lines. Hence, the argument is valid: any valuation for which all the premises are true is a valuation for which the conclusion is true.

It might be worth illustrating the tactic again. Let us check whether the following argument is valid
$$A\eor B, \enot (A\eand C), \enot (B \eand \enot D) \proves (\enot C\eor D)$$
At the first stage, we determine the truth value of the conclusion. Since this is a disjunction, it is true whenever either disjunct is true, so we can speed things along a bit. We can then ignore every line apart from the few lines where the conclusion is false. (Notice that the negation in the conclusion is determined for just those lines were $D$ is false.)
\begin{center}
\begin{tabular}[t]{c c c c | c   c   c  c  d e e f }
$A$ & $B$ & $C$ & $D$ & $A\eor B$, & $\enot (A\eand C)$, & $\enot (B\eand \enot D)$&\proves & $(\enot$ &$C$& $\eor$ & $D)$\\
\hline
T & T & T & T & & &			&\checkmark		&  &&  \TTbf{T} & \Tstrut\\
T & T & T & F & ? & ? & ? 	&?		& F & &  \circled{\TTbf{F}} & \\
T & T & F & T &  & &			&\checkmark  & & &  \TTbf{T} & \\
T & T & F & F &  &  &   		&\checkmark	& T & &  \TTbf{T} &\\
T & F & T & T &  &  &  		&\checkmark	& & &  \TTbf{T} & \\
T & F & T & F & ? & ? & ?  	&?	& F &  &  \circled{\TTbf{F}} &\\
T & F & F & T & & & 			&\checkmark	& & & \TTbf{T} &\\
T & F & F & F & & & 			&\checkmark	& T &  & \TTbf{T} & \\
F & T & T & T & & & 			&\checkmark	& & & \TTbf{T} & \\
F & T & T & F & ? & ? & ? 	&?	& F &  & \circled{\TTbf{F}} &\\
F & T & F & T & & &  			&\checkmark	& & & \TTbf{T} & \\
F & T & F & F & & & 			&\checkmark	&T & & \TTbf{T} & \\
F & F & T & T & & & 			&\checkmark	& & & \TTbf{T} & \\
F & F & T & F & ? & ? & ? 	&?	& F & & \circled{\TTbf{F}} & \\
F & F & F & T & & & 			&\checkmark	& & & \TTbf{T} & \\
F & F & F & F & & & 			&\checkmark	& T& & \TTbf{T} & \\
\end{tabular}
\end{center}
We must now evaluate the premises. The first premise is the simplest, and so we start there. Of the four lines where the conclusion is false, there are three where $A \eor B$ is true. So the truth values for the next premise have to be determined for those three lines. (The second premise is simpler to evaluate than the third, so it's next.) Knowing the truth value for $\enot(A \eand C)$ leaves us with one line where the first two premise are true. A little bit more work tells us that the third premise is false on that line. There is no line where the premises are true and the conclusion is false! The argument is valid.
\begin{center}
\begin{tabular}[t]{c c c c | d e f    d e e f    d e e e f    c    d e e f }
$A$ & $B$ & $C$ & $D$ & $A$ & $\eor$ & $B$, & $\enot$ & $(A$ &$\eand$ &$ C)$, & $\enot$ & $(B$ & $\eand$ & $\enot$ & $D)$&\proves & $(\enot$ &$C$& $\eor$ & $D)$\\
\hline
T & T & T & T & & && & && & && & & 								&\checkmark	& &  &  \TTbf{T} & \Tstrut\\ 
T & T & T & F & &\TTbf{T}& & \TTbf{F}& &T& & & & & & 		&\checkmark	& F & &  \TTbf{F} & \\
T & T & F & T & & && & && & &&  & &   							&\checkmark	& & &  \TTbf{T} & \\
T & T & F & F & & && & && & &&  &  &   							&\checkmark	& T & &  \TTbf{T} & \\
T & F & T & T & & && & && & &&  &  &  							&\checkmark	& & &  \TTbf{T} & \\
T & F & T & F & &\TTbf{T}& &\TTbf{F}& &T& &  && & & 		&\checkmark	& F & & \TTbf{F} & \\
T & F & F & T & & && & && & && & & 								&\checkmark	& & & \TTbf{T} & \\
T & F & F & F & & && & && & && & & 								&\checkmark	& T &  & \TTbf{T} & \\
F & T & T & T& & && & && & & & & & 								&\checkmark	& & & \TTbf{T} & \\
F & T & T & F & &\TTbf{T}& & \TTbf{T}& & F& & \TTbf{F}& & T& T&  			&\checkmark	& F &  & \TTbf{F} & \\
F & T & F & T & & && & && & && & &  								&\checkmark	& & & \TTbf{T} & \\
F & T & F & F& & && & && & && & & 								&\checkmark	&T & & \TTbf{T} & \\
F & F & T & T & & && & && & && & & 								&\checkmark	& & & \TTbf{T} & \\
F & F & T & F & & \TTbf{F} & & & & & & &&  &  &  				&\checkmark	& F & & \TTbf{F} & \\
F & F & F & T & & && & && & && & & 								&\checkmark	& & & \TTbf{T} & \\
F & F & F & F & & && & && & && & & 								&\checkmark	& T& & \TTbf{T} & \\
\end{tabular}
\end{center}
If we had used no shortcuts, we would have had to write 256 `T's or `F's on this table. Using shortcuts, we only had to write 37. We have saved ourselves a \emph{lot} of work.



\section{Partial truth tables}\label{s:PartialTruthTable}

Sometimes, we do not need to know what happens on every line of a truth table. Sometimes, just a line or two will do. 

\paragraph{Tautology.} 
In order to show that a sentence is a tautology, we need to show that it is true on every valuation. That is to say, we need to know that it comes out true on every line of the truth table. So we need a complete truth table. 

To show that a sentence is \emph{not} a tautology, however, we only need one line: a line on which the sentence is false. Therefore, in order to show that some sentence is not a tautology, it is enough to provide a single valuation---a single line of the truth table---which makes the sentence false. 

Suppose that we want to show that the sentence `$(U \eand T) \eif (S \eand W)$' is \emph{not} a tautology. We set up a \define{partial truth table}:
\begin{center}
\begin{tabular}{c c c c |d e e e e e f}
$S$&$T$&$U$&$W$&$(U$&\eand&$T)$&\eif    &$(S$&\eand&$W)$\\
\hline
   &   &   &   &    &   &    &\TTbf{F}&    &   &\Tstrut\\
\end{tabular}
\end{center}
We have only left space for one line, rather than 16, since we are only looking for one line on which the sentence is false. For just that reason, we have filled in `F' for the entire sentence. 

The main logical operator of the sentence is a conditional. In order for the conditional to be false, the antecedent must be true and the consequent must be false. So we fill these in on the table:
\begin{center}
\begin{tabular}{c c c c |d e e e e e f}
$S$&$T$&$U$&$W$&$(U$&\eand&$T)$&\eif    &$(S$&\eand&$W)$\\
\hline
   &   &   &   &    &  T  &    &\TTbf{F}&    &   F &\Tstrut\\  
\end{tabular}
\end{center}
In order for the `$(U\eand T)$' to be true, both `$U$' and `$T$' must be true.
\begin{center}
\begin{tabular}{c c c c|d e e e e e f}
$S$&$T$&$U$&$W$&$(U$&\eand&$T)$&\eif    &$(S$&\eand&$W)$\\
\hline
   & T & T &   &  T &  T  & T  &\TTbf{F}&    &   F &\Tstrut\\   
\end{tabular}
\end{center}
Now we just need to make `$(S\eand W)$' false. To do this, we need to make at least one of `$S$' and `$W$' false. We can make both `$S$' and `$W$' false if we want. All that matters is that the whole sentence turns out false on this line. Making an arbitrary decision, we finish the table in this way:
\begin{center}
\begin{tabular}{c c c c|d e e e e e f}
$S$&$T$&$U$&$W$&$(U$&\eand&$T)$&\eif    &$(S$&\eand&$W)$\\
\hline
 F & T & T & F &  T &  T  & T  &\TTbf{F}&  F &   F & F\Tstrut\\  
\end{tabular}
\end{center}
We now have a partial truth table, which shows that `$(U \eand T) \eif (S \eand W)$' is not a tautology. Put otherwise, we have shown that there is a valuation which makes `$(U \eand T) \eif (S \eand W)$' false, namely, the valuation which makes `$S$' false, `$T$' true, `$U$' true and `$W$' false. 

\paragraph{Contradiction.}
Showing that something is a contradiction requires a complete truth table: we need to show that there is no valuation which makes the sentence true; that is, we need to show that the sentence is false on every line of the truth table. 

However, to show that something is \emph{not} a contradiction, all we need to do is find a valuation which makes the sentence true, and a single line of a truth table will suffice. We can illustrate this with the same example.
\begin{center}
\begin{tabular}{c c c c|d e e e e e f}
$S$&$T$&$U$&$W$&$(U$&\eand&$T)$&\eif    &$(S$&\eand&$W)$\\
\hline
  &  &  &  &   &   &   &\TTbf{T}&  &  &\Tstrut\\
\end{tabular}
\end{center}
To make the sentence true, it will suffice to ensure that the antecedent is false. Since the antecedent is a conjunction, we can just make one of them false. For no particular reason, we choose to make `$U$' false; and then we can assign whatever truth value we like to the other atomic sentences.
\begin{center}
\begin{tabular}{c c c c|d e e e e e f}
$S$&$T$&$U$&$W$&$(U$&\eand&$T)$&\eif    &$(S$&\eand&$W)$\\
\hline
 F & T & F & F &  F &  F  & T  &\TTbf{T}&  F &   F & F\Tstrut\\
\end{tabular}
\end{center}

\paragraph{Truth functional equivalence.}
To show that two sentences are logically equivalent, we must show that the sentences have the same truth value on every valuation. So this requires a  complete truth table.

To show that two sentences are \emph{not} logically equivalent, we only need to show that there is a valuation on which they have different truth values. So this requires only a one-line partial truth table: make the table so that one sentence is true and the other false.

\paragraph{Consistency.}
To show that some sentences are jointly consistent, we must show that there is a valuation which makes all of the sentences true,so this requires only a partial truth table with a single line. 

To show that some sentences are jointly inconsistent, we must show that there is no valuation which makes all of the sentence true. So this requires a complete truth table: You must show that on every row of the table at least one of the sentences is false.

\paragraph{Validity.}
To show that an argument is valid, we must show that there is no valuation which makes all of the premises true and the conclusion false. So this  requires a complete truth table.  (Likewise for entailment.)

To show that argument is \emph{invalid}, we must show that there is a valuation which makes all of the premises true and the conclusion false. So this requires only a one-line partial truth table on which all of the premises are true and the conclusion is false. (Likewise for a failure of entailment.)

\begin{table*}\centering\sffamily\footnotesize
\ra{1.25}
\begin{tabular}{@{}l l l@{}}\toprule
\textsc{To check} & \textsc{that it is} & \textsc{that it is not}\\\midrule
tautology & complete & one-line partial \\
contradiction &  complete & one-line partial \\
equivalent & complete  & one-line partial \\
consistent & one-line partial & complete \\
valid & complete & one-line partial \\
entailment & complete & one-line partial\\
\bottomrule
\end{tabular}
\caption{The kind of truth table required to check each of these logical notions.}\label{table.CompleteVsPartial}
\end{table*}




\practiceproblems
\solutions

\solutions
\problempart
\label{pr.TT.equiv3}
Use complete or partial truth tables (as appropriate) to determine whether these pairs of sentences are logically equivalent:
\begin{earg}
\item $A$, $\enot A$ %No
\item $A$, $A \eor A$ %Yes
\item $A\eif A$, $A \eiff A$ %Yes
\item $A \eor \enot B$, $A\eif B$ %No
\item $A \eand \enot A$, $\enot B \eiff B$ %Yes
\item $\enot(A \eand B)$, $\enot A \eor \enot B$ %Yes
\item $\enot(A \eif B)$, $\enot A \eif \enot B$ %No
\item $(A \eif B)$, $(\enot B \eif \enot A)$ %Yes
\end{earg}

\solutions
\problempart
\label{pr.TT.consistent4}
Use complete or partial truth tables (as appropriate) to determine whether these sentences are jointly consistent, or jointly inconsistent:
\begin{earg}
\item $A \eand B$, $C\eif \enot B$, $C$ %inconsistent
\item $A\eif B$, $B\eif C$, $A$, $\enot C$ %inconsistent
\item $A \eor B$, $B\eor C$, $C\eif \enot A$ %consistent
\item $A$, $B$, $C$, $\enot D$, $\enot E$, $F$ %consistent
\item $A \eand (B \eor C)$, $\enot(A \eand C)$, $\enot(B \eand C)$ %consistent
\item $A \eif B$, $B \eif C$, $\enot(A \eif C)$ %inconsistent
\end{earg}

\solutions
\problempart
\label{pr.TT.valid4}
Use complete or partial truth tables (as appropriate) to determine whether each argument is valid or invalid:
\begin{earg}
\item $A\eor\bigl[A\eif(A\eiff A)\bigr] \therefore A$ %invalid
\item $A\eiff\enot(B\eiff A) \therefore A$ %invalid
\item $A\eif B, B \therefore A$ %invalid
\item $A\eor B, B\eor C, \enot B \therefore A \eand C$ %valid
\item $A\eiff B, B\eiff C \therefore A\eiff C$ %valid
\end{earg}

\problempart
\label{pr.TT.TTorC3}
Determine whether each sentence is a tautology, a contradiction, or a contingent sentence. Justify your answer with a complete or partial truth table where appropriate.

% truth tables in LaTeX generated by http://www.curtisbright.com/logic/. Be sure to give him a shout out.

\begin{earg}
\item  $A \eif \enot A$ \vspace{.5ex}							

%{\color{red}
%$
%\begin{array}{c|cccc}
%A&A&\eif&\enot&A\\\hline
%T&T&\mathbf{F}&F&T\\
%F&F&\mathbf{T}&T&F
%\end{array}
%$ 
%
%Contingent	 \vspace{6pt}
%}
%	T letter, 2 connectives
\item $A \eif (A \eand (A \eor B))$ \vspace{.5ex}	

%{\color{red}
%$
%\begin{array}{cc|ccc@{}ccc@{}ccc@{}c@{}c}
%A&B&A&\eif&(&A&\eand&(&A&\eor&B&)&)\\\hline
%T&T&T&\mathbf{T}&&T&T&&T&T&T&&\\
%T&F&T&\mathbf{T}&&T&T&&T&T&F&&\\
%F&T&F&\mathbf{T}&&F&F&&F&T&T&&\\
%F&F&F&\mathbf{T}&&F&F&&F&F&F&&
%\end{array}
%$
%
%Tautology \vspace{6pt}
%}
%			2 letters, 3 connectives

\item $(A \eif B) \eiff (B \eif A)$ 	\vspace{.5ex}				%
%
%{\color{red}
%$
%\begin{array}{cc|c@{}ccc@{}ccc@{}ccc@{}c}
%a&b&(&a&\rightarrow&b&)&\leftrightarrow&(&b&\rightarrow&a&)\\\hline
%T&T&&T&T&T&&\mathbf{T}&&T&T&T&\\
%T&F&&T&F&F&&\mathbf{F}&&F&T&T&\\
%F&T&&F&T&T&&\mathbf{F}&&T&F&F&\\
%F&F&&F&T&F&&\mathbf{T}&&F&T&F&
%\end{array}
%$
%
%Contingent \vspace{6pt}
%
%}
%		2 letters, 3 connectives

\item $A \eif \enot(A \eand (A \eor B)) $	\vspace{.5ex}	

%{\color{red}
%$
%\begin{array}{cc|cccc@{}ccc@{}ccc@{}c@{}c}
%a&b&a&\rightarrow&\enot&(&a&\eand&(&a&\eor&b&)&)\\\hline
%T&T&T&\mathbf{F}&F&&T&T&&T&T&T&&\\
%T&F&T&\mathbf{F}&F&&T&T&&T&T&F&&\\
%F&T&F&\mathbf{T}&T&&F&F&&F&T&T&&\\
%F&F&F&\mathbf{T}&T&&F&F&&F&F&F&&
%\end{array}
%$
%
%Contingent	\vspace{6pt}
%
%}
%
% 2 letters, 4 connectives

\item $\enot B \eif [(\enot A \eand A) \eor B]$\vspace{.5ex} 

%{\color{red}
%$
%\begin{array}{cc|cccc@{}c@{}cccc@{}ccc@{}c}
%a&b&\enot&b&\rightarrow&(&(&\enot&a&\eand&a&)&\eor&b&)\\\hline
%T&T&F&T&\mathbf{T}&&&F&T&F&T&&T&T&\\
%T&F&T&F&\mathbf{F}&&&F&T&F&T&&F&F&\\
%F&T&F&T&\mathbf{T}&&&T&F&F&F&&T&T&\\
%F&F&T&F&\mathbf{F}&&&T&F&F&F&&F&F&
%\end{array}
%$
%Contingent	 \vspace{6pt}
%
%}
%	2 letters, 5 connectives

\item $\enot(A \eor B) \eiff (\enot A \eand \enot B)$ \vspace{.5ex}

%{\color{red}
%$
%\begin{array}{cc|cc@{}ccc@{}ccc@{}ccccc@{}c}
%a&b&\enot&(&a&\eor&b&)&\leftrightarrow&(&\enot&a&\eand&\enot&b&)\\\hline
%T&T&F&&T&T&T&&\mathbf{T}&&F&T&F&F&T&\\
%T&F&F&&T&T&F&&\mathbf{T}&&F&T&F&T&F&\\
%F&T&F&&F&T&T&&\mathbf{T}&&T&F&F&F&T&\\
%F&F&T&&F&F&F&&\mathbf{T}&&T&F&T&T&F&
%\end{array}
%$
%
%Tautology \vspace{6pt}
%}
%2 letters, 6 connectives

\item $[(A \eand B) \eand C] \eif B$\vspace{.5ex}							
%
%{\color{red}
%$
%\begin{array}{ccc|c@{}c@{}ccc@{}ccc@{}ccc}
%a&b&c&(&(&a&\eand&b&)&\eand&c&)&\rightarrow&b\\\hline
%T&T&T&&&T&T&T&&T&T&&\mathbf{T}&T\\
%T&T&F&&&T&T&T&&F&F&&\mathbf{T}&T\\
%T&F&T&&&T&F&F&&F&T&&\mathbf{T}&F\\
%T&F&F&&&T&F&F&&F&F&&\mathbf{T}&F\\
%F&T&T&&&F&F&T&&F&T&&\mathbf{T}&T\\
%F&T&F&&&F&F&T&&F&F&&\mathbf{T}&T\\
%F&F&T&&&F&F&F&&F&T&&\mathbf{T}&F\\
%F&F&F&&&F&F&F&&F&F&&\mathbf{T}&F
%\end{array}
%$
%
%Tautology \vspace{6pt}
%}
%
%3 letters, 3 connectives

\item $\enot\bigl[(C\eor A) \eor B\bigr]$\vspace{.5ex} 						
%
%{\color{red}
%$
%\begin{array}{ccc|cc@{}c@{}ccc@{}ccc@{}c}
%a&b&c&\enot&(&(&c&\eor&a&)&\eor&b&)\\\hline
%T&T&T&\mathbf{F}&&&T&T&T&&T&T&\\
%T&T&F&\mathbf{F}&&&F&T&T&&T&T&\\
%T&F&T&\mathbf{F}&&&T&T&T&&T&F&\\
%T&F&F&\mathbf{F}&&&F&T&T&&T&F&\\
%F&T&T&\mathbf{F}&&&T&T&F&&T&T&\\
%F&T&F&\mathbf{F}&&&F&F&F&&T&T&\\
%F&F&T&\mathbf{F}&&&T&T&F&&T&F&\\
%F&F&F&\mathbf{T}&&&F&F&F&&F&F&
%\end{array}
%$
%
%Contingent \vspace{6pt}
%
%}
%	 	3 letters, 3 connectives

\item $\bigl[(A\eand B) \eand\enot(A\eand B)\bigr] \eand C$ \vspace{.5ex}	
%
%{\color{red}
%$
%\begin{array}{ccc|c@{}c@{}ccc@{}cccc@{}ccc@{}c@{}ccc}
%a&b&c&(&(&a&\eand&b&)&\eand&\enot&(&a&\eand&b&)&)&\eand&c\\\hline
%T&T&T&&&T&T&T&&F&F&&T&T&T&&&\mathbf{F}&T\\
%T&T&F&&&T&T&T&&F&F&&T&T&T&&&\mathbf{F}&F\\
%T&F&T&&&T&F&F&&F&T&&T&F&F&&&\mathbf{F}&T\\
%T&F&F&&&T&F&F&&F&T&&T&F&F&&&\mathbf{F}&F\\
%F&T&T&&&F&F&T&&F&T&&F&F&T&&&\mathbf{F}&T\\
%F&T&F&&&F&F&T&&F&T&&F&F&T&&&\mathbf{F}&F\\
%F&F&T&&&F&F&F&&F&T&&F&F&F&&&\mathbf{F}&T\\
%F&F&F&&&F&F&F&&F&T&&F&F&F&&&\mathbf{F}&F
%\end{array}
%$
%
%Contradiction \vspace{6pt}
%
%}
%
%% 	3 letters, 5 connectives
%
\item $(A \eand B) ]\eif[(A \eand C) \eor (B \eand D)]$ \vspace{.5ex}		
%
%{\color{red}
%$
%\begin{array}{cccc|c@{}c@{}ccc@{}c@{}ccc@{}c@{}ccc@{}ccc@{}ccc@{}c@{}c}
%a&b&c&d&(&(&a&\eand&b&)&)&\eif&(&(&a&\eand&c&)&\eor&(&b&\eand&d&)&)\\\hline
%T&T&T&T&&&T&T&T&&&\mathbf{T}&&&T&T&T&&T&&T&T&T&&\\
%T&T&F&F&&&T&T&T&&&\mathbf{F}&&&T&F&F&&F&&T&F&F&&\\
%\end{array}
%$
%
%Contingent \vspace{6pt}
%}
%
%	4 letters, 5 connectives
\end{earg}

\noindent\problempart
\label{pr.TT.TTorC4}
Determine whether each sentence is a tautology, a contradiction, or a contingent sentence. Justify your answer with a complete or partial truth table where appropriate.
\begin{earg}
\item  $\enot (A \eor A)$\vspace{.5ex}							%	Contradiction		1 letter, 2 connectives
\item $(A \eif B) \eor (B \eif A)$\vspace{.5ex}					%	Tautology			2 letters, 2 connectives
\item $[(A \eif B) \eif A] \eif A$\vspace{.5ex}					%	Tautology			2 letters, 3 connectives
\item $\enot[( A \eif B) \eor (B \eif A)]$\vspace{.5ex}			%	Contradiction		2 letters, 4 connectives
\item $(A \eand B) \eor (A \eor B)$\vspace{.5ex} 				%	Contingent		2 letters, 5 connectives
\item $\enot(A\eand B) \eiff A$\vspace{.5ex} 					%contingent			2 letters, 3 connectives
\item $A\eif(B\eor C)$\vspace{.5ex} 							%contingent			3 letters, 2 connectives
\item $(A \eand\enot A) \eif (B \eor C)$\vspace{.5ex} 			%tautology			3 letters, 4 connectives 
\item $(B\eand D) \eiff [A \eiff(A \eor C)]$\vspace{.5ex}			%contingent			4 letters, 4 connectives
\item $\enot[(A \eif B) \eor (C \eif D)]$\vspace{.5ex} 			% Contingent. 		4 letters, 4 connectives
\end{earg}



\noindent\problempart
Determine whether each the following pairs of sentences are logically equivalent using complete truth tables. If the two sentences really are logically equivalent, write ``equivalent.'' Otherwise write, ``not equivalent.''
\begin{earg}
\item $A$ and $A \eor A$
\item $A$ and $A \eand A$
\item $A \eor \enot B$ and $A\eif B$
\item $(A \eif B)$ and $(\enot B \eif \enot A)$
\item $\enot(A \eand B)$ and $\enot A \eor \enot B$
\item $ ((U \eif (X \eor X)) \eor U)$ and $\enot (X \eand (X \eand U))$
\item $ ((C \eand (N \eiff C)) \eiff C)$ and $(\enot \enot \enot N \eif C)$
\item $[(A \eor B) \eand C]$ and $[A \eor (B \eand C)]$
\item $((L \eand C) \eand I)$ and $L \eor C$
\end{earg}


\noindent\problempart
\label{pr.TT.consistent5}
Determine whether each collection of sentences is jointly consistent or jointly inconsistent. Justify your answer with a complete or partial truth table where appropriate.
\begin{earg}
\item $A\eif A$, $\enot A \eif \enot A$, $A\eand A$, $A\eor A$ %consistent
\item $A \eif \enot A$, $\enot A \eif A$%inconsistent. 
\item $A\eor B$, $A\eif C$, $B\eif C$ %consistent
\item $A \eor B$, $A \eif C$, $B \eif C$, $\enot C$ %	Inconsistent
\item $B\eand(C\eor A)$, $A\eif B$, $\enot(B\eor C)$  %inconsistent
\item $(A \eiff B) \eif B$,  $B \eif \enot (A \eiff B)$, $A \eor B$  %	Consistent
\item $A\eiff(B\eor C)$, $C\eif \enot A$, $A\eif \enot B$ %consistent
\item  $A \eiff B$,  $\enot B \eor \enot A$,  $A \eif  B$ % Consistent
\item $A \eiff B$, $A \eif C$, $B \eif D$, $\enot(C \eor D)$ %consitent
\item $\enot (A \eand \enot B)$,  $B \eif \enot A$, $\enot B$   %Consistent
\end{earg}

\noindent\problempart Determine whether each argument is valid or invalid. Justify your answer with a complete or partial truth table where appropriate.
\label{pr.TT.valid5} 
\begin{earg}
\item $A\eif(A\eand\enot A)\therefore \enot A$% valid
\item $A \eor B$, $A \eif B$, $B \eif A \therefore  A \eiff B$  % Valid
\item $A\eor(B\eif A)\therefore \enot A \eif \enot B$ %valid
\item $A \eor B$, $A \eif B$, $ B \eif A \therefore  A \eand B$ %valid
\item $(B\eand A)\eif C$, $(C\eand A)\eif B\therefore (C\eand B)\eif A$ % invalid
\item $\enot (\enot A \eor \enot B)$, $A \eif \enot C \therefore  A \eif (B \eif C)$ % invalid.
\item $A \eand (B \eif C)$, $\enot C \eand (\enot B \eif \enot A)\therefore C \eand \enot C$ % valid
\item $A \eand B$, $\enot A \eif \enot C$, $B \eif \enot D \therefore  A \eor B$ % Invalid
\item $A \eif B\therefore (A \eand B) \eor (\enot A \eand \enot B)$ % invalid
\item $\enot A \eif B$,$ \enot B \eif C $,$ \enot C \eif A \therefore  \enot A \eif (\enot B \eor \enot C) $% Invalid

\end{earg}

\noindent\problempart Determine whether each argument is valid or invalid. Justify your answer with a complete or partial truth table where appropriate.
\label{pr.TT.valid6} 
\begin{earg}
\item $A\eiff\enot(B\eiff A)\therefore A$ % invalid
\item $A\eor B$, $B\eor C$, $\enot A\therefore B \eand C$ % invalid
\item $A \eif C$, $E \eif (D \eor B)$, $B \eif \enot D\therefore (A \eor C) \eor (B \eif (E \eand D))$ % invalid
\item $A \eor B$, $C \eif A$, $C \eif B\therefore A \eif (B \eif C)$ % invalid
\item $A \eif B$, $\enot B \eor A\therefore A \eiff B$ % valid
\end{earg}

