%!TEX root = forallx.tex
\chapter[Quick reference]{Quick reference}

%\addcontentsline{toc}{chapter}{C\ Quick Reference}
%\pagestyle{plain}
%{\LARGE \bf Quick Reference}

\section{Characteristic Truth Tables}
\label{app.CharacteristicTTs}
\bigskip

%\hfill
\begin{tabular}[t]{d | f}
\meta{A} & \enot\meta{A}\\
\hline
T & F\Tstrut\\
F & T 
\end{tabular}
\hspace{1cm}
\begin{tabular}[t]{d d | g | g | g | f}
\meta{A} & \meta{B} & \meta{A}~\eand~\meta{B} & \meta{A}~\eor~\meta{B} & \meta{A}~\eif~\meta{B} & \meta{A}~\eiff~\meta{B}\\
\hline
T & T & T & T & T & T\Tstrut\\
T & F & F & T & F & F\\
F & T & F & T & T & F\\
F & F & F & F & T & T
\end{tabular}
%\hfill

\vspace{4em}
%%%%%%%%%%%%%%%%%%%%%%%%%%


%\hfill
\noindent\begin{tabular}[t]{d | f}
\meta{A} & \enot\meta{A}\\
\hline
$\top$ & $\bot$\Tstrut\\
$\bot$ & $\top$ 
\end{tabular}
\hspace{1cm}
\begin{tabular}[t]{d d | g | g | g | f}
\meta{A} & \meta{B} & \meta{A}~\eand~\meta{B} & \meta{A}~\eor~\meta{B} & \meta{A}~\eif~\meta{B} & \meta{A}~\eiff~\meta{B}\\
\hline
$\top$ & $\top$ & $\top$ & $\top$ & $\top$ & $\top$\Tstrut\\
$\top$ & $\bot$ & $\bot$ & $\top$ & $\bot$ & $\bot$\\
$\bot$ & $\top$ & $\bot$ & $\top$ & $\top$ & $\bot$\\
$\bot$ & $\bot$ & $\bot$ & $\bot$ & $\top$ & $\top$
\end{tabular}
%\hfill

\vspace{4em}
%%%%%%%%%%%%%%%%%%%%%%%%%%

%\hfill
\noindent\begin{tabular}[t]{d | f}
\meta{A} & \enot\meta{A}\\
\hline
1 & 0\Tstrut\\
0 & 1 
\end{tabular}
\hspace{1cm}
\begin{tabular}[t]{d d | g | g | g | f}
\meta{A} & \meta{B} & \meta{A}~\eand~\meta{B} & \meta{A}~\eor~\meta{B} & \meta{A}~\eif~\meta{B} & \meta{A}~\eiff~\meta{B}\\
\hline
1 & 1 & 1 & 1 & 1 & 1\Tstrut\\
1 & 0 & 0 & 1 & 0 & 0\\
0 & 1 & 0 & 1 & 1 & 0\\
0 & 0 & 0 & 0 & 1 & 1
\end{tabular}
%\hfill

\vfill


%\section{Symbolization}
%\medskip

%\begin{center}
%\label{app.symbolization}
%\begin{tabular*}{\textwidth}{rl}
%\multicolumn{2}{c}{\textsc{Sentential Connectives}}\\ \\
%It is not the case that $P$. & $\enot P$\\
%Either $P$, or $Q$. & $(P \eor Q)$\\
%Neither $P$, nor $Q$. & $\enot(P \eor Q)$\ or \ $(\enot P \eand \enot Q)$\\
%Both $P$, and $Q$. & $(P \eand Q)$\\
%If $P$, then $Q$. & $(P \eif Q)$\\
%$P$ only if $Q$. & $(P \eif Q)$\\
%$P$ if and only if $Q$. & $(P \eiff Q)$\\
%$P$ unless $Q$. & $(P \eor Q)$\\
%\end{tabular*}
%\end{center}


%  BEGIN: Rules of proof

% change margins so that all the rules will fit
%\setlength{\topmargin}{0 in}
%\setlength{\headheight}{0 in}
%\setlength{\headsep}{0 in}
%\setlength{\textheight}{9 in}
%\setlength{\evensidemargin}{0.25 in}
%\setlength{\oddsidemargin}{0.25 in}
%\setlength{\textwidth}{6 in}

\newpage

% eliminate page numbers
%\pagestyle{empty}

\section{Rules of derivation for TFL}\label{ProofRules}

When you have what is in \textcolor{blue}{blue}, then, on a new line, you can put what is in \textcolor{red2}{red}. $m$, $n$, $p$, and $q$ stand for lines numbers. $m$ and $n$ don't have to be consecutive line numbers. The $p$ and $q$ in the negation-introduction and negation-elimination rules are consecutive line numbers.

\setlength{\columnsep}{2.25cm}
\begin{multicols}{2}
%\twocolumn

\noindent\textsc{Conjunction Intro}

\begin{proof}
	\have[m]{a}{\textcolor{blue}{\meta{A}}}
	\have[n]{b}{\textcolor{blue}{\meta{B}}}
	\have[\ ]{c}{\textcolor{red2}{\meta{A}\eand\meta{B}}} \ai{a, b}
\end{proof}

\begin{proof}
	\have[m]{a}{\textcolor{blue}{\meta{A}}}
	\have[n]{b}{\textcolor{blue}{\meta{B}}}
	\have[\ ]{c}{\textcolor{red2}{\meta{B}\eand\meta{A}}} \ai{a, b}
\end{proof}
\bigskip

\noindent\textsc{Conjunction Elim}

\begin{proof}
	\have[m]{ab}{\textcolor{blue}{\meta{A}\eand\meta{B}}}
	\have[\ ]{a}{\textcolor{red2}{\meta{A}}} \ae{ab}
\end{proof}

\begin{proof}
	\have[m]{ab}{\textcolor{blue}{\meta{A}\eand\meta{B}}}
	\have[\ ]{b}{\textcolor{red2}{\meta{B}}} \ae{ab}
\end{proof}
\bigskip



\vfill\null
\columnbreak
%%%%%%%%%%%%%%%%%%%% column break

\noindent\textsc{Disjunction Intro}

\begin{proof}
	\have[m]{a}{\textcolor{blue}{\meta{A}}}
	\have[\ ]{ab}{\textcolor{red2}{\meta{A}\eor\meta{B}}} \oi{a}
\end{proof}

\begin{proof}
	\have[m]{a}{\textcolor{blue}{\meta{A}}}
	\have[\ ]{ba}{\textcolor{red2}{\meta{B}\eor\meta{A}}} \oi{a}
\end{proof}
\bigskip


\noindent\textsc{Disjunction Elim}

\begin{proof}
	\have[m]{ab}{\textcolor{blue}{\meta{A}\eor\meta{B}}}
	\have[n]{nb}{\textcolor{blue}{\enot\meta{B}}}
	\have[\ ]{a}{\textcolor{red2}{\meta{A}}} \oe{ab,nb}
\end{proof}

\begin{proof}
	\have[m]{ab}{\textcolor{blue}{\meta{A}\eor\meta{B}}}
	\have[n]{na}{\textcolor{blue}{\enot\meta{A}}}
	\have[\ ]{b}{\textcolor{red2}{\meta{B}}} \oe{ab,nb}
\end{proof}
\bigskip

\noindent\textsc{Double Negation}

\begin{proof}
	\have[m]{a}{\textcolor{blue}{\meta{A}}}
	\have[\ ]{c}{\textcolor{red2}{\enot\enot\meta{A}}} \dn{a}
\end{proof}


\begin{comment} %%%%%%%%%% begin commented out
\begin{proof}
	\have[m]{a}{\textcolor{blue}{\enot\enot\meta{A}}}
	\have[\ ]{c}{\textcolor{red2}{\meta{A}}} \dn{a}
\end{proof}
\end{comment} %%%%%%%%%% end commented out

\newpage %%%


\noindent\textsc{Conditional Elim}

\begin{proof}
	\have[m]{ab}{\textcolor{blue}{\meta{A}\eif\meta{B}}}
	\have[n]{a}{\textcolor{blue}{\meta{A}}}
	\have[\ ]{b}{\textcolor{red2}{\meta{B}}} \ce{ab,a}
\end{proof}
\bigskip


\noindent\textsc{Biconditional Intro}

\begin{proof}
	\have[m]{a1}{\textcolor{blue}{\meta{A}\eif\meta{B}}} 
	\have[n]{a2}{\textcolor{blue}{\meta{B}\eif\meta{A}}}
	\have[\ ]{b1}{\textcolor{red2}{\meta{A}\eiff\meta{B}}} \bi{a1,a2}
	\end{proof}

\begin{proof}
	\have[m]{a1}{\textcolor{blue}{\meta{A}\eif\meta{B}}} 
	\have[n]{a2}{\textcolor{blue}{\meta{B}\eif\meta{A}}}
	\have[\ ]{b1}{\textcolor{red2}{\meta{B}\eiff\meta{A}}} \bi{a1,a2}
	\end{proof}
\bigskip


\noindent\textsc{Biconditional Elim}

\begin{proof}
	\have[m]{ab}{\textcolor{blue}{\meta{A}\eiff\meta{B}}}
	\have[n]{a}{\textcolor{blue}{\meta{B}}}
	\have[\ ]{b}{\textcolor{red2}{\meta{A}}} \be{ab,a}
\end{proof}

\begin{proof}
	\have[m]{ab}{\textcolor{blue}{\meta{A}\eiff\meta{B}}}
	\have[n]{a}{\textcolor{blue}{\meta{A}}}
	\have[\ ]{b}{\textcolor{red2}{\meta{B}}} \be{ab,a}
\end{proof}
\bigskip



\vfill\null
\columnbreak
%%%%%%%%%%%%%%%%%%%% column break


\noindent\textsc{Conditional Intro}

\nopagebreak
\begin{proof}
	\open
		\hypo[m]{a}{\textcolor{blue}{\meta{A}}} \as{}
		\have[n]{b}{\textcolor{blue}{\meta{B}}}
	\close
	\have[\ ]{ab}{\textcolor{red2}{\meta{A}\eif\meta{B}}} \ci{a-b}
\end{proof}
\bigskip


\noindent\textsc{Negation Intro}

\begin{proof}
	\open
		\hypo[m]{a}{\textcolor{blue}{\meta{A}}} \as{}
		\have[p]{b}{\textcolor{blue}{\meta{B}}}
		\have[q]{nb}{\textcolor{blue}{\enot\meta{B}}}
	\close
	\have[\ ]{na}{\textcolor{red2}{\enot\meta{A}}} \ni{a-nb}
\end{proof}
\bigskip


\noindent\textsc{Negation Elim}\label{ProofRules-end}

\begin{proof}
	\open
		\hypo[m]{na}{\textcolor{blue}{\enot\meta{A}}} \as{}
		\have[p]{b}{\textcolor{blue}{\meta{B}}}
		\have[q]{nb}{\textcolor{blue}{\enot\meta{B}}}
	\close
	\have[\ ]{a}{\textcolor{red2}{\meta{A}}} \ne{na-nb}
\end{proof}
\bigskip


\noindent\textsc{Reiteration}

\begin{proof}
	\have[m]{a}{\textcolor{blue}{\meta{A}}}
	\have[\ ]{c}{\textcolor{red2}{\meta{A}}} \reit{a}
\end{proof}


\vfill\null

%%%%%%%%%% begin commented out
\begin{comment}
\section{Derived rules for TFL}

\textsc{Dilemma}

\begin{proof}
	\have[m]{ab}{\meta{A}\eor\meta{B}}
	\have[n]{ac}{\meta{A}\eif\meta{C}}
	\have[p]{bc}{\meta{B}\eif\meta{C}}
	\have[\ ]{a}{\meta{C}} \by{DIL}{ab,ac,bc}
\end{proof}
\bigskip

\textsc{Modus Tollens}

\begin{proof}
	\have[m]{ab}{\meta{A}\eif\meta{B}}
	\have[n]{a}{\enot\meta{B}}
	\have[\ ]{b}{\enot\meta{A}} \by{MT}{ab,a}
\end{proof}
\bigskip

\textsc{Hypothetical Syllogism}

\begin{proof}
	\have[m]{ab}{\meta{A}\eif\meta{B}}
	\have[n]{bc}{\meta{B}\eif\meta{C}}
	\have[\ ]{ac}{\meta{A}\eif\meta{C}}\by{HS}{ab,bc}
\end{proof}

\end{comment}
%%%%%%%%%% end commented out

\end{multicols}

